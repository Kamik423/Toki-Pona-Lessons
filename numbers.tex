%%%%%%%%%%%%%%%%%%%%%%%%%%%%%%%%%%%%%%%%%%%%%%%%%%%%%%%%%%%%%%%%%%%%%%%%%%
\section{Numbers}
%
%%%%%%%%%%%%%%%%%%%%%%%%%%%%%%%%%%%%%%%%%%%%%%%%%%%%%%%%%%%%%%%%%%%%%%%%%%
\subsection*{Vocabulary}
%%%%%%%%%%%%%%%%%%%%%%%%%%%%%%%%%%%%%%%%%%%%%%%%%%%%%%%%%%%%%%%%%%%%%%%%%%
%
\index{\textit{ala}}
\index{\textit{wan}}
\index{\textit{tu}}
\index{\textit{luka}}
\index{\textit{nanpa}}
\index{\textit{weka}}
\index{\textit{mute}}
\index{\textit{ali}}
\index{esun}
\index{5}
\index{2}
\index{1}
\index{0}
\index{eliminate}
\index{remove}
\index{away}
\index{number}
\index{five}
\index{divide}
\index{two}
\index{unite}
\index{one}
\index{marketplace}
\index{store}
\index{20}
\index{100}
\begin{supertabular}{p{5,5cm}|ll}
ala && zero \\
wan && one; to unite \\
tu && two; to divide \\
luka && five \\
mute && many, several, very, 20 (Is rarely used.) \\
ali && all, every, complete, 100 (Is rarely used.)\\
nanpa && number, numeral, to count, to reckon     \\
weka && away; to remove, to eliminate \\
esun && marketplace, store \\
\end{supertabular} 
%
%%%%%%%%%%%%%%%%%%%%%%%%%%%%%%%%%%%%%%%%%%%%%%%%%%%%%%%%%%%%%%%%%%%%%%%%%%
\subsection*{Use Numbers Sparingly!}
%%%%%%%%%%%%%%%%%%%%%%%%%%%%%%%%%%%%%%%%%%%%%%%%%%%%%%%%%%%%%%%%%%%%%%%%%%
%
The method that you're about to learn for making higher numbers should be avoided as much as possible. 
%
%%%%%%%%%%%%%%%%%%%%%%%%%%%%%%%%%%%%%%%%%%%%%%%%%%%%%%%%%%%%%%%%%%%%%%%%%%
\index{number!cardinal}
\index{number!higher}
\subsection*{Cardinal Numbers}
%%%%%%%%%%%%%%%%%%%%%%%%%%%%%%%%%%%%%%%%%%%%%%%%%%%%%%%%%%%%%%%%%%%%%%%%%%
%
There are only few number words in Toki Pona.
When we need to make higher numbers, we combine these numbers together. 

\begin{supertabular}{p{5,5cm}|ll}
wan && 1 \\ 
tu  && 2 \\ 
tu wan && 2 + 1 = 3 \\
tu tu && 2 + 2 = 4 \\
luka && 5 \\
luka wan && 5 + 1 = 6 \\
luka tu && 5 + 2 = 7 \\
luka tu wan && 5 + 2 + 1 = 8 \\
luka tu tu && 5 + 2 + 2 = 9 \\
luka luka && 5 + 5 = 10 \\
luka luka wan && 5 + 5 + 1 = 11 \\
luka luka tu && 5 + 5 + 2 = 12 \\
luka luka tu wan && 5 + 5 + 2 + 1 = 13 \\
luka luka tu tu && 5 + 5 + 2 + 2 = 14 \\
luka luka luka && 5 + 5 + 5 = 15 \\
mute wan && 20 + 1 = 21 (Is rarely used.) \\
ali tu && 100 + 2 = 102 (Is rarely used.) \\
\end{supertabular} 

\index{adjective!numbers}
\index{number!adjective}
These numbers are added onto nouns just like \textbf{adjectives}. 
When used together with other adjectives, numbers are inserted at the end.
Only pronouns can used after numbers.
Optionally you can insert a comma before numbers. 

\begin{supertabular}{p{5,5cm}|ll}
jan, \textbf{luka tu} && 7 people \\
jan lili, \textbf{tu wan} && 3 children \\
\end{supertabular} 

As you can see, it can get very confusing if you want to talk about numbers higher than 14 or so.
However, Toki Pona is simply not intended for such high numbers. 
It is a simple language. 
%
%%%%%%%%%%%%%%%%%%%%%%%%%%%%%%%%%%%%%%%%%%%%%%%%%%%%%%%%%%%%%%%%%%%%%%%%%%
\newpage
\index{\textit{mute}}
\subsection*{Use \textit{mute}. Conserve the Numbers.}
%%%%%%%%%%%%%%%%%%%%%%%%%%%%%%%%%%%%%%%%%%%%%%%%%%%%%%%%%%%%%%%%%%%%%%%%%%
%
Okay, so it's a bad idea to use the numbers when you don't absolutely need them. 
So, instead, we use \textit{mute} for any number higher than two.

\begin{supertabular}{p{5,5cm}|ll}
jan \textbf{mute} li kama. && Many people came. \\
\end{supertabular} 

Of course, this is still pretty vague. 
\textit{mute} in the above sentence could mean 3 or it could mean 3 000. 
Fortunately, \textit{mute} is just an adjective, and so we can attach other adjectives after it. 
We have learned that you should not repeat a word. \textit{mute} and \textit{lili} are exceptions some people repeat it up to three times to represent higher numbers. 
This is not a good style. 
Better is to use \textit{mute kin}. 

\index{\textit{lili}!\textit{mute}}
\index{\textit{mute}!\textit{lili}}
\index{\textit{mute}!\textit{mute}}
\index{\textit{mute}!\textit{kin}}
\index{\textit{lili}!\textit{kin}}
\begin{supertabular}{p{5,5cm}|ll}
jan \textbf{mute kin} li kama! && Many, many, many people are coming! \\
\end{supertabular} 

\index{\textit{lili}!\textit{mute}}
\index{\textit{mute}!\textit{lili}}
More than likely, that sentence is saying that at least a thousand people are coming.  
Now suppose that you had more than two people but still not very many. 
Let's say that the number is around 4 or 5. Here's how you'd say that. 

\begin{supertabular}{p{5,5cm}|ll}
jan \textbf{mute lili} li kama. && A small amount (of) people are coming. \\
\end{supertabular} 
%
%%%%%%%%%%%%%%%%%%%%%%%%%%%%%%%%%%%%%%%%%%%%%%%%%%%%%%%%%%%%%%%%%%%%%%%%%%
\index{number!ordinal}
\subsection*{Ordinal Numbers}
%%%%%%%%%%%%%%%%%%%%%%%%%%%%%%%%%%%%%%%%%%%%%%%%%%%%%%%%%%%%%%%%%%%%%%%%%%
%
If you understood how the cardinal numbers work, the ordinal numbers only require one more step. 
Like I said, if you understood the cardinal numbers, it's easy because you just stick \textit{nanpa} in between the noun and the number. 

\begin{supertabular}{p{5,5cm}|ll}
jan \textbf{nanpa} tu tu && 4th person \\
ni li jan lili ona \textbf{nanpa} tu. && This is her second child. \\
meli mi \textbf{nanpa} wan li nasa. && My first girlfriend was crazy. \\
\end{supertabular} 
%
%%%%%%%%%%%%%%%%%%%%%%%%%%%%%%%%%%%%%%%%%%%%%%%%%%%%%%%%%%%%%%%%%%%%%%%%%%
\index{\textit{wan}!other uses}
\index{\textit{tu}!other uses}
\index{verb!\textit{wan}}
\index{verb!\textit{tu}}
\subsection*{Other Uses of \textit{wan} and \textit{tu}}
%%%%%%%%%%%%%%%%%%%%%%%%%%%%%%%%%%%%%%%%%%%%%%%%%%%%%%%%%%%%%%%%%%%%%%%%%%
%
\index{unite}
\textit{wan} can be used as a \textbf{verb}. 
It means "\textbf{to unite}". 

\index{marry}
\index{wedding}
\begin{supertabular}{p{5,5cm}|ll}
mi en meli mi li \textbf{wan}. && My girlfriend and I got married. \\
jan pali pi ma ali o wan! && Proletarians of all countries, unite! \\

\end{supertabular} 

\textbf{\textit{tu}} used as a \textbf{verb} means "\textbf{to split}" or "\textbf{to divide}". 

\index{split}
\begin{supertabular}{p{5,5cm}|ll}
o \textbf{tu} e palisa ni. && Split this stick.  \\
\end{supertabular} 

%%%%%%%%%%%%%%%%%%%%%%%%%%%%%%%%%%%%%%%%%%%%%%%%%%%%%%%%%%%%%%%%%%%%%%%%%%
\index{The Meaning of Life}
\index{42}
\subsection*{The Meaning of Life}
%%%%%%%%%%%%%%%%%%%%%%%%%%%%%%%%%%%%%%%%%%%%%%%%%%%%%%%%%%%%%%%%%%%%%%%%%%

\begin{supertabular}{p{5,5cm}|ll}
ali li seme? &&  The Ultimate Question of Life, the Universe and Everything. \\
ni li \textbf{mute mute tu}.  && The answer is 42. \\
\end{supertabular} 

%
%%%%%%%%%%%%%%%%%%%%%%%%%%%%%%%%%%%%%%%%%%%%%%%%%%%%%%%%%%%%%%%%%%%%%%%%%%
\newpage
\subsection*{Miscellaneous}
%%%%%%%%%%%%%%%%%%%%%%%%%%%%%%%%%%%%%%%%%%%%%%%%%%%%%%%%%%%%%%%%%%%%%%%%%%
%
\index{\textit{weka}}
\index{get rid of}
\index{remove}
\index{undress}
Today's word is \textit{weka}. 
As a verb, it just means "to get rid of", "to remove", etc. 

\begin{supertabular}{p{5,5cm}|ll}
o \textbf{weka} e len sina. && Remove your clothes. \\
o \textbf{weka} e jan lili, tan ni. && Remove the kid from here \\ 
ona li wile ala kute e ni. && He shouldn't hear this. \\ 
\end{supertabular} 

\index{\textit{weka}!adjective}
\index{\textit{weka}!adverb}
\index{adjective!\textit{weka}}
\index{adverb!\textit{weka}}
\textit{weka} is also used very often as an \textbf{adjective} and an \textbf{adverb}. 

\begin{supertabular}{p{5,5cm}|ll}
mi \textbf{weka}. && I was away. \\
mi wile tawa \textbf{weka}. && I want to leave. \\
\end{supertabular} 

\index{far}
\index{distant}
It can also be used to mean the equivalent of "\textbf{far}" or "\textbf{distant}". 

\begin{supertabular}{p{5,5cm}|ll}
tomo mi li \textbf{weka}, tan ni. && My house is away from here. \\
ma Elopa li \textbf{weka}, tan ma Mewika. && Europe is away from the USA. \\
\end{supertabular} 

\index{closeby}
And add \textbf{\textit{ala}} to mean that it's somewhere \textbf{closeby}. 

\begin{supertabular}{p{5,5cm}|ll}
ma Mewika li \textbf{weka ala}, tan ma Kupa. && The USA is not away from Cuba. \\
\end{supertabular} 

\index{\textit{esun}}
\index{supermarket}
\index{store}
\subsubsection*{\textit{esun}}

\begin{supertabular}{p{5,5cm}|ll}
mi nanpa e mani mi, lon esun suli. && I count my money at a supermarket. \\
\end{supertabular}
%
%%%%%%%%%%%%%%%%%%%%%%%%%%%%%%%%%%%%%%%%%%%%%%%%%%%%%%%%%%%%%%%%%%%%%%%%%%
\subsection*{Practice 16 (Answers: Page~\pageref{'numbers'})}
%%%%%%%%%%%%%%%%%%%%%%%%%%%%%%%%%%%%%%%%%%%%%%%%%%%%%%%%%%%%%%%%%%%%%%%%%%
%
Try translating these sentences.

\begin{supertabular}{p{5,5cm}|ll}
   I saw three birds.    \\ % no-dictionary
   Many people are coming. \\   % no-dictionary
   The first person is here. \\   % no-dictionary
   I own two cars.   \\ % no-dictionary
   Some (but not a lot) of people are coming. \\  % no-dictionary  
   Unite!    \\ % no-dictionary
 && \\ % no-dictionary
   mi weka e ijo tu ni.   \\ % no-dictionary
   o tu.   \\ % no-dictionary
   mi lukin e soweli luka. \\   % no-dictionary 
   mi weka.   \\ % no-dictionary
\end{supertabular}
%%%%%%%%%%%%%%%%%%%%%%%%%%%%%%%%%%%%%%%%%%%%%%%%%%%%%%%%%%%%%%%%%%%%%%%%%%
% eof
