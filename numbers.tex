%%%%%%%%%%%%%%%%%%%%%%%%%%%%%%%%%%%%%%%%%%%%%%%%%%%%%%%%%%%%%%%%%%%%%%%%%%
\section{Numbers}
%
\index{Number}
%
%%%%%%%%%%%%%%%%%%%%%%%%%%%%%%%%%%%%%%%%%%%%%%%%%%%%%%%%%%%%%%%%%%%%%%%%%%
\subsection*{Vocabulary}
%%%%%%%%%%%%%%%%%%%%%%%%%%%%%%%%%%%%%%%%%%%%%%%%%%%%%%%%%%%%%%%%%%%%%%%%%%
%
\begin{supertabular}{p{2,5cm}|ll}
%
\index{ala}
\textbf{\dots ala} && \textit{adjective numeral}: 0 \\ % no-dictionary
 && \\ % no-dictionary
%
\index{wan}
\textbf{\dots wan} && \textit{adjective numeral}: 1 \\ % no-dictionary
\textbf{wan} && \textit{noun}: unit, element, particle, part, piece \\ % no-dictionary
\textbf{wan (e \dots)} && \textit{verb transitive}: to unite, to make one \\ % no-dictionary
 && \\ % no-dictionary
%
\index{tu}
\textbf{\dots tu} && \textit{adjective numeral}: 2 \\ % no-dictionary
\textbf{tu} && \textit{noun}: duo, pair \\ % no-dictionary
\textbf{tu (e \dots)} && \textit{verb transitive}: to divide, to double, to separate, to cut in two \\ % no-dictionary
 && \\ % no-dictionary
%
\index{luka}
\textbf{\dots luka} && \textit{adjective numeral}: 5 \\ % no-dictionary
 && \\ % no-dictionary
%
\index{mute}
\textbf{\dots mute} && \textit{adjective numeral}: 20 (official Toki Pona book) \\ % no-dictionary
 && \\ % no-dictionary
%
\index{ale}
\textbf{\dots ale} && \textit{adjective numeral}: 100 (official Toki Pona book) \\ % no-dictionary
 && \\ % no-dictionary
%
\index{esun}
\textbf{\dots esun} && \textit{adjective}: commercial, trade, marketable, for sale, salable, deductible \\  % no-dictionary
\textbf{esun} && \textit{noun}: market, shop, fair, bazaar, business, transaction \\ % no-dictionary
\textbf{esun (e \dots)} && \textit{verb transitive}: to buy, to sell, to barter, to swap \\ % no-dictionary
 && \\ % no-dictionary
%
\index{nanpa}
\textbf{nanpa \dots} && \textit{adjective numeral}: To build ordinal numbers. \\ % no-dictionary
\textbf{nanpa} && \textit{noun}: number, numeral \\ % no-dictionary
\textbf{nanpa (e \dots)} && \textit{verb transitive}: to count, to reckon,  to number \\ % no-dictionary
 && \\ % no-dictionary
%
\index{weka}
\textbf{\dots weka} && \textit{adjective}: absent, away, ignored \\ % no-dictionary
\textbf{weka} && \textit{noun}: absence \\ % no-dictionary
\textbf{weka (e \dots)} && \textit{verb transitive}: to remove, to eliminate, to throw away, to get rid of \\ % no-dictionary
 && \\ % no-dictionary 
%
\index{\#}
\textbf{\#} && \textit{unofficial}: Number sign  \\ % no-dictionary
%
\end{supertabular} \\
%
%
%
%%%%%%%%%%%%%%%%%%%%%%%%%%%%%%%%%%%%%%%%%%%%%%%%%%%%%%%%%%%%%%%%%%%%%%%%%%
\newpage
%
\subsection*{Numbers Are Adjectives}
%
%%%%%%%%%%%%%%%%%%%%%%%%%%%%%%%%%%%%%%%%%%%%%%%%%%%%%%%%%%%%%%%%%%%%%%%%%%
%
Numbers can only be adjectives and not adverbs. 
As can be seen again, in Toki Pona adjectives are more complex than adverbs. 
%
%%%%%%%%%%%%%%%%%%%%%%%%%%%%%%%%%%%%%%%%%%%%%%%%%%%%%%%%%%%%%%%%%%%%%%%%%%

\subsection*{Cardinal Numbers}
%%%%%%%%%%%%%%%%%%%%%%%%%%%%%%%%%%%%%%%%%%%%%%%%%%%%%%%%%%%%%%%%%%%%%%%%%%
%
There are only few number words in Toki Pona.
However, with adjectives \textit{ala}, \textit{wan}, \textit{tu}, \textit{luka}, \textit{mute} and \textit{ale} numbers can be formed.

\begin{supertabular}{p{5,5cm}|ll}
ala && 0 \\
wan && 1 \\ 
tu  && 2 \\ 
tu wan && 2 + 1 = 3 \\
tu tu && 2 + 2 = 4 \\
luka && 5 \\
luka wan && 5 + 1 = 6 \\
luka tu && 5 + 2 = 7 \\
luka tu wan && 5 + 2 + 1 = 8 \\
luka tu tu && 5 + 2 + 2 = 9 \\
luka luka && 5 + 5 = 10 \\
luka luka wan && 5 + 5 + 1 = 11 \\
luka luka tu && 5 + 5 + 2 = 12 \\
luka luka tu wan && 5 + 5 + 2 + 1 = 13 \\
luka luka tu tu && 5 + 5 + 2 + 2 = 14 \\
luka luka luka && 5 + 5 + 5 = 15 \\
mute wan && 20 + 1 = 21 (Is rarely used.) \\
ali tu && 100 + 2 = 102 (Is rarely used.) \\
\end{supertabular} 

When numbers used together with other adjectives, numbers are inserted at the end.
Only possessive pronouns can used after numbers to build compound nouns.
You can insert unofficially a \# before numbers. 

\begin{supertabular}{p{5,5cm}|ll}
jan \# luka tu && 7 people \\
jan lili \# tu wan && 3 children \\
\end{supertabular} 

As you can see, it can get very confusing if you want to talk about numbers higher than 14 or so.
However, Toki Pona is simply not intended for such high numbers. 
It is a simple language. 
There are also natural languages that do not have larger numbers. 
For example the language of the Pirah\'{a} (\cite{www:piraha:01}). 

%
%%%%%%%%%%%%%%%%%%%%%%%%%%%%%%%%%%%%%%%%%%%%%%%%%%%%%%%%%%%%%%%%%%%%%%%%%%
%
\subsection*{Amounts}
%
\index{amounts}
%%%%%%%%%%%%%%%%%%%%%%%%%%%%%%%%%%%%%%%%%%%%%%%%%%%%%%%%%%%%%%%%%%%%%%%%%%
%
With the conjunction \textitit{en} it is possible to connect (compound) nouns or pronouns.
This can also be used to calculate totals. 

\begin{supertabular}{p{5,5cm}|ll}
kili tu en kili wan li ' kili tu wan. && Two apples and one apple are three apples. \\
kili tu tu en kili wan li ' kili seme? && Two apples and one apple is how many apples?  \\
kili seme en kili wan li ' kili \# luka && Five apples minus one apple are how many apples? \\
\end{supertabular}

%
%%%%%%%%%%%%%%%%%%%%%%%%%%%%%%%%%%%%%%%%%%%%%%%%%%%%%%%%%%%%%%%%%%%%%%%%%%
%
\subsection*{Numbers as Predicate Adjectives}
%
\index{42}
%%%%%%%%%%%%%%%%%%%%%%%%%%%%%%%%%%%%%%%%%%%%%%%%%%%%%%%%%%%%%%%%%%%%%%%%%%

\begin{supertabular}{p{5,5cm}|ll}
ali li ' seme? &&  The Ultimate Question of Life, the Universe and Everything. \\
ni li ' \# mute mute tu.  && The answer is 42. \\
\end{supertabular} 

This philosophical answer from 'The Hitchhiker's Guide to the Galaxy' shows that numbers can also be predicate adjectives. 
%
%%%%%%%%%%%%%%%%%%%%%%%%%%%%%%%%%%%%%%%%%%%%%%%%%%%%%%%%%%%%%%%%%%%%%%%%%%
% \newpage
%
\subsection*{Use the Adjective \textit{mute} for Large Numbers.}
%
\index{\textit{mute}}
%%%%%%%%%%%%%%%%%%%%%%%%%%%%%%%%%%%%%%%%%%%%%%%%%%%%%%%%%%%%%%%%%%%%%%%%%%
%

The method that you're about to learn for making higher numbers should be avoided as much as possible. 
We use the adjective \textit{mute} ('many') for large numbers. 

\begin{supertabular}{p{5,5cm}|ll}
jan mute li kama. && Many people came. \\
\end{supertabular} 

Of course, this is still pretty vague. 
The adjective \textit{mute} in the above sentence could mean 3 or it could mean 3 000. 
Fortunately, \textit{mute} is just an adjective, and so we can attach other adjectives after it. 
We have learned that you should not repeat a word. 
The adjectives \textit{mute} and \textit{lili} are exceptions some people repeat it up to three times to represent higher numbers. 
This is not a good style. 
Better is to use \textit{mute kin}. 

\begin{supertabular}{p{5,5cm}|ll}
jan mute kin li kama! && Many, many, many people are coming! \\
\end{supertabular} 

More than likely, that sentence is saying that at least a thousand people are coming.  
Now suppose that you had more than two people but still not very many. 
Let's say that the number is around 4 or 5. Here's how you'd say that. 

\begin{supertabular}{p{5,5cm}|ll}
jan mute lili li kama. && A small amount (of) people are coming. \\
\end{supertabular} 
%
%%%%%%%%%%%%%%%%%%%%%%%%%%%%%%%%%%%%%%%%%%%%%%%%%%%%%%%%%%%%%%%%%%%%%%%%%%
%
\subsection*{Ordinal Numbers}
%
%%%%%%%%%%%%%%%%%%%%%%%%%%%%%%%%%%%%%%%%%%%%%%%%%%%%%%%%%%%%%%%%%%%%%%%%%%
%
If you understood how the cardinal numbers work, the ordinal numbers only require one more step. 
Like I said, if you understood the cardinal numbers, it's easy because you just stick the adjective \textit{nanpa} in between the noun and the number. 

\begin{supertabular}{p{5,5cm}|ll}
jan nanpa tu tu && 4th person \\
ni li jan lili ona nanpa tu. && This is her second child. \\
meli mi nanpa wan li ' nasa. && My first girlfriend was crazy. \\
\end{supertabular} 

%%%%%%%%%%%%%%%%%%%%%%%%%%%%%%%%%%%%%%%%%%%%%%%%%%%%%%%%%%%%%%%%%%%%%%%%%%
%
\subsection*{The Noun \textit{wan}}
%
\index{\textit{wan}!noun}
%%%%%%%%%%%%%%%%%%%%%%%%%%%%%%%%%%%%%%%%%%%%%%%%%%%%%%%%%%%%%%%%%%%%%%%%%%
%
The noun \textit{wan} means 'unity' or also 'marriage'.

\begin{supertabular}{p{5,5cm}|ll}
mi en meli mi li ' wan. && My girlfriend and I got married. \\
\end{supertabular} 

%%%%%%%%%%%%%%%%%%%%%%%%%%%%%%%%%%%%%%%%%%%%%%%%%%%%%%%%%%%%%%%%%%%%%%%%%%
%
\subsection*{The Transitive Verb \textit{wan}}
%
\index{\textit{wan}!verb}
%%%%%%%%%%%%%%%%%%%%%%%%%%%%%%%%%%%%%%%%%%%%%%%%%%%%%%%%%%%%%%%%%%%%%%%%%%
%
The transitive verb \textit{wan} means 'unite'. 

\begin{supertabular}{p{5,5cm}|ll}
jan pali pi ma ali o wan e ona.! && Proletarians of all countries, unite! \\
\end{supertabular} 

%%%%%%%%%%%%%%%%%%%%%%%%%%%%%%%%%%%%%%%%%%%%%%%%%%%%%%%%%%%%%%%%%%%%%%%%%%
%
\subsection*{The Noun \textit{tu}}
%
\index{\textit{tu}!noun}
%%%%%%%%%%%%%%%%%%%%%%%%%%%%%%%%%%%%%%%%%%%%%%%%%%%%%%%%%%%%%%%%%%%%%%%%%%
%
The noun \textit{tu} means 'duo' or 'pair'.

\begin{supertabular}{p{5,5cm}|ll}
tu pi ona en sina pi kalama musi li ' pona. && Your music duo is good. \\ 
\end{supertabular} 

%
%%%%%%%%%%%%%%%%%%%%%%%%%%%%%%%%%%%%%%%%%%%%%%%%%%%%%%%%%%%%%%%%%%%%%%%%%%
%
\subsection*{The Transitive Verb \textit{tu}}
%
\index{\textit{tu}!verb}
%%%%%%%%%%%%%%%%%%%%%%%%%%%%%%%%%%%%%%%%%%%%%%%%%%%%%%%%%%%%%%%%%%%%%%%%%%
%
The transitive verb \textit{tu} means 'to split' or 'to divide'. 

\begin{supertabular}{p{5,5cm}|ll}
o tu e palisa ni. && Split this stick.  \\
\end{supertabular} 

%
%%%%%%%%%%%%%%%%%%%%%%%%%%%%%%%%%%%%%%%%%%%%%%%%%%%%%%%%%%%%%%%%%%%%%%%%%%
%
\subsection*{The Transitive Verb \textit{nanpa}}
%
\index{\textit{nanpa}!verb}
%%%%%%%%%%%%%%%%%%%%%%%%%%%%%%%%%%%%%%%%%%%%%%%%%%%%%%%%%%%%%%%%%%%%%%%%%%
%

\begin{supertabular}{p{5,5cm}|ll}
ona li nanpa e jan. && He counts people. \\
\end{supertabular} 
%
%
%
%
%%%%%%%%%%%%%%%%%%%%%%%%%%%%%%%%%%%%%%%%%%%%%%%%%%%%%%%%%%%%%%%%%%%%%%%%%%
%
\subsection*{Miscellaneous}
%%%%%%%%%%%%%%%%%%%%%%%%%%%%%%%%%%%%%%%%%%%%%%%%%%%%%%%%%%%%%%%%%%%%%%%%%%
%
%
%%%%%%%%%%%%%%%%%%%%%%%%%%%%%%%%%%%%%%%%%%%%%%%%%%%%%%%%%%%%%%%%%%%%%%%%%%
\subsubsection*{The Noun \textit{weka}}
%
\index{\textit{weka}!noun}
%%%%%%%%%%%%%%%%%%%%%%%%%%%%%%%%%%%%%%%%%%%%%%%%%%%%%%%%%%%%%%%%%%%%%%%%%%

\begin{supertabular}{p{5,5cm}|ll}
weka sina li ' ike, tawa mi. && Your absence is not good to me. \\
\end{supertabular} 
%
%%%%%%%%%%%%%%%%%%%%%%%%%%%%%%%%%%%%%%%%%%%%%%%%%%%%%%%%%%%%%%%%%%%%%%%%%%
\subsubsection*{The Adjective \textit{weka}}
%
\index{\textit{weka}!adjective}
%%%%%%%%%%%%%%%%%%%%%%%%%%%%%%%%%%%%%%%%%%%%%%%%%%%%%%%%%%%%%%%%%%%%%%%%%%

\begin{supertabular}{p{5,5cm}|ll}
jan weka li kama. && The absentee is coming. \\
\end{supertabular} 

%
%%%%%%%%%%%%%%%%%%%%%%%%%%%%%%%%%%%%%%%%%%%%%%%%%%%%%%%%%%%%%%%%%%%%%%%%%%
\subsubsection*{The Transitive Verb \textit{weka}}
%
\index{\textit{weka}!verb}
%%%%%%%%%%%%%%%%%%%%%%%%%%%%%%%%%%%%%%%%%%%%%%%%%%%%%%%%%%%%%%%%%%%%%%%%%%

\begin{supertabular}{p{5,5cm}|ll}
o weka e len sina. && Remove your clothes. \\
o weka e jan lili, tan ni. && Remove the kid from here \\ 
\end{supertabular} 

%
%%%%%%%%%%%%%%%%%%%%%%%%%%%%%%%%%%%%%%%%%%%%%%%%%%%%%%%%%%%%%%%%%%%%%%%%%%
\subsubsection*{The Adverb \textit{weka}}
%
\index{\textit{weka}!adverb}
%%%%%%%%%%%%%%%%%%%%%%%%%%%%%%%%%%%%%%%%%%%%%%%%%%%%%%%%%%%%%%%%%%%%%%%%%%

\begin{supertabular}{p{5,5cm}|ll}
mi tawa weka e mi. && I'm moving away. \\
o tawa weka ala e sina! && Don't move away! \\
\end{supertabular} 

%
%
%
%%%%%%%%%%%%%%%%%%%%%%%%%%%%%%%%%%%%%%%%%%%%%%%%%%%%%%%%%%%%%%%%%%%%%%%%%%
%
\subsubsection*{The Noun \textit{esun}}
%
\index{\textit{esun!noun}}
%%%%%%%%%%%%%%%%%%%%%%%%%%%%%%%%%%%%%%%%%%%%%%%%%%%%%%%%%%%%%%%%%%%%%%%%%%

\begin{supertabular}{p{5,5cm}|ll}
mi nanpa e mani mi, lon esun suli. && I count my money at a supermarket. \\
\end{supertabular}

%
%%%%%%%%%%%%%%%%%%%%%%%%%%%%%%%%%%%%%%%%%%%%%%%%%%%%%%%%%%%%%%%%%%%%%%%%%%
%
\subsubsection*{The Adjective \textit{esun}}
%
\index{\textit{esun!adjective}}
%%%%%%%%%%%%%%%%%%%%%%%%%%%%%%%%%%%%%%%%%%%%%%%%%%%%%%%%%%%%%%%%%%%%%%%%%%

\begin{supertabular}{p{5,5cm}|ll}
meli esun li pana e pan, tawa mi. && The salesgirl gives me the bread. \\
\end{supertabular}

%
%%%%%%%%%%%%%%%%%%%%%%%%%%%%%%%%%%%%%%%%%%%%%%%%%%%%%%%%%%%%%%%%%%%%%%%%%%
%
\subsubsection*{The Transitive Verb \textit{esun}}
%
\index{\textit{esun!verb}}
%%%%%%%%%%%%%%%%%%%%%%%%%%%%%%%%%%%%%%%%%%%%%%%%%%%%%%%%%%%%%%%%%%%%%%%%%%

\begin{supertabular}{p{5,5cm}|ll}
o esun ala e ilo moli! && Don't trade in guns! \\
\end{supertabular}

%
%
%
%%%%%%%%%%%%%%%%%%%%%%%%%%%%%%%%%%%%%%%%%%%%%%%%%%%%%%%%%%%%%%%%%%%%%%%%%%
\newpage
%
\subsection*{Practice (Answers: Page~\pageref{'numbers'})}
%%%%%%%%%%%%%%%%%%%%%%%%%%%%%%%%%%%%%%%%%%%%%%%%%%%%%%%%%%%%%%%%%%%%%%%%%%
%
Please write down your answers and check them afterwards. 

\begin{supertabular}{p{5,5cm}|ll}
How are ordinal numbers formed? &&   \\ % no-dictionary
Can a number be placed directly after the separator \textit{li}? &&  \\ % no-dictionary 
Which word type are used to form numbers? &&  \\ % no-dictionary
How are large numbers formed? &&  \\ % no-dictionary
Which word type can be used in a compound noun after numbers? && \\ % no-dictionary
How to make sums? &&  \\ % no-dictionary 
\end{supertabular} 

Which word types can represent the respective word in the sentence after the hyphen?
Example:

\begin{supertabular}{p{5,5cm}|ll}
pona - mi pona e ni. && transitive verb \\ % no-dictionary
\end{supertabular}

\begin{supertabular}{p{5,5cm}|ll}
nanpa - ona li ' jan nanpa wan. &&  \\ % no-dictionary
wan - mi wan. &&  \\ % no-dictionary
luka - ni li ' luka tu. && \\ % no-dictionary
luka - ni li ' luka # tu. &&  \\ % no-dictionary
nanpa - sina nanpa e kili. &&  \\ % no-dictionary
weka - sina tawa weka e sina. &&  \\ % no-dictionary
esun - o esun e ni! &&  \\ % no-dictionary
\end{supertabular}

Try to translate these sentences. 
You can use the tool \textit{Toki Pona Parser} (\cite{www:rowa:02}) for spelling and grammar check. 

\begin{supertabular}{p{5,5cm}|ll}
I saw three birds.  &&   \\ % no-dictionary
Many people are coming. &&  \\   % no-dictionary
The first person is here. && \\   % no-dictionary
I own two cars.  &&  \\ % no-dictionary
Some (but not a lot) of people are coming. && \\  % no-dictionary  
Unite!  &&   \\ % no-dictionary
Is this a part? &&  \\ % no-dictionary
\end{supertabular}

\begin{supertabular}{p{5,5cm}|ll}
mi weka e ijo tu ni. &&   \\ % no-dictionary
o tu.  &&  \\ % no-dictionary
mi lukin e soweli luka. && \\   % no-dictionary 
mi ' weka.  &&  \\ % no-dictionary
ona li sike ala sike? && \\ % no-dictionary
\end{supertabular}
%%%%%%%%%%%%%%%%%%%%%%%%%%%%%%%%%%%%%%%%%%%%%%%%%%%%%%%%%%%%%%%%%%%%%%%%%%
% eof
