%%%%%%%%%%%%%%%%%%%%%%%%%%%%%%%%%%%%%%%%%%%%%%%%%%%%%%%%%%%%%%%%%%%%%%%%%%
\section{Adjectives and Adverbs}
%%%%%%%%%%%%%%%%%%%%%%%%%%%%%%%%%%%%%%%%%%%%%%%%%%%%%%%%%%%%%%%%%%%%%%%%%%
\index{\textit{ike}}
\index{\textit{jaki}}
\index{\textit{lawa}}         
\index{\textit{len}}
\index{\textit{lili}}
\index{\textit{mute}}
\index{\textit{nasa}}
\index{\textit{seli}}
\index{\textit{sewi}}
\index{\textit{tomo}}
\index{\textit{utala}}
\index{complicated}
\index{evil}
\index{bad}
\index{trash}
\index{nasty}
\index{dirty}
\index{head}
\index{leading}
\index{main}
\index{clothing}
\index{little}
\index{a lot}
\index{many}
\index{weird}
\index{silly}
\index{stupid}
\index{crazy}
\index{drunk}
\index{strange}
\index{hot}
\index{warm}
\index{sky}
\index{superior}
\index{high}
\index{building}
\index{house}
\index{construction}
\index{fight}
\index{battle}
\index{war}
\index{burn}
\subsection*{Vocabulary}
%%%%%%%%%%%%%%%%%%%%%%%%%%%%%%%%%%%%%%%%%%%%%%%%%%%%%%%%%%%%%%%%%%%%%%%%%%
\begin{supertabular}{p{2,5cm}|ll}
\textbf{\dots ike} && \textit{adjective}: bad, negative, wrong, evil, overly complex \\ % no-dictionary
\textbf{\dots ike} && \textit{adverb}: bad, negative, wrong, evil, overly complex \\ % no-dictionary
\textbf{ike} && \textit{noun}: negativity, badness, evil \\ % no-dictionary
\textbf{ike} && \textit{verb intransitive}: to be bad, to suck \\ % no-dictionary
\textbf{ike (e \dots)} && \textit{verb transitive}: to make bad, to worsen \\ % no-dictionary
 && \\ % no-dictionary
\textbf{\dots jaki} && \textit{adjective}: dirty, gross, filthy, obscene \\ % no-dictionary
\textbf{\dots jaki} && \textit{adverb}: dirty, gross, filthy \\ % no-dictionary
\textbf{jaki} && \textit{noun}: dirt, pollution, garbage, filth, feces \\ % no-dictionary
\textbf{jaki (e \dots)} && \textit{verb transitive}: to pollute, to dirty \\ % no-dictionary
 && \\ % no-dictionary
\textbf{\dots lawa} && \textit{adjective}: main, leading, in charge \\ % no-dictionary
\textbf{\dots lawa} && \textit{adverb}: main, leading, in charge \\ % no-dictionary
\textbf{lawa} && \textit{noun}: head, mind \\ % no-dictionary
\textbf{lawa (e \dots)} && \textit{verb transitive}: to lead, to control, to rule, to steer \\ % no-dictionary
 && \\ % no-dictionary
\textbf{\dots len} && \textit{adjective}: dressed, clothed, costumed, dressed up \\ % no-dictionary
\textbf{len} && \textit{noun}: clothing, cloth, fabric, network, internet \\ % no-dictionary
\textbf{len (e \dots)} && \textit{verb transitive}: to wear, to be dressed, to dress \\ % no-dictionary
 && \\ % no-dictionary
\textbf{\dots lili} && \textit{adjective}: small, little, young, a bit, short, few, less \\ % no-dictionary
\textbf{\dots lili} && \textit{adverb}: small, little, young, a bit, short, few, less \\ % no-dictionary
\textbf{lili} && \textit{noun}: smallness, youth, immaturity \\ % no-dictionary
\textbf{lili (e \dots)} && \textit{verb transitive}: to reduce, to shorten, to shrink, to lessen \\ % no-dictionary
 && \\ % no-dictionary
\textbf{\dots mute} && \textit{adjective}: many, very, much, several, a lot, abundant, numerous, more \\ % no-dictionary
\textbf{\dots mute} && \textit{adverb}: many, very, much, several, a lot, abundant, numerous, more \\ % no-dictionary
\textbf{mute} && \textit{noun}: amount, quantity \\ % no-dictionary
\textbf{mute (e \dots)} && \textit{verb transitive}: to make many or much \\ % no-dictionary
 && \\ % no-dictionary
\textbf{\dots nasa} && \textit{adjective}: silly, crazy, foolish, drunk, strange, stupid, weird \\ % no-dictionary
\textbf{\dots nasa} && \textit{adverb}: silly, crazy, foolish, drunk, strange, stupid, weird \\ % no-dictionary
\textbf{nasa} && \textit{noun}: stupidity, foolishness, silliness, nonsense, idiocy, obtuseness, muddler \\ % no-dictionary
\textbf{nasa (e \dots)} && \textit{verb transitive}: to drive crazy, to make weird \\ % no-dictionary
 && \\ % no-dictionary
\textbf{\dots seli} && \textit{adjective}: hot, warm, cooked \\ % no-dictionary
\textbf{\dots seli} && \textit{adverb}: hot, warm, cooked \\ % no-dictionary
\textbf{seli} && \textit{noun}: fire, warmth, heat \\ % no-dictionary
\textbf{seli (e \dots)} && \textit{verb transitive}: to heat, to warm up, to cook \\ % no-dictionary
 && \\ % no-dictionary
\textbf{\dots sewi} && \textit{adjective}: superior, elevated, religious, formal \\ % no-dictionary
\textbf{\dots sewi} && \textit{adverb}: superior, elevated, religious, formal \\ % no-dictionary
\textbf{sewi} && \textit{noun}: high, up, above, top, over, on \\ % no-dictionary
\textbf{sewi} && \textit{verb intransitive}: to get up \\ % no-dictionary
\textbf{sewi (e \dots)} && \textit{verb transitive}: to lift \\ % no-dictionary
 && \\ % no-dictionary
\textbf{\dots tomo} && \textit{adjective}: urban, domestic, household \\ % no-dictionary
\textbf{\dots tomo} && \textit{adverb}: urban, domestic, household \\ % no-dictionary
\textbf{tomo} && \textit{noun}: indoor constructed space, e.g. house, home, room, building \\ % no-dictionary
\textbf{tomo (e \dots)} && \textit{verb transitive}: to build, to construct, to engineer \\ % no-dictionary
 && \\ % no-dictionary
\textbf{\dots utala} && \textit{adjective}: fighting \\ % no-dictionary
\textbf{\dots utala} && \textit{adverb}: fighting \\ % no-dictionary
\textbf{utala} && \textit{noun}: conflict, disharmony, fight, war, battle, attack, violence \\ % no-dictionary
\textbf{utala (e \dots)} && \textit{verb transitive}: to hit, to strike, to attack, to compete against \\ % no-dictionary
\end{supertabular} \\
% 
%%%%%%%%%%%%%%%%%%%%%%%%%%%%%%%%%%%%%%%%%%%%%%%%%%%%%%%%%%%%%%%%%%%%%%%%%%
\index{adjective}
\index{noun!compound}
\subsection*{Adjectives}
%%%%%%%%%%%%%%%%%%%%%%%%%%%%%%%%%%%%%%%%%%%%%%%%%%%%%%%%%%%%%%%%%%%%%%%%%%

Toki Pona has a very minimal vocabulary. 
Many words do not exist in this language. 
Therfore, we often have to combine various words together.
For example, there is no word that means "friend". 

\begin{supertabular}{p{5,5cm}|ll}
jan pona && friend (good person) \\
\end{supertabular} 

As you can see \textbf{the adjective} (which was \textit{pona} in the above example) \textbf{goes after the main noun}. 
This will undoubtedly seem incredibly awkward to you if you only speak English. 
However, many languages do this. 

\index{victim}
\index{fork}
\index{spoon}
\index{knife}
\begin{supertabular}{p{5,5cm}|ll}
jan pakala && an injured person, victim, etc. \\
ilo moku && an eating utensil (fork/spoon/knife) \\
\end{supertabular} 

\index{adjective!more than one}
You can add until three adjective to a noun. 
One adjective should not be used more than once.

\index{soldier}
\begin{supertabular}{p{5,5cm}|ll}
jan utala && soldier  \\
jan utala nasa && stupid soldier  \\
jan utala nasa \textbf{mute} && many stupid soldiers  \\
jan utala nasa \textbf{ni} && this stupid soldier  \\
\end{supertabular} 

\index{adjective!more than one!\textit{ni}}
\index{adjective!more than one!\textit{mute}}
As you might have noticed, \textbf{\textit{ni} and \textit{mute} as adjectives come at the end of the phrase}. 
The reason for this is that the phrases build as you go along, so the adjectives must be put into an organized, logical order. 
Notice the differences in these two phrases.

\index{sidekick}
\begin{supertabular}{p{5,5cm}|ll}
jan utala nasa && stupid soldier  \\
jan nasa utala && fighting fool \\
\end{supertabular}

\index{adjective!combination}
Here are some handy adjective combinations using words that you've already learned and that are fairly common.

\index{ugly}
\index{enemy}
\index{leader}
\index{child}
\index{god}
\index{adult}
\index{lover}
\index{prostitute}
\index{mud}
\index{swamp}
\index{city}
\index{town}
\index{we}
\index{us}
\index{they}
\index{them}
\index{pretty}
\index{attractive}
\index{alcohol}
\index{beer}
\index{wine}
\index{restroom}
\index{pretty}
\index{chat room}
\index{conference room}
\begin{supertabular}{p{5,5cm}|ll}
ike lukin && ugly  \\
pona lukin && pretty, attractive \\
jan ni li pona lukin && That person is pretty. \\
jan ike && enemy \\
jan lawa && leader \\
jan lili && child \\
jan sewi && saint, God, Flying Spaghetti Monster \\
jan suli && adult \\
jan unpa && lover, prostitute \\
ma telo && mud, swamp \\
ma tomo && city, town \\
mi mute && we, us \\
ona mute && they, them \\
telo nasa && alcohol, beer, wine \\
tomo telo && restroom \\
ilo suno && flashlight \\ 
\end{supertabular} 
 
%%%%%%%%%%%%%%%%%%%%%%%%%%%%%%%%%%%%%%%%%%%%%%%%%%%%%%%%%%%%%%%%%%%%%%%%%%
\index{possessive}
\index{pronoun}
\index{pronoun!\textit{mi}}
\index{pronoun!\textit{sina}}
\subsection*{Possessives}
%%%%%%%%%%%%%%%%%%%%%%%%%%%%%%%%%%%%%%%%%%%%%%%%%%%%%%%%%%%%%%%%%%%%%%%%%%

You use pronouns (my, your, his, hers, its,) like any other adjective. 
If there are several adjectives the pronoun stands at the end.

\begin{supertabular}{p{5,5cm}|ll}
tomo pona \textbf{mi} && my nice house \\
ma \textbf{sina} && your country \\
telo \textbf{ona} && his/her/its water \\
\end{supertabular} 

Other words are treated the same way. 

\begin{supertabular}{p{5,5cm}|ll}
len \textbf{jan} && somebody's clothes \\
seli \textbf{suno} && the sun's heat \\
\end{supertabular} 

%%%%%%%%%%%%%%%%%%%%%%%%%%%%%%%%%%%%%%%%%%%%%%%%%%%%%%%%%%%%%%%%%%%%%%%%%%
\index{adverb}
\subsection*{Adverbs}
%%%%%%%%%%%%%%%%%%%%%%%%%%%%%%%%%%%%%%%%%%%%%%%%%%%%%%%%%%%%%%%%%%%%%%%%%%
\textbf{The adverb simply follows the verb that it modifies}. 
You can add until three adverb to a verb. 
One adverb should not be used more than once.

\begin{supertabular}{p{5,5cm}|ll}
mi lawa \textbf{pona} e jan. && I lead people well. \\
mi utala \textbf{ike}. && I fight badly. \\
sina lukin \textbf{sewi} e suno. && You look up at the sun. \\
ona li wile \textbf{mute} e ni. && He wants that a lot. \\
mi mute li lukin \textbf{lili} e ona. && We barely saw it. \\
\end{supertabular} 

%%%%%%%%%%%%%%%%%%%%%%%%%%%%%%%%%%%%%%%%%%%%%%%%%%%%%%%%%%%%%%%%%%%%%%%%%%
\newpage
\subsection*{Practice 5 (Answers: Page~\pageref{'adjectives_compund_nouns_adverbs'})}
%%%%%%%%%%%%%%%%%%%%%%%%%%%%%%%%%%%%%%%%%%%%%%%%%%%%%%%%%%%%%%%%%%%%%%%%%%
Firstly, see how well you can read the following poem. 

\begin{supertabular}{p{5,5cm}|ll}
mi jo e kili. && \\ % no-dictionary
ona li ' pona li ' lili. && \\ % no-dictionary
mi moku lili e kili lili. && \\ % no-dictionary
\end{supertabular} 

Try translating these sentences. 

\begin{supertabular}{p{5,5cm}|ll}
The leader drank dirty water. &&   \\ % no-dictionary
I need a fork.   &&   \\ % no-dictionary
An enemy is attacking them.   &&   \\ % no-dictionary
That bad person has strange clothes.   &&  \\  % no-dictionary
We drank a lot of vodka.   &&   \\ % no-dictionary
Children watch adults.   &&   \\ % no-dictionary
 && \\ % no-dictionary
mi lukin sewi e tomo suli.  &&    \\ % no-dictionary
seli suno li seli e tomo mi.  &&   \\ % no-dictionary
jan lili li wile e telo kili.  &&  \\ % no-dictionary
ona mute li nasa e jan suli. * &&  \\ % no-dictionary
\end{supertabular} 

* Notice how even though \textit{nasa} is typically an adjective, it is used as a verb here. 
%%%%%%%%%%%%%%%%%%%%%%%%%%%%%%%%%%%%%%%%%%%%%%%%%%%%%%%%%%%%%%%%%%%%%%%%%%
% eof
