%%%%%%%%%%%%%%%%%%%%%%%%%%%%%%%%%%%%%%%%%%%%%%%%%%%%%%%%%%%%%%%%%%%%%%%%%%
%
\section{Relative Location Information}
%
%%%%%%%%%%%%%%%%%%%%%%%%%%%%%%%%%%%%%%%%%%%%%%%%%%%%%%%%%%%%%%%%%%%%%%%%%%
%
\subsection*{Vocabulary}
%%%%%%%%%%%%%%%%%%%%%%%%%%%%%%%%%%%%%%%%%%%%%%%%%%%%%%%%%%%%%%%%%%%%%%%%%%
%
\begin{supertabular}{p{2,5cm}|ll}
%
\index{anpa}
\textbf{\dots anpa} && \textit{adjective}: low, lower, bottom, down \\ % no-dictionary
\textbf{\dots anpa} && \textit{adverb}: downstairs, below, deep, low, deeply \\ % no-dictionary
\textbf{anpa} && \textit{noun}: bottom, lower part, under, below, floor, beneath \\ % no-dictionary
\textbf{anpa} && \textit{verb intransitive}: to prostrate oneself \\ % no-dictionary
\textbf{anpa (e \dots)} && \textit{verb transitive}: to defeat, to beat, to vanquish, to conquer, to enslave \\ % no-dictionary
 && \\ % no-dictionary
%
\index{insa}
\textbf{\dots insa} && \textit{adjective}: inner, internal \\ % no-dictionary
\textbf{insa} && \textit{noun}: inside, inner world, centre, stomach \\ % no-dictionary
 && \\ % no-dictionary
%
\index{monsi}
\textbf{\dots monsi} && \textit{adjective}: back, rear \\ % no-dictionary
\textbf{monsi} && \textit{noun}: back, rear end, butt, behind \\ % no-dictionary
 && \\ % no-dictionary
%
\index{noka}
\textbf{\dots noka} && \textit{adjective}: foot-, lower, bottom \\  % no-dictionary
\textbf{ \dots noka } && \textit{adverb}: on foot \\  % no-dictionary
\textbf{noka} && \textit{noun}: leg, foot; organ of locomotion; bottom, lower part \\ % no-dictionary
 && \\ % no-dictionary
%
\index{poka}
\textbf{\dots poka} && \textit{adjective}: neighbouring \\ % no-dictionary
\textbf{poka} && \textit{noun}: side, hip, next to \\ % no-dictionary
% \textbf{\dots , poka \dots} && \textit{preposition}: in the accompaniment of, with \\ % no-dictionary
 && \\ % no-dictionary
%
\index{sewi}
\textbf{\dots sewi} && \textit{adjective}: superior, elevated, religious, formal \\ % no-dictionary
\textbf{\dots sewi} && \textit{adverb}: superior, elevated, religious, formal \\ % no-dictionary
\textbf{sewi} && \textit{noun}: high, up, above, top, over, on \\ % no-dictionary
\textbf{sewi} && \textit{verb intransitive}: to get up \\ % no-dictionary
\textbf{sewi (e \dots)} && \textit{verb transitive}: to lift \\ % no-dictionary
 && \\ % no-dictionary
%
\index{sinpin}
\textbf{\dots sinpin} && \textit{adjective}: facial, frontal, anterior, vertical \\ % no-dictionary
\textbf{sinpin} && \textit{noun}: face, foremost, front, wall, chest, torso \\ % no-dictionary
\end{supertabular} \\
%
%
%%%%%%%%%%%%%%%%%%%%%%%%%%%%%%%%%%%%%%%%%%%%%%%%%%%%%%%%%%%%%%%%%%%%%%%%%%
\newpage
%
\subsection*{The Spatial Nouns \textit{anpa}, \textit{insa}, \textit{monsi}, \textit{noka}, \textit{poka}, \textit{sewi} and \textit{sinpin}}
%
\index{noun!spatial}
\index{spatial noun}
\index{\textit{anpa}!spatial noun}
\index{\textit{insa}!spatial noun}
\index{\textit{monsi}!spatial noun}
\index{\textit{noka}!spatial noun}
\index{\textit{poka}!spatial noun}
\index{\textit{sewi}!spatial noun}
\index{\textit{sinpin}!spatial noun}
%%%%%%%%%%%%%%%%%%%%%%%%%%%%%%%%%%%%%%%%%%%%%%%%%%%%%%%%%%%%%%%%%%%%%%%%%%
%

In Toki Pona relative location information is formed with special nouns. 
These special nouns are called 'spatial nouns'. 
In addition to the noun, adjectives, possessive pronouns or demonstrative pronouns are required for the relative location information. 

A spatial noun is preceded by either an intransitive verb or a preposition. 
This means that relative location information is either in an indirect object or a prepositional object and is therefore part of a predicate phrase.

%
\subsubsection*{Spatial Nouns in an Indirect Object}
%
\index{spatial noun!indirect object}
%%%%%%%%%%%%%%%%%%%%%%%%%%%%%%%%%%%%%%%%%%%%%%%%%%%%%%%%%%%%%%%%%%%%%%%%%%
%
\index{\textit{lon}!intransitive verb}
\index{\textit{lon}!preposition}
%
Usually, the word \textit{lon} is used as an intransitive verb or preposition before a spatial noun. 
If no further verb is present, \textit{lon} cannot be a preposition. 
Then \textit{lon} can only be an intransitive verb.

\begin{supertabular}{p{5,5cm}|ll}
pipi li lon anpa mi.       && The bug is underneath me. \\
telo suli li lon monsi mi. && The sea is behind me.  \\
moku li lon insa mi.       && Food is inside me. \\
ma li lon noka mi.         && Land is under my feet. \\
ona li lon sewi mi.        && He is in my above, i.e. he is above me. \\
tomo li lon sinpin mi.     && The house is in front of me. \\
\end{supertabular} 

%
\subsubsection*{Spatial Nouns in a Prepositional Object}
%
\index{spatial noun!prepositional object}
%%%%%%%%%%%%%%%%%%%%%%%%%%%%%%%%%%%%%%%%%%%%%%%%%%%%%%%%%%%%%%%%%%%%%%%%%%
%

The following examples contain a verb. 
So \textit{lon} can only be a preposition. 

\begin{supertabular}{p{5,5cm}|ll}
mi moku, lon poka sina.    && I'm eating beside [or with] you. \\
ona li pona e ilo, lon tomo ona. && He repairs the tools in his house. \\
\end{supertabular} 

Besides \textit{lon} other intransitive verbs and prepositions can also be used. 
In this sentence the second \textit{tawa} is a preposition and stands before the spatial noun \textit{noka}. 

\begin{supertabular}{p{5,5cm}|ll}
mi tawa e mi, tawa noka sina. && I bow before you. \\
\end{supertabular} 

%
%
%
%%%%%%%%%%%%%%%%%%%%%%%%%%%%%%%%%%%%%%%%%%%%%%%%%%%%%%%%%%%%%%%%%%%%%%%%%%
\subsection*{Further meanings of these words}
%
\index{\textit{poka}!adjective}
%%%%%%%%%%%%%%%%%%%%%%%%%%%%%%%%%%%%%%%%%%%%%%%%%%%%%%%%%%%%%%%%%%%%%%%%%%
%
%
%%%%%%%%%%%%%%%%%%%%%%%%%%%%%%%%%%%%%%%%%%%%%%%%%%%%%%%%%%%%%%%%%%%%%%%%%%
\subsubsection*{The transitive Verb \textit{anpa}}
%
\index{\textit{anpa}!verb}
%%%%%%%%%%%%%%%%%%%%%%%%%%%%%%%%%%%%%%%%%%%%%%%%%%%%%%%%%%%%%%%%%%%%%%%%%%

\begin{supertabular}{p{5,5cm}|ll}
mi anpa e jan utala. && I defeated the warrior. \\
\end{supertabular} 

%
%%%%%%%%%%%%%%%%%%%%%%%%%%%%%%%%%%%%%%%%%%%%%%%%%%%%%%%%%%%%%%%%%%%%%%%%%%
\subsubsection*{The 'normal' noun \textit{poka}}
%
\index{\textit{poka}!noun}
%%%%%%%%%%%%%%%%%%%%%%%%%%%%%%%%%%%%%%%%%%%%%%%%%%%%%%%%%%%%%%%%%%%%%%%%%%

\begin{supertabular}{p{5,5cm}|ll}
poka telo && water side, the beach \\
\end{supertabular} 

%
%%%%%%%%%%%%%%%%%%%%%%%%%%%%%%%%%%%%%%%%%%%%%%%%%%%%%%%%%%%%%%%%%%%%%%%%%%
\subsubsection*{The Adjektive \textit{poka}}
%
\index{\textit{poka}!adjective}
%%%%%%%%%%%%%%%%%%%%%%%%%%%%%%%%%%%%%%%%%%%%%%%%%%%%%%%%%%%%%%%%%%%%%%%%%%

\begin{supertabular}{p{5,5cm}|ll}
jan poka && neighbor, someone who is beside you \\
\end{supertabular} 

%
%
%
%%%%%%%%%%%%%%%%%%%%%%%%%%%%%%%%%%%%%%%%%%%%%%%%%%%%%%%%%%%%%%%%%%%%%%%%%%
\newpage
%
\subsection*{Practice (Answers: Page~\pageref{'other_prepositions'})}
%%%%%%%%%%%%%%%%%%%%%%%%%%%%%%%%%%%%%%%%%%%%%%%%%%%%%%%%%%%%%%%%%%%%%%%%%%
%
Please write down your answers and check them afterwards. 

\begin{supertabular}{p{5,5cm}|ll}
How do you create relative location information in Toki Pona? &&  \\ % no-dictionary
What is a possessive pronoun? && \\ % no-dictionary
Where is a slot for a substantive demonstrative pronoun possible? &&   \\ % no-dictionary
Which separator is at the end of a declarative sentence? &&  \\ % no-dictionary
What is a predicate adjective? &&  \\ % no-dictionary
Which sentence phrases can contain spatial nouns be found? &&   \\ % no-dictionary
\end{supertabular}

Try to translate these sentences. 
You can use the tool \textit{Toki Pona Parser} (\cite{www:rowa:02}) for spelling and grammar check. 

\begin{supertabular}{p{5,5cm}|ll}
My friend is beside me. && \\ % no-dictionary
The sun is above me. && \\ % no-dictionary
The land is beneath me. && \\ % no-dictionary
Bad things are behind me. && \\ % no-dictionary
I'm okay because I'm alive. * && \\ % no-dictionary
I look at the land with you. && \\ % no-dictionary
\end{supertabular}

\begin{supertabular}{p{5,5cm}|ll}
poka mi li ' pakala. && \\ % no-dictionary
mi kepeken poki li kepeken ilo moku. && \\ % no-dictionary
jan li lon insa tomo. && \\ % no-dictionary
\end{supertabular} 

* \textit{lon} as a verb by itself means to exist, to be real, etc. 
% 
%%%%%%%%%%%%%%%%%%%%%%%%%%%%%%%%%%%%%%%%%%%%%%%%%%%%%%%%%%%%%%%%%%%%%%%%%%
% eof
