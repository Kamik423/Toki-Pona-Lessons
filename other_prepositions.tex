%%%%%%%%%%%%%%%%%%%%%%%%%%%%%%%%%%%%%%%%%%%%%%%%%%%%%%%%%%%%%%%%%%%%%%%%%%
\section{Spatial Nouns}
%%%%%%%%%%%%%%%%%%%%%%%%%%%%%%%%%%%%%%%%%%%%%%%%%%%%%%%%%%%%%%%%%%%%%%%%%%
\index{preposition}
\subsection*{Vocabulary}
%%%%%%%%%%%%%%%%%%%%%%%%%%%%%%%%%%%%%%%%%%%%%%%%%%%%%%%%%%%%%%%%%%%%%%%%%%
%
% \index{\textit{kan}}
\index{\textit{sama}}
\index{\textit{tan}}
\index{\textit{anpa}} 
\index{\textit{insa}}
\index{\textit{monsi}}
\index{\textit{poka}}
\index{\textit{sewi}}
\index{\textit{pipi}}
\index{with}
\index{among}
\index{company of}
\index{same}
\index{similar}
\index{reason}
\index{cause}
\index{because of}
\index{ground}
\index{low}
\index{deep}
\index{lower}
\index{defeat}
\index{inside}
\index{stomach}
\index{rear}
\index{back}
\index{behind}
\index{side}
\index{hip}
\index{nearby}
\index{high}
\index{above}
\index{roof}
\index{top}
\index{sky}
\index{\textit{sona}}
\begin{supertabular}{p{2,5cm}|ll}
\textbf{\dots anpa} && \textit{adjective}: low, lower, bottom, down \\ % no-dictionary
\textbf{\dots anpa} && \textit{adverb}: downstairs, below, deep, low, deeply \\ % no-dictionary
\textbf{anpa} && \textit{noun}: bottom, lower part, under, below, floor, beneath \\ % no-dictionary
\textbf{anpa} && \textit{verb intransitive}: to prostrate oneself \\ % no-dictionary
\textbf{anpa (e \dots)} && \textit{verb transitive}: to defeat, to beat, to vanquish, to conquer, to enslave \\ % no-dictionary
 && \\ % no-dictionary
\textbf{\dots insa} && \textit{adjective}: inner, internal \\ % no-dictionary
\textbf{insa} && \textit{noun}: inside, inner world, centre, stomach \\ % no-dictionary
 && \\ % no-dictionary
\textbf{\dots monsi} && \textit{adjective}: back, rear \\ % no-dictionary
\textbf{monsi} && \textit{noun}: back, rear end, butt, behind \\ % no-dictionary
 && \\ % no-dictionary
\textbf{\dots noka} && \textit{adjective}: foot-, lower, bottom \\  % no-dictionary
\textbf{ \dots noka } && \textit{adverb}: on foot \\  % no-dictionary
\textbf{noka} && \textit{noun}: leg, foot; organ of locomotion; bottom, lower part \\ % no-dictionary
 && \\ % no-dictionary
\textbf{\dots poka} && \textit{adjective}: neighbouring \\ % no-dictionary
\textbf{poka} && \textit{noun}: side, hip, next to \\ % no-dictionary
% \textbf{\dots , poka \dots} && \textit{preposition}: in the accompaniment of, with \\ % no-dictionary
 && \\ % no-dictionary
\textbf{\dots sewi} && \textit{adjective}: superior, elevated, religious, formal \\ % no-dictionary
\textbf{\dots sewi} && \textit{adverb}: superior, elevated, religious, formal \\ % no-dictionary
\textbf{sewi} && \textit{noun}: high, up, above, top, over, on \\ % no-dictionary
\textbf{sewi} && \textit{verb intransitive}: to get up \\ % no-dictionary
\textbf{sewi (e \dots)} && \textit{verb transitive}: to lift \\ % no-dictionary
 && \\ % no-dictionary
\textbf{\dots sinpin} && \textit{adjective}: facial, frontal, anterior, vertical \\ % no-dictionary
\textbf{sinpin} && \textit{noun}: face, foremost, front, wall, chest, torso \\ % no-dictionary
\end{supertabular} \\
%
%
%%%%%%%%%%%%%%%%%%%%%%%%%%%%%%%%%%%%%%%%%%%%%%%%%%%%%%%%%%%%%%%%%%%%%%%%%%
\newpage
\index{spatial noun}
\index{\textit{anpa}}
\index{\textit{insa}}
\index{\textit{monsi}}
\index{\textit{poka}}
\index{\textit{sewi}}
\index{\textit{noka}}
\index{\textit{poka}}
\index{\textit{sinpin}}
\index{\textit{anpa}!\textit{noun}}
\index{\textit{insa}!\textit{noun}}
\index{\textit{monsi}!\textit{noun}}
\index{\textit{sewi}!\textit{noun}}
\index{\textit{noka}!spatial noun}
\index{\textit{poka}!spatial noun}
\index{\textit{sinpin}!spatial noun}
\index{\textit{noun}!\textit{anpa}}
\index{\textit{noun}!\textit{insa}}
\index{\textit{noun}!\textit{monsi}}
\index{\textit{noun}!\textit{sewi}}
\index{\textit{noun}!\textit{noka}}
\index{\textit{noun}!\textit{poka}}
\index{\textit{noun}!\textit{sinpin}}
\subsection*{The Spatial Nouns \textit{anpa}, \textit{insa}, \textit{monsi}, \textit{noka}, \textit{poka}, \textit{sewi} and \textit{sinpin}}
%%%%%%%%%%%%%%%%%%%%%%%%%%%%%%%%%%%%%%%%%%%%%%%%%%%%%%%%%%%%%%%%%%%%%%%%%%
%
Although you might be tempted to use these words as prepositions, but they are not. 
But they are nouns to describe a place.
They are often used with the in transitive verb \textit{lon}. 

\begin{supertabular}{p{5,5cm}|ll}
pipi li lon anpa mi.       && The bug is underneath me. \\
telo suli li lon monsi mi. && The sea is behind me.  \\
moku li lon insa mi.       && Food is inside me. \\
ma li lon noka mi.         && Land is under my feet. \\
mi moku, lon poka sina.    && I'm eating beside [or with] you. \\
ona li lon sewi mi.        && He is in my above, i.e. he is above me. \\
tomo li lon sinpin mi.     && The house is in front of me. \\
\end{supertabular} 
%

%
%%%%%%%%%%%%%%%%%%%%%%%%%%%%%%%%%%%%%%%%%%%%%%%%%%%%%%%%%%%%%%%%%%%%%%%%%%
\subsection*{Further meanings of these words}

\textit{anpa} can also be used as a transitive verb.

\begin{supertabular}{p{5,5cm}|ll}
mi anpa e jan utala. && I defeated the warrior. \\
\end{supertabular} 


\textit{poka} can be used as a 'normal' noun.

\begin{supertabular}{p{5,5cm}|ll}
poka telo && water side, the beach \\
\end{supertabular} 

\textbf{\textit{poka} as an adjective} \\
\begin{supertabular}{p{5,5cm}|ll}
jan poka && neighbor, someone who is beside you \\
\end{supertabular} 
%

%
%%%%%%%%%%%%%%%%%%%%%%%%%%%%%%%%%%%%%%%%%%%%%%%%%%%%%%%%%%%%%%%%%%%%%%%%%%
% \newpage
\subsection*{Practice (Answers: Page~\pageref{'other_prepositions'})}
%%%%%%%%%%%%%%%%%%%%%%%%%%%%%%%%%%%%%%%%%%%%%%%%%%%%%%%%%%%%%%%%%%%%%%%%%%
%
Please write down your answers and check them afterwards. 










Try to translate these sentences. 
You can use the tool \textit{Toki Pona Parser} (\cite{www:rowa:02}) for spelling and grammar check. 

\begin{supertabular}{p{5,5cm}|ll}
My friend is beside me. && \\ % no-dictionary
The sun is above me. && \\ % no-dictionary
The land is beneath me. && \\ % no-dictionary
Bad things are behind me. && \\ % no-dictionary
I'm okay because I'm alive. * && \\ % no-dictionary
I look at the land with you. && \\ % no-dictionary
 && \\ % no-dictionary
poka mi li ' pakala. && \\ % no-dictionary
mi kepeken poki li kepeken ilo moku. && \\ % no-dictionary
jan li lon insa tomo. && \\ % no-dictionary
\end{supertabular} 

* \textit{lon} as a verb by itself means to exist, to be real, etc. 
% 
%%%%%%%%%%%%%%%%%%%%%%%%%%%%%%%%%%%%%%%%%%%%%%%%%%%%%%%%%%%%%%%%%%%%%%%%%%
% eof
