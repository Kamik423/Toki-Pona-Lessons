%%%%%%%%%%%%%%%%%%%%%%%%%%%%%%%%%%%%%%%%%%%%%%%%%%%%%%%%%%%%%%%%%%%%%%%%%%
\section{Other Prepositions and Spatial Nouns}
%%%%%%%%%%%%%%%%%%%%%%%%%%%%%%%%%%%%%%%%%%%%%%%%%%%%%%%%%%%%%%%%%%%%%%%%%%
\index{preposition}
\subsection*{Vocabulary}
%%%%%%%%%%%%%%%%%%%%%%%%%%%%%%%%%%%%%%%%%%%%%%%%%%%%%%%%%%%%%%%%%%%%%%%%%%
%
% \index{\textit{kan}}
\index{\textit{sama}}
\index{\textit{tan}}
\index{\textit{anpa}} 
\index{\textit{insa}}
\index{\textit{monsi}}
\index{\textit{poka}}
\index{\textit{sewi}}
\index{with}
\index{among}
\index{company of}
\index{same}
\index{similar}
\index{reason}
\index{cause}
\index{because of}
\index{ground}
\index{low}
\index{deep}
\index{lower}
\index{defeat}
\index{inside}
\index{stomach}
\index{rear}
\index{back}
\index{behind}
\index{side}
\index{hip}
\index{nearby}
\index{high}
\index{above}
\index{roof}
\index{top}
\index{sky}
\index{\textit{sona}}
\begin{supertabular}{p{5,5cm}|ll}
% kan && with, among, in the company of \\
sama && same, similar \\
tan && reason, cause; because of, from \\
sona && to know, to know how to; wisdom \\
anpa && ground; low, deep; to lower, to defeat \\ 
insa && inside, stomach \\
monsi && rear, back, behind \\
poka && side, hip, nearby \\
sewi && high, above, roof, top, sky \\
\end{supertabular} 
%
%%%%%%%%%%%%%%%%%%%%%%%%%%%%%%%%%%%%%%%%%%%%%%%%%%%%%%%%%%%%%%%%%%%%%%%%%%
\index{\textit{same}}
\subsection*{\textit{sama}}
%%%%%%%%%%%%%%%%%%%%%%%%%%%%%%%%%%%%%%%%%%%%%%%%%%%%%%%%%%%%%%%%%%%%%%%%%%
% This word can be used for several different parts of speech, but I don't think that it's too difficult to understand. 
%
\begin{supertabular}{p{5,5cm}|ll}
jan ni li ' \textbf{sama} mi. && That person is like me. \\
ona li lukin, \textbf{sama} pipi. && He looks like a bug. \\
\textbf{sama} li ' ike. && Equality is bad. \\
\end{supertabular} 
%
%%%%%%%%%%%%%%%%%%%%%%%%%%%%%%%%%%%%%%%%%%%%%%%%%%%%%%%%%%%%%%%%%%%%%%%%%%
\index{\textit{tan}!preposition}
\index{preposition!\textit{tan}}
\subsection*{tan}
%%%%%%%%%%%%%%%%%%%%%%%%%%%%%%%%%%%%%%%%%%%%%%%%%%%%%%%%%%%%%%%%%%%%%%%%%%
\textbf{\textit{tan} as a preposition} 

\begin{supertabular}{p{5,5cm}|ll}
mi moku, \textbf{tan} ni: mi wile moku. &&  I eat because I'm hungry. \\
% mi \textbf{tan} ma ike.   &&       I'm from a bad country. \\
\end{supertabular} 

\textbf{\textit{tan} as a noun}

\begin{supertabular}{p{5,5cm}|ll}
mi sona e \textbf{tan}. && I know the reason. / I know why. \\
\end{supertabular} 
%
%
\index{\textit{tan}!nounn}
\index{nounn!\textit{tan}}

When used as a noun, tan means cause or reason. 
That can be helpful when you want to translate a sentence such as "I don't know why". 
However, you have to rephrase that sentence a little.
%
%
%
\index{\textit{poka}}
\index{\textit{poka}!\textit{noun}}
\index{\textit{noun}!\textit{poka}}
\subsection*{\textit{poka}}
%%%%%%%%%%%%%%%%%%%%%%%%%%%%%%%%%%%%%%%%%%%%%%%%%%%%%%%%%%%%%%%%%%%%%%%%%%
%
poka is rather unique in that it can act both as a noun/adjective and also as a preposition. 
% Let's look at each of these uses separately.

\textbf{\textit{poka} as a noun/adjective} \\
This use is the same as you saw earlier in anpa, insa, monsi, and sewi. 
% Here are some examples.

\begin{supertabular}{p{5,5cm}|ll}
% ona li lon poka mi. && He is at my side. He is beside me. \\
jan \textbf{poka} && neighbor, someone who is beside you \\
\textbf{poka} telo && water side, the beach \\
\end{supertabular} 

\textbf{\textit{poka} as a preposition} \\

\begin{supertabular}{p{5,5cm}|ll}
mi moku, \textbf{poka} jan pona mi. && I ate beside my friend. \\
\end{supertabular} 
%   
% So, you see, you can treat poka either as a noun/adjective or as a preposition. It's up to you, but whichever way you choose you can still express your thought. 
%
% \begin{supertabular}{p{5,5cm}|ll}
% mi utala e jan ike poka jan nasa. && I fought an enemy with a drunk guy. \\
% mi utala e jan ike lon poka pi jan nasa. && I fought an enemy at the side of a drunk guy. \\
% \end{supertabular} 
%
% That's not so bad, now is it?
%
%
%%%%%%%%%%%%%%%%%%%%%%%%%%%%%%%%%%%%%%%%%%%%%%%%%%%%%%%%%%%%%%%%%%%%%%%%%%
\newpage
\index{\textit{anpa}}
\index{\textit{insa}}
\index{\textit{monsi}}
\index{\textit{poka}}
\index{\textit{sewi}}
\subsection*{The Spatial Nouns \textit{anpa}, \textit{insa}, \textit{monsi} and \textit{sewi}}
%%%%%%%%%%%%%%%%%%%%%%%%%%%%%%%%%%%%%%%%%%%%%%%%%%%%%%%%%%%%%%%%%%%%%%%%%%
%
\index{\textit{anpa}!\textit{noun}}
\index{\textit{insa}!\textit{noun}}
\index{\textit{monsi}!\textit{noun}}
\index{\textit{sewi}!\textit{noun}}
\index{\textit{noun}!\textit{anpa}}
\index{\textit{noun}!\textit{insa}}
\index{\textit{noun}!\textit{monsi}}
\index{\textit{noun}!\textit{sewi}}
\textbf{nouns} \\
Although you might be tempted to use these words as prepositions, but they are not. 
% \textbf{You have to use another preposition along with these words.} 

\begin{supertabular}{p{5,5cm}|ll}
ona li \textbf{lon} sewi mi.    &&  He is in my above, i.e. he is above me. \\
pipi li \textbf{lon} anpa mi.   &&  The bug is underneath me. \\
moku li \textbf{lon} insa mi.   &&  Food is inside me. \\
mi moku \textbf{lon} poka sina. &&  I'm eating beside [or with] you. \\
\end{supertabular} 

So you see, these words as used here are nouns, and \textit{mi} is a possessive pronoun meaning "my". 
And since these words are merely nouns, you must still have a verb; in the above examples, \textit{lon} is acting as the verb. 
\textbf{Don't forget to include a verb!} 

Since these words aren't prepositions, they are free for other uses, just like any other noun/adjective/verb.

\textbf{\textit{monsi} as a body part} \\
\begin{supertabular}{p{5,5cm}|ll}
monsi && back, butt \\
\end{supertabular} 

\index{\textit{anpa}!verb}
\index{verb!\textit{anpa}}
\index{defeat}
\textbf{\textit{anpa} as a verb} \\
\begin{supertabular}{p{5,5cm}|ll}
mi \textbf{anpa} e jan utala. && I defeated the warrior. \\
\end{supertabular} 
%
%%%%%%%%%%%%%%%%%%%%%%%%%%%%%%%%%%%%%%%%%%%%%%%%%%%%%%%%%%%%%%%%%%%%%%%%%%
\subsection*{Practice 7 (Answers: Page~\pageref{'other_prepositions'})}
%%%%%%%%%%%%%%%%%%%%%%%%%%%%%%%%%%%%%%%%%%%%%%%%%%%%%%%%%%%%%%%%%%%%%%%%%%
%
Try translating these sentences.

\begin{supertabular}{p{5,5cm}|ll}
My friend is beside me. && \\ % no-dictionary
The sun is above me. && \\ % no-dictionary
The land is beneath me. && \\ % no-dictionary
Bad things are behind me. && \\ % no-dictionary
I'm okay because I'm alive. * && \\ % no-dictionary
I look at the land with my friend. && \\ % no-dictionary
People look like ants. && \\ % no-dictionary
 && \\ % no-dictionary
poka mi li ' pakala. && \\ % no-dictionary
mi kepeken poki li kepeken ilo moku. && \\ % no-dictionary
jan li lon insa tomo. && \\ % no-dictionary
\end{supertabular} 

* \textit{lon} as a verb by itself means to exist, to be real, etc. 
% 
%%%%%%%%%%%%%%%%%%%%%%%%%%%%%%%%%%%%%%%%%%%%%%%%%%%%%%%%%%%%%%%%%%%%%%%%%%
% eof
