%%%%%%%%%%%%%%%%%%%%%%%%%%%%%%%%%%%%%%%%%%%%%%%%%%%%%%%%%%%%%%%%%%%%%%%%%%
\section{Conditional Sentences with the Separator \textit{la}}
%%%%%%%%%%%%%%%%%%%%%%%%%%%%%%%%%%%%%%%%%%%%%%%%%%%%%%%%%%%%%%%%%%%%%%%%%%
%
%%%%%%%%%%%%%%%%%%%%%%%%%%%%%%%%%%%%%%%%%%%%%%%%%%%%%%%%%%%%%%%%%%%%%%%%%%
\subsection*{Vocabulary}
%%%%%%%%%%%%%%%%%%%%%%%%%%%%%%%%%%%%%%%%%%%%%%%%%%%%%%%%%%%%%%%%%%%%%%%%%%
%
\index{\textit{la}}
\index{sonditional sentences}
\index{\textit{mun}}
\index{\textit{open}}
\index{\textit{pini}}
\index{\textit{tenpo}}
\index{\textit{kipisi}}
\index{separate!adverb}
\index{adverb!separate}
\index{moon}
\index{lunar}
\index{open}
\index{begin}
\index{turn on}
\index{end}
\index{stop}
\index{turn off}
\index{time}
\index{cut}
\begin{supertabular}{p{2,5cm}|ll}
\textbf{\dots la \dots} && \textit{separator}: A "la" is between the context phrase and the main sentence. \\ && A context phrase can be sentence, half sentence or noun. \\ && Don't use "la" before or after \\ && the other separators "e", "li", "pi", ".", "!", "?", ":", ",".  \\ % no-dictionary
 && \\ % no-dictionary
\textbf{\dots mun} && \textit{adjective}: lunar \\ % no-dictionary
\textbf{mun} && \textit{noun}: moon, lunar, night sky object, star \\ % no-dictionary
 && \\ % no-dictionary
\textbf{\dots open} && \textit{adjective}: initial, starting, opening \\ % no-dictionary
\textbf{open} && \textit{noun}: start, beginning, opening \\ % no-dictionary
\textbf{open la \dots} && \textit{noun}: at the opening, in the beginning  \\ % no-dictionary
\textbf{open (e \dots)} && \textit{verb transitive}: to open, to start, to begin, to turn on \\ % no-dictionary
\textbf{open \dots } && \textit{auxiliary verb}: to begin, to start \\ % no-dictionary
 && \\ % no-dictionary
\textbf{\dots pini} && \textit{adjective}: completed, finished, past, done \\ % no-dictionary
\textbf{\dots pini} && \textit{adverb}: ago, past, perfectly \\ % no-dictionary
\textbf{pini} && \textit{noun}: end, tip \\ % no-dictionary
\textbf{pini (e \dots)} && \textit{verb transitive}: to end, to stop, to turn off, to finish, to close \\ % no-dictionary
\textbf{pini \dots } && \textit{auxiliary verb}: to stop, to finish, to end, to interrupt \\ % no-dictionary
 && \\ % no-dictionary
\textbf{\dots tenpo} && \textit{adjective}: temporal, chronological, chronologic \\ % no-dictionary
\textbf{\dots tenpo} && \textit{adverb}: chronologically \\ % no-dictionary
\textbf{tenpo} && \textit{noun}: time, period of time, moment, duration, situation, occasion \\ % no-dictionary
 && \\ % no-dictionary
\textbf{kipisi } && \textit{noun}: section, fragment, slice \\ % no-dictionary
\textbf{kipisi (e \dots)} && \textit{verb transitive}: to cut \\ % no-dictionary
 && \\ % no-dictionary
 \textbf{ante la \dots} && \textit{noun}: if difference, if variance, if disagreement  \\ % no-dictionary
 && \\ % no-dictionary
\textbf{ike la \dots} && \textit{noun}: if negativity, if badness, if evil \\ % no-dictionary
 && \\ % no-dictionary
\textbf{ken la \dots} && \textit{noun}: if possibility, if ability, if permission \\ % no-dictionary
 && \\ % no-dictionary
\textbf{kin la \dots} && \textit{noun}: if reality, if fact \\  % no-dictionary
 && \\ % no-dictionary
\textbf{pona la \dots} && \textit{noun}: if good, if simplicity, if positivity \\ % no-dictionary
\end{supertabular} \\
%
%%%%%%%%%%%%%%%%%%%%%%%%%%%%%%%%%%%%%%%%%%%%%%%%%%%%%%%%%%%%%%%%%%%%%%%%%%
\newpage
\index{\textit{la}}
\subsection*{Conditional Phrases}
%%%%%%%%%%%%%%%%%%%%%%%%%%%%%%%%%%%%%%%%%%%%%%%%%%%%%%%%%%%%%%%%%%%%%%%%%%
%
With the help of the separator \textit{la} a conditional sentence is formed.
In front of the separator \textit{la} there is the conditional phrase. 
This is the condition. 
In the English language, a condition is formed using the word' if'.
In Toki Pona the separator \textit{la} serves for this purpose. 
After \textit{la} a complete main sentence begins.

%%%%%%%%%%%%%%%%%%%%%%%%%%%%%%%%%%%%%%%%%%%%%%%%%%%%%%%%%%%%%%%%%%%%%%%%%%
\newpage
\index{maybe}
\index{\textit{la}!\textit{ken}}
\index{\textit{ken}!\textit{la}}
\subsubsection*{Conditional phrases with a noun or pronoun}
%%%%%%%%%%%%%%%%%%%%%%%%%%%%%%%%%%%%%%%%%%%%%%%%%%%%%%%%%%%%%%%%%%%%%%%%%%
%
A conditional phrase can have different structures. 
In the simplest case, a conditional phrase consists of a single word. 
This single word can only be a noun or pronoun. 

\begin{supertabular}{p{5,5cm}|ll}
ilo li ' pakala. && The tool is broken. \\
ken la ilo li ' pakala. && Maybe the tool is broken. \\
\end{supertabular} 

The noun \textit{ken} means 'possibility'.
\textit{ken la} therefore means 'If there is a possibility' or better 'Maybe'.
 
\begin{supertabular}{p{5,5cm}|ll}
ken la jan Lisa li jo e ona. && Maybe Lisa has it. \\
ken la ona li lape. && Maybe he's alseep. \\
ken la mi ken tawa ma Elopa. && Maybe I can go to Europe. \\
\end{supertabular} 

%
%%%%%%%%%%%%%%%%%%%%%%%%%%%%%%%%%%%%%%%%%%%%%%%%%%%%%%%%%%%%%%%%%%%%%%%%%%
\index{\textit{la}!time}
\index{time!\textit{la}}
\index{\textit{la}!\textit{tenpo}}
\index{\textit{tenpo}!\textit{la}}
\subsubsection*{Composite Noun or Pronouns as Conditional Phrases}
%%%%%%%%%%%%%%%%%%%%%%%%%%%%%%%%%%%%%%%%%%%%%%%%%%%%%%%%%%%%%%%%%%%%%%%%%%
%
A conditional phrase can be also a composite noun or Pronoun. 
That is, the noun or pronoun followed by one or more adjectives or \textit{pi} phrases.
Optionally, a conjunct (\textit{anu}, \textit{en}, \textit{taso}) can be used before the noun or pronoun. 

Typical examples of this are time specifications. 
Time Specifications are formed with the noun \textit{tenpo} and adjectives. 

\index{day}
\index{night}
\index{present}
\index{today}
\index{tonight}
\index{future}
\index{soon}
\index{past}
\index{yesterday}
\index{last night}
\index{night!last}
\index{tomorrow}
\index{often}
\begin{supertabular}{p{5,5cm}|ll}
tenpo suno && day (sun time) \\
tenpo pimeja && night (dark time) \\
tenpo ni && the present (this time) \\
tenpo suno ni && today (this sun time) \\
tenpo pimeja ni && tonight (this dark time) \\
tenpo kama && the future (coming time) \\
tenpo kama lili && soon (little coming time) \\
tenpo pini && the past (past time) \\
tenpo suno pini && yesterday (past sun time) \\
tenpo pimeja pini && last night (past dark time) \\
tenpo suno kama && tomorrow (coming sun time) \\
tenpo mute && often (many times) \\
\end{supertabular} 

Time specifications as a conditional phrase define the time in which the statement of the main record takes place. 
Literally translated, it would mean something like this: 'If time... is, then happens...'. 

\index{drunk}
\begin{supertabular}{p{5,5cm}|ll}
tenpo pini la mi ' weka. && In the past, I was away. \\
tenpo ni la mi lon. && At this time, I am here. \\
tenpo kama la mi lape. && In the future, I'll sleep. \\
taso tenpo pimeja pini la mi kama nasa. && But, Last night, I became drunk. \\
\end{supertabular} 

\index{age}
\index{old}
\index{how!old}
\index{year}
\index{birthday}

With a question pronoun \textit{seme} in a conditional-phrase it is possible to ask for age.

\begin{supertabular}{p{5,5cm}|ll}
tenpo pi mute seme la sina sike e suno? && How old are you? \\
\end{supertabular} 

Birthdays come once a year, and each time you have a birthday, you have gone around the sun one complete time. 
To answer and tell someone how old you are, just replace the \textit{pi mute seme} with your age.

\begin{supertabular}{p{5,5cm}|ll}
tenpo tu tu la mi sike e suno. && Four times (\textit{la}) I circled the sun. \\
\end{supertabular} 

%
%%%%%%%%%%%%%%%%%%%%%%%%%%%%%%%%%%%%%%%%%%%%%%%%%%%%%%%%%%%%%%%%%%%%%%%%%%
\index{if!\textit{la}}
\index{when!\textit{la}}
\index{\textit{la}!if}
\index{\textit{la}!when}
\subsubsection*{Complete Sentences as Conditional Phrases}
%%%%%%%%%%%%%%%%%%%%%%%%%%%%%%%%%%%%%%%%%%%%%%%%%%%%%%%%%%%%%%%%%%%%%%%%%%
%
A conditional phrase can also be a complete sentence.

\begin{supertabular}{p{5,5cm}|ll}
mama mi li ' moli la mi pilin ike. && My parents die, I feel bad. \\
\end{supertabular} 

In casual English, we'd translate this as "If my parents were to die, I'd be sad." 
All conditional phrases follow the same order as that example. 
If you like formulas and patterns, here's a good way to think about it. 

\begin{supertabular}{p{5,5cm}|ll}
1 la 2. && If/when 1 happens, then 2 happens also. \\  % no-dictionary
\end{supertabular} 

\begin{supertabular}{p{5,5cm}|ll}
mi lape la ali li ' pona. && When I'm asleep, everything is good. \\
sina moku e telo nasa la sina nasa. && If you drink beer, you'll be silly. \\
sina ' moli la sina ken ala toki. && If you are dead, you can't speak. \\
mi pali mute la mi pilin ike. && When I work a lot, I feel bad. \\
sama pi ni en ona la mi wile jo e ni tu. &&  If this and that is the same, I want both.\\
tawa mi la mi pilin pona. && Am I in motion, I feel good. \\
tan ni la mi sona e nasin. && If this is the cause, we know the solution. \\
lon ona la mi ken lukin e ona. && If it has suchness, we can see it. \\
\end{supertabular} 

Commas together with the separator \textit{la} are neither necessary nor useful. 

%
%%%%%%%%%%%%%%%%%%%%%%%%%%%%%%%%%%%%%%%%%%%%%%%%%%%%%%%%%%%%%%%%%%%%%%%%%%
\index{\textit{la}!several}
\subsubsection*{Several Conditional Phrases in one sentence}
%%%%%%%%%%%%%%%%%%%%%%%%%%%%%%%%%%%%%%%%%%%%%%%%%%%%%%%%%%%%%%%%%%%%%%%%%%
%
It is possible to use \textit{la} two times in a sentence. 
But please not more than two. 

\begin{supertabular}{p{5,5cm}|ll}
ken la tenpo pimeja la ni li ' pona. && Maybe in the night it will be ok. \\  
\end{supertabular} 

%
%%%%%%%%%%%%%%%%%%%%%%%%%%%%%%%%%%%%%%%%%%%%%%%%%%%%%%%%%%%%%%%%%%%%%%%%%%
\subsubsection*{Conditional Phrases versus Prepositional Objects with \textit{lon} }
%%%%%%%%%%%%%%%%%%%%%%%%%%%%%%%%%%%%%%%%%%%%%%%%%%%%%%%%%%%%%%%%%%%%%%%%%%
%
The (compound) noun of the prepositional object with the prepostion \textit{lon} can be placed before \textit{la} with the same meaning.
This only applies if the sentence contains only one predicate phrase with only one prepositional object. 

\begin{supertabular}{p{5,5cm}|ll}
mi moku e telo, lon tenpo ni. && I drink now. \\
tenpo ni la mi moku e telo.  && I drink now. \\
\end{supertabular} 

%
%%%%%%%%%%%%%%%%%%%%%%%%%%%%%%%%%%%%%%%%%%%%%%%%%%%%%%%%%%%%%%%%%%%%%%%%%%
\index{comparative}
\index{superlative}
\index{adjective!comparative}
\index{adjective!superlative}
\subsection*{Miscellaneous}
%%%%%%%%%%%%%%%%%%%%%%%%%%%%%%%%%%%%%%%%%%%%%%%%%%%%%%%%%%%%%%%%%%%%%%%%%%
\subsubsection*{comparative and superlative} 
%%%%%%%%%%%%%%%%%%%%%%%%%%%%%%%%%%%%%%%%%%%%%%%%%%%%%%%%%%%%%%%%%%%%%%%%%%
%
Now to use this concept in Toki Pona, you have to split your idea up into two separate sentences. 
Here's how you'd say "Lisa is better than Susan."

\begin{supertabular}{p{5,5cm}|ll}
jan Lisa li ' pona mute. ...  && Lisa is very good. ... \\ % no-dictionary
... jan Susan li ' pona lili. && ... Susan is a little good. \\ % no-dictionary
\end{supertabular} 

Make sense? 
You say that one thing is very much of something, while you use another object as the basis for comparison and say that it's only a little bit of something. 

\begin{supertabular}{p{5,5cm}|ll}
mi ' suli mute. sina ' suli lili. && I'm bigger than you. \\
mi moku mute. sina moku lili. && I eat more than you. \\
\end{supertabular} 
%
%%%%%%%%%%%%%%%%%%%%%%%%%%%%%%%%%%%%%%%%%%%%%%%%%%%%%%%%%%%%%%%%%%%%%%%%%%
\index{headlines}
\subsubsection*{Headlines} 
%%%%%%%%%%%%%%%%%%%%%%%%%%%%%%%%%%%%%%%%%%%%%%%%%%%%%%%%%%%%%%%%%%%%%%%%%%
%
Headings can be incomplete sentences and do not end with a punctuation mark.

\textit{tenpo mun nanpa luka luka wan} \\
tenpo ni li ike kin, lon ma Tosi. \\
suno li suli lili kin. \\ 
telo li kama, lon sewi. \\
kasi li moli. \\
waso li tawa. \\
tenpo seli o kama! 

%%%%%%%%%%%%%%%%%%%%%%%%%%%%%%%%%%%%%%%%%%%%%%%%%%%%%%%%%%%%%%%%%%%%%%%%%%
\newpage
\subsection*{Practice (Answers: Page~\pageref{'la'})}
%%%%%%%%%%%%%%%%%%%%%%%%%%%%%%%%%%%%%%%%%%%%%%%%%%%%%%%%%%%%%%%%%%%%%%%%%%
%
Please write down your answers and check them afterwards. 

\begin{supertabular}{p{5,5cm}|ll}

\end{supertabular} 

Try to translate these sentences. 
You can use the tool \textit{Toki Pona Parser} (\cite{www:rowa:02}) for spelling and grammar check. 

\begin{supertabular}{p{5,5cm}|ll}
Maybe Susan will come.  && \\ % no-dictionary
Last night I watched X-Files.  &&   \\ % no-dictionary
If the enemy comes, burn these papers.  &&   \\ % no-dictionary
Maybe he's in school.  &&   \\ % no-dictionary
I have to work tomorrow.  &&   \\ % no-dictionary
When it's hot, I sweat. *  &&  \\ % no-dictionary
Open the door.   &&  \\ % no-dictionary
The moon is big tonight.   &&  \\ % no-dictionary
 && \\ % no-dictionary
ken la jan lili li wile moku e telo.  &&   \\ % no-dictionary
tenpo ali la o kama sona!   &&  \\ % no-dictionary
sina sona e toki ni la sina sona e toki pona!   &&  \\ % no-dictionary
\end{supertabular}

* Think: "Heat is present, I emit fluid from my skin."
%%%%%%%%%%%%%%%%%%%%%%%%%%%%%%%%%%%%%%%%%%%%%%%%%%%%%%%%%%%%%%%%%%%%%%%%%%
% eof
