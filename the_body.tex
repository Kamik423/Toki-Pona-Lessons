%%%%%%%%%%%%%%%%%%%%%%%%%%%%%%%%%%%%%%%%%%%%%%%%%%%%%%%%%%%%%%%%%%%%%%%%%%
\section{The Body}
%
%%%%%%%%%%%%%%%%%%%%%%%%%%%%%%%%%%%%%%%%%%%%%%%%%%%%%%%%%%%%%%%%%%%%%%%%%%
\subsection*{Vocabulary}
%%%%%%%%%%%%%%%%%%%%%%%%%%%%%%%%%%%%%%%%%%%%%%%%%%%%%%%%%%%%%%%%%%%%%%%%%%
%
\begin{supertabular}{p{2,5cm}|ll}
%
\index{ko}
\textbf{ko} && \textit{noun}: semi-solid or squishy substance; clay, dough, glue, paste, powder, gum \\ % no-dictionary
\textbf{ko (e \dots)} && \textit{verb transitive}: to squash, to pulverize \\ % no-dictionary
 && \\ % no-dictionary
%
\index{kute}
\textbf{\dots kute} && \textit{adjective}: auditory, hearing \\ % no-dictionary
\textbf{kute} && \textit{noun}: hearing, ear \\ % no-dictionary
\textbf{kute (e \dots)} && \textit{verb transitive}: to hear, to listen, \\ % no-dictionary
 && \\ % no-dictionary
%
\index{linja}
\textbf{\dots linja} && \textit{adjective}: elongated, oblong, long \\ % no-dictionary
\textbf{linja} && \textit{noun}: long and flexible thing; string, rope, hair, thread, cord, chain, line, yarn \\ % no-dictionary
 && \\ % no-dictionary
%
\index{luka}
\textbf{\dots luka} && \textit{adjective}: tangible, palpable \\ % no-dictionary
\textbf{luka} && \textit{noun}: arm, hand, tacticle organ \\ % no-dictionary
 && \\ % no-dictionary
%
\index{lupa}
\textbf{\dots lupa} && \textit{adjective}:) hole-, holey, full of holes \\ % no-dictionary
\textbf{lupa} && \textit{noun}: hole, orifice, door, window \\ % no-dictionary
\textbf{lupa (e \dots)} && \textit{verb transitive}: to pierce, to stab, to perforate \\ % no-dictionary
 && \\ % no-dictionary
%
\index{nena}
\textbf{\dots nena} && \textit{adjective}: hilly, undulating, mountainous, hunchbacked, humpbacked, bumpy \\ % no-dictionary
\textbf{nena} && \textit{noun}: bump, hill, extrusion, button, mountain, nose, protuberance \\ % no-dictionary
 && \\ % no-dictionary
%
\index{oko}
\textbf{\dots oko} && \textit{adjective}: optical, eye- \\ % no-dictionary
\textbf{oko} && \textit{noun}: eye \\ % no-dictionary
 && \\ % no-dictionary
%
\index{palisa}
\textbf{\dots palisa} && \textit{adjective}: long \\ % no-dictionary
\textbf{palisa} && \textit{noun}: long hard thing; branch, rod, stick, pointy thing \\ % no-dictionary
\textbf{palisa (e \dots)} && \textit{verb transitive}: to stretch, to beat, to poke, to stab, to sexually arouse \\ % no-dictionary
 && \\ % no-dictionary
%
\index{selo}
\textbf{selo} && \textit{noun}: skin, outer form, bark, peel, shell, skin, boundary, shape \\ % no-dictionary
\textbf{selo (e \dots)} && \textit{verb transitive}: to shelter, to protect, to guard \\ % no-dictionary
 && \\ % no-dictionary
%
\index{sijelo}
\textbf{\dots sijelo} && \textit{adjective}: physical, bodily, corporal, corporeal, material, carnal \\ % no-dictionary
\textbf{\dots sijelo} && \textit{adverb}: physically, bodily \\ % no-dictionary
\textbf{sijelo} && \textit{noun}: body (of person or animal), physical state, torso \\ % no-dictionary
\textbf{sijelo (e \dots)} && \textit{verb transitive}: to heal, to heal up, to cure \\ % no-dictionary
 && \\ % no-dictionary
%
\index{sike}
\textbf{\dots sike} && \textit{adjective}: round, cyclical, of one year \\ % no-dictionary
\textbf{\dots sike} && \textit{adverb}: rotated \\ % no-dictionary
\textbf{sike} && \textit{noun}: circle, ball, cycle, sphere, wheel; round or circular thing \\ % no-dictionary
\textbf{sike (e \dots)} && \textit{verb transitive}: to orbit, to circle, to revolve, to circle around, to rotate \\ % no-dictionary
 && \\ % no-dictionary
%
\index{uta}
\textbf{\dots uta} && \textit{adjective}: oral \\ % no-dictionary
\textbf{\dots uta} && \textit{adverb}: orally \\ % no-dictionary
\textbf{uta} && \textit{noun}: mouth, lips, oral cavity, jaw, beak \\ % no-dictionary
\textbf{uta (e \dots)} && \textit{verb transitive}: to kiss, to osculate, to oral stimulate, to suck \\ % no-dictionary


\end{supertabular} \\
%
%%%%%%%%%%%%%%%%%%%%%%%%%%%%%%%%%%%%%%%%%%%%%%%%%%%%%%%%%%%%%%%%%%%%%%%%%%
\newpage
%
\subsection*{Body Parts}
%%%%%%%%%%%%%%%%%%%%%%%%%%%%%%%%%%%%%%%%%%%%%%%%%%%%%%%%%%%%%%%%%%%%%%%%%%
%
With the above nouns and optional adjectives body parts can be described. 
However, some of the words have other uses as well. 
%

\begin{supertabular}{p{5,5cm}|ll}
oko && eye \\
nena kute && ear \\
nena kon && nose \\ 
uta && mouth �\\
ijo uta walo && teeth \\
linja lawa && hair (of head) \\
lawa && head \\
anpa lawa && neck (bottom of head) \\
luka && hand, arm \\
len luka && gloves, mittens \\
poka && hip \\
noka && leg, foot \\
len noka && shoe, pants \\
sinpin && chest, abdomen, face \\
nena sike meli && female breasts \\
lupa meli && vagina \\
palisa mije && penis \\
sike mije && man's testicles \\
monsi && a person's back \\ 
selo && skin \\
\end{supertabular} 
%
%
%%%%%%%%%%%%%%%%%%%%%%%%%%%%%%%%%%%%%%%%%%%%%%%%%%%%%%%%%%%%%%%%%%%%%%%%%%
% \newpage
%
\subsection*{Bodily Fluids and Wastes}
%
\index{\textit{ko}}
%%%%%%%%%%%%%%%%%%%%%%%%%%%%%%%%%%%%%%%%%%%%%%%%%%%%%%%%%%%%%%%%%%%%%%%%%%
%
With the noun \textit{telo} and corresponding adjectives body fluids and excretions are described.
The noun \textit{ko} is often combined with the adjective \textit{jaki}.

\begin{supertabular}{p{5,5cm}|ll}
telo walo mije && The fluid that a man releases during \textit{unpa}. \\
telo sijelo loje && blood (red bodily fluid) \\ 
telo jelo && urine (yellow fluid) \\
mi pana e telo jelo. && I peed. \\
ko jaki && feces \\
mi pana e ko jaki. && I crapped. \\
\end{supertabular} 

%%%%%%%%%%%%%%%%%%%%%%%%%%%%%%%%%%%%%%%%%%%%%%%%%%%%%%%%%%%%%%%%%%%%%%%%%%
%
\subsection*{The Transitive Verb \textit{kute}}
\index{\textit{kute}}
%%%%%%%%%%%%%%%%%%%%%%%%%%%%%%%%%%%%%%%%%%%%%%%%%%%%%%%%%%%%%%%%%%%%%%%%%%

\textbf{\textit{kute} can also be used a verb} \\ 
\begin{supertabular}{p{5,5cm}|ll}
mi kute e toki sina. && I hear your talking. \\ 
mi kute e kalama musi. && I'm listening to music. \\
\end{supertabular} 

%%%%%%%%%%%%%%%%%%%%%%%%%%%%%%%%%%%%%%%%%%%%%%%%%%%%%%%%%%%%%%%%%%%%%%%%%%
%
\subsection*{A Song}
%
%%%%%%%%%%%%%%%%%%%%%%%%%%%%%%%%%%%%%%%%%%%%%%%%%%%%%%%%%%%%%%%%%%%%%%%%%%

Here the version of 'Heads, shoulders, knees and toes' translated in Toki Pona from jan Mali and used in her nice video Toki Pona lessons \cite{www:astrodonunt:02}}.
As you can see these are not exact grammar sentences because it is lyric.

lawa, sewi luka, palisa noka, palisa noka \\
lawa, sewi luka, palisa noka, palisa noka \\
en oko en nena kute en uta en nena kon \\
lawa, sewi luka, palisa noka, palisa noka \\

%
%%%%%%%%%%%%%%%%%%%%%%%%%%%%%%%%%%%%%%%%%%%%%%%%%%%%%%%%%%%%%%%%%%%%%%%%%%
\newpage
%
\subsection*{Practice (Answers: Page~\pageref{'the_body'})}
%
%%%%%%%%%%%%%%%%%%%%%%%%%%%%%%%%%%%%%%%%%%%%%%%%%%%%%%%%%%%%%%%%%%%%%%%%%%

Please write down your answers and check them afterwards. 

Which word type can represent the respective word in the sentence after the hyphen?
Example:

\begin{supertabular}{p{5,5cm}|ll}
pona - mi pona e ni. && transitive verb \\ % no-dictionary
\end{supertabular}

\begin{supertabular}{p{5,5cm}|ll}
kepeken - mi kepeken ilo. &&  \\ % no-dictionary
sina - sina pona ala pona? &&  \ % no-dictionary
kama - mi kama jo e tomo tawa. &&  \\ % no-dictionary
lon - mi lon tomo. &&  \\ % no-dictionary
kepeken - mi pali e ni, kepeken ilo. &&  \\ % no-dictionary
\end{supertabular}

Try to translate these sentences. 
You can use the tool \textit{Toki Pona Parser} (\cite{www:rowa:02}) for spelling and grammar check. 

\begin{supertabular}{p{5,5cm}|ll}
Kiss me. *  && \\ % no-dictionary
I need to pee. &&  \\ % no-dictionary
My hair is wet. && \\ % no-dictionary
Something is in my eye.  &&  \\ % no-dictionary
I can't hear your talking. &&  \\ % no-dictionary
I need to crap.  &&  \\ % no-dictionary
That hole is big. &&   \\ % no-dictionary
Is it a chain? &&  \\ % no-dictionary
 && \\ % no-dictionary
selo pi jelo en laso pi akesi lili li ' pona, tawa mi. && \\ % no-dictionary
a! telo sijelo loje li kama, tan nena kute mi!  &&  \\ % no-dictionary
selo mi li wile e ni: mi pilin e ona. ** &&   \\ % no-dictionary
o pilin e nena. &&   \\ % no-dictionary
o moli e pipi, kepeken palisa.  &&  \\ % no-dictionary
luka mi li ' jaki. mi wile telo e ona. &&   \\ % no-dictionary
o pana e sike, tawa mi. &&   \\ % no-dictionary
mi pilin e seli sijelo sina. &&    \\ % no-dictionary
ona li selo ala selo? &&  \\ % no-dictionary
\end{supertabular} 

* We sorta have an idiom for this. 
Think: 'Touch my mouth using your mouth.' \\
** This sentence is sorta idiomatic. 
Look at the answer if you can't figure it out. 
%
%%%%%%%%%%%%%%%%%%%%%%%%%%%%%%%%%%%%%%%%%%%%%%%%%%%%%%%%%%%%%%%%%%%%%%%%%%
% eof

