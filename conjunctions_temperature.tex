%%%%%%%%%%%%%%%%%%%%%%%%%%%%%%%%%%%%%%%%%%%%%%%%%%%%%%%%%%%%%%%%%%%%%%%%%%
\section{Conjunctions and Temperature}
%%%%%%%%%%%%%%%%%%%%%%%%%%%%%%%%%%%%%%%%%%%%%%%%%%%%%%%%%%%%%%%%%%%%%%%%%%
%
%%%%%%%%%%%%%%%%%%%%%%%%%%%%%%%%%%%%%%%%%%%%%%%%%%%%%%%%%%%%%%%%%%%%%%%%%%
\subsection*{Vocabulary}
%%%%%%%%%%%%%%%%%%%%%%%%%%%%%%%%%%%%%%%%%%%%%%%%%%%%%%%%%%%%%%%%%%%%%%%%%%
%
\index{\textit{ante}}
\index{\textit{anu}}
\index{\textit{en}}
\index{\textit{kin}}
\index{\textit{lete}}
\index{\textit{lipu}}
\index{\textit{mani}}
\index{\textit{pilin}}
\index{\textit{taso}}
\index{different}
\index{other}
\index{or}
\index{and}
\index{indeed}
\index{still}
\index{too}
\index{cold}
\index{freeze}
\index{paper}
\index{sheet}
\index{page}
\index{ticket}
\index{money}
\index{currency}
\index{feel}
\index{think}
\index{but}
\index{only}
\begin{supertabular}{p{2,5cm}|ll}
\textbf{\dots anu \dots} && \textit{conjunction}: or (used for decision questions) \\ % no-dictionary
 && \\ % no-dictionary
\textbf{\dots en \dots} && \textit{conjunction}: and (used to coordinate head nouns) \\ % no-dictionary
 && \\ % no-dictionary
\textbf{\dots taso} && \textit{adjective}: only, sole \\ % no-dictionary
\textbf{\dots taso} && \textit{adverb}: only, just, merely, simply, solely, singly \\ % no-dictionary
\textbf{\dots taso \dots} && \textit{conjunction}: but, however \\ % no-dictionary
 && \\ % no-dictionary
\textbf{\dots kin} && \textit{adjective}: indeed, still, too \\ && kin can be the very last word in an adjective group. \\ % no-dictionary
\textbf{\dots kin} && \textit{adverb}: actually, indeed, in fact, really, objectively, \\ && kin can be the very last word in an adverb group. \\% no-dictionary
\textbf{kin} && \textit{noun}: reality, fact \\  % no-dictionary
\textbf{kin!} && \textit{interjection}: really! \\ % no-dictionary
 && \\ % no-dictionary
\textbf{\dots ante} && \textit{adjective}: different, dissimilar, changed, other, unequal, differential \\ % no-dictionary
\textbf{ante} && \textit{noun}: difference, distinction, differential, variation, variance, disagreement \\ % no-dictionary
\textbf{ante (e \dots)} && \textit{verb transitive}: to change, to alter, to modify \\ % no-dictionary
 && \\ % no-dictionary
\textbf{\dots pilin} && \textit{adjective}: sensitive, feeling, empathic \\ % no-dictionary
\textbf{\dots pilin} && \textit{adverb}: perceptively \\ % no-dictionary
\textbf{pilin} && \textit{noun}: feelings, emotion, feel, think, sense, touch, \\ % no-dictionary
\textbf{pilin} && \textit{verb intransitive}: to feel, to sense \\ % no-dictionary
\textbf{pilin (e \dots)} && \textit{verb transitive}: to feel, to think, to touch, to fumble, to fiddle \\ % no-dictionary
 && \\ % no-dictionary
\textbf{\dots lete} && \textit{adjective}: cold, cool, uncooked, raw, perishing \\ % no-dictionary
\textbf{\dots lete} && \textit{adverb}: bleakly \\ % no-dictionary
\textbf{lete} && \textit{noun}: cold, chill, bleakness \\ % no-dictionary
\textbf{lete (e \dots)} && \textit{verb transitive}: to cool down, to chill \\ % no-dictionary
 && \\ % no-dictionary
\textbf{\dots lipu} && \textit{adjective}: book-, paper-, card-, ticket-, sheet-, page,- \\ % no-dictionary
\textbf{lipu} && \textit{noun}: paper, book, card, ticket, sheet, (web-)page, list ; flat and bendable thing \\ % no-dictionary
 && \\ % no-dictionary
\textbf{\dots mani} && \textit{adjective}: financial, financially, monetary, pecuniary \\ % no-dictionary
\textbf{\dots mani} && \textit{adverb}: financially \\ % no-dictionary
\textbf{mani} && \textit{noun}: money, material wealth, currency, dollar, capital \\ % no-dictionary
\end{supertabular} \\
% 
%%%%%%%%%%%%%%%%%%%%%%%%%%%%%%%%%%%%%%%%%%%%%%%%%%%%%%%%%%%%%%%%%%%%%%%%%%
\index{conjunction}
\index{conjunction!\textit{anu}}
\index{conjunction!\textit{en}}
\index{conjunction!\textit{taso}}
\subsection*{The conjunctions \textit{anu} and \textit{en}}
%%%%%%%%%%%%%%%%%%%%%%%%%%%%%%%%%%%%%%%%%%%%%%%%%%%%%%%%%%%%%%%%%%%%%%%%%%
%
%%%%%%%%%%%%%%%%%%%%%%%%%%%%%%%%%%%%%%%%%%%%%%%%%%%%%%%%%%%%%%%%%%%%%%%%%%
\index{\textit{anu}!conjunction}
\index{\textit{anu}!question}
\index{question!\textit{anu}}
\index{question!choice}
\index{choice!\textit{anu}}
\subsubsection*{\textit{anu}}
%%%%%%%%%%%%%%%%%%%%%%%%%%%%%%%%%%%%%%%%%%%%%%%%%%%%%%%%%%%%%%%%%%%%%%%%%%
%
\textit{anu} is used to make questions when there is a choice between two options. 

\begin{supertabular}{p{5,5cm}|ll}
jan Susan anu jan Lisa li moku e suwi?  && Susan or Lisa ate the cookies? \\
sina jo e kili anu telo nasa? && Do you have the fruit, \\ && or is it the wine that you have? \\
sina toki, tawa mi anu ona? && Are you talking to me, \\ && or are you talking to him? \\
ona anu jan ante li ' ike? && Is he bad, or is it the \\ && other person who's bad? \\
\end{supertabular} 

\index{what!or what}
\index{what!\textit{anu}}
\index{\textit{anu}!or what}
Do you know how sometimes in English we say stuff like, 'So are you coming or what?'. 
% We can do the same thing in Toki Pona. 

\begin{supertabular}{p{5,5cm}|ll}
sina kama anu seme? && Are you coming or what? \\ 
sina wile moku anu seme? && Do you want to eat or what? \\
sina wile e mani anu seme? && Do you want the money or what? \\
\end{supertabular} 

Yes/No questions with predicate nouns or predicate adjectives

We had learned that yes/no questions with the adverb \textit{ala} require a verb. 
That there is no verb in Toki Pona, the verb slot can remain empty. 
The predicate is then formed by a predicate adjective or predicate adjective.
Yes/no questions with the adverb \textit{ala} are not possible. 
To form yes/no questions with predicate nouns or predicate adjectives \textit{anu seme} is used. 

\begin{supertabular}{p{5,5cm}|ll}
sina ' pona anu seme? && Are you OK (or what)? \\
ona li ' mama anu seme? && Is she a mother (or what)? \\
\end{supertabular} 

\textit{anu} can be used in declarative sentences also.

\begin{supertabular}{p{5,5cm}|ll}
mi lukin e mije anu meli. && I see a man or a women. \\ 
\end{supertabular} 
%
%%%%%%%%%%%%%%%%%%%%%%%%%%%%%%%%%%%%%%%%%%%%%%%%%%%%%%%%%%%%%%%%%%%%%%%%%%
\index{\textit{en}!conjunction}
\index{and}
\subsubsection*{\textit{en}}
%%%%%%%%%%%%%%%%%%%%%%%%%%%%%%%%%%%%%%%%%%%%%%%%%%%%%%%%%%%%%%%%%%%%%%%%%%
%
This word simply means 'and'. 
It is used to join two nouns in the subject of a sentence together. 

\begin{supertabular}{p{5,5cm}|ll}
mi en sina li ' jan pona. && You and I are friends. \\
jan lili en jan suli li toki. && The child and the adult are talking. \\
kalama musi en meli li ' pona, tawa mi. && I like music and girls. \\
\end{supertabular} 

\index{sentence!compound}
\index{sentence!compound!li}
\index{\textit{li}!compound sentences}
\index{\textit{e}!compound sentences}
\index{sentence!compound!e}
\index{connect!objects}
\index{object!connect}
\index{multiple!\textit{e}}
\index{multiple!\textit{li}}
\index{\textit{li}!multiple}
\index{\textit{e}!multiple}
Note that \textit{en} is not intended to connect two direct objects.} 
For that, use the multiple-\textit{e} technique (Page~\pageref{'multiple_e'). 

\begin{supertabular}{p{5,5cm}|ll}
mi wile e moku e telo. && I want food and water. \\
\end{supertabular} 

Also note that \textit{en} is not used to connect two whole sentences, even though this is common in English. 
Instead, use the multiple-\textit{li} technique  (Page~\pageref{'multiple_li'}) or split the sentence into two sentences. 

\begin{supertabular}{p{5,5cm}|ll}
mi moku e kili ... && I'm eating fruit, ... \\ % no-dictionary
... li toki, kepeken toki pona.  && ... and I'm speaking in/using Toki Pona. \\ % no-dictionary
% / && \\ % no-dictionary
mi moku e kili.  && I'm eating fruit, ... \\ % no-dictionary
mi toki, kepeken toki pona. && ... and I'm speaking in/using Toki Pona. \\ % no-dictionary
\end{supertabular} 

\textit{en} can also be used with \textit{pi} if two people own something together. 
With \textit{en} you can avoid several \textit{pi} phrases.

\begin{supertabular}{p{5,5cm}|ll}
% tomo pi jan Keli en mije ona li suli. && The house of Keli and her boyfriend is big. \\
jan lili pi jan Ken en jan Lisa li ' suwi. && Ken and Lisa's baby is sweet. \\
\end{supertabular} 
%
%%%%%%%%%%%%%%%%%%%%%%%%%%%%%%%%%%%%%%%%%%%%%%%%%%%%%%%%%%%%%%%%%%%%%%%%%%
\index{\textit{taso}!conjunction}
\index{conjunction!\textit{taso}}
\subsection*{\textit{taso}}
\subsubsection*{\textit{taso} as conjunction}
%%%%%%%%%%%%%%%%%%%%%%%%%%%%%%%%%%%%%%%%%%%%%%%%%%%%%%%%%%%%%%%%%%%%%%%%%%
%
% If you don't understand these examples below, it's because you have forgotten other concepts; \textit{taso} itself is common sense. 
% The only thing you need to remember is 
Start a new sentence when you want to use \textit{taso}.
Do not use a comma!

\begin{supertabular}{p{5,5cm}|ll}
mi wile moku. taso mi jo ala e moku. && I want to eat. But I don't have food. \\ 
mi wile lukin e tomo mi. taso mi lon ma ante. && I want to see my house. But I'm in a different country. \\ 
mi ' pona. taso meli mi li ' pakala. && I'm okay. But my girlfriend is injured. \\
\end{supertabular} 
%
%%%%%%%%%%%%%%%%%%%%%%%%%%%%%%%%%%%%%%%%%%%%%%%%%%%%%%%%%%%%%%%%%%%%%%%%%%
\index{\textit{taso}!adjective}
\index{adjective!\textit{taso}}
\subsubsection*{\textit{taso} as adjective}
%%%%%%%%%%%%%%%%%%%%%%%%%%%%%%%%%%%%%%%%%%%%%%%%%%%%%%%%%%%%%%%%%%%%%%%%%%
%
\begin{supertabular}{p{5,5cm}|ll}
jan Lisa taso li kama. && Only Lisa came. \\
mi sona e ni taso. && I know only that. (That's all I know.) \\
\end{supertabular} 
%
%%%%%%%%%%%%%%%%%%%%%%%%%%%%%%%%%%%%%%%%%%%%%%%%%%%%%%%%%%%%%%%%%%%%%%%%%%
\index{\textit{taso}!adverb}
\index{adverb!\textit{taso}}
\subsubsection*{\textit{taso} as adverb}
%%%%%%%%%%%%%%%%%%%%%%%%%%%%%%%%%%%%%%%%%%%%%%%%%%%%%%%%%%%%%%%%%%%%%%%%%%
%
\begin{supertabular}{p{5,5cm}|ll}
mi musi taso. && I'm just joking. \\
mi pali taso. && I just work. (All I ever do is work.) \\ 
mi lukin taso e meli ni! ali li ' pona. && I only looked at that girl! Everything's okay. \\
\end{supertabular} 
%
%%%%%%%%%%%%%%%%%%%%%%%%%%%%%%%%%%%%%%%%%%%%%%%%%%%%%%%%%%%%%%%%%%%%%%%%%%
\index{\textit{kin}}
\index{also}
\index{still}
\index{indeed}
\subsection*{\textit{kin}}
%%%%%%%%%%%%%%%%%%%%%%%%%%%%%%%%%%%%%%%%%%%%%%%%%%%%%%%%%%%%%%%%%%%%%%%%%%
%
\textit{kin} it can occur after almost any word in a adjective or adverb group. 

\begin{supertabular}{p{5,5cm}|ll}
A: mi tawa, tawa ma Elopa. && I went to Europe. \\
B: pona! mi tawa kin. && Cool! I went too. \\
A: mi mute o tawa. && Let's go. \\
B: mi ken ala. mi moku kin. && I can't. I'm still eating. \\
A: a! sina lukin ala lukin e ijo nasa ni? && Whoa! Do you see that weird thing? \\
B: mi lukin kin e ona. && I see it indeed. \\
\end{supertabular} 
%
%%%%%%%%%%%%%%%%%%%%%%%%%%%%%%%%%%%%%%%%%%%%%%%%%%%%%%%%%%%%%%%%%%%%%%%%%%
% \newpage
\index{temperature!\textit{pilin}}
\index{\textit{pilin}!temperature}
\index{\textit{seli}}
\index{\textit{lete}}
\index{hot}
\index{heat}
\index{cold}
\subsection*{Temperature and \textit{pilin}}
%%%%%%%%%%%%%%%%%%%%%%%%%%%%%%%%%%%%%%%%%%%%%%%%%%%%%%%%%%%%%%%%%%%%%%%%%%
%
If you've forgotten, \textit{seli} means 'hot' or 'heat'. 
We can use this word to talk about the weather. 
In this lesson, you also should have learned that \textit{lete} means 'cold'. 
Now the thing about these phrases is that they're only used to talk about the temperature of the surroundings in general. 

\begin{supertabular}{p{5,5cm}|ll}
seli li lon. && It's hot. \\
lete li lon. && It's cold. \\
\end{supertabular} 

\index{\textit{seli}!\textit{mute}}
\index{\textit{lete}!\textit{mute}}
\index{\textit{mute}!\textit{seli}}
\index{\textit{mute}!\textit{lete}}
You can also use \textit{lili} and \textit{mute} to be more specific. 
 
\begin{supertabular}{p{5,5cm}|ll}
seli mute li lon. && It's very hot. \\
seli lili li lon. && It's warm. \\
lete mute li lon. && It's very cold. \\
lete lili li lon. && It's cool. \\
\end{supertabular} 

If you're referring to a certain object that is cold, irregardless of the surrounding environment, you use \textit{pilin}.  
Suppose you grab an axe and you discover that the handle is cold. 

\begin{supertabular}{p{5,5cm}|ll}
ilo ni li ' lete pilin. && This axe feels cold. \\
\end{supertabular} \\

Also, just like with the \textit{lon} phrases, the \textit{pilin} phrases can use \textit{mute} and \textit{lili} to intensify the descriptions. 

\begin{supertabular}{p{5,5cm}|ll}
ni li ' lete pilin mute. && This is very cold. \\
ni li ' seli pilin lili. && This feels warm. \\
\end{supertabular} 
%
%%%%%%%%%%%%%%%%%%%%%%%%%%%%%%%%%%%%%%%%%%%%%%%%%%%%%%%%%%%%%%%%%%%%%%%%%%
\index{\textit{pilin}!feeling}
\index{feeling!\textit{pilin}}
\index{feeling!happy}
\index{feeling!sad}
\index{happy}
\index{sad}
\subsection*{Other uses of \textit{pilin}}
%%%%%%%%%%%%%%%%%%%%%%%%%%%%%%%%%%%%%%%%%%%%%%%%%%%%%%%%%%%%%%%%%%%%%%%%%%
%
You also use \textit{pilin} to describe how a person or an animal is feeling. 

\begin{supertabular}{p{5,5cm}|ll}
mi pilin pona. && I feel good. / I feel happy. \\
mi pilin ike. && I feel bad. / I feel sad. \\
sina pilin seme? && How do you feel? \\ 
\end{supertabular} 

\index{think}
It can also mean 'to think'.

\begin{supertabular}{p{5,5cm}|ll}
mi pilin e ni: sina ike. && I think this: You're bad. \\ 
\end{supertabular} 

When you ask someone 'What are you thinking about?' in Toki Pona, the 'about' part is removed. 

\begin{supertabular}{p{5,5cm}|ll}
sina pilin e seme? && What are you thinking? \\
\end{supertabular} 

However, when you answer to say what you're thinking about, the 'about' part gets added back in. 
Use \textit{pi} if needed (see Page~\pageref{'mistakes_with_pi'}).

\begin{supertabular}{p{5,5cm}|ll}
mi pilin ijo. && I'm thinking (about) something. \\
mi pilin pi meli ni. && I'm thinking about that woman. \\
\end{supertabular} 
% 
%%%%%%%%%%%%%%%%%%%%%%%%%%%%%%%%%%%%%%%%%%%%%%%%%%%%%%%%%%%%%%%%%%%%%%%%%%
\newpage
\subsection*{Practice (Answers: Page~\pageref{'conjunctions_temperature'})}
%%%%%%%%%%%%%%%%%%%%%%%%%%%%%%%%%%%%%%%%%%%%%%%%%%%%%%%%%%%%%%%%%%%%%%%%%%
%
Please write down your answers and check them afterwards. 

Try to translate these sentences. 
You can use the tool \textit{Toki Pona Parser} (\cite{www:rowa:02}) for spelling and grammar check. 


\begin{supertabular}{p{5,5cm}|ll}
   Do you want to come or what?    \\ % no-dictionary
   Do you want food, or do you want water?    \\ % no-dictionary
   I still want to go to my house.    \\ % no-dictionary
   This paper feels cold.    \\ % no-dictionary
   I like currency of other nations.   \\  % no-dictionary
   I want to go, but I can't.    \\ % no-dictionary
   I'm alone. *  \\ % no-dictionary
   Do you like me?   &&   \\ % no-dictionary
 && \\ % no-dictionary
   mi olin kin e sina.    \\ % no-dictionary
   mi pilin e ni: ona li jo ala e mani.    \\ % no-dictionary
   mi wile lukin e ma ante.    \\ % no-dictionary
   mi wile ala e ijo. mi lukin taso.    \\ % no-dictionary
   mi pilin lete.    \\ % no-dictionary
   sina wile toki, tawa mije anu meli?    \\ % no-dictionary
\end{supertabular} 

* Think: 'Only I am present.'
%
%%%%%%%%%%%%%%%%%%%%%%%%%%%%%%%%%%%%%%%%%%%%%%%%%%%%%%%%%%%%%%%%%%%%%%%%%%
% eof
