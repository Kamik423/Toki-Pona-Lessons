%%%%%%%%%%%%%%%%%%%%%%%%%%%%%%%%%%%%%%%%%%%%%%%%%%%%%%%%%%%%%%%%%%%%%%%%%%
\section{Tricky Things}

Words are accented on the first syllable.

\begin{supertabular}{p{5,5cm}|ll}
Toki Pona && toki pona \\ % & English - Toki Pona
hello && toki! \\ % & English - Toki Pona
good-bye && mi tawa (said by person leaving); tawa pona (said by person staying) \\ % & English - Toki Pona
thank you && pona \\ % & English - Toki Pona
that one && ni \\ % & English - Toki Pona
how much? && mute seme? \\ % & English - Toki Pona
English && toki Inli \\ % & English - Toki Pona
yes && lon \\ % & English - Toki Pona
no && lon ala \\ % & English - Toki Pona
generic toast && telo nasa pona! \\ % & English - Toki Pona
Do you speak English? && sina sona ala sona e toki Inli? \\ % & English - Toki Pona
\end{supertabular} 





We use \textit{pona moku} to mean "tasty" or "delicious", by the way. 

\subsubsection*{Using \textit{lon} as an action verb}

What you are about to learn is a minor detail and is rarely used in Toki Pona. 
It's also going to use some vocabulary that we haven't learned yet. 
Unless you're just highly interested, you should probably skip this section and come back later if you want to, because this feature is not very important at all. 

... So you're still here? Good! 
Firstly, we need to cover two vocabulary words. 
\textit{lape} (sleep), \textit{pini} (stop, end). 
These words will be covered in later lessons, so you don't have to memorize them now if you do not wish to, but they are necessary for what you are about to learn. 
Now, with this new vocabulary, you can talk about waking someone up.

\begin{supertabular}{p{5,5cm}|ll}
mi pini e lape sina. && I ended your sleep. \\ && I woke you up. \\
\end{supertabular} 

That is plain enough. 
However, you can also express this using lon.

\begin{supertabular}{p{5,5cm}|ll}
mi \textbf{lon} e sina. && I made you aware of reality. \\  && I forced you to be to present and alert. \\
\end{supertabular} 

Note, however, that you could not say, "\textit{sina lon e mi}" ("You woke me up"). 
The person who was sleeping was in his own private existance of sleep. 
When he woke up, he would not feel that he had brought into reality because, to him, sleep was the reality. 
He was simply moved from one existance to another one. 
"\textit{mi lon e sina}" only works because, to the waker, it seems as if the sleeper is not present in the waker's reality; the sleeper seems absent, and so waking him up brings him back to reality. 
Make sense? 

If you didn't quite understand that, don't worry. 
It's a very minor diversion included for anyone who happens to be interested. 
For most situations, it'd be best to use the \textit{pini e lape} phrase. 

\subsection*{\textit{ali} and \textit{ale}}

\begin{supertabular}{p{5,5cm}|ll}
ale / ali && everything, all \\
\end{supertabular}  

Before we learn how to use this word, I suppose you're wondering why there are two words that mean the exact same thing. 
Well, the explanation involves phonetics. 
The word \textit{ali} has been created fairly recently. 
Before it, there was only \textit{ale}. 
However, as you might have already noticed, the words \textit{ale} and \textit{ala} are very close in pronunciation. 
It's easy to confuse these two words in your head, and it's also easy to hear the word wrong when it is spoken aloud. 
People became more concerned that \textit{ala} and \textit{ale} could be confused too easily, and so \underline{Marraskuu}, the creator of Toki Pona, suggested the word \textit{ali}. 
It means the exact same thing as \textit{ale}; however, it's separated from \textit{ala} more and is easier to differentiate. 
Both \textit{ale} and \textit{ali} will exist until it seems that one has become more popular than the other. 
You can use whichever word you like more; however, I personally prefer \textit{ali} and so I'll use it in these lessons. 
So, anyhow, now let's talk about how to use this word. 

%%%%%%%%%%%%%%%%%%%%%%%%%%%%%%%%%%%%%%%%%%%%%%%%%%%%%%%%%%%%%%%%%%%%%%%%%%
% eof
