%%%%%%%%%%%%%%%%%%%%%%%%%%%%%%%%%%%%%%%%%%%%%%%%%%%%%%%%%%%%%%%%%%%%%%%%%%
\section{Questions using \textit{seme}}
%%%%%%%%%%%%%%%%%%%%%%%%%%%%%%%%%%%%%%%%%%%%%%%%%%%%%%%%%%%%%%%%%%%%%%%%%%
\subsection*{Vocabulary}
%%%%%%%%%%%%%%%%%%%%%%%%%%%%%%%%%%%%%%%%%%%%%%%%%%%%%%%%%%%%%%%%%%%%%%%%%%
%
\index{\textit{olin}}
\index{\textit{seme}}
\index{\textit{sin}}
\index{\textit{supa}}
\index{\textit{suwi}}
\index{love}
\index{what}
\index{which}
\index{new}
\index{another}
\index{more}
\index{furniture}
\index{sweet}
\index{cute}
\index{candy}
\index{cookie}
\begin{supertabular}{p{2,5cm}|ll}
\textbf{seme} && \textit{question word}: what, which, wh- (question word) \\ % no-dictionary
 && \\ % no-dictionary
\textbf{\dots olin} && \textit{adjective}: love \\ % no-dictionary
\textbf{olin} && \textit{noun}: love \\ % no-dictionary
\textbf{olin (e \dots)} && \textit{verb transitive}: to love (a person) \\ % no-dictionary
 && \\ % no-dictionary
\textbf{\dots sin} && \textit{adjective}: new, fresh, another, more \\ % no-dictionary
\textbf{ \dots sin } && \textit{adverb}: regenerative \\ % no-dictionary
\textbf{sin} && \textit{noun}: news, novelty, innovation, newness, new release \\ % no-dictionary
\textbf{sin (e \dots)} && \textit{verb transitive}: to renew, to renovate, to freshen \\ % no-dictionary 
 && \\ % no-dictionary
\textbf{\dots supa} && \textit{adjective}: flat, shallow, flat-bottomed, horizontal \\ % no-dictionary
\textbf{supa} && \textit{noun}: horizontal surface, e.g furniture, table, chair, pillow, floor \\ % no-dictionary
 && \\ % no-dictionary
\textbf{\dots suwi} && \textit{adjective}: sweet, cute \\ % no-dictionary
\textbf{suwi} && \textit{noun}: candy, sweet food \\ % no-dictionary
\textbf{suwi (e \dots)} && \textit{verb transitive}: to sweeten \\ % no-dictionary
\end{supertabular} \\
%
%%%%%%%%%%%%%%%%%%%%%%%%%%%%%%%%%%%%%%%%%%%%%%%%%%%%%%%%%%%%%%%%%%%%%%%%%%
% \newpage
\index{question!\textit{seme}}
\index{\textit{seme}!question}
\index{what!subject}
\index{subject!what}
\subsection*{Questions, Questions}
%
\begin{supertabular}{p{5,5cm}|ll}
\textbf{seme}? && Pardon? \\
\end{supertabular} 
%
%%%%%%%%%%%%%%%%%%%%%%%%%%%%%%%%%%%%%%%%%%%%%%%%%%%%%%%%%%%%%%%%%%%%%%%%%%
\subsubsection*{What -- Subject}
%%%%%%%%%%%%%%%%%%%%%%%%%%%%%%%%%%%%%%%%%%%%%%%%%%%%%%%%%%%%%%%%%%%%%%%%%%
%
We talked about how to ask questions that can be answered with a "yes" or "no". 
However, we didn't talk about questions that require more in-depth answers. 
Well, to ask questions like these in Toki Pona, we have to use the word \textit{seme}. 

\begin{supertabular}{p{5,5cm}|ll}
\textbf{seme} li utala e sina? && What attacked you? \\
\textbf{seme} li moku e kili mi? && What is eating my fruit? \\
\textbf{seme} li lon poka mi? && What is beside me? \\
\textbf{seme} li lon tomo mi? && What is in my house? \\
\textbf{seme} li ' pona, tawa sina? && What do you like? \\ 
\end{supertabular} 
%
%%%%%%%%%%%%%%%%%%%%%%%%%%%%%%%%%%%%%%%%%%%%%%%%%%%%%%%%%%%%%%%%%%%%%%%%%%
\index{where!object}
\index{what!object}
\index{object!what}
\subsubsection*{What / Where -- Object}
%%%%%%%%%%%%%%%%%%%%%%%%%%%%%%%%%%%%%%%%%%%%%%%%%%%%%%%%%%%%%%%%%%%%%%%%%%
%
Okay, this next part is probably going to seem rather confusing at first, and so I'm going to ease you into it by using things that you've already learned. 

\begin{supertabular}{p{5,5cm}|ll}
sina lukin e pipi. && You're watching a bug. \\
\end{supertabular} 

Now we're going to turn that sentence into a question. 

\begin{supertabular}{p{5,5cm}|ll}
sina lukin e \textbf{seme}? && What are you watching? \\
\end{supertabular} 

Both \textit{pipi} and \textit{seme} are treated as plain nouns, and the word order of the sentence does not change even when the sentence is a question. 

\begin{supertabular}{p{5,5cm}|ll}
sina pakala e \textbf{seme}? && What did you hurt? \\
sina lon \textbf{seme}? && Where are you? \\ 
ona li jo e \textbf{seme}? && What does he have? \\
\end{supertabular} 
%
%%%%%%%%%%%%%%%%%%%%%%%%%%%%%%%%%%%%%%%%%%%%%%%%%%%%%%%%%%%%%%%%%%%%%%%%%%
\index{who}
\index{whom}
\index{\textit{seme}!\textit{jan}}
\index{\textit{jan}!\textit{seme}}
\subsubsection*{Who(m)}
%%%%%%%%%%%%%%%%%%%%%%%%%%%%%%%%%%%%%%%%%%%%%%%%%%%%%%%%%%%%%%%%%%%%%%%%%%
%
Since you're asking about a person, you use \textbf{\textit{jan}}, then add \textbf{\textit{seme}} to make into a question word. 

\begin{supertabular}{p{5,5cm}|ll}
\textbf{jan seme} li moku? && Who is eating? \\
\textbf{jan seme} li tawa, poka sina? && Who went with you? \\
sina lukin e \textbf{jan seme}? && Who\textbf{m} did you see? \\
sina toki, tawa \textbf{jan seme}? && Who\textbf{m} are you talking to? \\
\end{supertabular} 
%
%%%%%%%%%%%%%%%%%%%%%%%%%%%%%%%%%%%%%%%%%%%%%%%%%%%%%%%%%%%%%%%%%%%%%%%%%%
\index{which}
\index{trick!question}
\index{question!trick}
\subsubsection*{Which}
%%%%%%%%%%%%%%%%%%%%%%%%%%%%%%%%%%%%%%%%%%%%%%%%%%%%%%%%%%%%%%%%%%%%%%%%%%
%
\begin{supertabular}{p{5,5cm}|ll}
ma \textbf{seme} li ' pona, tawa sina? && Which countries do you like? \\
sina kama, tan ma \textbf{seme}? && Which country do you come from? \\ 
\end{supertabular} 

The only reason that this concept might seem difficult is because you're tempted to move the word orders around, because many languages (including English) do it. 
\textbf{One neat little trick you can do to check a translation is to think of the question as a plain statement, and then replace the word \textit{seme} with \textit{ni}.} 
%
%%%%%%%%%%%%%%%%%%%%%%%%%%%%%%%%%%%%%%%%%%%%%%%%%%%%%%%%%%%%%%%%%%%%%%%%%%
% \newpage
\index{why}
\index{\textit{tan}!\textit{seme}}
\index{\textit{seme}!\textit{tan}}
\subsubsection*{Why}
%%%%%%%%%%%%%%%%%%%%%%%%%%%%%%%%%%%%%%%%%%%%%%%%%%%%%%%%%%%%%%%%%%%%%%%%%%
%
\textit{seme} is also used to make what equals the English word "why". 
Also, don't forget that \textbf{\textit{tan}} as a preposition means "because of".

\begin{supertabular}{p{5,5cm}|ll}
sina kama, \textbf{tan seme}? && Why did you come? \\
\end{supertabular} 

%
%%%%%%%%%%%%%%%%%%%%%%%%%%%%%%%%%%%%%%%%%%%%%%%%%%%%%%%%%%%%%%%%%%%%%%%%%%
\subsection*{Miscellaneous}
%%%%%%%%%%%%%%%%%%%%%%%%%%%%%%%%%%%%%%%%%%%%%%%%%%%%%%%%%%%%%%%%%%%%%%%%%%
%
%%%%%%%%%%%%%%%%%%%%%%%%%%%%%%%%%%%%%%%%%%%%%%%%%%%%%%%%%%%%%%%%%%%%%%%%%%
\index{\textit{supa}}
\index{bed}
\index{chair}
\index{sofa}
\index{table}
\subsubsection*{\textit{supa}}
%%%%%%%%%%%%%%%%%%%%%%%%%%%%%%%%%%%%%%%%%%%%%%%%%%%%%%%%%%%%%%%%%%%%%%%%%%
%
\textit{supa} means \textbf{any type of horizontal surface or furniture}. 

\begin{supertabular}{p{5,5cm}|ll}
\textbf{supa}  && table, chair, sofa, ...  \\
\textbf{supa} lape &&  bed \\
\end{supertabular} 
%
%%%%%%%%%%%%%%%%%%%%%%%%%%%%%%%%%%%%%%%%%%%%%%%%%%%%%%%%%%%%%%%%%%%%%%%%%%
\index{\textit{suwi}}
\index{sweet}
\index{cute}
\index{sexy}
\index{attractive}
\index{candy}
\subsubsection*{\textit{suwi}}
%%%%%%%%%%%%%%%%%%%%%%%%%%%%%%%%%%%%%%%%%%%%%%%%%%%%%%%%%%%%%%%%%%%%%%%%%%
%
\textit{suwi} means "\textbf{sweet}", "\textbf{cute}", "\textbf{candy}" or some other type of sweet food. 
It \textbf{don't mean that it's sexy, attractive,} or anything like that. 

\begin{supertabular}{p{5,5cm}|ll}
jan lili sina li ' \textbf{suwi}. && Your baby is cute. \\
telo kili ni li ' \textbf{suwi}. && This fruit drink is sweet. \\
mi wile e \textbf{suwi}! && I want a cookie! \\
\end{supertabular} 
%
%%%%%%%%%%%%%%%%%%%%%%%%%%%%%%%%%%%%%%%%%%%%%%%%%%%%%%%%%%%%%%%%%%%%%%%%%%
\index{\textit{sin}}
\index{another}
\index{more}
\subsubsection*{\textit{sin}}
%%%%%%%%%%%%%%%%%%%%%%%%%%%%%%%%%%%%%%%%%%%%%%%%%%%%%%%%%%%%%%%%%%%%%%%%%%
%
sin as an \textbf{adjective} and simply means "\textbf{another}" or "\textbf{more}". 

\begin{supertabular}{p{5,5cm}|ll}
jan \textbf{sin} li kama. && More people are coming. \\
mi wile e suwi \textbf{sin}! && I want another/more cookie(s)! \\
\end{supertabular} 
%
%%%%%%%%%%%%%%%%%%%%%%%%%%%%%%%%%%%%%%%%%%%%%%%%%%%%%%%%%%%%%%%%%%%%%%%%%%
\index{\textit{olin}}
\index{love}
\subsubsection*{\textit{olin}}
%%%%%%%%%%%%%%%%%%%%%%%%%%%%%%%%%%%%%%%%%%%%%%%%%%%%%%%%%%%%%%%%%%%%%%%%%%
%
This word is used to mean "\textbf{love}". 
However, it only refers to affectionate love, like \textbf{loving people}. 
For example, you might \textit{olin} your girlfriend or your parents, but you don't \textit{olin} baseball. 
You can't \textit{olin} things or objects. 

\begin{supertabular}{p{5,5cm}|ll}
mi \textbf{olin} e sina. && I love you. \\
\end{supertabular}  
%
%%%%%%%%%%%%%%%%%%%%%%%%%%%%%%%%%%%%%%%%%%%%%%%%%%%%%%%%%%%%%%%%%%%%%%%%%%
\subsection*{Practice 10 (Answers: Page~\pageref{'questions_using_seme'})}
%%%%%%%%%%%%%%%%%%%%%%%%%%%%%%%%%%%%%%%%%%%%%%%%%%%%%%%%%%%%%%%%%%%%%%%%%%
%
\begin{supertabular}{p{5,5cm}|ll}
What do you want to do? &&   \\ % no-dictionary
Who loves you? &&   \\ % no-dictionary
Does it sweeten?  &&  \\ % no-dictionary
I'm going to bed. && \\  % no-dictionary
Are more people coming? &&  \\  % no-dictionary
Give me a lollipop! && \\   % no-dictionary
Who's there? &&   \\ % no-dictionary
Which bug hurt you?  &&  \\ % no-dictionary
Whom did you go with? &&  \\  % no-dictionary
He loves to eat. * &&  \\ % no-dictionary
 && \\ % no-dictionary
%%\end{supertabular} 
%%
%%And now try changing these sentences from Toki Pona into English: \\
%%
%%\begin{supertabular}{p{5,5cm}|ll}
jan Ken o, mi olin e sina.  && \\  % no-dictionary
ni li ' jan seme?  && \\  % no-dictionary
sina lon seme?   && \\  % no-dictionary
mi lon tan seme?  && \\  % no-dictionary
jan seme li ' meli sina?   && \\    % no-dictionary
sina tawa ma tomo, tan seme?    && \\   % no-dictionary
sina wile tawa, tawa  ma seme?      && \\  % no-dictionary
\end{supertabular} 

* Think carefully! This one is tricky. 
