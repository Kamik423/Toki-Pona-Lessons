%%%%%%%%%%%%%%%%%%%%%%%%%%%%%%%%%%%%%%%%%%%%%%%%%%%%%%%%%%%%%%%%%%%%%%%%%%
\section{Questions}
%%%%%%%%%%%%%%%%%%%%%%%%%%%%%%%%%%%%%%%%%%%%%%%%%%%%%%%%%%%%%%%%%%%%%%%%%%
\subsection*{Vocabulary}
%%%%%%%%%%%%%%%%%%%%%%%%%%%%%%%%%%%%%%%%%%%%%%%%%%%%%%%%%%%%%%%%%%%%%%%%%%
%
\begin{supertabular}{p{2,5cm}|ll}
%
\index{olin}
\textbf{\dots olin} && \textit{adjective}: love \\ % no-dictionary
\textbf{olin} && \textit{noun}: love \\ % no-dictionary
\textbf{olin (e \dots)} && \textit{verb transitive}: to love (a person) \\ % no-dictionary
 && \\ % no-dictionary
%
\index{seme}
\textbf{seme} && \textit{question pronoun}: what, which, wh- (question word) \\ % no-dictionary
 && \\ % no-dictionary
%
\index{sin}
\textbf{\dots sin} && \textit{adjective}: new, fresh, another, more \\ % no-dictionary
\textbf{ \dots sin } && \textit{adverb}: regenerative \\ % no-dictionary
\textbf{sin} && \textit{noun}: news, novelty, innovation, newness, new release \\ % no-dictionary
\textbf{sin (e \dots)} && \textit{verb transitive}: to renew, to renovate, to freshen \\ % no-dictionary 
 && \\ % no-dictionary
%
\index{supa}
\textbf{\dots supa} && \textit{adjective}: flat, shallow, flat-bottomed, horizontal \\ % no-dictionary
\textbf{supa} && \textit{noun}: horizontal surface, e.g furniture, table, chair, pillow, floor \\ % no-dictionary
 && \\ % no-dictionary
%
\index{suwi}
\textbf{\dots suwi} && \textit{adjective}: sweet, cute \\ % no-dictionary
\textbf{suwi} && \textit{noun}: candy, sweet food \\ % no-dictionary
\textbf{suwi (e \dots)} && \textit{verb transitive}: to sweeten \\ % no-dictionary
\end{supertabular} \\
%
%%%%%%%%%%%%%%%%%%%%%%%%%%%%%%%%%%%%%%%%%%%%%%%%%%%%%%%%%%%%%%%%%%%%%%%%%%
\newpage
%
\subsection*{The Question Pronoun \textit{seme}}
%
\index{\textit{seme}}
\index{question!\textit{seme}}
\index{what!subject}
\index{subject!what}
\index{question pronoun}
\index{pronoun!question}
\index{interrogative pronoun}
\index{pronoun!interrogative}
\index{question mark}
%%%%%%%%%%%%%%%%%%%%%%%%%%%%%%%%%%%%%%%%%%%%%%%%%%%%%%%%%%%%%%%%%%%%%%%%%%
%
We talked about how to ask questions that can be answered with a 'yes' or 'no'. 
However, we didn't talk about questions that require more in-depth answers. 
Well, to ask questions like these in Toki Pona, we have to use the question pronoun (interrogative pronoun) \textit{seme}. 
As you know, pronouns are proxies for different types of words.
The question pronoun \textit{seme} replaced the word or the part of a sentence which is inquired.
Depending on in what slot(s) \textit{seme} is used, it can represent different kinds of words or parts of sentences. 
Separators cannot be represented by a question pronoun \textit{seme}. 
At a question with \textit{seme} the sequence of word slots does not change. 
%
%%%%%%%%%%%%%%%%%%%%%%%%%%%%%%%%%%%%%%%%%%%%%%%%%%%%%%%%%%%%%%%%%%%%%%%%%%
\subsubsection*{Pardon?}
%
\index{pardon}
%%%%%%%%%%%%%%%%%%%%%%%%%%%%%%%%%%%%%%%%%%%%%%%%%%%%%%%%%%%%%%%%%%%%%%%%%%
%
If with the question pronoun \textit{seme} a complete question is made, nothing was understood. 

\begin{supertabular}{p{5,5cm}|ll}
seme? && Pardon? \\
\end{supertabular} 
%
%%%%%%%%%%%%%%%%%%%%%%%%%%%%%%%%%%%%%%%%%%%%%%%%%%%%%%%%%%%%%%%%%%%%%%%%%%
\subsubsection*{Who/What -- Subject}
%%%%%%%%%%%%%%%%%%%%%%%%%%%%%%%%%%%%%%%%%%%%%%%%%%%%%%%%%%%%%%%%%%%%%%%%%%
%
At questions who or what the subject is, in its place the question pronoun \textit{seme} is put in the sentence.
As you know this is the first position in the sentence. 

\begin{supertabular}{p{5,5cm}|ll}
seme li utala e sina? && Who/What attacked you? \\
seme li moku e kili mi? && Who/What is eating my fruit? \\
seme li lon poka mi? && Who/What is beside me? \\
seme li lon tomo mi? && Who/What is in my house? \\
seme li ' pona, tawa sina? && Who/What do you like? \\ 
\end{supertabular} 
%
%%%%%%%%%%%%%%%%%%%%%%%%%%%%%%%%%%%%%%%%%%%%%%%%%%%%%%%%%%%%%%%%%%%%%%%%%%
\subsubsection*{What / Where -- direct Object}
%
\index{where!object}
\index{what!object}
\index{object!what}
\index{object!where}
%%%%%%%%%%%%%%%%%%%%%%%%%%%%%%%%%%%%%%%%%%%%%%%%%%%%%%%%%%%%%%%%%%%%%%%%%%
%
At questions on direct object (recipient of action) the question pronoun \textit{seme} is used at the position of the direct object.
To simplify matters, we are taking a step-by-step approach.
Here's a statement:

\begin{supertabular}{p{5,5cm}|ll}
sina lukin e pipi. && You're watching a bug. \\
\end{supertabular} 

Now we're going to turn that sentence into a question. 

\begin{supertabular}{p{5,5cm}|ll}
sina lukin e seme? && What are you watching? \\
\end{supertabular} 

Here the question pronoun \textit{seme} represents the noun \textit{pipi}. 
The word order of the sentence does not change even when the sentence is a question. 

\begin{supertabular}{p{5,5cm}|ll}
sina pakala e seme? && What did you hurt? \\
ona li jo e seme? && What does he have? \\
\end{supertabular} 
%
%%%%%%%%%%%%%%%%%%%%%%%%%%%%%%%%%%%%%%%%%%%%%%%%%%%%%%%%%%%%%%%%%%%%%%%%%%
\subsubsection*{What -- Indirect Object}
%
\index{was!indirect Objekt}
\index{indirect Objekt!was}
\index{Objekt!was}
%%%%%%%%%%%%%%%%%%%%%%%%%%%%%%%%%%%%%%%%%%%%%%%%%%%%%%%%%%%%%%%%%%%%%%%%%%

If the question pronoun \textit{seme} is used after an intransitive verb, one asks for an indirect object. 

\begin{supertabular}{p{5,5cm}|ll}
sina kepeken seme? && What are you using? \\
pipi li mu seme? && What do bugs communicate? \\
\end{supertabular} 
%
%%%%%%%%%%%%%%%%%%%%%%%%%%%%%%%%%%%%%%%%%%%%%%%%%%%%%%%%%%%%%%%%%%%%%%%%%%
% \newpage
%
\subsubsection*{What -- Prepositional Object}

\index{what!prepositional object}
\index{prepositional object!what}
\index{object!what}
%%%%%%%%%%%%%%%%%%%%%%%%%%%%%%%%%%%%%%%%%%%%%%%%%%%%%%%%%%%%%%%%%%%%%%%%%%
%
If the question pronoun \textit{seme} is set after a preposition, a question (what) is possible for the prepositional object. \\

\begin{supertabular}{p{5,5cm}|ll}
sina pali e ni, kepeken seme? &&  What did you use to work on this? \\
\end{supertabular} 

%%%%%%%%%%%%%%%%%%%%%%%%%%%%%%%%%%%%%%%%%%%%%%%%%%%%%%%%%%%%%%%%%%%%%%%%%%
\subsubsection*{Why}
%
\index{why}
%%%%%%%%%%%%%%%%%%%%%%%%%%%%%%%%%%%%%%%%%%%%%%%%%%%%%%%%%%%%%%%%%%%%%%%%%%
%
The preposition \textit{tan} and the question pronoun \textit{seme} are used to formulate' why'. 
Both words form a prepositional object here.

\begin{supertabular}{p{5,5cm}|ll}
sina kama, tan seme? && Why did you come? \\
\end{supertabular} 

%%%%%%%%%%%%%%%%%%%%%%%%%%%%%%%%%%%%%%%%%%%%%%%%%%%%%%%%%%%%%%%%%%%%%%%%%%
\subsubsection*{Who(m)}
%
\index{who}
\index{whom}
%%%%%%%%%%%%%%%%%%%%%%%%%%%%%%%%%%%%%%%%%%%%%%%%%%%%%%%%%%%%%%%%%%%%%%%%%%
%
At questions on one person the question pronoun \textit{seme} represents an adjective after the noun \textit{jan}. 

\begin{supertabular}{p{5,5cm}|ll}
jan seme li moku? && Who is eating? \\
jan seme li tawa, lon poka sina? && Who went with you? \\
sina lukin e jan seme? && Whom did you see? \\
sina toki, tawa jan seme? && Whom are you talking to? \\
\end{supertabular} 
%
%%%%%%%%%%%%%%%%%%%%%%%%%%%%%%%%%%%%%%%%%%%%%%%%%%%%%%%%%%%%%%%%%%%%%%%%%%
\subsubsection*{Which}
%
\index{which}
%%%%%%%%%%%%%%%%%%%%%%%%%%%%%%%%%%%%%%%%%%%%%%%%%%%%%%%%%%%%%%%%%%%%%%%%%%
%
At questions on things the question pronoun \textit{seme} represents adjective after the corresponding noun. 

\begin{supertabular}{p{5,5cm}|ll}
ma seme li ' pona, tawa sina? && Which countries do you like? \\
sina kama, tan ma seme? && Which country do you come from? \\ 
\end{supertabular} 

The only reason that this concept might seem difficult is because you're tempted to move the word orders around, because many languages (including English) do it. 
One neat little trick you can do to check a translation is to think of the question as a plain statement, and then replace the question pronoun \textit{seme} with the pronoun \textit{ni}. 
%
%
\newpage
%%%%%%%%%%%%%%%%%%%%%%%%%%%%%%%%%%%%%%%%%%%%%%%%%%%%%%%%%%%%%%%%%%%%%%%%%%
\subsection*{Miscellaneous}
%%%%%%%%%%%%%%%%%%%%%%%%%%%%%%%%%%%%%%%%%%%%%%%%%%%%%%%%%%%%%%%%%%%%%%%%%%
%
%%%%%%%%%%%%%%%%%%%%%%%%%%%%%%%%%%%%%%%%%%%%%%%%%%%%%%%%%%%%%%%%%%%%%%%%%%
\subsubsection*{\textit{supa}}
%
\index{\textit{supa}!noun}
%%%%%%%%%%%%%%%%%%%%%%%%%%%%%%%%%%%%%%%%%%%%%%%%%%%%%%%%%%%%%%%%%%%%%%%%%%
%
\textit{supa} means any type of horizontal surface or furniture. 

\begin{supertabular}{p{5,5cm}|ll}
supa  && table, chair, sofa, ...  \\
supa lape &&  bed \\
\end{supertabular} 

%
%%%%%%%%%%%%%%%%%%%%%%%%%%%%%%%%%%%%%%%%%%%%%%%%%%%%%%%%%%%%%%%%%%%%%%%%%%
\subsubsection*{\textit{suwi}}
%
\index{\textit{suwi}!adjective}
\index{\textit{suwi}!noun}
%%%%%%%%%%%%%%%%%%%%%%%%%%%%%%%%%%%%%%%%%%%%%%%%%%%%%%%%%%%%%%%%%%%%%%%%%%
%
As an adjective \textit{suwi} means 'sweet' or 'cute'.
It don't mean that it's sexy, attractive, or anything like that. 
As a noun \textit{suwi} means  'candy' or some other type of sweet food. 

\begin{supertabular}{p{5,5cm}|ll}
jan lili sina li ' suwi. && Your baby is cute. \\
telo kili ni li ' suwi. && This fruit drink is sweet. \\
mi wile e suwi! && I want a cookie! \\
\end{supertabular} 

%
%%%%%%%%%%%%%%%%%%%%%%%%%%%%%%%%%%%%%%%%%%%%%%%%%%%%%%%%%%%%%%%%%%%%%%%%%%
\subsubsection*{\textit{sin}}
%
\index{\textit{sin}!adjective}
%%%%%%%%%%%%%%%%%%%%%%%%%%%%%%%%%%%%%%%%%%%%%%%%%%%%%%%%%%%%%%%%%%%%%%%%%%
%
\textit{sin} as an adjective and simply means 'another' or 'more'. 

\begin{supertabular}{p{5,5cm}|ll}
jan sin li kama. && More people are coming. \\
mi wile e suwi sin! && I want another/more cookie(s)! \\
\end{supertabular} 

%
%%%%%%%%%%%%%%%%%%%%%%%%%%%%%%%%%%%%%%%%%%%%%%%%%%%%%%%%%%%%%%%%%%%%%%%%%%
\subsubsection*{\textit{olin}}
%
\index{\textit{olin}}
%%%%%%%%%%%%%%%%%%%%%%%%%%%%%%%%%%%%%%%%%%%%%%%%%%%%%%%%%%%%%%%%%%%%%%%%%%
%
This word is used to mean 'love'. 
However, it only refers to affectionate love, like loving people. 
For example, you might \textit{olin} your girlfriend or your parents.
In this example \textit{olin} is a transitive verb.

\begin{supertabular}{p{5,5cm}|ll}
mi olin e sina. && I love you. \\
\end{supertabular}  
 
You can't \textit{olin} things or objects. 
Then the familiar pattern is used:

\begin{supertabular}{p{5,5cm}|ll}
ni li pona tawa mi. && I like this.\\
\end{supertabular} 
%
%
%%%%%%%%%%%%%%%%%%%%%%%%%%%%%%%%%%%%%%%%%%%%%%%%%%%%%%%%%%%%%%%%%%%%%%%%%%
\newpage{}
%
\subsection*{Practice (Answers: Page~\pageref{'questions_using_seme'})}
%%%%%%%%%%%%%%%%%%%%%%%%%%%%%%%%%%%%%%%%%%%%%%%%%%%%%%%%%%%%%%%%%%%%%%%%%%
%
Please write down your answers and check them afterwards. 

\begin{supertabular}{p{5,5cm}|ll}
How does the sentence structure change for a question in \textit{toki pona}? &&  \\ % no-dictionary
What kind of word has the word \textit{seme}? &&  \\ % no-dictionary
What is a reflexive pronoun? &&  \\ % no-dictionary
What can represent the word \textit{seme}? &&   \\ % no-dictionary
How do you ask for a person (who, whom)? &&  \\ % no-dictionary
How is a Why question asked? &&  \\ % no-dictionary
How do you ask for an indirect object? &&  \\ % no-dictionary
How to ask for a prepositional object? &&  \\ % no-dictionary
Are there nested subordinate clauses in \textit{toki pona}? && \\ % no-dictionary
\end{supertabular}

Try to translate these sentences. 
You can use the tool \textit{Toki Pona Parser} (\cite{www:rowa:02}) for spelling and grammar check. 

\begin{supertabular}{p{5,5cm}|ll}
What do you want to do? &&   \\ % no-dictionary
Who loves you? &&   \\ % no-dictionary
Does it sweeten?  &&  \\ % no-dictionary
I'm going to bed. && \\  % no-dictionary
Are more people coming? &&  \\  % no-dictionary
Give me a lollipop! && \\   % no-dictionary
Who's there? &&   \\ % no-dictionary
Which bug hurt you?  &&  \\ % no-dictionary
He loves to eat. * &&  \\ % no-dictionary
\end{supertabular}

\begin{supertabular}{p{5,5cm}|ll}
jan Ken o, mi olin e sina.  && \\  % no-dictionary
ni li ' jan seme?  && \\  % no-dictionary
sina lon seme?   && \\  % no-dictionary
mi lon tan seme?  && \\  % no-dictionary
jan seme li ' meli sina?   && \\    % no-dictionary
sina tawa ma tomo, tan seme?    && \\   % no-dictionary
sina wile tawa, tawa  ma seme?      && \\  % no-dictionary
\end{supertabular} 

* Think carefully! This one is tricky. 
