%%%%%%%%%%%%%%%%%%%%%%%%%%%%%%%%%%%%%%%%%%%%%%%%%%%%%%%%%%%%%%%%%%%%%%%%%%
\section{Gender,Unofficial Words,Addressing People,\\Interjections, Commands} 
%%%%%%%%%%%%%%%%%%%%%%%%%%%%%%%%%%%%%%%%%%%%%%%%%%%%%%%%%%%%%%%%%%%%%%%%%%
%
%%%%%%%%%%%%%%%%%%%%%%%%%%%%%%%%%%%%%%%%%%%%%%%%%%%%%%%%%%%%%%%%%%%%%%%%%%
\subsection*{Vocabulary}
%%%%%%%%%%%%%%%%%%%%%%%%%%%%%%%%%%%%%%%%%%%%%%%%%%%%%%%%%%%%%%%%%%%%%%%%%%
%
\index{\textit{a}}
\index{\textit{awen}}
\index{\textit{mama}}
\index{\textit{mije}}
\index{\textit{meli}}
\index{\textit{mu}}
\index{\textit{nimi}}
\index{\textit{o}}
%\index{\textit{pata}}
\index{\textit{pona}}
\index{\textit{toki}}
\index{wait}
\index{pause}
\index{remaining}
\index{parent}
\index{man}
\index{husband}
\index{boyfriend}
\index{male}
\index{woman}
\index{wife}
\index{girlfriend}
\index{female}
\index{sound!animal}
\index{animal!sound}
\index{name}
\index{word}
\index{vocative}
\index{imperative}
%\index{sibling}
\index{yay}
\index{cool}
\index{good}
\index{hey}
\index{language}
\begin{supertabular}{p{2,5cm}|ll}
\textbf{a} && \textit{interjection}: ah, ha, uh, oh, ooh, aw, well (emotion word) \\ % no-dictionary
\textbf{a a a!} && \textit{interjection}: laugh \\ % no-dictionary
 && \\ % no-dictionary
\textbf{\dots awen} && \textit{adjective}: remaining, stationary, permanent, sedentary \\ % no-dictionary
\textbf{\dots awen} && \textit{adverb}: still, yet \\ % no-dictionary
\textbf{awen} && \textit{noun}: inertia, continuity, continuum, stay \\ % no-dictionary
\textbf{awen} && \textit{verb intransitive}: to stay, to wait,to remain \\ % no-dictionary
\textbf{awen (e \dots)} && \textit{verb transitive}: to keep \\ % no-dictionary
 && \\ % no-dictionary
\textbf{\dots mama} && \textit{adjective}: of the parent, parental, maternal, fatherly, motherly, mumsy \\ % no-dictionary
\textbf{mama} && \textit{noun}: parent, mother, father \\ % no-dictionary
\textbf{mama (e \dots)} && \textit{verb transitive}: to mother sb., to wet-nurse, mothering \\ % no-dictionary
 && \\ % no-dictionary
\textbf{\dots meli} && \textit{adjective}: female, feminine, womanly \\ % no-dictionary
\textbf{meli} && \textit{noun}: woman, female, girl, wife, girlfriend \\ % no-dictionary
 && \\ % no-dictionary
\textbf{\dots mije} && \textit{adjective}: male, masculine, manly \\ % no-dictionary
\textbf{mije} && \textit{noun}: man, male, husband, boyfriend \\ % no-dictionary
 && \\ % no-dictionary
\textbf{\dots mu} && \textit{adjective}: animal nois- \\ % no-dictionary
\textbf{\dots mu} && \textit{adverb}: animal nois- \\ % no-dictionary
\textbf{mu!} && \textit{interjection}: woof! meow! moo! etc. (cute animal noise) \\ % no-dictionary
\textbf{mu} && \textit{noun}: animal noise \\ % no-dictionary
\textbf{mu (e \dots)} && \textit{verb transitive}: to make animal nois \\ % no-dictionary
 && \\ % no-dictionary
\textbf{nimi} && \textit{noun}: word, name \\ % no-dictionary
\textbf{nimi (e \dots )} && \textit{verb transitive}: to name \\ % no-dictionary
 && \\ % no-dictionary
\textbf{o!} && \textit{interjection}: hey! (calling somebody's attention) \\ % no-dictionary
\textbf{\dots o, \dots} && \textit{interjection}: adressing people \\ % no-dictionary
\textbf{o \dots !} && \textit{subject}: An 'o' is used for imperative (commands). 'o' is the subject.  \\ % no-dictionary
\textbf{\dots o \dots !} && \textit{separator}: An 'o' is used for imperative (commands): 'o' replace 'li'. \\ % no-dictionary
 && \\ % no-dictionary
\textbf{ala!} && \textit{interjection}: no! \\ % no-dictionary
 && \\ % no-dictionary
\textbf{ike!} && \textit{interjection}: oh dear! woe! alas! \\ % no-dictionary
 && \\ % no-dictionary
\textbf{jaki!} && \textit{interjection}: ew! yuck! \\ % no-dictionary
 && \\ % no-dictionary
\textbf{pakala!} && \textit{interjection}: damn! fuck! \\ % no-dictionary
 && \\ % no-dictionary
\textbf{pona!} && \textit{interjection}: great! good! thanks! OK! cool! yay! \\ % no-dictionary
 && \\ % no-dictionary
\textbf{toki!} && \textit{interjection}: hello, hi, good morning, \\ % no-dictionary
 && \\ % no-dictionary
\textbf{"} && \textit{separator}: Quotation marks are used for words with original spelling or for quotes. \\ % no-dictionary
\end{supertabular} \\
%
%%%%%%%%%%%%%%%%%%%%%%%%%%%%%%%%%%%%%%%%%%%%%%%%%%%%%%%%%%%%%%%%%%%%%%%%%%
\index{gender}
\subsection*{Gender}
%%%%%%%%%%%%%%%%%%%%%%%%%%%%%%%%%%%%%%%%%%%%%%%%%%%%%%%%%%%%%%%%%%%%%%%%%%
%
Toki Pona doesn't have any grammatical gender like in most Western languages.  
However, some words in Toki Pona (such as \textit{mama}) don't tell you which gender a person is, and so we use \textit{mije} and \textit{meli} to distinguish. 

\index{mother}
\index{father}
\index{sister}
\index{brother}
\begin{supertabular}{p{5,5cm}|ll}
mama && a parent in general (mother or father) \\
mama meli && mother \\
mama mije && father \\
% pata && a sibling (brother or sister) \\
% pata meli && sister \\
% pata mije && brother \\
\end{supertabular} 

%
%%%%%%%%%%%%%%%%%%%%%%%%%%%%%%%%%%%%%%%%%%%%%%%%%%%%%%%%%%%%%%%%%%%%%%%%%%
\index{unofficial words}
\index{words!unofficial}
\index{nation!name of}
\index{country!name of}
\index{languages!name of}
\index{religion!name of}
\index{person!name of}
\index{woman!name of}
\index{man!name of}
\index{name!of a nation}
\index{name!of a country}
\index{name!of a languages}
\index{name!of a religion}
\index{name!of a person}
\index{name!of a woman}
\index{name!of a man}
\subsection*{Unofficial words}
%%%%%%%%%%%%%%%%%%%%%%%%%%%%%%%%%%%%%%%%%%%%%%%%%%%%%%%%%%%%%%%%%%%%%%%%%%
%
Take a moment to look over the Toki Pona dictionary in the Appendix (Page~\pageref{'unofficial_words'}). 
There are no words for the names of nations; there are also aren't any words for religions or even other languages. 
The reason that these words aren't in the dictionary is because they are 'unofficial'. 
Before using an unofficial word, we often adapt the word to fit into Toki Pona's phonetic rules. 
So, for example, America becomes \textit{Mewika}, Canada becomes \textit{Kanata}, and so on. 
Now the thing about these unofficial words is that they can never be used by themselves. 
They are always treated like adjectives. 
Either the unofficial word is used directly after the noun or as predicate adjective after \textit{li}. 

\textbf{country names}

After the noun \textit{ma} an unofficial word is used as the country name. 

\begin{supertabular}{p{5,5cm}|ll}
ma Kanata li ' pona. && Canada is good. \\
ma Italija li ' pona lukin. && Italy is beautiful. \\
mi wile tawa, tawa ma Tosi. && I want to go to Germany. \\
\end{supertabular} 

Here is an example of an unofficial word as predicate adjective.

\begin{supertabular}{p{5,5cm}|ll}
ma mi li ' Tosi. && My homeland is Germany.  \\
\end{supertabular}

\index{city}
As we have learnt the combination of the noun \textit{ma} and the adjective \textit{tomo} mean 'city'.
After this combination, an unofficial word is used as a city name. 

\begin{supertabular}{p{5,5cm}|ll}
ma tomo Lantan li ' suli. && London is big. \\
ma tomo Pelin && Berlin \\
ma tomo Alenta && Atlanta \\
ma tomo Loma && Rome \\
mi kama, tan ma tomo Pelin. && I'm from Berlin. \\
\end{supertabular} 

Here is an example of an unofficial word as predicate adjective.

\begin{supertabular}{p{5,5cm}|ll}
ma tomo mi li ' Pelin. && My homecity is Berlin.  \\
\end{supertabular}

If you want to talk about a language, you simply use the noun\textit{toki} and then attach the unofficial word onto it. 

\begin{supertabular}{p{5,5cm}|ll}
toki Inli li ' pona. && The English language is good. \\
ma Inli li ' pona. && England is good. \\
toki Kanse && French language \\
toki Epelanto li ' pona. && Esperanto ist einfach. \\
\end{supertabular} 

Here is an example of an unofficial word as predicate adjective.

\begin{supertabular}{p{5,5cm}|ll}
toki mi li ' Tosi. && My mother tongue is German. \\
\end{supertabular}

A resident of a country is named by nouns \textit{jan}, \textit{meli} or \textit{mije} and the unofficial Word.

\begin{supertabular}{p{5,5cm}|ll}
jan Kanata && Canadian person \\
jan Mesiko && Mexican person \\
meli Italija && Italian woman \\
mije Epanja && Spanish man \\
\end{supertabular} 

Now suppose you want to talk about someone using their name. 
For example, what if you want to say 'Lisa is cool'? 
To say a person's name in Toki Pona, you just say the noun \textit{jan} and then the person's name. 

\begin{supertabular}{p{5,5cm}|ll}
jan Lisa li ' pona. && Lisa is cool. \\
\end{supertabular} 

Like for the names of countries, we often adapt a person's name to fit into Toki Pona's phonetic rules. 
Keep in mind that no one is going to pressure you to adopt a tokiponized name; it's just for fun. 

\begin{supertabular}{p{5,5cm}|ll}
jan Pentan li pana e sona, tawa mi. && Brandon teaches to me. \\
jan Mewi li toki, tawa mi. && Mary's talking to me. \\
jan Nesan li ' musi. && Nathan is funny. \\
jan Eta li ' jan unpa. && Heather is a whore. \\
pana e sona && to teach (give knowledge) \\
\end{supertabular} 

This is the way to say your name. 

\begin{supertabular}{p{5,5cm}|ll}
mi ' jan Pepe. && I am Pepe. \\
nimi mi li ' Pepe. && My name is Pepe. 
\end{supertabular} 

Nobody is forcing you to use a name in Toki Pona style.
This is pure fun.
If you use the original spelling, please put it in quotation marks.

\begin{supertabular}{p{5,5cm}|ll}
mi ' jan "Robert". && I'm Robert. \\
\end{supertabular} 
%
%%%%%%%%%%%%%%%%%%%%%%%%%%%%%%%%%%%%%%%%%%%%%%%%%%%%%%%%%%%%%%%%%%%%%%%%%%
% \newpage
%\subsection*{Addressing People, Commands, Interjections}
\index{addressing!\textit{o}}
\index{\textit{o}!addressing people}
\index{addressing people}
\index{person!addressing}
\index{person!attention}
\index{attention!a person}
\subsection*{Addressing People}
%%%%%%%%%%%%%%%%%%%%%%%%%%%%%%%%%%%%%%%%%%%%%%%%%%%%%%%%%%%%%%%%%%%%%%%%%%
%
Sometimes you need to get a person's attention before you can talk to him. 
When you want to address someone like that before saying the sentence, you just follow this same pattern. 
\textit{jan} (name) \textit{o,} (sentence). 
Note the comma behind the \textit{o}. 

\begin{supertabular}{p{5,5cm}|ll}
jan Ken o, pipi li lon len sina. && Ken, a bug is on your shirt. \\
jan Keli o, sina ' pona lukin. && Kelly, you are pretty. \\
jan Mawen o, sina wile ala wile moku? && Marvin, are you hungry? \\
jan Tepani o, sina ' ike, tawa mi. && Steffany, I don't like you. \\
\end{supertabular} 
%
%%%%%%%%%%%%%%%%%%%%%%%%%%%%%%%%%%%%%%%%%%%%%%%%%%%%%%%%%%%%%%%%%%%%%%%%%%
\index{command}
\index{command!\textit{o}}
\index{\textit{o}!command}
\subsection*{Commands}
%%%%%%%%%%%%%%%%%%%%%%%%%%%%%%%%%%%%%%%%%%%%%%%%%%%%%%%%%%%%%%%%%%%%%%%%%%
%
The command form is introduced with \textit{o} and ends with an exclamation mark.
The interjection word \textit{o} is the subject here.

\begin{supertabular}{p{5,5cm}|ll}
o pali! && Work! \\
o awen! && Wait! \\
o ' pona! && Be good! \\
o lukin e ni! && Watch this! \\
o tawa, tawa ma tomo, lon poka jan pona sina! && Go to the city with your friend! \\
\end{supertabular} 

We've learned how to address people and how to make commands; now let's put these two concepts together. 
Suppose you want to address someone and tell them to do something. 
Notice how one of the \textit{o}'s got dropped, as did the comma. 

\begin{supertabular}{p{5,5cm}|ll}
jan San o, ...  && John, ... \\ % no-dictionary
 ... o tawa tomo sina! && ... go to your house! \\ % no-dictionary
jan San o tawa tomo sina!  && John, go to your house! \\
 && \\ % no-dictionary
jan Ta o toki ala, tawa mi! && Todd, don't talk to me! \\
jan Sesi o moku e kili ni! && Jessie, eat this fruit!. \\
\end{supertabular} 

The separator \textit{o} replaces the separator \textit{li}. After \textit{mi} and \textit{sina} also the separator \textit{o} is used. 

\begin{supertabular}{p{5,5cm}|ll}
sina o telo e sina! && Wash yourself! \\
\end{supertabular} 

\index{let's go}
This structure can also be used to make sentences like 'Let's go'.

\begin{supertabular}{p{5,5cm}|ll}
mi mute o tawa! && Let's go. \\
mi mute o ' musi! && Let's have fun. \\
\end{supertabular} 
%
%%%%%%%%%%%%%%%%%%%%%%%%%%%%%%%%%%%%%%%%%%%%%%%%%%%%%%%%%%%%%%%%%%%%%%%%%%
% \newpage
\index{interjection}
\index{interjection!of an animal}
\index{interjection!laughter}
\index{laughter}
\index{Hahaha}
\index{Hello}
\index{f-ck}
\index{d-mm}
\index{cool}
\subsection*{Interjections}
%%%%%%%%%%%%%%%%%%%%%%%%%%%%%%%%%%%%%%%%%%%%%%%%%%%%%%%%%%%%%%%%%%%%%%%%%%
%
An interjection sentence makes conveys excitement.
Interjections sentences often consist only of a noun or an interjection word (e. g.: \textit{a}) and end with an exclamation mark.

\begin{supertabular}{p{5,5cm}|ll}
jan Lisa o, toki! && Hello Lisa! \\
pona! && Yay! Good! Hoorah! \\
ike! && Oh no! Uh! oh! Alas! \\
pakala! && F-ck! D-mn! \\
mu! && Meow! Woof! Grrr! Moo! (sounds made by animals) \\
a! && Ooh, Ahh! Unh! Oh! \\
a a a! && Hahaha! (laughter) \\
\end{supertabular} 

The interjection word \textit{a} adds emotion or stress. 
It can be used at the end of a sentence.
Use the Interjektion-Word \textit{a} sparingly!

\begin{supertabular}{p{5,5cm}|ll}
sina ' suli a! && You are so tall! \\
\end{supertabular} 

The interjection words \textit{o} and \textit{a} only used when the person makes you feel really emotional. 
For example, if you haven't seen a person for a long time or if you have sex with them and you still speak perfect Toki Pona. 

\begin{supertabular}{p{5,5cm}|ll}
jan Epi o a! && Oh Abbie! \\
\end{supertabular} 

\index{Good day}
\index{Sleep well}
\index{Enjoy your meal}
\index{Bye}
\index{Good bye}
\index{Welcome}
\index{Have fun}
%
The second group of interjections are kind like salutations.
They usually consist of a noun, an optional adjective and an exclamation mark. 

\begin{supertabular}{p{5,5cm}|ll}
toki! && Hello!, Hi! \\
suno pona! && Good sun! Good day! \\
lape pona! && Sleep well! Have a good night! \\
moku pona! && Good food! Enjoy your meal! \\
mi tawa && I'm going. Bye! \\
tawa pona! && (in reply) Go well! Good bye! \\
kama pona! && Come well! Welcome! \\
musi pona! && Good fun! Have fun! \\
\end{supertabular}  

They can also consist of a complete sentence with an exclamation mark.

\begin{supertabular}{p{5,5cm}|ll}
jan Lisa o, toki! && Hello Lisa! \\
mi tawa && I'm going. Bye! \\
\end{supertabular}
%
%%%%%%%%%%%%%%%%%%%%%%%%%%%%%%%%%%%%%%%%%%%%%%%%%%%%%%%%%%%%%%%%%%%%%%%%%%
\newpage
\subsection*{Practice 9 (Answers: Page~\pageref{'gender_unofficial_words_etc'})}
%%%%%%%%%%%%%%%%%%%%%%%%%%%%%%%%%%%%%%%%%%%%%%%%%%%%%%%%%%%%%%%%%%%%%%%%%%
%
Try to translate these sentences. 
You can use the tool \textit{Toki Pona Parser} (\cite{www:rowa:02}) for spelling and grammar check. 

\begin{supertabular}{p{5,5cm}|ll}
Susan is crazy. &&   \\ % no-dictionary
Go!  &&  \\ % no-dictionary
Mama, wait!  &&  \\ % no-dictionary
I come from Europe. * &&  \\ % no-dictionary
Hahaha! That's funny!  &&  \\ % no-dictionary
My name is Ken. &&  \\  % no-dictionary
Hello Lisa!  && \\  % no-dictionary
F-ck!  &&   \\ % no-dictionary
I want to go to Australia. **  &&  \\  % no-dictionary
Bye! *** &&   \\ % no-dictionary
 && \\ % no-dictionary
mu! &&   \\ % no-dictionary
mi wile kama sona e toki Inli. **** &&  \\ % no-dictionary
jan Ana o pana e moku tawa mi!  &&  \\ % no-dictionary
o tawa musi, lon poka mi! ***** &&  \\ % no-dictionary
jan Mose o lawa e mi mute, tawa ma pona! &&   \\ % no-dictionary
tawa pona!  &&   \\ % no-dictionary
\end{supertabular} 

* Europe = Elopa \\
** Australia = Oselija \\
*** spoken by the person who's leaving \\
****  \textit{kama sona} = learn. This is similar to how \textit{kama jo} = 'get', like we learned in lesson six. \\
% ***** If you've forgotten what \textit{tawa musi} means, we learned it in lesson eight. 
%%%%%%%%%%%%%%%%%%%%%%%%%%%%%%%%%%%%%%%%%%%%%%%%%%%%%%%%%%%%%%%%%%%%%%%%%%
% eof
