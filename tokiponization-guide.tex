%%%%%%%%%%%%%%%%%%%%%%%%%%%%%%%%%%%%%%%%%%%%%%%%%%%%%%%%%%%%%%%%%%%%%%%%%%
\section{Tokiponization Guidelines}
%
\label{'phonet_trans'}
%%%%%%%%%%%%%%%%%%%%%%%%%%%%%%%%%%%%%%%%%%%%%%%%%%%%%%%%%%%%%%%%%%%%%%%%%%
%
To create the Toki Pona version of a foreign name, you may use the following guidelines.
Also see Alphabet and sounds (Page~\pageref{'pronunciation_alphabet'}) for rules on what Toki Pona syllables and words are possible.
You can find a tool for transliterate of names in tokipona.net \cite{www:tokipona.net:01}.

\begin{itemize}
    \item
          It is always better to translate the "idea" of a foreign word before attempting to create a new phonetic transcription that may not be recognizable by everyone.
          (Example: Jean Chr�tien, Prime Minister of Canada = jan lawa pi ma Kanata, rather than jan Kesijen)
    \item
          Use the native pronunciation as a basis, rather than the spelling.
    \item
          If more than one language is spoken locally, use the dominant one.
    \item
          If it does not belong to any one language, use an international form.
          (Example: Atlantik = Alansi)
    \item
          Use the colloquial pronunciation that locals actually and commonly use, rather than the "proper" or standard pronunciation.
          (Example: Toronto = Towano, not Tolonto)
    \item
          If a person chooses to have a Toki Pona name, he can choose whatever he wants and does not necessarily have to follow these guidelines.
    \item
          Names of nations, languages, religions have already been established.
          If one is missing from the official list, make a suggestion on the Toki Pona discussion list.
    \item
          If possible, find a common root between the name of the nation, the language and the people.
          (Example: England, English, English(wo)man = Inli)
    \item
          Cities and locations can be given a Toki Pona name, but they will only have an official name if they are internationally known.
    \item
          If full Tokiponization would compromise intelligibility, you can always leave a foreign name as is.
\end{itemize}

\subsection*{Syllables of Unoffial Words}

\begin{itemize}
    \item
          Every syllable consists of a consonant and a vovel, plus an optional \textit{n}.
    \item
          The next syllable after a optional \textit{n} cannot start with a \textit{n}.
    \item
          The first syllable of a word does not need to beginn with a consonant.
    \item
          The syllables \textit{ti} and \textit{tin} become \textit{si} and \textit{sin}.
    \item
          The consonant \textit{w} cannot appear before \textit{o} or \textit{u}.
    \item
          The consonant \textit{j} cannot appear before \textit{i}.
\end{itemize}

%
%%%%%%%%%%%%%%%%%%%%%%%%%%%%%%%%%%%%%%%%%%%%%%%%%%%%%%%%%%%%%%%%%%%%%%%%%%
\newpage
\subsection*{Phonetic Guidelines}
%%%%%%%%%%%%%%%%%%%%%%%%%%%%%%%%%%%%%%%%%%%%%%%%%%%%%%%%%%%%%%%%%%%%%%%%%%
%
\begin{itemize}
    \item
          Voiced plosives become voiceless. (Example: b = p, d = t, g = k)
    \item
          v becomes w.
    \item
          f becomes p.
    \item
          The trilled or tapped [r] of most world languages becomes l.
    \item
          The approximant r of languages like English becomes w.
    \item
          Any uvular or velar consonant becomes k, including the French or German r.
    \item
          At the end of a word, The sh sound may be converted to si.
          (Example: Lush = Lusi)
    \item
          The schwa can become any vowel in Toki Pona and is often influenced by neighbouring vowels for cute reduplication.
    \item
          It is better to keep the same number of syllables and drop a consonant than add a new vowel.
          (Example: Chuck = Sa, not Saku)
    \item
          When dealing with consonant clusters, the dominant plosive is generally kept, dropping fricatives such as [s] and laterals such al [l] first.
          (Example: Esperanto = Epelanto)
          You may also choose to keep the consonant at the head of the new syllable (Example: Atling = Alin).
    \item
          Approximants like [j] and [w] in consonant clusters can either be converted into a syllable of their own
          (Swe = Suwe; Pju = Piju) or dropped entirely (Swe = Se; Pju = Pu).
    \item
          In some cases, it is better to change the letter order slightly, rather than dropping a sound.
          (Ex: Lubnan = Lunpan, not Lupan or Lunan)
    \item
          Dental fricatives such as English th can either convert to t or s.
    \item
          The illegal syllables ti, wo and wu convert to si, o and u.
          (Example: Antarktika = Antasika)
    \item
          Affricates generally convert to fricatives.
          (Example: John = San, not Tan)
    \item
          Any nasal consonant at the end of a syllable converts to n.
          (Example: Fam = Pan)
    \item
          Nasal vowels (in French and Portuguese) also convert to syllable-final n.
    \item
          If necessary to preserve proper syllable structure, the consonant w or j can be inserted as a euphonic glide. (Example: Tai = Tawi; Nihon = Nijon; Eom = Ejon) It may also be possible to relocate a consonant that would have otherwise been dropped in the conversion. (Example: Monkeal = Monkela, not Monkeja; Euska = Esuka)
    \item
          Voiceless lateral consonants convert to s.
    \item
          If necessary, you may want to tweak a word to avoid a potentially misleading homonym.
          (Example: Allah = jan sewi Ila, not jan sewi Ala no God).
          If possible, use a related word in the source language rather than introducing an arbitrary change.
          (In Arabic, Allah actually means the God, whereas Illah means God.)
\end{itemize}
%
%%%%%%%%%%%%%%%%%%%%%%%%%%%%%%%%%%%%%%%%%%%%%%%%%%%%%%%%%%%%%%%%%%%%%%%%%%
% eof
