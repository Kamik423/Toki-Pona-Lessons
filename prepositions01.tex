%%%%%%%%%%%%%%%%%%%%%%%%%%%%%%%%%%%%%%%%%%%%%%%%%%%%%%%%%%%%%%%%%%%%%%%%%%
\section{Prepositions, Transitive and Intransitive Verbs}
%%%%%%%%%%%%%%%%%%%%%%%%%%%%%%%%%%%%%%%%%%%%%%%%%%%%%%%%%%%%%%%%%%%%%%%%%%
\index{preposition}
\subsection*{Vocabulary}
%
\index{\textit{lon}}
\index{\textit{kepeken}}
\index{\textit{tawa}}
\index{\textit{kama}}       
\index{\textit{kiwen}}
\index{\textit{kon}}
% \index{\textit{leko}}
\index{\textit{pana}}
\index{\textit{poki}}
\index{\textit{toki}}
\index{exist}
\index{on}
\index{at}
\index{in}
\index{be!in}
\index{be!at}
\index{be!on}
\index{with}
\index{using}
\index{go}
\index{move}
\index{for}
\index{come}
\index{happen}
\index{cause}
\index{stone}
\index{rock}
\index{air}
\index{atmosphere}
\index{spirit}
\index{wind}
\index{square}
\index{block}
\index{stairs}
\index{give}
\index{send}
\index{release}
\index{emit}
\index{container}
\index{bowl}
\index{glass}
\index{cup}
\index{box}
\index{language}
\index{talk}
\index{speak}
\begin{supertabular}{p{2,5cm}|ll}
\textbf{\dots lon} && \textit{adjective}: true, existing, correct, real, genuine \\ % no-dictionary
\textbf{lon} && \textit{noun}: existence, being, presence \\ % no-dictionary
\textbf{\dots , lon \dots} && \textit{preposition}: be (located) in/at/on \\ % no-dictionary
\textbf{lon} && \textit{verb intransitive}: to be there, to be present, to be real/true, to exist \\ % no-dictionary
 && \\ % no-dictionary
\textbf{kepeken} && \textit{noun}: use, usage, tool \\ % no-dictionary
\textbf{\dots , kepeken \dots} && \textit{preposition}: with \\ % no-dictionary
\textbf{kepeken} && \textit{verb intransitive}: to use \\ % no-dictionary
\textbf{kepeken \dots} && \textit{verb pre}: to use \\ % no-dictionary
 && \\ % no-dictionary
\textbf{\dots tawa} && \textit{adjective}: moving, mobile \\ % no-dictionary
\textbf{\dots tawa} && \textit{adverb}: moving, mobile \\ % no-dictionary
\textbf{tawa} && \textit{noun}: movement, transportation \\ % no-dictionary
\textbf{\dots , tawa \dots} && \textit{preposition}: to, in order to, towards, for, until \\ % no-dictionary
\textbf{tawa} && \textit{verb intransitive}: go to, walk, travel, move, leave \\ % no-dictionary
\textbf{tawa (e \dots)} && \textit{verb transitive}: to move, to displace \\ % no-dictionary
 && \\ % no-dictionary
\textbf{\dots kama} && \textit{adjective}: coming, future \\ % no-dictionary
\textbf{\dots kama} && \textit{adverb}: coming, future \\ % no-dictionary
\textbf{kama} && \textit{noun}: event, happening, chance, arrival, beginning \\ % no-dictionary
\textbf{kama} && \textit{verb intransitive}: to come, to become, to arrive, to happen \\ % no-dictionary
\textbf{kama \dots} && \textit{verb pre}: to become, to mange to \\ % no-dictionary
\textbf{kama (e \dots)} && \textit{verb transitive}: to bring about, to summon \\ % no-dictionary
kama \textbf{jo (e \dots)} && \textit{verb transitive}: to get \\ % no-dictionary
 && \\ % no-dictionary
\textbf{\dots kiwen} && \textit{adjective}: hard, solid, stone-like, made of stone or metal \\ % no-dictionary
\textbf{\dots kiwen} && \textit{adverb}: hard, solid, stone-like, made of stone or metal \\ % no-dictionary
\textbf{kiwen} && \textit{noun}: hard thing, rock, stone, metal, mineral, clay \\ % no-dictionary
\textbf{kiwen (e \dots)} && \textit{verb transitive}: to solidify, to harden, to petrify, to fossilize \\ % no-dictionary
 && \\ % no-dictionary
\textbf{\dots kon} && \textit{adjective}: air-like, ethereal, gaseous \\ % no-dictionary
\textbf{\dots kon} && \textit{adverb}: air-like, ethereal, gaseous \\ % no-dictionary
\textbf{kon} && \textit{noun}: air, wind, smell, soul \\ % no-dictionary
\textbf{kon} && \textit{verb intransitive:}: to breathe \\ % no-dictionary
\textbf{kon (e \dots)} && \textit{verb transitive}: to blow away something, to puff away something \\ % no-dictionary
 && \\ % no-dictionary
\textbf{\dots pana} && \textit{adjective}: generous \\ % no-dictionary
\textbf{pana} && \textit{noun}: giving, transfer, exchange \\ % no-dictionary
\textbf{pana (e \dots)} && \textit{verb transitive}: to give, to put, to send, to place, to release, to emit, to cause \\ % no-dictionary
 && \\ % no-dictionary
\textbf{poki} && \textit{noun}: container, box, bowl, cup, glass \\ % no-dictionary
\textbf{poki (e \dots)} && \textit{verb transitive}: to box up, to put in, to can, to bottle \\ % no-dictionary
 && \\ % no-dictionary
\textbf{\dots toki} && \textit{adjective}: speaking, eloquent, linguistic, verbal, grammatical \\ % no-dictionary
\textbf{\dots toki} && \textit{adverb}: speaking, eloquent, linguistic, verbal, grammatical \\ % no-dictionary
\textbf{toki} && \textit{noun}: language, speech, tongue, lingo, jargon, \\ % no-dictionary
\textbf{toki} && \textit{verb intransitive}: to talk, to chat, to communicate \\ % no-dictionary
\textbf{toki (e \dots)} && \textit{verb transitive}: to speak, to talk, to say, to pronounce, to discourse \\ % no-dictionary
\end{supertabular} \\
%
%%%%%%%%%%%%%%%%%%%%%%%%%%%%%%%%%%%%%%%%%%%%%%%%%%%%%%%%%%%%%%%%%%%%%%%%%%
\subsection*{Prepositions}
%%%%%%%%%%%%%%%%%%%%%%%%%%%%%%%%%%%%%%%%%%%%%%%%%%%%%%%%%%%%%%%%%%%%%%%%%%
%
A preposition describes a relationship between other words in a sentence  and usually stand in front of nouns. 
In Toki Pona prepositions start a prepositional object. 
A prepositional object stands at the end of a sentence. 
It is recommended that you put a comma before a preposition.
%
%%%%%%%%%%%%%%%%%%%%%%%%%%%%%%%%%%%%%%%%%%%%%%%%%%%%%%%%%%%%%%%%%%%%%%%%%%
\subsection*{Transitive Verbs vs. Intransitive Verbs}
%%%%%%%%%%%%%%%%%%%%%%%%%%%%%%%%%%%%%%%%%%%%%%%%%%%%%%%%%%%%%%%%%%%%%%%%%%
%
In the sentence "I'm going to my house," the speaker simply went home.
He did not do anything to his house. 
"go" is an intransitive verb here.
In the sentence "I'm building a house." is an \textbf{object} (house) and "build" is an transitive verb.
In Toki Pona an \textbf{\textit{e} follow a transitive verb}. 
%
%%%%%%%%%%%%%%%%%%%%%%%%%%%%%%%%%%%%%%%%%%%%%%%%%%%%%%%%%%%%%%%%%%%%%%%%%%
\index{preposition!lon}
\index{verb!\textit{lon}}
\index{\textit{lon}!preposition}
\index{\textit{lon}!verb}
\subsection*{\textit{lon}}
%%%%%%%%%%%%%%%%%%%%%%%%%%%%%%%%%%%%%%%%%%%%%%%%%%%%%%%%%%%%%%%%%%%%%%%%%%
%
\begin{supertabular}{p{5,5cm}|ll}
mi \textbf{lon} tomo. && I'm in the house. \\
mi moku, \textbf{lon} tomo. && I eat in the house. \\
\end{supertabular} 

In the first example, \textit{lon} is used as an intransitive verb; in the second, it is used as a preposition. 
\textit{lon} can be used as both a \textbf{verb} and a \textbf{preposition}. 
Optionally, a comma can be inserted before a preposition.

\begin{supertabular}{p{5,5cm}|ll}
suno li \textbf{lon} sewi. && The sun is in the sky. \\
mi telo e mi, \textbf{lon} tomo telo. && I bathe myself in the restroom. \\
kili li \textbf{lon} poki. && The fruit is in the basket. \\
\end{supertabular} 
%
%%%%%%%%%%%%%%%%%%%%%%%%%%%%%%%%%%%%%%%%%%%%%%%%%%%%%%%%%%%%%%%%%%%%%%%%%%
\index{\textit{lon}!\textit{wile}}
\index{\textit{wile}!\textit{lon}}
\subsubsection*{Using \textit{lon} with \textit{wile}}
%%%%%%%%%%%%%%%%%%%%%%%%%%%%%%%%%%%%%%%%%%%%%%%%%%%%%%%%%%%%%%%%%%%%%%%%%%
%
Because \textit{lon} can be used as either a preposition or a intrantitive verb, the meaning of the sentence can be a bit confusing when used with \textit{wile}. 

\begin{supertabular}{p{5,5cm}|ll}
mi \textbf{wile lon} tomo. && I want to be at home. / I want in a house. \\
\end{supertabular} 

\index{trick!\textit{e ni}}
\index{\textit{e}!\textit{ni}}
\index{\textit{ni}!\textit{e}}
The sentence has two possible translations. 
The first translation states that the speaker wishes he were at home. 
The second translation states that the speaker wants to do something in a house. 
\textbf{It's best to split this sentence up by a colon}.

\begin{supertabular}{p{5,5cm}|ll}
mi \textbf{wile} e ni\textbf{:} mi \textbf{lon} tomo. && I want this: I'm at home. \\
\end{supertabular} 

\textbf{Toki Pona often uses this \textit{e ni:} trick. 
Before and after the colon has to be complete sentences. }
Toki Pona has no nested subordinate clauses.

\begin{supertabular}{p{5,5cm}|ll}
sina toki \textbf{e ni} tawa mi\textbf{:} sina moku. && You told me that you are eating. \\
\end{supertabular} 

%%%%%%%%%%%%%%%%%%%%%%%%%%%%%%%%%%%%%%%%%%%%%%%%%%%%%%%%%%%%%%%%%%%%%%%%%%
% \newpage
\index{preposition!\textit{kepeken}}
\index{verb!\textit{kepeken}}
\index{\textit{kepeken}!preposition}
\index{\textit{kepeken}!verb}
\subsection*{\textit{kepeken}}
%%%%%%%%%%%%%%%%%%%%%%%%%%%%%%%%%%%%%%%%%%%%%%%%%%%%%%%%%%%%%%%%%%%%%%%%%%
%
%%%%%%%%%%%%%%%%%%%%%%%%%%%%%%%%%%%%%%%%%%%%%%%%%%%%%%%%%%%%%%%%%%%%%%%%%%
\subsubsection*{Using \textit{kepeken} as an intransitive verb}
%%%%%%%%%%%%%%%%%%%%%%%%%%%%%%%%%%%%%%%%%%%%%%%%%%%%%%%%%%%%%%%%%%%%%%%%%%

Don't use an \textbf{e} after \textit{kepeken}.
 
\begin{supertabular}{p{5,5cm}|ll}
mi \textbf{kepeken} ilo. && I'm using tools. \\
sina wile \textbf{kepeken} ilo. && You have to use tools. \\
mi \textbf{kepeken} poki ni. && I'm using that cup. \\
\end{supertabular} 
%
%%%%%%%%%%%%%%%%%%%%%%%%%%%%%%%%%%%%%%%%%%%%%%%%%%%%%%%%%%%%%%%%%%%%%%%%%%
\subsubsection*{Using \textit{kepeken} as a preposition}
%%%%%%%%%%%%%%%%%%%%%%%%%%%%%%%%%%%%%%%%%%%%%%%%%%%%%%%%%%%%%%%%%%%%%%%%%%
%
\begin{supertabular}{p{5,5cm}|ll}
mi moku, \textbf{kepeken} ilo moku. && I eat using a fork/spoon/ \\ && any type of eating utensil. \\
mi lukin, \textbf{kepeken} ilo suno. && I look using a flashlight.  \\
\end{supertabular} 

%%%%%%%%%%%%%%%%%%%%%%%%%%%%%%%%%%%%%%%%%%%%%%%%%%%%%%%%%%%%%%%%%%%%%%%%%%
\index{preposition!\textit{tawa}}
\index{verb!\textit{tawa}}
\index{\textit{tawa}!preposition}
\index{\textit{tawa}!verb}
\subsection*{\textit{tawa}}
%%%%%%%%%%%%%%%%%%%%%%%%%%%%%%%%%%%%%%%%%%%%%%%%%%%%%%%%%%%%%%%%%%%%%%%%%%
%
%%%%%%%%%%%%%%%%%%%%%%%%%%%%%%%%%%%%%%%%%%%%%%%%%%%%%%%%%%%%%%%%%%%%%%%%%%
\subsubsection*{Using \textit{tawa} as a intransitive verb} \\
%%%%%%%%%%%%%%%%%%%%%%%%%%%%%%%%%%%%%%%%%%%%%%%%%%%%%%%%%%%%%%%%%%%%%%%%%%
%
Like \textit{lon}, \textbf{\textit{tawa} as a intransitive verb doesn't have an \textit{e} after it}. \\

\begin{supertabular}{p{5,5cm}|ll}
mi \textbf{tawa} tomo mi. && I'm going to my house. \\
ona mute li \textbf{tawa} utala && They're going to the war. \\
sina wile \textbf{tawa} telo suli. && You want to go to the ocean. \\
ona li \textbf{tawa} sewi kiwen. && She's going up the rock. \\
\end{supertabular} 

%%%%%%%%%%%%%%%%%%%%%%%%%%%%%%%%%%%%%%%%%%%%%%%%%%%%%%%%%%%%%%%%%%%%%%%%%%
\index{\textit{tawa}!verb}
\index{verb!\textit{tawa}}
\subsubsection*{\textit{tawa} as a transitive verb}
%%%%%%%%%%%%%%%%%%%%%%%%%%%%%%%%%%%%%%%%%%%%%%%%%%%%%%%%%%%%%%%%%%%%%%%%%%
%
Do you remember how I said \textit{tawa} doesn't have an \textit{e} after it because nothing is being done to an object? 
Well, that's true, \textbf{but \textit{tawa} can have objects}, like this.

\begin{supertabular}{p{5,5cm}|ll}
mi \textbf{tawa e} kiwen. && I'm moving the rock. \\
ona li \textbf{tawa e} len mi. && She moved my clothes. \\
\end{supertabular} 

\textit{tawa} can be used as an action verb in these situations because there is an object. 
Something has done an action on something else; that was not the case with \textit{tawa} in other example sentences. 

%
%%%%%%%%%%%%%%%%%%%%%%%%%%%%%%%%%%%%%%%%%%%%%%%%%%%%%%%%%%%%%%%%%%%%%%%%%%
\index{\textit{tawa}!to}
\index{to!\textit{tawa}}
\subsubsection*{Using \textit{tawa} to mean "to"}
%%%%%%%%%%%%%%%%%%%%%%%%%%%%%%%%%%%%%%%%%%%%%%%%%%%%%%%%%%%%%%%%%%%%%%%%%%
%
As I said a moment ago, \textit{tawa} is often used as a \textbf{preposition}. 

\begin{supertabular}{p{5,5cm}|ll}
mi toki, \textbf{tawa} sina. && I talk to you. \\
ona li lawa e jan, \textbf{tawa} ma pona. && He led people to the good land. \\
ona li kama, \textbf{tawa} ma mi. && He's coming to my country. \\
\end{supertabular} 

\index{\textit{li}!\textit{li pona tawa mi}}
\index{\textit{li}!\textit{li ike tawa mi}}
\index{like!I like}
\index{I like}
In Toki Pona, \textbf{to say that you (don't) like something, we have pattern}, and the pattern uses \textit{tawa} as a preposition.

\begin{supertabular}{p{5,5cm}|ll}
ni li ' \textbf{pona, tawa mi}. && That is good to me. / I like that. \\
ni li ' \textbf{ike, tawa} mi && That is bad to me. / I don't like that. \\
\end{supertabular} 

\begin{supertabular}{p{5,5cm}|ll}
kili li ' pona, \textbf{tawa} mi. && I like fruit. \\
toki li ' pona, \textbf{tawa} mi. && I like talking. / I like languages. \\
utala li ' ike, \textbf{tawa} mi. && I don't like wars. \\
telo suli li ' ike, \textbf{tawa} mi. && I don't like the ocean. \\
\end{supertabular} 

\index{clauses}
Keep in mind that \textit{e} can only come after the verb of the sentence. 
\textbf{Toki Pona does not use clauses.} 
So for example, if you wanted to say "I like watching the countryside," it's best to split this into two sentences.

\begin{supertabular}{p{5,5cm}|ll}
mi lukin e ma. ni li ' pona, \textbf{tawa} mi. && I'm watching the countryside. This is good to me.\\
\end{supertabular} 

Of course, you could choose to say this same sentence using other techniques. 

\begin{supertabular}{p{5,5cm}|ll}
ma li pona lukin. && The countryside is good to look at. \\
\end{supertabular} 

%%%%%%%%%%%%%%%%%%%%%%%%%%%%%%%%%%%%%%%%%%%%%%%%%%%%%%%%%%%%%%%%%%%%%%%%%%
\index{\textit{tawa}!for}
\subsubsection*{Using \textit{tawa} to mean "for"}
%%%%%%%%%%%%%%%%%%%%%%%%%%%%%%%%%%%%%%%%%%%%%%%%%%%%%%%%%%%%%%%%%%%%%%%%%%
%
Okay, so \textit{tawa} essentially means "to go" to or simply "to", right? 
Not exactly. 
It can also mean "for", as in this sentence.
 
\begin{supertabular}{p{5,5cm}|ll}
mi pona e tomo, \textbf{tawa} jan pakala. && I fixed the house for the disabled man. \\
\end{supertabular} 

Unfortunately, the trick of letting \textit{tawa} mean both "to" and "for" isn't without its drawbacks. 
Keep reading to see why. 

%%%%%%%%%%%%%%%%%%%%%%%%%%%%%%%%%%%%%%%%%%%%%%%%%%%%%%%%%%%%%%%%%%%%%%%%%%
\index{\textit{tawa}!adjective}
\index{adjective!\textit{tawa}}
\index{car}
\index{boat}
\index{ship}
\index{airplane}
\index{helicopter}
\subsubsection*{Using \textit{tawa} as an adjective}
%%%%%%%%%%%%%%%%%%%%%%%%%%%%%%%%%%%%%%%%%%%%%%%%%%%%%%%%%%%%%%%%%%%%%%%%%%
%
\textit{tawa} is used as an adjective to make the phrase we use for "car", "boat" or "airplane".

\begin{supertabular}{p{5,5cm}|ll}
tomo \textbf{tawa} && car (moving construction) \\
tomo \textbf{tawa} telo && boat, ship \\
tomo \textbf{tawa} kon && airplane, helicopter \\
\end{supertabular} 
%
%%%%%%%%%%%%%%%%%%%%%%%%%%%%%%%%%%%%%%%%%%%%%%%%%%%%%%%%%%%%%%%%%%%%%%%%%%
\subsubsection*{Ambiguity of \textit{tawa}}
%%%%%%%%%%%%%%%%%%%%%%%%%%%%%%%%%%%%%%%%%%%%%%%%%%%%%%%%%%%%%%%%%%%%%%%%%%
%
I want you to think about the following sentence for a few moments before continuing on. 
See if you can think of different meanings that it might have. 

\begin{supertabular}{p{5,5cm}|ll}
mi pana e tomo \textbf{tawa} sina. && ? \\   % no-dictionary
\end{supertabular} 

If \textit{tawa} is used as an adjective, then this sentence says "I gave your car." 
If it is used as a preposition, though, it could mean, "I gave the house to you." 
% So, how do you tell the difference? 
% You don't! (Insert evil, mocking laugh here.) 
You can put a comma before \textit{tawa}. 
Better is to write it in a different way.

\begin{supertabular}{p{5,5cm}|ll}
mi jo e tomo tawa sina. mi pana e ni tawa sina. && I have your car. I give it to you. \\
ni li tomo. mi pana e ni tawa sina. && This is a house. I give it to you. \\
\end{supertabular} 

%
%%%%%%%%%%%%%%%%%%%%%%%%%%%%%%%%%%%%%%%%%%%%%%%%%%%%%%%%%%%%%%%%%%%%%%%%%%
% \newpage
\index{\textit{kama}}
\index{\textit{kama}!\textit{tawa}}
\index{\textit{tawa}!\textit{kama}}
\subsection*{\textit{kama}}
%%%%%%%%%%%%%%%%%%%%%%%%%%%%%%%%%%%%%%%%%%%%%%%%%%%%%%%%%%%%%%%%%%%%%%%%%%
\subsubsection*{\textit{kama} as a intransitive verb}
%%%%%%%%%%%%%%%%%%%%%%%%%%%%%%%%%%%%%%%%%%%%%%%%%%%%%%%%%%%%%%%%%%%%%%%%%%
%
We've already briefly touched on this word, but it is quite common, and so we need to look at it a little more closely. 
First, \textbf{it's used with \textit{tawa}}, like this.

\begin{supertabular}{p{5,5cm}|ll}
ona li \textbf{kama tawa} tomo mi. && He came to my house. \\
\end{supertabular} 
%
%%%%%%%%%%%%%%%%%%%%%%%%%%%%%%%%%%%%%%%%%%%%%%%%%%%%%%%%%%%%%%%%%%%%%%%%%%
\subsubsection*{\textit{kama} as a transitive verb}
%%%%%%%%%%%%%%%%%%%%%%%%%%%%%%%%%%%%%%%%%%%%%%%%%%%%%%%%%%%%%%%%%%%%%%%%%%
%
\index{cause}
\index{get}
\index{\textit{kama}!verb}
\index{verb!\textit{kama}}
It can also be used as an action verb. 
It means "to cause" or "to bring about".

\begin{supertabular}{p{5,5cm}|ll}
mi \textbf{kama e} pakala. && I caused an accident. \\
sina \textbf{kama e} ni: mi wile moku. && You caused this: I want to eat. \\ && You made me hungry. \\
\end{supertabular} 
%
%%%%%%%%%%%%%%%%%%%%%%%%%%%%%%%%%%%%%%%%%%%%%%%%%%%%%%%%%%%%%%%%%%%%%%%%%%
\subsubsection*{\textit{kama} as Auxiliary Verb}
%%%%%%%%%%%%%%%%%%%%%%%%%%%%%%%%%%%%%%%%%%%%%%%%%%%%%%%%%%%%%%%%%%%%%%%%%%
%
It can also be used as auxiliary verb. 
One of the most common infinitives that it is used with is \textit{jo}, so we'll just cover it for now. 

\begin{supertabular}{p{5,5cm}|ll}
kama jo && get \\
mi \textbf{kama jo} e telo. && I got the water. \\
\end{supertabular} 
%
%%%%%%%%%%%%%%%%%%%%%%%%%%%%%%%%%%%%%%%%%%%%%%%%%%%%%%%%%%%%%%%%%%%%%%%%%%
\subsection*{Practice 6 (Answers: Page~\pageref{'prepositions01'})}
%%%%%%%%%%%%%%%%%%%%%%%%%%%%%%%%%%%%%%%%%%%%%%%%%%%%%%%%%%%%%%%%%%%%%%%%%%
%
Try translating these sentences. 

\begin{supertabular}{p{5,5cm}|ll}
I fixed the flashlight using a small tool.  &&  \\ % no-dictionary
I like Toki Pona.   &&   \\ % no-dictionary
We gave them food. * &&    \\ % no-dictionary
This is for my friend.   &&   \\ % no-dictionary
The tools are in the container.   &&   \\ % no-dictionary
That bottle is in the dirt.  &&    \\ % no-dictionary
I want to go to his house using my car.  &&   \\  % no-dictionary
They are arguing.   &&   \\ % no-dictionary
&& \\ % no-dictionary
sina wile kama tawa tomo toki.   &&   \\ % no-dictionary
jan li toki, kepeken toki pona lon tomo toki.  &&    \\ % no-dictionary
mi tawa tomo toki. ona li ' pona tawa mi.  &&   \\  % no-dictionary
sina kama jo e jan pona, lon ni. **  &&   \\ % no-dictionary
\end{supertabular} 

\index{here}
\index{there}
* If you got that one wrong, think of the sentence like this: 
"We gave food to them." It means the same thing. \\
** \textit{lon ni} means either "here" or "there". 
Can you figure out what it literally means? 
%
%%%%%%%%%%%%%%%%%%%%%%%%%%%%%%%%%%%%%%%%%%%%%%%%%%%%%%%%%%%%%%%%%%%%%%%%%%
% eof
