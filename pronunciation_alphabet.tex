%%%%%%%%%%%%%%%%%%%%%%%%%%%%%%%%%%%%%%%%%%%%%%%%%%%%%%%%%%%%%%%%%%%%%%%%%%
\label{'pronunciation_alphabet'}
\section{Pronunciation and the Alphabet}
%%%%%%%%%%%%%%%%%%%%%%%%%%%%%%%%%%%%%%%%%%%%%%%%%%%%%%%%%%%%%%%%%%%%%%%%%%
%
% There are \textbf{fourteen letters} in the Toki Pona alphabet.
%%%%%%%%%%%%%%%%%%%%%%%%%%%%%%%%%%%%%%%%%%%%%%%%%%%%%%%%%%%%%%%%%%%%%%%%%%
\index{Alphabet}
\index{Pronunciation}
\index{Consonants}
\subsection*{Consonants}
%%%%%%%%%%%%%%%%%%%%%%%%%%%%%%%%%%%%%%%%%%%%%%%%%%%%%%%%%%%%%%%%%%%%%%%%%%
%
Except for \textit{j}, all the consonants are pronounced like in English. 
\textit{j} is always pronounced just like the letter \textit{y}. 

\begin{supertabular}{p{2,5cm}|ll}
\textbf{letter}   &&    \textbf{pronounced as in} \\ % no-dictionary
k && \textbf{k}ill \\ % no-dictionary
l && \textbf{l}et \\ % no-dictionary
m && \textbf{m}et \\ % no-dictionary
n && \textbf{n}et \\ % no-dictionary
p && \textbf{p}it \\ % no-dictionary
s && \textbf{s}ink \\ % no-dictionary
t && \textbf{t}oo \\ % no-dictionary
w && \textbf{w}et \\ % no-dictionary
j && \textbf{y}et \\ % no-dictionary
\end{supertabular} 

%%%%%%%%%%%%%%%%%%%%%%%%%%%%%%%%%%%%%%%%%%%%%%%%%%%%%%%%%%%%%%%%%%%%%%%%%%
\index{Vowels}
\subsection*{Vowels}
%%%%%%%%%%%%%%%%%%%%%%%%%%%%%%%%%%%%%%%%%%%%%%%%%%%%%%%%%%%%%%%%%%%%%%%%%%
%
Toki Pona's vowels are quite unlike English's. Whereas vowels in English are quite arbitrary and can be pronounced tons of different ways depending on the word, Toki Pona's vowels are all regular and never change pronunciation. 
If you're familiar with Italian, Spanish, Esperanto, or certain other languages, then your work is already cut out for you. The vowels are the same in Toki Pona as they are in these languages. 

\begin{supertabular}{p{2,5cm}|ll}
\textbf{letter}   &&    \textbf{pronounced as in} \\ % no-dictionary
a   &&    f\textbf{a}ther \\ % no-dictionary
e   &&    m\textbf{e}t \\ % no-dictionary
i   &&    p\textbf{ee}l \\ % no-dictionary
o   &&    m\textbf{o}re \\ % no-dictionary
u   &&    f\textbf{oo}d \\ % no-dictionary
\end{supertabular} \\
%
%%%%%%%%%%%%%%%%%%%%%%%%%%%%%%%%%%%%%%%%%%%%%%%%%%%%%%%%%%%%%%%%%%%%%%%%%%
\subsection*{The More Advanced Stuff}
%%%%%%%%%%%%%%%%%%%%%%%%%%%%%%%%%%%%%%%%%%%%%%%%%%%%%%%%%%%%%%%%%%%%%%%%%%
%
\textbf{All official Toki Pona words} 
\textbf{are never capitalized}. They are lowercase even at the beginning of the sentence! 
The only time that capital letters are used is when you are using unofficial words, like the names of people or places or religions. 
%
%
%%%%%%%%%%%%%%%%%%%%%%%%%%%%%%%%%%%%%%%%%%%%%%%%%%%%%%%%%%%%%%%%%%%%%%%%%%
\subsection*{Special Characters}
%%%%%%%%%%%%%%%%%%%%%%%%%%%%%%%%%%%%%%%%%%%%%%%%%%%%%%%%%%%%%%%%%%%%%%%%%%
%
Please always pay attention to correct punctuation marks. Wrong or missing
Punctuation marks impair the intelligibility.

\begin{supertabular}{p{2,5cm}|ll}
\textbf{.} && \textit{separator}: A declarative sentence ends with a full stop. \\ % no-dictionary
\textbf{!} && \textit{separator}: An imperative or an interjection sentence ends with an exclamation mark. \\ % no-dictionary
\textbf{?} && \textit{separator}: An questions always ends in a question mark. \\ % no-dictionary
\textbf{:} && \textit{separator}: A colon is between an hint sentences and a sentences. \\  % no-dictionary
\textbf{,} && \textit{separator}: A comma is used after an 'o' to addressing people. \\ % no-dictionary
\end{supertabular} \\
%
%%%%%%%%%%%%%%%%%%%%%%%%%%%%%%%%%%%%%%%%%%%%%%%%%%%%%%%%%%%%%%%%%%%%%%%%%%
% eof
