%!TEX root = ../toki-pona.tex

\section{Direct Objects}
\subsection*{Vocabulary}
\index{e@\texttp{e}}
\index{ijo@\texttp{ijo}}
\index{ilo@\texttp{ilo}}
\index{jo@\texttp{jo}}
\index{kili@\texttp{kili}}
\index{lukin@\texttp{lukin}}
\index{ma@\texttp{ma}}
\index{pakala@\texttp{pakala}}
\index{unpa@\texttp{unpa}}
\index{wile@\texttp{wile}}

\begin{vocabularytable}
    \texttp{\dots{} e \dots{}}  & \wordtype{separator}: An \texttp{e} introduces a direct object. Don't use \texttp{e} before or after the other separators. \\
    \wordrule %%%%%%%%%%%%%%%%%%%%%%%%%%%%%%%%%%%%%%
    \texttp{ijo}                & \wordtype{noun}: thing, something, stuff, anything, object                                                                 \\
    \texttp{\dots{} ijo}        & \wordtype{adjective}, \wordtype{adverb}: of something                                                                      \\
    \texttp{ijo (e \dots{})}    & \wordtype{verb transitive}: to objectify                                                                                   \\
    \wordrule %%%%%%%%%%%%%%%%%%%%%%%%%%%%%%%%%%%%%%
    \texttp{ilo}                & \wordtype{noun}: tool, device, machine, thing used for a specific purpose                                                  \\
    \texttp{\dots{} ilo}        & \wordtype{adjective}: useful                                                                                               \\
                                & \wordtype{adverb}: usefully                                                                                                \\
    \wordrule %%%%%%%%%%%%%%%%%%%%%%%%%%%%%%%%%%%%%%
    \texttp{jo}                 & \wordtype{noun}: having, possessions, content                                                                              \\
    \texttp{\dots{} jo}         & \wordtype{adjective}: private, personal                                                                                    \\
    \texttp{jo (e \dots{})}     & \wordtype{verb transitive}: to have, to contain                                                                            \\
    \wordrule %%%%%%%%%%%%%%%%%%%%%%%%%%%%%%%%%%%%%%
    \texttp{kili}               & \wordtype{noun}: fruit, pulpy vegetable, mushroom                                                                          \\
    \texttp{\dots{} kili}       & \wordtype{adjective}, \wordtype{adverb}: fruity                                                                            \\
    \wordrule %%%%%%%%%%%%%%%%%%%%%%%%%%%%%%%%%%%%%%
    \texttp{lukin}              & \wordtype{noun}: view, look, glance, sight, gaze, glimpse, seeing, vision                                                  \\
    \texttp{\dots{} lukin}      & \wordtype{adjective}: visual                                                                                               \\
                                & \wordtype{adverb}: visually                                                                                                \\
                                & \wordtype{verb intransitive}: to look, to watch out, to pay attention                                                      \\
    \texttp{lukin (e \dots{})}  & \wordtype{verb transitive}: to see, to look at, to watch, to read                                                          \\
    \texttp{lukin \dots{}}      & \wordtype{auxiliary verb}: to seek to, try to, look for                                                                    \\
    \wordrule %%%%%%%%%%%%%%%%%%%%%%%%%%%%%%%%%%%%%%
    \texttp{ma}                 & \wordtype{noun}: land, earth, country, (outdoor) area                                                                      \\
    \texttp{\dots{} ma}         & \wordtype{adjective}: countrified, outdoor, alfresco, open-air                                                             \\
    \wordrule %%%%%%%%%%%%%%%%%%%%%%%%%%%%%%%%%%%%%%
    \texttp{pakala}             & \wordtype{noun}: blunder, accident, mistake, destruction, damage, breaking                                                 \\
    \texttp{\dots{} pakala}     & \wordtype{adjective}, \wordtype{adverb}: destroyed, ruined, demolished, shattered, wrecked                                 \\
                                & \wordtype{verb intransitive}: to screw up, to fall apart, to break                                                         \\
    \texttp{pakala (e \dots{})} & \wordtype{verb transitive}: to screw up, to ruin, to break, to hurt, to injure, to damage                                  \\
    \wordrule %%%%%%%%%%%%%%%%%%%%%%%%%%%%%%%%%%%%%%
    \texttp{unpa}               & \wordtype{noun}: sex, sexuality                                                                                            \\
    \texttp{\dots{} unpa}       & \wordtype{adjective}, \wordtype{adverb}: erotic, sexual                                                                    \\
                                & \wordtype{verb intransitive}: to have sex                                                                                  \\
    \texttp{unpa (e \dots{})}   & \wordtype{verb transitive}: to have sex with, to sleep with, to fuck                                                       \\
    \wordrule %%%%%%%%%%%%%%%%%%%%%%%%%%%%%%%%%%%%%%
    \texttp{wile}               & \wordtype{noun}: desire, need, will                                                                                        \\
    \texttp{wile (e \dots{})}   & \wordtype{verb transitive}: to want, need, wish, have to, must, will, should                                               \\
    \texttp{wile \dots{}}       & \wordtype{auxiliary verb}: to want, need, wish, have to, must, will, should                                                \\
\end{vocabularytable}

\subsection*{Transitive Verbs, the Separator \texttp{e}, and Direct Objects}
\index{object!direct}
\index{verb!transitive}
\index{transitive verb}
\index{e@\texttp{e}}
We saw how phrases such as \texttp{mi moku} could have two potential meanings.
``I'm eating'' or ``I am food''.
There is one way to specify that you want to say.

\begin{translationtable}
    \texttp{mi moku e kili.} & I eat fruit. \\
\end{translationtable}
%
Also we discussed how \texttp{sina pona}, like \texttp{mi moku}, has two possible meanings.
``You are good'' or ``You're fixing''.
Normally, it would mean ``You are good'' simply because no one really says ``I'm fixing'' without actually telling what it is that they are trying to fix.

\begin{translationtable}
    \texttp{ona li pona e ilo.} & She's fixing the machine. \\
    \texttp{mi pona e ijo.}     & I'm fixing something.     \\
\end{translationtable}
%
Only a (composite) verb can stand in front of \texttp{e}.
More specifically, it is a slot for a transitive verb.
Transitive verbs are verbs after which a direct object (accusative object) can stand.
A transitive verb does something to the direct object.

The separator \texttp{e} preface the direct object.
An object is an optional record supplement.
A direct object is most strongly influenced by the action (i.e\@. the predicate).
Your can ask for direct object (accusative object) by ``Who'' or ``What'' (``What does she repair?'').
The direct object is part of the predicate phrase.

In the direct object is the first slot after the separator \texttp{e} always a noun or pronoun slot.
In the above examples the noun slots were filled with \texttp{kili} and \texttp{ijo}.

\subsection*{Reflexive Pronouns}
\index{reflexive pronoun}
\index{pronoun!reflexive}
\index{pronoun!personal}
\index{mi@\texttp{mi}!reflexive pronoun}
\index{sina@\texttp{sina}!reflexive pronoun}
\index{ona@\texttp{ona}!reflexive pronoun}
A reflexive pronoun represents the subject in the direct object.
So a slot for a reflective pronoun is located after the separator \texttp{e}.
In the following example, \texttp{ona} is a reflexive pronoun, since it refers to the subject \texttp{jan}.

\begin{translationtable}
    \texttp{jan li telo e ona.} & A person washes himself. \\
\end{translationtable}
%
In this sentence the first \texttp{mi} is a personal pronoun.
The \texttp{mi} after the \texttp{e} is a reflexive pronoun.

\begin{translationtable}
    \texttp{mi telo e mi.} & I wash myself. \\
\end{translationtable}
%
Here a sentence with \texttp{sina} as personal and reflective pronouns

\begin{translationtable}
    \texttp{sina telo e sina.} & You wash yourself. \\
\end{translationtable}
%
Here a sentence with \texttp{ona} as personal and reflective pronouns

\begin{translationtable}
    \texttp{ona li telo e ona.} & She washes herself. \\
\end{translationtable}
%
\subsection*{Compound Sentences}
\index{sentence!compound}
There are two ways to make compound sentences in Toki Pona; one way involves using \texttp{li}, and the other way involves using \texttp{e}.
Since you've now studied both of these words, we'll cover how to use both of them to make compound sentences.

\subsubsection*{Several \texttp{li} Separators for Several Predicate Phrases}
\label{sssec:multiple_li}
\index{predicate phrases!several}
\index{li@\texttp{li}!several}
It is possible to use the separator \texttp{li} several times in a sentence.
Each separator \texttp{li} starts a new predicate phrase.
This allows you to assign several actions or properties to one subject.

\begin{translationtable}
    \texttp{ona li ' pona li unpa.} & He's awesome and has sex. \\
\end{translationtable}
%
In the next example the separator \texttp{li} is still omitted before \texttp{moku} because the subject of the sentence is the personal pronoun \texttp{mi}, we still use it before the second predicate, \texttp{pakala}.
Without the separator \texttp{li} there, the sentence would be chaotic and confusing.
Compound sentences with personal pronoun \texttp{sina} as subject follow this same pattern.

\begin{translationtable}
    \texttp{mi moku li pakala.} & I eat and destroy. \\
\end{translationtable}
%
Predicate phrases are not nested. You can change the order:
\texttp{ona li moku li ' pona.} = \texttp{ona li ' pona li moku.}
Each predicate phrase can of course contain direct objects.

\begin{translationtable}
    \texttp{mi moku e moku li lukin e ma.} & I eat the food and look at the landscape. \\
\end{translationtable}
%
The official Toki Pona book recommends to use only one predicate phrase for the personal pronouns \texttp{mi} or \texttp{sina} as subject.

\subsubsection*{Several \texttp{e} Separators for Several direct Objects}
\label{sssec:multiple_e}
\index{object!several}
\index{e@\texttp{e}!several}
For the other type of compound sentences, one predicate phrase has several direct objects.
In other words, the action of a transitive verb refers to several things.

\begin{translationtable}
    \texttp{mi moku e kili e telo.}     & I eat fruit and [drink] water.      \\
    \texttp{mi wile lukin e ma e suno.} & I want to see the land and the sun. \\
\end{translationtable}
%
\texttp{e} phrases are not nested. You can change the order.
\texttp{mi moku e moku e telo.} = \texttp{mi moku e telo e moku.}

We can combine several \texttp{li} and \texttp{e}.
We have two predicate phrases with two direct objects each.
However, it is better to use several short sentences.

\begin{translationtable}
    \texttp{mi moku e kili e telo li lukin e ma e jan.} & I eat fruits and water and see land and people. \\
\end{translationtable}

\practice{
    How to ask for the direct object?                                                   & \answer{With ``whom'' or ``what''.}                            \\\wordrule
    What word type has a predicate before the separator \texttp{e}?                     & \answer{It is always a transitive verb.}                       \\\wordrule
    To which phrase in the sentence belongs a direct object?                            & \answer{To the predicate phrase.}                              \\\wordrule
    What kinds of words are possible after the separator \texttp{e}?                    & \answer{A noun or pronoun.}                                    \\\wordrule
    What is a predicate noun?                                                           & \answer{A noun used as a predicate.}                           \\\wordrule
    Where are possible slots for reflexive pronouns?                                    & \answer{After the separator \texttp{e}.}                       \\\wordrule
    Is it possible to describe several properties of a subject with several \texttp{e}? & \answer{No, because \texttp{e} comes after a transitive verb.} \\\wordrule
    How can you create multiple predicate phrases in a sentence?                        & \answer{With several separators \texttp{li}.}                  \\
}

\practice[Which word types are the bold words?]{
    \texttp{\textbf{mi} moku.}           & \showanswer{personal pronoun} \\\wordrule
    \texttp{mi moku \textbf{e} kili.}    & \answer{separator}            \\\wordrule
    \texttp{mi \textbf{pona} e ijo.}     & \answer{transitive verb}      \\\wordrule
    \texttp{sina telo e \textbf{sina}.}  & \answer{reflexive pronoun}    \\\wordrule
    \texttp{ona li pona e \textbf{ilo}.} & \answer{noun}                 \\
}

\practice[Try to translate these sentences.]{
    \answer{\texttp{mi jo e ilo.}}              & I have a tool.                           \\\wordrule
    \answer{\texttp{ona li moku e kili.}}       & She's eating fruit.                      \\\wordrule
    \answer{\texttp{ijo li lukin e mi.}}        & Something is watching me.                \\\wordrule
    \answer{\texttp{kili li ' moku li ' pona.}} & Pineapple is a food and is good.         \\\wordrule
    \answer{\texttp{ona li telo e ona.}}        & He washes himself.                       \\\wordrule
    \texttp{mi ' jan li ' suli.}                & \answer{I am somebody and am important.} \\
}
