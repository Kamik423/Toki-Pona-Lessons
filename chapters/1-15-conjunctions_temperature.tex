%%%%%%%%%%%%%%%%%%%%%%%%%%%%%%%%%%%%%%%%%%%%%%%%%%%%%%%%%%%%%%%%%%%%%%%%%%
\section{Conjunctions and Temperature}
%%%%%%%%%%%%%%%%%%%%%%%%%%%%%%%%%%%%%%%%%%%%%%%%%%%%%%%%%%%%%%%%%%%%%%%%%%
%
%%%%%%%%%%%%%%%%%%%%%%%%%%%%%%%%%%%%%%%%%%%%%%%%%%%%%%%%%%%%%%%%%%%%%%%%%%
\subsection*{Vocabulary}
%%%%%%%%%%%%%%%%%%%%%%%%%%%%%%%%%%%%%%%%%%%%%%%%%%%%%%%%%%%%%%%%%%%%%%%%%%
%
\begin{supertabular}{p{2,5cm}|ll}
    %
    \index{ante}
    \textbf{\dots ante}       &  & \textit{adjective}: different, dissimilar, changed, other, unequal, differential            \\ % no-dictionary
    \textbf{ante}             &  & \textit{noun}: difference, distinction, differential, variation, variance, disagreement     \\ % no-dictionary
    \textbf{ante (e \dots)}   &  & \textit{verb transitive}: to change, to alter, to modify                                    \\ % no-dictionary
                              &  &                                                                                             \\ % no-dictionary
    %
    \index{anu}
    \textbf{\dots anu \dots}  &  & \textit{conjunction}: or (used for decision questions)                                      \\ % no-dictionary
                              &  &                                                                                             \\ % no-dictionary
    %
    \index{en}
    \textbf{\dots en \dots}   &  & \textit{conjunction}: and (used to coordinate head nouns)                                   \\ % no-dictionary
                              &  &                                                                                             \\ % no-dictionary
    %
    \index{kin}
    \textbf{\dots kin}        &  & \textit{adjective}: indeed, still, too                                                      \\ && kin can be the very last word in an adjective group. \\ % no-dictionary
    \textbf{\dots kin}        &  & \textit{adverb}: actually, indeed, in fact, really, objectively,                            \\ && kin can be the very last word in an adverb group. \\% no-dictionary
    \textbf{kin}              &  & \textit{noun}: reality, fact                                                                \\  % no-dictionary
    \textbf{kin!}             &  & \textit{interjection}: really!                                                              \\ % no-dictionary
                              &  &                                                                                             \\ % no-dictionary
    %
    \index{lete}
    \textbf{\dots lete}       &  & \textit{adjective}: cold, cool, uncooked, raw, perishing                                    \\ % no-dictionary
    \textbf{\dots lete}       &  & \textit{adverb}: bleakly                                                                    \\ % no-dictionary
    \textbf{lete}             &  & \textit{noun}: cold, chill, bleakness                                                       \\ % no-dictionary
    \textbf{lete (e \dots)}   &  & \textit{verb transitive}: to cool down, to chill                                            \\ % no-dictionary
                              &  &                                                                                             \\ % no-dictionary
    %
    \index{lipu}
    \textbf{\dots lipu}       &  & \textit{adjective}: book-, paper-, card-, ticket-, sheet-, page,-                           \\ % no-dictionary
    \textbf{lipu}             &  & \textit{noun}: paper, book, card, ticket, sheet, (web-)page, list ; flat and bendable thing \\ % no-dictionary
                              &  &                                                                                             \\ % no-dictionary
    %
    \index{mani}
    \textbf{\dots mani}       &  & \textit{adjective}: financial, financially, monetary, pecuniary                             \\ % no-dictionary
    \textbf{\dots mani}       &  & \textit{adverb}: financially                                                                \\ % no-dictionary
    \textbf{mani}             &  & \textit{noun}: money, material wealth, currency, dollar, capital                            \\ % no-dictionary
    %
    \index{pilin}
    \textbf{\dots pilin}      &  & \textit{adjective}: sensitive, feeling, empathic                                            \\ % no-dictionary
    \textbf{\dots pilin}      &  & \textit{adverb}: perceptively                                                               \\ % no-dictionary
    \textbf{pilin}            &  & \textit{noun}: feelings, emotion, feel, think, sense, touch,                                \\ % no-dictionary
    \textbf{pilin}            &  & \textit{verb intransitive}: to feel, to sense                                               \\ % no-dictionary
    \textbf{pilin (e \dots)}  &  & \textit{verb transitive}: to feel, to think, to touch, to fumble, to fiddle                 \\ % no-dictionary
                              &  &                                                                                             \\ % no-dictionary
    %
    \index{taso}
    \textbf{\dots taso}       &  & \textit{adjective}: only, sole                                                              \\ % no-dictionary
    \textbf{\dots taso}       &  & \textit{adverb}: only, just, merely, simply, solely, singly                                 \\ % no-dictionary
    \textbf{\dots taso \dots} &  & \textit{conjunction}: but, however                                                          \\ % no-dictionary
                              &  &                                                                                             \\ % no-dictionary
    %
\end{supertabular} \\
%
%%%%%%%%%%%%%%%%%%%%%%%%%%%%%%%%%%%%%%%%%%%%%%%%%%%%%%%%%%%%%%%%%%%%%%%%%%
\newpage
%%%%%%%%%%%%%%%%%%%%%%%%%%%%%%%%%%%%%%%%%%%%%%%%%%%%%%%%%%%%%%%%%%%%%%%%%%
\subsection*{Conjunctions}
%
\index{conjunction}
%%%%%%%%%%%%%%%%%%%%%%%%%%%%%%%%%%%%%%%%%%%%%%%%%%%%%%%%%%%%%%%%%%%%%%%%%%
%
Conjunctions connect words and phrases.
Conjunctions have similar tasks to prepositions.
% Unlike prepositions, conjunctions do not cause cases.
In \textit{toki pona} there are conjunctions \textit{anu} (or),  \textit{en} (and) and \textit{taso} (but, however).

%
%%%%%%%%%%%%%%%%%%%%%%%%%%%%%%%%%%%%%%%%%%%%%%%%%%%%%%%%%%%%%%%%%%%%%%%%%%
\subsubsection*{Alternative-questions with the Conjunction \textit{anu}}
%
\index{\textit{anu}}
\index{question!choice}
\index{choice}
%%%%%%%%%%%%%%%%%%%%%%%%%%%%%%%%%%%%%%%%%%%%%%%%%%%%%%%%%%%%%%%%%%%%%%%%%%
%
The conjunction \textit{anu} is used to make alternative-questions.
The alternative-question is the combination of two (or rarely more) choices.
In the following questions there is a choice between two subjects.
Between these subjects there is the conjunction \textit{anu}.

\begin{supertabular}{p{5,5cm}|ll}
    jan Susan anu jan Lisa li moku e suwi? &  & Susan or Lisa ate the cookies? \\
    ona anu jan ante li ' ike?             &  & Is he bad, or is it the        \\ && other person who's bad? \\
\end{supertabular}

In the following question the decision is made between two direct objects.

\begin{supertabular}{p{5,5cm}|ll}
    sina jo e kili anu telo nasa? &  & Do you have the fruit, \\ && or is it the wine that you have? \\
\end{supertabular}

In the following question, the decision is made between two prepositional objects.
The preposition is only used once.

\begin{supertabular}{p{5,5cm}|ll}
    sina toki, tawa mi anu ona? &  & Are you talking to me, \\ && or are you talking to him? \\
\end{supertabular}

%
%%%%%%%%%%%%%%%%%%%%%%%%%%%%%%%%%%%%%%%%%%%%%%%%%%%%%%%%%%%%%%%%%%%%%%%%%%
\subsubsection*{Answer-Questions with the conjunction \textit{anu}}
%
\index{answer-question}
\index{question!answer-}
\index{what!or what}
\index{or what}
%%%%%%%%%%%%%%%%%%%%%%%%%%%%%%%%%%%%%%%%%%%%%%%%%%%%%%%%%%%%%%%%%%%%%%%%%%
%
In answer-questions is the answer already included in the question.
A confirmation or denial is expected as an answer.
In English there is the saying '... or what?' or '... isn't it?'.
In Toki Pona answer questions are formed by adding the conjunction \textit{anu} and the question pronoun \textit{seme} after the statement.

\begin{supertabular}{p{5,5cm}|ll}
    sina kama anu seme?        &  & Are you coming or what?        \\
    sina wile moku anu seme?   &  & Do you want to eat or what?    \\
    sina wile e mani anu seme? &  & Do you want the money or what? \\
\end{supertabular}

%
%%%%%%%%%%%%%%%%%%%%%%%%%%%%%%%%%%%%%%%%%%%%%%%%%%%%%%%%%%%%%%%%%%%%%%%%%%
\subsubsection*{Yes/No questions with predicate nouns or predicate adjectives}
%
\index{yes/no question!predicate noun}
\index{yes/no question!predicate adjective}
\index{predicate noun!yes/no question}
\index{predicate adjective!yes/no question}
%%%%%%%%%%%%%%%%%%%%%%%%%%%%%%%%%%%%%%%%%%%%%%%%%%%%%%%%%%%%%%%%%%%%%%%%%%
%
We had learned that yes/no questions with the adverb \textit{ala} require a verb.
That there is no verb in Toki Pona, the verb slot can remain empty.
The predicate is then formed by a predicate adjective or predicate adjective.
Yes/no questions with the adverb \textit{ala} are not possible.
To form yes/no questions with predicate nouns or predicate adjectives \textit{anu seme} is used.
A answer-question is therefore formulated.

\begin{supertabular}{p{5,5cm}|ll}
    sina ' pona anu seme?   &  & Are you OK (or what)?      \\
    ona li ' mama anu seme? &  & Is she a mother (or what)? \\
\end{supertabular}

%
%%%%%%%%%%%%%%%%%%%%%%%%%%%%%%%%%%%%%%%%%%%%%%%%%%%%%%%%%%%%%%%%%%%%%%%%%%
\subsubsection*{Declarative Sentences with the Conjunction \textit{anu}}
%
\index{declarative sentence!\textit{anu}}
\index{sentence!\textit{anu}}
\index{\textit{anu}!in a declarative sentence}
%%%%%%%%%%%%%%%%%%%%%%%%%%%%%%%%%%%%%%%%%%%%%%%%%%%%%%%%%%%%%%%%%%%%%%%%%%

The conjunction \textit{anu} can be used in declarative sentences also.

\begin{supertabular}{p{5,5cm}|ll}
    mi lukin e mije anu meli. &  & I see a man or a women. \\
\end{supertabular}

%
%%%%%%%%%%%%%%%%%%%%%%%%%%%%%%%%%%%%%%%%%%%%%%%%%%%%%%%%%%%%%%%%%%%%%%%%%%
\subsubsection*{The Conjunction \textit{en} Connects Nouns and Pronouns}
%
\index{\textit{en}}
\index{sentence!compound}
\index{connected!objects}
\index{object!connected}
\index{\textit{li}!multiple}
\index{\textit{e}!multiple}
\index{\textit{pi}!avoid multiple}
%%%%%%%%%%%%%%%%%%%%%%%%%%%%%%%%%%%%%%%%%%%%%%%%%%%%%%%%%%%%%%%%%%%%%%%%%%
%
The conjunction \textit{en} is used to connect two (composite) nouns or pronouns.
In the following examples, one subject is formed in each case.

\begin{supertabular}{p{5,5cm}|ll}
    mi en sina li ' jan pona.               &  & You and I are friends.               \\
    jan lili en jan suli li toki.           &  & The child and the adult are talking. \\
    kalama musi en meli li ' pona, tawa mi. &  & I like music and girls.              \\
\end{supertabular}

The conjunction \textit{en} can be used with the separator \textit{pi} to form complex compound nouns.
With \textit{en} you can avoid several \textit{pi} phrases.
Such complex nouns are unknown in many languages.
In the first sentence \textit{jan lili pi jan Ken en jan Lisa} is one complex noun.

\begin{supertabular}{p{5,5cm}|ll}
    jan lili pi jan Ken en jan Lisa li ' suwi. &  & Ken and Lisa's baby is sweet.               \\
    tomo pi jan Keli en mije ona li suli.      &  & The house of Keli and her boyfriend is big. \\
\end{supertabular}
%

Note that \textit{en} is not used to connect two whole sentences, even though this is common in English.
Instead, use the multiple-\textit{li} technique  (Page~\pageref{'multiple_li'}) or split the sentence into two sentences.

Also note that \textit{en} is not intended to connect two direct objects.
For that, use the multiple-\textit{e} technique (Page~\pageref{'multiple_e'}).

%
%%%%%%%%%%%%%%%%%%%%%%%%%%%%%%%%%%%%%%%%%%%%%%%%%%%%%%%%%%%%%%%%%%%%%%%%%%
\subsubsection*{The Conjunction \textit{taso} }
%
\index{\textit{taso}!conjunction}
%%%%%%%%%%%%%%%%%%%%%%%%%%%%%%%%%%%%%%%%%%%%%%%%%%%%%%%%%%%%%%%%%%%%%%%%%%
%
If you use the conjunction \textit{taso} at the beginning of a sentence you refer to the previous sentence.
Separate these sentences not with a comma, but with a full stop.
Also do not use a comma after the conjunction \textit{taso}.
This mistake is usually made by people who are native English speakers.

\begin{supertabular}{p{5,5cm}|ll}
    mi wile moku. taso mi jo ala e moku.          &  & I want to eat. But I don't have food.                   \\
    mi wile lukin e tomo mi. taso mi lon ma ante. &  & I want to see my house. But I'm in a different country. \\
    mi ' pona. taso meli mi li ' pakala.          &  & I'm okay. But my girlfriend is injured.                 \\
\end{supertabular}

%
%%%%%%%%%%%%%%%%%%%%%%%%%%%%%%%%%%%%%%%%%%%%%%%%%%%%%%%%%%%%%%%%%%%%%%%%%%
\subsubsection*{A conjunction at the beginning of a sentence}

As we have just learned, the confunction \textit{taso} can be at the beginning of a sentence.
So a slot for a conjunction is possible at the beginning of a sentence.
Such a conjunction does not connect main clauses.
Otherwise no period would end the sentence before it.
With such a conjunction, the sentence refers to the previous sentence.

\begin{supertabular}{p{5,5cm}|ll}
    A: mi wile moku.        &  & I want to eat.         \\
    B: en mi wile moku kin. &  & And I want to eat too. \\
\end{supertabular}

%
%
%
%%%%%%%%%%%%%%%%%%%%%%%%%%%%%%%%%%%%%%%%%%%%%%%%%%%%%%%%%%%%%%%%%%%%%%%%%%
\subsection*{Miscellaneous}
%
%
%%%%%%%%%%%%%%%%%%%%%%%%%%%%%%%%%%%%%%%%%%%%%%%%%%%%%%%%%%%%%%%%%%%%%%%%%%
\subsubsection*{The Adjective \textit{taso}}
%
\index{\textit{taso}!adjective}
%%%%%%%%%%%%%%%%%%%%%%%%%%%%%%%%%%%%%%%%%%%%%%%%%%%%%%%%%%%%%%%%%%%%%%%%%%
%
\begin{supertabular}{p{5,5cm}|ll}
    jan Lisa taso li kama. &  & Only Lisa came.                        \\
    mi sona e ni taso.     &  & I know only that. (That's all I know.) \\
\end{supertabular}
%
%%%%%%%%%%%%%%%%%%%%%%%%%%%%%%%%%%%%%%%%%%%%%%%%%%%%%%%%%%%%%%%%%%%%%%%%%%
\subsubsection*{The Adverb \textit{taso}}
%
\index{\textit{taso}!adverb}
%%%%%%%%%%%%%%%%%%%%%%%%%%%%%%%%%%%%%%%%%%%%%%%%%%%%%%%%%%%%%%%%%%%%%%%%%%
%
\begin{supertabular}{p{5,5cm}|ll}
    mi musi taso.            &  & I'm just joking.                      \\
    mi pali taso.            &  & I just work. (All I ever do is work.) \\
    mi lukin taso e meli ni! &  & I only looked at that girl!           \\
\end{supertabular}

%
%%%%%%%%%%%%%%%%%%%%%%%%%%%%%%%%%%%%%%%%%%%%%%%%%%%%%%%%%%%%%%%%%%%%%%%%%%
\subsection*{The Noun \textit{kin}}
%
\index{\textit{kin}!noun}
%%%%%%%%%%%%%%%%%%%%%%%%%%%%%%%%%%%%%%%%%%%%%%%%%%%%%%%%%%%%%%%%%%%%%%%%%%
%

\begin{supertabular}{p{5,5cm}|ll}
    kin ni li kama, tawa suno. &  & This fact comes to light. \\
\end{supertabular}

%
%%%%%%%%%%%%%%%%%%%%%%%%%%%%%%%%%%%%%%%%%%%%%%%%%%%%%%%%%%%%%%%%%%%%%%%%%%
\subsection*{The Adjective \textit{kin}}
%
\index{\textit{kin}!adjective}
%%%%%%%%%%%%%%%%%%%%%%%%%%%%%%%%%%%%%%%%%%%%%%%%%%%%%%%%%%%%%%%%%%%%%%%%%%
%
The adjective \textit{kin} is at the end of an adjective group and emphasizes it.

\begin{supertabular}{p{5,5cm}|ll}
    jan pona mi kin li lon ni. &  & My good friend is here. \\
\end{supertabular}

%%%%%%%%%%%%%%%%%%%%%%%%%%%%%%%%%%%%%%%%%%%%%%%%%%%%%%%%%%%%%%%%%%%%%%%%%%
\subsection*{The Adverb \textit{kin}}
%
\index{\textit{kin}!adverb}
%%%%%%%%%%%%%%%%%%%%%%%%%%%%%%%%%%%%%%%%%%%%%%%%%%%%%%%%%%%%%%%%%%%%%%%%%%
%

The adverb \textit{kin} is at the end of an adverb group and emphasizes it.

\begin{supertabular}{p{5,5cm}|ll}
    A: mi tawa, tawa ma Elopa.                &  & I went to Europe.                   \\
    mi tawa kin e mi, tawa ma Elopa.          &  & I went to Europe too.               \\
    A: mi mute o tawa.                        &  & Let's go.                           \\
    B: mi ken ala. mi moku kin e moku.        &  & I can't. I'm still eating the food. \\
    A: a! sina lukin ala lukin e ijo nasa ni? &  & Whoa! Do you see that weird thing?  \\
    B: mi lukin kin e ona.                    &  & I see it indeed.                    \\
\end{supertabular}

%
%
%
%%%%%%%%%%%%%%%%%%%%%%%%%%%%%%%%%%%%%%%%%%%%%%%%%%%%%%%%%%%%%%%%%%%%%%%%%%
% \newpage
%
\subsection*{Temperatures}
%
\index{temperature}
\index{\textit{seli}}
\index{\textit{lete}}
\index{\textit{pilin}}
%%%%%%%%%%%%%%%%%%%%%%%%%%%%%%%%%%%%%%%%%%%%%%%%%%%%%%%%%%%%%%%%%%%%%%%%%%
%
As nouns \textit{seli} mean 'heat' and \textit{lete} 'cold'.
The adjectives \textit{lilili} and \textit{mute} relativize these nouns.
We can use these words to express weather temperatures.
\textit{lon} is here an intransitive verb.

\begin{supertabular}{p{5,5cm}|ll}
    seli li lon.      &  & It's hot.       \\
    lete li lon.      &  & It's cold.      \\
    seli mute li lon. &  & It's very hot.  \\
    seli lili li lon. &  & It's warm.      \\
    lete mute li lon. &  & It's very cold. \\
    lete lili li lon. &  & It's cool.      \\
\end{supertabular}

%
%%%%%%%%%%%%%%%%%%%%%%%%%%%%%%%%%%%%%%%%%%%%%%%%%%%%%%%%%%%%%%%%%%%%%%%%%%
\subsection*{The Intransitive Verb \textit{pilin}}
%
\index{\textit{pilin}!verb}
%%%%%%%%%%%%%%%%%%%%%%%%%%%%%%%%%%%%%%%%%%%%%%%%%%%%%%%%%%%%%%%%%%%%%%%%%%

If one wants to describe the temperature of an object, one uses \textit{seli} or \textit{lete} as predicate nouns.

\begin{supertabular}{p{5,5cm}|ll}
    ilo ni li ' lete mute , tawa mi. &  & This axe feels very cold. \\
    ni li ' seli lili, tawa mi.      &  & This feels warm.          \\
\end{supertabular}

When one freezes or sweats, one says this with the intransitive verb \textit{pilin} and the adverbs \textit{seli} and \textit{lete}.

\begin{supertabular}{p{5,5cm}|ll}
    mi pilin lete mute. &  & I'm very cold. \\
\end{supertabular}

The intransitive verb \textit{pilin} can generally describe feelings of a person or an animal.

\begin{supertabular}{p{5,5cm}|ll}
    mi pilin pona.   &  & I feel good. / I feel happy. \\
    mi pilin ike.    &  & I feel bad. / I feel sad.    \\
    sina pilin seme? &  & How do you feel?             \\
\end{supertabular}

%
%%%%%%%%%%%%%%%%%%%%%%%%%%%%%%%%%%%%%%%%%%%%%%%%%%%%%%%%%%%%%%%%%%%%%%%%%%
\subsection*{The Transitive Verb \textit{pilin}}
%
\index{\textit{pilin}!verb}
%%%%%%%%%%%%%%%%%%%%%%%%%%%%%%%%%%%%%%%%%%%%%%%%%%%%%%%%%%%%%%%%%%%%%%%%%%

The transitive \textit{pilin} means 'to think'.

\begin{supertabular}{p{5,5cm}|ll}
    mi pilin e ni: sina ike. &  & I think this: You're bad.       \\
    sina pilin e seme?       &  & What are you thinking?          \\
    mi pilin e ijo.          &  & I'm thinking (about) something. \\
    mi pilin e meli ni.      &  & I'm thinking about that woman.  \\
\end{supertabular}
%
%
%
%%%%%%%%%%%%%%%%%%%%%%%%%%%%%%%%%%%%%%%%%%%%%%%%%%%%%%%%%%%%%%%%%%%%%%%%%%
\newpage
%
\subsection*{Practice (Answers: Page~\pageref{'conjunctions_temperature'})}
%%%%%%%%%%%%%%%%%%%%%%%%%%%%%%%%%%%%%%%%%%%%%%%%%%%%%%%%%%%%%%%%%%%%%%%%%%
%
Please write down your answers and check them afterwards.

\begin{supertabular}{p{5,5cm}|ll}
    What are conjunctions?                                                                              &  & \\ % no-dictionary
    What is an answer-question?                                                                         &  & \\ % no-dictionary
    % What is the difference between conjunctions and prepositions? &&  \\ % no-dictionary
    How is an answer-question formed in \textit{toki pona}?                                             &  & \\ % no-dictionary
    Is there a comma before or after the conjunction \textit{taso}?                                     &  & \\ % no-dictionary
    What are alternative-questions?                                                                     &  & \\ % no-dictionary
    What connects the conjunction \textit{taso}?                                                        &  & \\ % no-dictionary
    What connects the conjunction \textit{en}?                                                          &  & \\ % no-dictionary
    How is an alternative-question formed in \textit{toki pona}?                                        &  & \\ % no-dictionary
    How is a yes/no-question with predicate nouns or predicate adjectives formed in \textit{toki pona}? &  & \\ % no-dictionary
\end{supertabular}

Try to translate these sentences.
You can use the tool \textit{Toki Pona Parser} (\cite{www:rowa:02}) for spelling and grammar check.

\begin{supertabular}{p{5,5cm}|ll}
    Do you want to come or what?            \\ % no-dictionary
    Do you want food, or do you want water? \\ % no-dictionary
    I still want to go to my house.         \\ % no-dictionary
    This paper feels cold.                  \\ % no-dictionary
    I like currency of other nations.       \\  % no-dictionary
    I want to go, but I can't.              \\ % no-dictionary
    I'm alone. *                            \\ % no-dictionary
    Do you like me?    &  &                 \\ % no-dictionary
    This lake is cold. &  &                 \\ % & English - Toki Pona
\end{supertabular}

\begin{supertabular}{p{5,5cm}|ll}
    mi olin kin e sina.                  \\ % no-dictionary
    mi pilin e ni: ona li jo ala e mani. \\ % no-dictionary
    mi wile lukin e ma ante.             \\ % no-dictionary
    mi wile ala e ijo. mi lukin taso.    \\ % no-dictionary
    mi pilin lete.                       \\ % no-dictionary
    sina wile toki, tawa mije anu meli?  \\ % no-dictionary
\end{supertabular}

* Think: 'Only I am present.'
%
%%%%%%%%%%%%%%%%%%%%%%%%%%%%%%%%%%%%%%%%%%%%%%%%%%%%%%%%%%%%%%%%%%%%%%%%%%
% eof
