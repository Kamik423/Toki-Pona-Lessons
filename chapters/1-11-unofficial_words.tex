%!TEX root = ../toki-pona.tex

\section{Unofficial Words}
\subsection*{Vocabulary}
\index{nasin@\texttp{nasin}}
\index{nimi@\texttp{nimi}}
\index{Quotation mark}

\begin{vocabularytable}
    nasin            & \wordtype{noun}: way, manner, custom, road, path, doctrine, system, method                     \\
    \dots{} nasin    & \wordtype{adjective}: systematic, habitual, customary, doctrinal                               \\
    \wordrule %%%%%%%%%%%%%%%%%%%%%%%%%%%%%%%%%%%%%%%%%%%%%%%%%%%%%%%%%%%%%
    nimi             & \wordtype{noun}: word, name                                                                    \\
    nimi (e \dots{}) & \wordtype{verb transitive}: to name                                                            \\
    \wordrule %%%%%%%%%%%%%%%%%%%%%%%%%%%%%%%%%%%%%%%%%%%%%%%%%%%%%%%%%%%%%
    ``\dots{}''      & \wordtype{separator}: Quotation marks are used for words with original spelling or for quotes. \\
\end{vocabularytable}

\subsubsection*{Some Unofficial Words}

\begin{translationtable}
    ma suli Amelika      & The Americas  \\
    ma suli Amelika lete & North America \\
    ma suli Amelika seli & South America \\
    ma suli Antasika     & Antarctica    \\
    ma suli Apika        & Africa        \\
    ma suli Asija        & Asia          \\
    ma suli Elopa        & Europe        \\
    \wordrule %%%%%%%%%%%%%%%%%%%%%%%%%%%%%%%%%%%%%%%%%%%%%%%%%%%%%%%%%%%%%
    ma Epanja            & Spain         \\
    ma Tosi              & Germany       \\
    \wordrule %%%%%%%%%%%%%%%%%%%%%%%%%%%%%%%%%%%%%%%%%%%%%%%%%%%%%%%%%%%%%
    ma tomo Lanten       & London        \\
    ma tomo Sanpansiko   & San Francisco \\
    \wordrule %%%%%%%%%%%%%%%%%%%%%%%%%%%%%%%%%%%%%%%%%%%%%%%%%%%%%%%%%%%%%
    toki Inli            & English       \\
    toki Epelanto        & Esperanto     \\
    \wordrule %%%%%%%%%%%%%%%%%%%%%%%%%%%%%%%%%%%%%%%%%%%%%%%%%%%%%%%%%%%%%
    meli Mawija          & Maria         \\
    jan Santa            & Santa Claus   \\
\end{translationtable}

\subsection*{Names are Adjectives}
\index{proper name}
\index{unofficial word}
\index{name}
\index{words!unofficial}
\index{nation}
\index{country}
\index{language}
\index{religion}
\index{person}
\index{adjective!unofficial word}
\index{predicate adjective!unofficial word}
Names of people, countries, cities, languages, and ideologies do not exist as official words in Toki Pona.
Names are unofficial words and do not appear in the dictionary of Toki Pona.
Unofficial words are adjectives.
You know adjectives describe nouns.
This means that names also describe nouns and cannot be used without a corresponding noun in the record.
This is necessary to recognize what the name stands for.
For example, if it is a country name, the corresponding adjective is used after the noun \texttp{ma}.
As we already know there are adjective slots after a noun or after the separator \texttp{li} as a slot for a predicate adjective.
Names also fit into these slots.
Unofficial words can only be adjectives and not adverbs.
As can be seen again, in Toki Pona adjectives are more complex than adverbs.
In order to recognize names as unofficial words, they always begin with a capital letter.
If the original spelling of the name is used, place it in quotation marks.

Unofficial words can be adapted to the phonetic rules of Toki Pona.
The appendix (page \pageref{'phonet_trans'}) describes how to proceed.
For example, America becomes \texttp{Mewika}, Canada becomes \texttp{Kanata}.
In the appendix you will find a list of important unofficial words (page \pageref{unofficial_words}).

\clearpage

\subsubsection*{Countries}
After the noun \texttp{ma} an unofficial word (adjective) is used as the country name.

\begin{translationtable}
    ma Kanata li ' pona.        & Canada is good.          \\
    ma Italija li ' pona lukin. & Italy is beautiful.      \\
    mi wile tawa, tawa ma Tosi. & I want to go to Germany. \\
\end{translationtable}
%
Since unofficial words are adjectives, they can also be used as predicate adjectives.

\begin{translationtable}
    ma mi li ' Tosi. & My homeland is Germany. \\
\end{translationtable}

\subsubsection*{Continents}
Continents are formed with \texttp{ma}, optionally the adjective \texttp{suli}, and the corresponding unofficial word (adjective).

\begin{translationtable}
    ma suli Apika & Africa \\
\end{translationtable}

\subsubsection*{Cities}
\index{city}
As we have learnt the combination of the noun \texttp{ma} and the adjective \texttp{tomo} means ``city''.
After this combination, an unofficial word (adjective) is used as a city name.

\begin{translationtable}
    ma tomo Lantan li ' suli.   & London is big.   \\
    ma tomo Pelin               & Berlin           \\
    ma tomo Loma                & Rome             \\
    mi kama, tan ma tomo Pelin. & I'm from Berlin. \\
\end{translationtable}
%
Here is an example of an unofficial word as predicate adjective.

\begin{translationtable}
    ma tomo mi li ' Pelin. & My home city is Berlin. \\
\end{translationtable}

\subsubsection*{Languages}
If you want to talk about a language, you simply use the noun \texttp{toki} and then attach the unofficial word (adjective) onto it.

\begin{translationtable}
    toki Inli li ' pona.     & The English language is good. \\
    ma Inli li ' pona.       & England is good.              \\
    toki Kanse               & French language               \\
    toki Epelanto li ' pona. & Esperanto is simple.          \\
\end{translationtable}
%
Here is an example of an unofficial word as predicate adjective.

\begin{translationtable}
    toki mi li ' Tosi. & My mother tongue is German. \\
\end{translationtable}

\subsubsection*{Residents of a Country}
A resident of a country is named by nouns \texttp{jan}, \texttp{meli}, or \texttp{mije} and the unofficial word (adjective).

\begin{translationtable}
    jan Kanata   & Canadian person \\
    jan Mesiko   & Mexican person  \\
    meli Italija & Italian woman   \\
\end{translationtable}

\newpage

\subsubsection*{People}
Now suppose you want to talk about someone using their name.
For example, what if you want to say ``Lisa is cool''?
To say a person's name in Toki Pona, you just say the noun \texttp{jan} and then the person's name.

\begin{translationtable}
    jan Lisa li ' pona. & Lisa is cool. \\
\end{translationtable}
%
Like for the names of countries, we often adapt a person's name to fit into Toki Pona's phonetic rules.
Keep in mind that no one is going to pressure you to adopt a tokiponized name; it's just for fun.

\begin{translationtable}
    jan Pentan li pana e sona, tawa mi. & Brandon teaches to me.    \\
    jan Mewi li toki, tawa mi.          & Mary's talking to me.     \\
    jan Nesan li ' musi.                & Nathan is funny.          \\
    jan Eta li ' jan unpa.              & Heather is a whore.       \\
    pana e sona                         & to teach (give knowledge) \\
\end{translationtable}
%
This is the way to say your name.

\begin{translationtable}
    mi ' jan Pepe.     & I am Pepe.       \\
    nimi mi li ' Pepe. & My name is Pepe. \\
\end{translationtable}
%
If you choose not to tokiponize your name just surround it with quotation marks.

\begin{translationtable}
    mi ' jan ``Robert''. & I'm Robert. \\
\end{translationtable}

\subsubsection*{Ideologies, Religions}
Ideologies and religions are named with the noun \texttp{nasin}, the adjective \texttp{sewi} and the corresponding unofficial word (adjective).

\begin{translationtable}
    nasin sewi Patapali & Pastafari \\
\end{translationtable}

\practice{
    What are proper names in Toki Pona?                                 & \answer{Unofficial words, adjectives}         \\\wordrule
    Where are slots for predicate adjectives located?                   & \answer{After the separator \texttp{li}.}     \\\wordrule
    How are names in Toki Pona highlighted?                             & \answer{The first letter is a capital letter} \\\wordrule
    How is the original spelling of a name marked?                      & \answer{By quotation marks.}                  \\\wordrule
    Which slots can unofficial words fill?                              & \answer{Adjective slots.}                     \\\wordrule
    What kind of word type must unofficial words be used together with? & \answer{With a noun.}                         \\
}

\practice[Try to translate these sentences.]{
    \answer{\texttp{jan Susan li ' nasa.}}                & Susan is crazy.                          \\\wordrule
    \answer{\texttp{mi kama, tan ma suli Elopa.}}         & I come from Europe.                      \\\wordrule
    \answer{\texttp{mi ' jan Ken. / nimi mi li Ken.}}     & My name is Ken.                          \\\wordrule
    \texttp{mi wile tawa, tawa ma suli Oselija.}          & \answer{I want to go to Australia.}      \\\wordrule
    \texttp{mi wile kama sona e toki Inli.}               & \answer{I want to learn English.}        \\
}