%!TEX root = ../toki-pona.tex

\section{Indirect Objects}
\subsection*{Vocabulary}
\index{kepeken@\texttp{kepeken}}
\index{kiwen@\texttp{kiwen}}
\index{kon@\texttp{kon}}
\index{lon@\texttp{lon}}
\index{pana@\texttp{pana}}
\index{poki@\texttp{poki}}
\index{tawa@\texttp{tawa}}

\begin{vocabularytable}
    kepeken                  & \wordtype{noun}: use, usage, tool                                                             \\
    \dots{}, kepeken \dots{} & \wordtype{preposition}: with, using                                                           \\
    kepeken                  & \wordtype{verb intransitive}: to use                                                          \\
    \wordrule %%%%%%%%%%%%%%%%%%%%%%%%%%%%%%%%%%%%%%%%%%%%%%%%%%%%%%%%%%%%%
    kiwen                    & \wordtype{noun}: hard thing, rock, stone, metal, mineral, clay                                \\
    \dots{} kiwen            & \wordtype{adjective}, \wordtype{adverb}: hard, solid, stone-like, made of stone or metal      \\
    kiwen (e \dots{})        & \wordtype{verb transitive}: to solidify, to harden, to petrify, to fossilize                  \\
    \wordrule %%%%%%%%%%%%%%%%%%%%%%%%%%%%%%%%%%%%%%%%%%%%%%%%%%%%%%%%%%%%%
    kon                      & \wordtype{noun}: air, wind, smell, soul                                                       \\
    \dots{} kon              & \wordtype{adjective}, \wordtype{adverb}: air-like, ethereal, gaseous                          \\
    kon                      & \wordtype{verb intransitive}: to breathe                                                      \\
    kon (e \dots{})          & \wordtype{verb transitive}: to blow away something, to puff away something                    \\
    \wordrule %%%%%%%%%%%%%%%%%%%%%%%%%%%%%%%%%%%%%%%%%%%%%%%%%%%%%%%%%%%%%
    lon                      & \wordtype{noun}: existence, being, presence                                                   \\
    \dots{} lon              & \wordtype{adjective}: true, existing, correct, real, genuine                                  \\
    \dots{}, lon \dots{}     & \wordtype{preposition}: be (located) in/at/on                                                 \\
    lon                      & \wordtype{verb intransitive}: to be there, to be present, to be real/true, to exist           \\
    lon (e \dots{})          & \wordtype{verb transitive}: to give birth, to create                                          \\
    \wordrule %%%%%%%%%%%%%%%%%%%%%%%%%%%%%%%%%%%%%%%%%%%%%%%%%%%%%%%%%%%%%
    pana                     & \wordtype{noun}: giving, transfer, exchange                                                   \\
    \dots{} pana             & \wordtype{adjective}: generous                                                                \\
    pana (e \dots{})         & \wordtype{verb transitive}: to give, to put, to send, to place, to release, to emit, to cause \\
    \wordrule %%%%%%%%%%%%%%%%%%%%%%%%%%%%%%%%%%%%%%%%%%%%%%%%%%%%%%%%%%%%%
    poki                     & \wordtype{noun}: container, box, bowl, cup, glass                                             \\
    poki (e \dots{})         & \wordtype{verb transitive}: to box up, to put in, to can, to bottle                           \\
    \wordrule %%%%%%%%%%%%%%%%%%%%%%%%%%%%%%%%%%%%%%%%%%%%%%%%%%%%%%%%%%%%%
    tawa                     & \wordtype{noun}: movement, transportation                                                     \\
    \dots{} tawa             & \wordtype{adjective}, \wordtype{adverb}: moving, mobile                                       \\
    \dots{}, tawa \dots{}    & \wordtype{preposition}: to, in order to, towards, for, until                                  \\
    tawa                     & \wordtype{verb intransitive}: to walk, to travel, to move, to leave, to visit                 \\
    tawa (e \dots{})         & \wordtype{verb transitive}: to move, to displace                                              \\
\end{vocabularytable}

\newpage

\subsection*{Indirect Objects and Intransitive Verbs}
\index{object!indirect}
\index{verb!intransitive}
\index{predicate phrase}
We've already learned about direct objects.
A direct object is most strongly influenced by the action (i.e.\@ the transitive verb).
Your can ask for direct object (accusative object) by ``Who'' or ``What'' (``What does she repair?'').
But, in the sentence, ``I am in the house.'' the object ``in the house'' is an indirect object because you can't ask for it with ``Who'' or ``What''.
It is also not directly influenced by the predicate.
A indirect object is also part of the predicate phrase.
The first slot in the intransitive object is always a noun or pronoun slot.
After that there are optional slots for adjectives, possessive pronouns, and demonstrative pronouns.

We've already learned transitive verbs.
A transitive verb does something to the direct object.
On the other hand, verbs that do not affect an object are called intransitive verbs.
An intransitive verb is followed by either no object or an indirect object.
In the sentences, ``I am.'' and ``I am in the house.'' the word ``am'' is an intransitive verb.
There is no \texttp{e} between intransitive verb and indirect object.

\index{lon@\texttp{lon}!intransitive verb}
The intransitive verb \texttp{lon} means ``to be there'' or ``to exist''.
Since there is no other predicate before \texttp{lon} there must be a verb \texttp{lon}.

\begin{translationtable}
    suno li lon sewi. & The sun is in the sky.      \\
    kili li lon poki. & The fruit is in the basket. \\
    mi lon tomo.      & I'm in the house.           \\
\end{translationtable}
%
\index{kepeken@\texttp{kepeken}!intransitive verb}%
The intransitive verb \texttp{kepeken} means ``to use''.

\begin{translationtable}
    mi kepeken ilo.        & I'm using tools.       \\
    sina wile kepeken ilo. & You have to use tools. \\
    mi kepeken poki ni.    & I'm using that cup.    \\
\end{translationtable}
%
\texttp{kepeken} is used as a transitive verb in some other lessons.
This is because you can ask for the object after \texttp{kepken} with ``What''.
As however the object is not directly influenced by the verb \texttp{kepeken}, it is an indirect object and \texttp{kepeken} an intransitive verb.

\index{kon@\texttp{kon}!intransitive verb}%
The intransitive verb \texttp{kon} means ``to breathe''.

\begin{translationtable}
    jan ni li kon ike. & This person is breathing badly. \\
\end{translationtable}
%
In contrast, the transitive verb \texttp{kon} means ``to blow away something''.

\begin{translationtable}
    mi kon e ilo suno. & I blow out the candle. \\
\end{translationtable}
%
\index{kama@\texttp{kama}!intransitive verb}
The intransitive verb \texttp{kama} means ``to come'' or ``to arrive''.

\begin{translationtable}
    pona li kama. & The good will come. \\
\end{translationtable}
%
\index{pakala@\texttp{pakala}!intransitive verb}%
The intransitive verb \texttp{pakala} means ``to screw up'', ``to fall apart'' or ``to break''.

\begin{translationtable}
    tomo ni li pakala. & This house is falling apart. \\
\end{translationtable}
%
\index{sewi@\texttp{sewi}!intransitive verb}%
The intransitive verb \texttp{sewi} means ``to get up''.

\begin{translationtable}
    mi sewi. & I get up. \\
\end{translationtable}

\subsection*{Intransitive Verbs, Adverbs and Auxiliary Verbs}
\index{adverb}
\index{auxiliary verb}
\index{verb!auxiliary}
We have learned that a verb can be modified by an adverb.
This of course also applies to intransitive verbs.
In this example, the adverb \texttp{mute} modifies the intransitive verb \texttp{lon.}

\begin{translationtable}
    mi lon mute tomo. & I'm often in the house. \\
\end{translationtable}
%
An intransitive verb can of course be preceded by an auxiliary verb.

\begin{translationtable}
    mi wile lon tomo. & I want to be in the house. \\
\end{translationtable}

\newpage

\practice{
    How you can not ask for an indirect object?                      & \answer{You can't ask ``Who'' or ``What''.}                     \\\wordrule
    Which object type is strongly influenced by the predicate?       & \answer{The direct object.}                                     \\\wordrule
    Which phrase in the sentence does the indirect object belong to? & \answer{To the predicate phrase.}                               \\\wordrule
    What slot is in the first position in an indirect object?        & \answer{A noun or pronoun slot.}                                \\\wordrule
    What do you call verbs that don't affect an object?              & \answer{They are intransitive verbs.}                           \\\wordrule
    What precedes an indirect object in Toki Pona?                   & \answer{An intransitive verb.}                                  \\\wordrule
    Where is a slot for an adjective demonstrative pronoun possible? & \answer{After a noun.}                                          \\\wordrule
    Where's an auxiliary verb slot?                                  & \answer{An auxiliary verb is placed in front of the main verb.} \\
}

\practice[Try to translate these sentences.]{
    \answer{\texttp{ni li tawa jan pona mi.}}     & This is for my friend.             \\\wordrule
    \answer{\texttp{ilo li lon poki.}}            & The tools are in the container.    \\\wordrule
    \answer{\texttp{poki ni li lon jaki.}}        & That bottle is in the dirt.        \\\wordrule
    \answer{\texttp{ona mute li utala toki.}}     & They are arguing.                  \\\wordrule
    \answer{\texttp{meli li lon e jan lili ona.}} & The woman gave birth to her child. \\
}