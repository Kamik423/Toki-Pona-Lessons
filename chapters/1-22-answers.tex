%!TEX root = ../toki-pona.tex

\section{Answers}
\printsolutions

\section*{REMOVE ALL ANSWERS BELOW (in progress):}
\subsection*{Direct Objects}
\label{'direct_objects_compund_sentences'}

\begin{supertabular}{p{5,5cm}|ll}
    How to ask for the direct object?                                                   &  & With 'whom' or' what'.                                \\
    What word type has a predicate before the separator \textit{e}?                     &  & It is always a transitive verb.                       \\
    To which phrase in the sentence belongs a direct object?                            &  & To the predicate phrase.                              \\
    What kinds of words are possible after the separator \textit{e}?                    &  & A noun or pronoun.                                    \\
    What is a predicate noun?                                                           &  & A noun used as a predicate.                           \\
    Where are possible slots for reflexive pronouns?                                    &  & After the separator \textit{e}.                       \\
    Is it possible to describe several properties of a subject with several \textit{e}? &  & No, because \textit{e} comes after a transitive verb. \\
    How can you create multiple predicate phrases in a sentence?                        &  & With several separators \textit{li}.                  \\
\end{supertabular}

\begin{supertabular}{p{5,5cm}|ll}
    e - mi moku e kili.                 &  & separator         \\
    pona - mi pona e ijo.               &  & transitive verb   \\
    The second sina - sina telo e sina. &  & reflexive pronoun \\
    ilo - ona li pona e ilo.            &  & noun              \\
\end{supertabular}

\begin{supertabular}{p{5,5cm}|ll}
    I have a tool.                   &  & mi jo e ilo.              \\
    She's eating fruit.              &  & ona li moku e kili.       \\
    Something is watching me.        &  & ijo li lukin e mi.        \\
    Pineapple is a food and is good. &  & kili li ' moku li ' pona. \\
    He washes himself.               &  & ona li telo e ona.        \\
\end{supertabular}

\begin{supertabular}{p{5,5cm}|ll}
    mi ' jan li ' suli. &  & I am somebody and am important. \\
\end{supertabular}

\newpage

\subsection*{Verbs, Adverbs, Auxiliary Verbs}
\label{'adverbs'}

\begin{supertabular}{p{5,5cm}|ll}
    What are adverbs?                                           &  & Adverbs describe an action (verb).                 \\
    Can an adverb be ranked according to a predicate noun?      &  & No, this is not possible.                          \\
    Where are slots for adverbs located?                        &  & Only after verbs.                                  \\
    What kind of words describes an action?                     &  & Verbs.                                             \\
    When does a predicate phrase contain slots for adverbs?     &  & If the predicate phrase contains a verb.           \\
    What is an auxiliary verb used for?                         &  & It complements the main verb.                      \\
    Which phrase in the sentence can contain an auxiliary verb? &  & An auxiliary verb belongs to the predicate phrase. \\
\end{supertabular}

\begin{supertabular}{p{5,5cm}|ll}
    kama - mi kama jo e telo.  &  & auxiliary verb                  \\
    wile - mi wile lukin e ma. &  & auxiliary verb, transitive verb \\
    ike - mi lawa ike e jan.   &  & adverb                          \\
    jan - mi ' jan.            &  & adjective, noun                 \\
\end{supertabular}

\begin{supertabular}{p{5,5cm}|ll}
    jan li pona ilo e ilo.     &  & The guy improve useful the tool. \\
    sina lukin unpa mute e mi. &  & You're looking at me very sexy.  \\
    jaki li jaki lili e mi.    &  & The garbage dirtys me something. \\
    sina len nasa jaki e sina. &  & You dress disgustingly silly.    \\
    ilo li sewi e sewi.        &  & The machine raises up the roof.  \\
    ona li lawa utala e utala. &  & He leads fightingly the battle.  \\
    mi wile unpa e ona.        &  & I want to have sex with him/her. \\
    jan li wile jo e ma.       &  & People want to own land.         \\
\end{supertabular}

\begin{supertabular}{p{5,5cm}|ll}
    She increases the property very badly. &  & ona li mute ike mute e jo. \\
    I want to have a lot of sex with you.  &  & mi wile unpa mute e sina.  \\
    She was barely dressed.                &  & ona li len lili e ona.     \\
    The sun shines warmly on the land.     &  & suno li suno seli e ma.    \\
    She's good.                            &  & ona li ' pona.             \\
    He wants to destroy the tool.          &  & ona li wile pakala e ilo.  \\
    She is thirsty.                        &  & ona li wile moku e telo.   \\
\end{supertabular}

\newpage

\subsection*{Nouns, Adjectives}
\label{'adjectives'}

\begin{supertabular}{p{5,5cm}|ll}
    What does a possessive pronoun replace?                   &  & It replaces a adjective.                                             \\
    What types of demonstrative pronouns are there?           &  & Adjective and noun demonstrative pronouns.                           \\
    What is more complex in Toki Pona: adjectives or adverbs? &  & adjectives.                                                          \\
    By what kind of words are nouns described?                &  & By adjectives.                                                       \\
    What is the difference between adverbs and adjectives?    &  & Adverbs describe verbs and adjectives describe nouns.                \\
    Where are adjective slots located?                        &  & Only after nouns and as a predicate adjective in a predicate phrase. \\
    Can an adjective follow a predicate noun?                 &  & Yes, since a predicate noun is a noun.                               \\
\end{supertabular}

\begin{supertabular}{p{5,5cm}|ll}
    mi jo e kili.             &  & I have a fruit.                          \\
    ona li ' pona li ' lili.  &  & It is good and is small.                 \\
    mi moku lili e kili lili. &  & I nibble (eat a little) the small fruit. \\
\end{supertabular}

\begin{supertabular}{p{5,5cm}|ll}
    The leader drank dirty water.        &  & jan lawa li moku e telo jaki.     \\
    I need a fork.                       &  & mi wile e ilo moku.               \\
    An enemy is attacking them.          &  & jan ike li utala e ona mute.      \\
    That bad person has strange clothes. &  & jan ike ni li jo e len nasa.      \\
    We drank a lot of vodka.             &  & mi mute li moku e telo nasa mute. \\
    Children watch adults.               &  & jan lili li lukin e jan suli.     \\
\end{supertabular}

\begin{supertabular}{p{5,5cm}|ll}
    mi lukin e ni.                &  & I am looking at that.                \\
    mi lukin sewi e tomo suli.    &  & I am looking up at the big building. \\
    seli suno li seli e tomo mi.  &  & The sun's warmth heats my home.      \\
    jan lili li wile e telo kili. &  & Children want fruit juice.           \\
    ona mute li nasa e jan suli.  &  & They drove the adults crazy.         \\
    mi kama e pakala.             &  & I caused an accident.                \\
\end{supertabular}

\newpage

\subsection*{Indirect Objects}
\label{'indirect_objects'}

\begin{supertabular}{p{5,5cm}|ll}
    How you can not ask for an indirect object?                      &  & You can't ask 'who' or 'what'.                         \\
    Which object type is strongly influenced by the predicate?       &  & The direct object.                                     \\
    Which phrase in the sentence does the indirect object belong to? &  & To the predicate phrase.                               \\
    What slot is in the first position in an indirect object?        &  & A noun or pronoun slot.                                \\
    What do you call verbs that don't affect an object?              &  & They are intransitive verbs.                           \\
    What stands in front of an indirect object in Toki Pona?         &  & An intransitive verb.                                  \\
    Where is a slot for an adjective demonstrative pronoun possible? &  & After a noun.                                          \\
    Where's an auxiliary verb slot?                                  &  & An auxiliary verb is placed in front of the main verb. \\
\end{supertabular}

\begin{supertabular}{p{5,5cm}|ll}
    This is for my friend.             &  & ni li tawa jan pona mi.     \\
    The tools are in the container.    &  & ilo li lon poki.            \\
    That bottle is in the dirt.        &  & poki ni li lon jaki.        \\
    They are arguing.                  &  & ona mute li utala toki.     \\
    The woman gave birth to her child. &  & meli li lon e jan lili ona. \\
\end{supertabular}

\newpage

\subsection*{Prepositional Objects}
\label{'prepositional_objects'}

\begin{supertabular}{p{5,5cm}|ll}
    What is closely related to a preposition?                                    &  & A preposition is closely connected to the verb. \\
    Which phrase in the sentence does the prepositional object belong to?        &  & It is an optional part of a predicate phrase.   \\
    Where are preposition slots located?                                         &  & At the beginning of a prepositional object.     \\
    At which position in the sentence can a prepositional object be located?     &  & At the end of a sentence.                       \\
    Which separators can be used to form composite sentences?                    &  & With the separators \textit{li} and \textit{e}. \\
    Which slots are possible in the second position in the prepositional object? &  & A noun or pronoun slot.                         \\
\end{supertabular}

\begin{supertabular}{p{5,5cm}|ll}
    I fixed the flashlight using a small tool. &  & mi pona e ilo suno, kepeken ilo lili.        \\
    I like Toki Pona.                          &  & toki pona li ' pona, tawa mi.                \\
    We gave them food.                         &  & mi mute li pana e moku, tawa ona mute.       \\
    I want to go to his house using my car.    &  & mi wile tawa tomo ona, kepeken tomo tawa mi. \\
    People look like ants.                     &  & jan li lukin, sama pipi.                     \\
\end{supertabular}

\begin{supertabular}{p{5,5cm}|ll}
    sina wile kama, tawa tomo toki.                  &  & You should come to the chat room.                \\
    jan li toki, kepeken toki pona, lon tomo toki.   &  & People talk in/using Toki Pona in the chat room. \\
    mi tawa, tawa tomo toki. ona li ' pona, tawa mi. &  & I go the chat room. It is good for me.           \\
                                                     &  & I like to go to the chat room.                   \\
    sina kama jo e jan pona, lon ni.                 &  & You will get friends there.                      \\
    sama li ' ike.                                   &  & Equality is bad.                                 \\
    mi sona e tan.                                   &  & I know the reason. / I know why.                 \\
\end{supertabular}

\newpage

\subsection*{Relative Location Information}
\label{'other_prepositions'}

\begin{supertabular}{p{5,5cm}|ll}
    How do you create relative location information in Toki Pona?     &  & With an indirect verb or a preposition and a compound spatial noun. \\
    What is a possessive pronoun?                                     &  & A possessive pronoun expresses a characteristic or affiliation.     \\
    Where is a slot for a substantive demonstrative pronoun possible? &  & Instead of a noun.                                                  \\
    Which separator is at the end of a declarative sentence?          &  & A full stop.                                                        \\
    What is a predicate adjective?                                    &  & An adjective that is used as predicate.                             \\
    Which sentence phrases can contain spatial nouns be found?        &  & In an indirect object or prepositional object.                      \\
\end{supertabular}

\begin{supertabular}{p{5,5cm}|ll}
    My friend is beside me.      &  & jan pona mi li lon poka mi.   \\
    The sun is above me.         &  & suno li lon sewi mi.          \\
    The land is beneath me.      &  & ma li lon anpa mi.            \\
    Bad things are behind me.    &  & ijo ike li lon monsi mi.      \\
    I'm okay because I'm alive.  &  & mi ' pona, tan ni: mi lon.    \\
    I look at the land with you. &  & mi lukin e ma, lon poka sina. \\
\end{supertabular}

\begin{supertabular}{p{5,5cm}|ll}
    poka mi li ' pakala.                 &  & My hip hurts.                 \\
    mi kepeken poki li kepeken ilo moku. &  & I'm using a bowl and a spoon. \\
    jan li lon insa tomo.                &  & Somebody's inside the house.  \\
\end{supertabular}

\newpage

\subsection*{Negation Yes/No Questions}
\label{'negation_yes_no_questions'}

\begin{supertabular}{p{5,5cm}|ll}
    Which separator is at the end of a question?                    &  & A question mark.                                                                  \\
    How is a verb negated in Toki Pona?                             &  & By placing the adverb \textit{ala} after the verb.                                \\
    How do you answer in Toki Pona negative to a yes/no question?   &  & One repeats the predicate or the auxiliary of the question and adds \textit{ala}. \\
    How do you answer positively to a yes/no question in Toki Pona? &  & One repeats the predicate or the auxiliary of the question.                       \\
\end{supertabular}

\begin{supertabular}{p{5,5cm}|ll}
    You have to tell me why.     &  & sina wile toki e tan, tawa mi. \\
    Is a bug beside me?          &  & pipi li lon ala lon poka mi?   \\
    I can't sleep.               &  & mi ken ala lape.               \\
    I don't want to talk to you. &  & mi wile ala toki, tawa sina.   \\
    He didn't go to the lake.    &  & ona li tawa ala, tawa telo.    \\
\end{supertabular}

\begin{supertabular}{p{5,5cm}|ll}
    sina wile ala wile pali? wile ala.    &  & Do you want to work? No.          \\
    jan utala li seli ala seli e tomo?    &  & Is the warrior burning the house? \\
    jan lili li ken ala moku e telo nasa. &  & Children can't drink beer.        \\
    sina kepeken ala kepeken ni?          &  & Are you using that?               \\
    sina ken ala ken kama?                &  & Can you come?                     \\
    sina pona ala pona?                   &  & Are you OK?                       \\
\end{supertabular}

\newpage

\subsection*{Unofficial Words}
\label{'unofficial_words_answers'}

\begin{supertabular}{p{5,5cm}|ll}
    What are proper names in Toki Pona?                                 &  & Unofficial words, adjectives          \\
    Where are slots for predicate adjectives located?                   &  & After the separator \textit{li}.      \\
    How are names in \textit{toki pona} highlighted?                    &  & The first letter is a capital letter. \\
    How is the original spelling of a name marked?                      &  & By quotation marks.                   \\
    Which slots can unofficial words fill?                              &  & Adjective slots.                      \\
    What kind of word type must unofficial words be used together with? &  & With a noun.                          \\
\end{supertabular}

\begin{supertabular}{p{5,5cm}|ll}
    Susan is crazy.            &  & jan Susan li ' nasa.                \\
    I come from Europe.        &  & mi kama, tan ma suli Elopa.         \\
    My name is Ken.            &  & mi ' jan Ken. / nimi mi li Ken.     \\
    Hello, Lisa.               &  & jan Lisa o, toki!                   \\
    I want to go to Australia. &  & mi wile tawa, tawa ma suli Oselija. \\
\end{supertabular}

\begin{supertabular}{p{5,5cm}|ll}
    mi wile kama sona e toki Inli.           &  & I want to learn English.         \\
    jan Ana o pana e moku, tawa mi!          &  & Ana, give me food.               \\
    jan Mose o lawa e mi mute, tawa ma pona! &  & Moses, lead us to the good land. \\
\end{supertabular}

\newpage

\subsection*{Addressing People, Interjections, Commands}
\label{'commands_interjections'}

\begin{supertabular}{p{5,5cm}|ll}
    Which separator ends a command sentence (imperative)?                                      &  & With an exclamation mark.                               \\
    What is the subject of the command form if no one is addressed directly?                   &  & The interjection word \textit{o}.                       \\
    How do you address people by name?                                                         &  & \textit{jan Name o,.... }                               \\
    What do injections consist of?                                                             &  & A noun or an interjection word and an exclamation mark. \\
    Which separator stands bevor  the predicate if someone is directly addressed in a command? &  & The separator \textit{o}.                               \\
    Which separator ends an interjection (exclamation)?                                        &  & With an exclamation mark.                               \\
\end{supertabular}

\begin{supertabular}{p{5,5cm}|ll}
    Go!                   &  & o tawa!              \\
    Mama, wait.           &  & mama meli o awen!    \\
    Hahaha! That's funny. &  & a a a! ni li ' musi. \\
    F-ck!                 &  & pakala!              \\
    Bye!                  &  & mi tawa!             \\
\end{supertabular}

\begin{supertabular}{p{5,5cm}|ll}
    mu!                       &  & meow, woof, moo, etc.                         \\
    o tawa musi, lon poka mi! &  & Dance with me!                                \\
    tawa pona!                &  & Good bye (spoken by the person who's staying) \\
    o pu!                     &  & Buy and read the official Toki Pona book!     \\
\end{supertabular}

\newpage

\subsection*{Questions}
\label{'questions_using_seme'}

\begin{supertabular}{p{5,5cm}|ll}
    How does the sentence structure change for a question in \textit{toki pona}? &  & The sentence structure does not change.                                      \\
    What kind of word has the word \textit{seme}?                                &  & It is a question pronoun.                                                    \\
    What is a reflexive pronoun?                                                 &  & A reflexive pronoun represents the subject in the direct object.             \\
    What can represent the word \textit{seme}?                                   &  & Sentence parts or all word types (except separators).                        \\
    How do you ask for a person (who, whom)?                                     &  & With the noun \textit{jan} and \textit{seme}.                                \\
    How is a Why question asked?                                                 &  & With the preposition \textit{tan} and \textit{seme} as prepositional object. \\
    How do you ask for an indirect object?                                       &  & If \textit{seme} follows an intransitive verb.                               \\
    How to ask for a prepositional object?                                       &  & If \textit{seme} follows after a preposition.                                \\
    Are there nested subordinate clauses in \textit{toki pona}?                  &  & No, there are none.                                                          \\
\end{supertabular}

\begin{supertabular}{p{5,5cm}|ll}
    What do you want to do? &  & sina wile pali e seme?                 \\
    Who loves you?          &  & jan seme li olin e sina?               \\
    Does it sweeten?        &  & ni li suwi ala suwi?                   \\
    I'm going to bed.       &  & mi tawa supa lape.                     \\
    Are more people coming? &  & jan sin li kama ala kama?              \\
    Give me a lollipop!     &  & o pana e suwi, tawa mi!                \\
    Who's there?            &  & jan seme li lon? / jan seme li lon ni? \\
    Which bug hurt you?     &  & pipi seme li pakala e sina?            \\
    He loves to eat.        &  & moku li pona, tawa ona.                \\
    Pardon?                 &  & seme?                                  \\
    This is mine.           &  & mi jo e ni.                            \\
\end{supertabular}

\begin{supertabular}{p{5,5cm}|ll}
    jan Ken o, mi olin e sina.    &  & Ken, I love you.                 \\
    ni li ' jan seme?             &  & Who is that?                     \\
    sina lon seme?                &  & Where are you?                   \\
                                  &  & (lit: You in what?)              \\
    mi lon, tan seme?             &  & Why am I here?                   \\
                                  &  & (lit: I exist because-of what?)  \\
    jan seme li ' meli sina?      &  & Who is your girlfriend/wife?     \\
    sina tawa ma tomo, tan seme?  &  & Why did you go to the city?      \\
    sina wile tawa, tawa ma seme? &  & What place do you want to go to? \\
\end{supertabular}

\newpage

\subsection*{Compound Nouns}
\label{'pi'}

\begin{supertabular}{p{5,5cm}|ll}
    Can the separator \textit{pi} be used to separate adjectives?                             &  & No, it is not possible.                                                \\
    Where is the main noun in \textit{toki pona} of a compound noun?                          &  & At the beginning.                                                      \\
    How many words must at least be between the separator \textit{pi} and the next separator? &  & Two words.                                                             \\
    Where can adjective slots after the separator \textit{pi} be located?                     &  & On the second and following positions after the separator \textit{pi}. \\
    How do you ask for the owner of an item?                                                  &  & item + \textit{pi} + \textit{jan} + \textit{seme}                      \\
\end{supertabular}

\begin{supertabular}{p{5,5cm}|ll}
    Keli's child is funny.                    &  & jan lili pi jan Keli li ' musi.         \\
    I am a Toki Ponan.                        &  & mi ' jan pi toki pona.                  \\
    He is a good musician.                    &  & ona li ' jan pona pi kalama musi.       \\
    The captain of the ship is eating.        &  & jan lawa pi tomo tawa telo li moku.     \\
    Meow.                                     &  & mu!                                     \\
    Enya's music is good.                     &  & kalama musi pi jan Enja li ' pona.      \\
    Which people of this group are important? &  & jan seme pi kulupu ni li suli?          \\
    Our house is messed up.                   &  & tomo pi mi mute li ' pakala.            \\
    How did she make that?                    &  & ona li pali e ni, kepeken nasin seme?   \\
    I look at the land with my friend.        &  & mi lukin e ma, lon poka pi jan pona mi. \\
    Whom did you go with?                     &  & sina tawa, lon poka pi jan seme?        \\
\end{supertabular}

\begin{supertabular}{p{5,5cm}|ll}
    pipi pi ma mama mi li ' lili.                              &  & The insects of my homeland are small.      \\
    kili pi jan Linta li ' ike.                                &  & Linda's fruit is bad.                      \\
    len pi jan Susan li ' jaki.                                &  & Susan's clothes are dirty.                 \\
    mi sona ala e nimi pi ona mute.                            &  & I don't know their names.                  \\
    mi wile toki meli.                                         &  & I want to talk about girls.                \\
    sina pakala e ilo, kepeken nasin seme?                     &  & How did you break the tool?                \\
    jan Wasintan [Washington] li ' jan lawa pona pi ma Mewika. &  & Washington was a good leader of America.   \\
    wile pi jan ike li pakala e ijo.                           &  & The desires of evil people mess things up. \\
\end{supertabular}

\newpage

\subsection*{Conjunctions \textit{kin} Temperature}
\label{'conjunctions_temperature'}

\begin{supertabular}{p{5,5cm}|ll}
    What are conjunctions?                                                                              &  & Conjunctions connect words and phrases.                                       \\
    What is an answer-question?                                                                         &  & The answer is already included in the question.                               \\
    % What is the difference between conjunctions and prepositions? && Conjunctions do not cause cases. \\
    How is an answer-question formed in \textit{toki pona}?                                             &  & The conjunction \textit{anu} and the question pronoun \textit{seme} is added. \\
    Is there a comma before or after the conjunction \textit{taso}?                                     &  & No, it is not.                                                                \\
    What are alternative-questions?                                                                     &  & A selection of several options is requested.                                  \\
    What connects the conjunction \textit{taso}?                                                        &  & It refers to the previous sentence.                                           \\
    What connects the conjunction \textit{en}?                                                          &  & It combines (composite) nouns or pronouns.                                    \\
    How is an alternative-question formed in \textit{toki pona}?                                        &  & With the conjunction \textit{anu}.                                            \\
    How is a yes/no-question with predicate nouns or predicate adjectives formed in \textit{toki pona}? &  & An answer question is formulated.                                             \\
\end{supertabular}

\begin{supertabular}{p{5,5cm}|ll}
    Do you want to come or what?            &  & sina wile kama anu seme?            \\
    Do you want food, or do you want water? &  & sina wile e moku anu telo?          \\
    I still want to go to my house.         &  & mi wile kin tawa, tawa tomo mi.     \\
    This paper feels cold.                  &  & lipu ni li ' lete, tawa mi.         \\
    I like currency of other nations.       &  & mani pi ma ante li ' pona, tawa mi. \\
    I want to go, but I can't.              &  & mi wile tawa. taso mi ken ala.      \\
    I'm alone.                              &  & mi taso li lon.                     \\
    Do you like me?                         &  & mi ' pona, tawa sina anu seme?      \\
    This lake is cold.                      &  & telo ni li ' lete, tawa mi.         \\
\end{supertabular}

\begin{supertabular}{p{5,5cm}|ll}
    mi olin kin e sina.                  &  & I still love you. / I love you too.      \\
    mi pilin e ni: ona li jo ala e mani. &  & I think that he doesn't have money.      \\
    mi wile lukin e ma ante.             &  & I want to see other countries.           \\
    mi wile ala e ijo. mi lukin taso.    &  & I don't want anything. I'm just looking. \\
    mi pilin lete.                       &  & I'm cold.                                \\
                                         &  & (lit. "I feel cold.")                    \\
    sina wile toki, tawa mije anu meli?  &  & Do you want to talk a male, or a female? \\
\end{supertabular}

\newpage

\subsection*{Colors}
\label{'colors'}

\begin{supertabular}{p{5,5cm}|ll}
    Which kinds of word are possible in the slot after the conjunction \textit{en}? &  & Noun or pronouns.                                              \\
    How are color pattern of an item described in \textit{toki pona}?               &  & Item + \textit{pi} + 1. colour + \textit{en} + 2. colour \dots \\
    How are color tones described for which there is no word in \textit{toki pona}? &  & Through several words.                                         \\
    Which kinds of word are possible in the slot after the separator \textit{pi}?   &  & Noun or pronouns.                                              \\
    What kinds of words have the words for colors in \textit{toki pona}?            &  & Adjectives and nouns.                                          \\
\end{supertabular}

\begin{supertabular}{p{5,5cm}|ll}
    I don't see the blue bag.                &  & mi lukin ala e poki laso.                    \\
    A little green person came from the sky. &  & jan laso jelo lili li kama, tan sewi. /      \\
    A little green person came from the sky. &  & jan jelo laso lili li kama, tan sewi.        \\
    I like the color purple.                 &  & kule loje laso li ' pona, tawa mi. /         \\
    I like the color purple.                 &  & kule laso loje li ' pona, tawa mi.           \\
    The sky is blue.                         &  & sewi li ' laso.                              \\
    Look at that red bug.                    &  & o lukin e pipi loje ni!                      \\
    I want the map.                          &  & mi wile e sitelen ma.                        \\
    Do you watch The X-Files?                &  & sina lukin ala lukin e sitelen tawa X-Files? \\
    Which color do you like?                 &  & kule seme li ' pona, tawa sina?              \\
    Is it red?                               &  & ona li ' loje anu seme?                      \\
\end{supertabular}

\begin{supertabular}{p{5,5cm}|ll}
    ni li pimeja ala pimeja e suno?                      &  & Does that darken the sun?                        \\
    suno li ' jelo.                                      &  & The sun is yellow.                               \\
    telo suli li ' laso.                                 &  & The big water [ocean] is blue.                   \\
    mi wile moku e kili loje.                            &  & I want to eat a red fruit.                       \\
    ona li kule e tomo tawa.                             &  & He's painting the car.                           \\
    len pi loje en laso pi meli sina li ' pona, tawa mi. &  & I like your wife's red and blue patterned dress. \\
\end{supertabular}

\begin{supertabular}{p{5,5cm}|ll}
    ma mi li ' pimeja. &  & My land is dark. \\
    kalama ala li lon  &  & No sound exists. \\
    mi lape. mi sona.  &  & I sleep. I know. \\
\end{supertabular}

\newpage

\subsection*{Living Things}
\label{'living_things'}

\begin{supertabular}{p{5,5cm}|ll}
    Which separator is at the end of a question?             &  & A question mark.                                                     \\
    In which cases is a comma used?                          &  & Addressing people: after \textit{o}. Optionally before prepositions. \\
    In which cases a colon is used?                          &  & A colon is between an hint sentences and a sentences.                \\
    Where are possible slots for prepositions in a sentence? &  & At the beginning of a prepositional object.                          \\
\end{supertabular}

\begin{supertabular}{p{5,5cm}|ll}
    Is this a mammal?                  &  & ni li ' soweli anu seme?                              \\
    I want a puppy.                    &  & mi wile e soweli lili.                                \\
    Ahh! The dinosaur wants to eat me! &  & a! akesi li wile moku e mi!                           \\
    The mosquito bit me.               &  & pipi li moku e mi.                                    \\
    Cows say moo.                      &  & soweli li toki e mu.                                  \\
    Birds fly in air.                  &  & waso li tawa, lon kon.                                \\
    Let's eat fish.                    &  & mi mute o moku e kala!                                \\
    Flowers are pretty.                &  & kasi kule li ' pona lukin.                            \\
    I like plants.                     &  & kasi li ' pona, tawa mi.                              \\
    Have you improved?                 &  & sina pona ala pona e sina? sina pona e sina anu seme? \\
\end{supertabular}

\begin{supertabular}{p{5,5cm}|ll}
    mama ona li kepeken kasi nasa. &  & His mother used pot.                      \\
    akesi li pana e telo moli.     &  & The snake emitted venom ("deadly fluid"). \\
    pipi li moku e kasi.           &  & Bugs eat plants.                          \\
    soweli mi li kama moli.        &  & My dog is dying.                          \\
    jan Pawe o, mi wile ala moli.  &  & Forrest, I don't want to die.             \\
    mi lon ma kasi.                &  & I'm in the forest.                        \\
    ona li kasi ala kasi?          &  & Is it growing?                            \\
\end{supertabular}

\newpage

\subsection*{The Body}
\label{'the_body'}

\begin{supertabular}{p{5,5cm}|ll}
    kepeken - mi kepeken ilo.            &  & intransitive verb, noun                    \\
    sina - sina pona ala pona?           &  & transitive verb                            \\
    kama - mi kama jo e tomo tawa.       &  & auxiliary verb                             \\
    lon - mi lon tomo.                   &  & intransitive verb, adverb, adjective, noun \\
    kepeken - mi pali e ni, kepeken ilo. &  & preposition                                \\
\end{supertabular}

\begin{supertabular}{p{5,5cm}|ll}
    Kiss me.                   &  & o pilin e uta mi, kepeken uta sina! \\
    I need to pee.             &  & mi wile pana e telo jelo.           \\
    My hair is wet.            &  & linja mi li ' telo.                 \\
    Something is in my eye.    &  & ijo li lon oko mi.                  \\
    I can't hear your talking. &  & mi ken ala kute e toki sina.        \\
    I need to crap.            &  & mi wile pana e ko jaki.             \\
    That hole is big.          &  & lupa ni li ' suli.                  \\
    Is it a chain?             &  & ona li ' linja anu seme?            \\
\end{supertabular}

\begin{supertabular}{p{5,5cm}|ll}
    selo pi jelo en laso pi akesi lili li ' pona, tawa mi. &  & I like the little lizard's green-blue skin. \\
    a! telo sijelo loje li kama tan nena kute mi!          &  & Ahh! Blood is coming from my ear!           \\
    selo mi li wile e ni: mi pilin e ona.                  &  & My skin wants this: I touch it.             \\
                                                           &  & This is how we say that our skin itches.    \\
    o pilin e nena.                                        &  & Touch the button.                           \\
    o moli e pipi, kepeken palisa.                         &  & Kill the roach with the stick.              \\
    luka mi li ' jaki. mi wile telo e ona.                 &  & My hands are dirty. I want to wash them.    \\
    o pana e sike, tawa mi.                                &  & Give the ball to me.                        \\
    mi pilin e seli sijelo sina.                           &  & I feel your bodily warmth.                  \\
    ona li selo ala selo?                                  &  & Is it protecting?                           \\
\end{supertabular}

\newpage

\subsection*{Numbers}
\label{'numbers'}

\begin{supertabular}{p{5,5cm}|ll}
    How are ordinal numbers formed?                                  &  & With the adjective \textit{nanpa} before numbers. \\
    Can a number be placed directly after the separator \textit{li}? &  & Yes, as predicate adjective.                      \\
    Which word type are used to form numbers?                        &  & Adjectives.                                       \\
    How are large numbers formed?                                    &  & With the adjective \textit{mute}.                 \\
    Which word type can be used in a compound noun after numbers?    &  & Possessive pronouns.                              \\
    How to make sums?                                                &  & With conjunction \textit{en}.                     \\
\end{supertabular}

\begin{supertabular}{p{5,5cm}|ll}
    nanpa - ona li ' jan nanpa wan. &  & adjective                                 \\
    wan - mi wan.                   &  & transitive verb, adjective (number), noun \\
    luka - ni li ' luka tu.         &  & adjective, adjective (number), noun       \\
    % luka - ni li ' luka # tu.       &  & adjective, noun                           \\
    nanpa - sina nanpa e kili.      &  & transitive verb                           \\
    weka - sina tawa weka e sina.   &  & adverb                                    \\
    esun - o esun e ni!             &  & transitive verb                           \\
\end{supertabular}

\begin{supertabular}{p{5,5cm}|ll}
    I saw three birds.                         &  & mi lukin e waso tu wan.  \\
    Many people are coming.                    &  & jan mute li kama.        \\
    The first person is here.                  &  & jan pi nanpa wan li lon. \\
    I own two cars.                            &  & mi jo e tomo tawa tu.    \\
    Some (but not a lot) of people are coming. &  & jan mute lili li kama.   \\
    Unite!                                     &  & o wan!                   \\
    Is this a part?                            &  & ni li ' wan anu seme?    \\
\end{supertabular}

\begin{supertabular}{p{5,5cm}|ll}
    mi weka e ijo tu ni.    &  & I got rid of those two things. \\
    o tu.                   &  & Break up. Split apart.         \\
    mi lukin e soweli luka. &  & I saw five mammals.            \\
    mi ' weka.              &  & I was away.                    \\
    ona li sike ala sike?   &  & Is it rotating?                \\
\end{supertabular}

\newpage

\subsection*{Conditional Sentences}
\label{'la'}

%\begin{supertabular}{p{5,5cm}|ll}
%la - ken la ni li pona. && separator \\
%ken - ken la mi tawa. && noun \\
%\end{supertabular}

\begin{supertabular}{p{5,5cm}|ll}
    What is a conditional phrase?                                      &  & It formulates a condition.                                        \\
    What follows the separator \textit{la}?                            &  & A complete main sentence.                                         \\
    What can a conditional phrase consist of?                          &  & It consists of a (composite) noun/pronoun or a complete sentence. \\
    Which word types can be at the beginning of a conditional phrase?  &  & Noun or pronoun. Optionally, there can be a conjunction before.   \\
    Can the question pronoun \textit{seme} be in a conditional phrase? &  & Yes, in a interrogative sentence.                                 \\
\end{supertabular}

\begin{supertabular}{p{5,5cm}|ll}
    Maybe Susan will come.                  &  & ken la jan Susan li kama.                             \\
    Last night I watched X-Files.           &  & tenpo pimeja pini la mi lukin e sitelen tawa X-Files. \\
    If the enemy comes, burn these papers.  &  & jan ike li kama la o seli e lipu ni!                  \\
    Maybe he's in school.                   &  & ken la ona li lon tomo sona.                          \\
    I have to work tomorrow.                &  & tenpo suno kama la mi wile pali.                      \\
    When it's hot, I sweat.                 &  & seli li lon la mi pana e telo, tan selo mi.           \\
    Open the door.                          &  & o open e lupa!                                        \\
    The moon is big tonight.                &  & tenpo pimeja ni la mun li ' suli.                     \\
    Is the moon big tonight?                &  & tenpo pimeja ni la mun li ' suli anu seme?            \\
    Under what conditions will you do this? &  & seme la sina pali e ni?                               \\
\end{supertabular}

\begin{supertabular}{p{5,5cm}|ll}
    tenpo suno ni la mun li pimeja ala pimeja e suno?             &  & Is there an eclipse today?                                        \\
    ken la jan lili li wile moku e telo.                          &  & Maybe the baby is thirsty.                                        \\
    tenpo ali la o kama sona!                                     &  & Always learn!                                                     \\
    sina sona e toki ni la sina sona e toki pona!                 &  & Figure this one out for yourself. :o)                             \\
    open la ala li lon!                                           &  & There was nothing in the beginning!                               \\
    ken la tomo pi ona en sina pi jelo en loje li ' ike, tawa mi. &  & Maybe I don't like the yellow-red patterned house of her and you. \\
    sina wile jo e ilo moli la sina wile moli e jan.              &  & If you want a gun, you want to kill people.                       \\
    jan nasa pi ilo moli li ken pana e ike.                       &  & Weapon fools can bring bad things.                                \\
\end{supertabular}

\begin{supertabular}{p{5,5cm}|ll}
    tenpo suno ni li tenpo suno pali nanpa luka.                                &  & Today is Friday.                                   \\
    tenpo suno ni la jan lili pi kama sona li tawa ala, tawa tomo pi kama sona. &  & Today the pupils don't go to school.               \\
    ona li wile e ni: jan li pakala ala e ma e telo e kon.                      &  & They don't want people to destroy the environment. \\
    tenpo kama la ona li wile lon kin.                                          &  & They also want to be able to live in the future.   \\
\end{supertabular}
