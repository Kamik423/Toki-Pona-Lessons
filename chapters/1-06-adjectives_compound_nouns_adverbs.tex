%!TEX root = ../toki-pona.tex

\section{Nouns, Adjectives, Pronouns}
\subsection*{Vocabulary}
\index{mama@\texttp{mama}}
\index{meli@\texttp{meli}}
\index{mije@\texttp{mije}}
\index{nasa@\texttp{nasa}}
\index{seli@\texttp{seli}}
\index{toki@\texttp{toki}}

\begin{vocabularytable}
    mama             & \wordtype{noun}: parent, mother, father                                                       \\
    \dots{} mama     & \wordtype{adjective}: of the parent, parental, maternal, fatherly, motherly, mumsy            \\
    mama (e \dots{}) & \wordtype{verb transitive}: to mother somebody, to wet-nurse, mothering                       \\
    \wordrule %%%%%%%%%%%%%%%%%%%%%%%%%%%%%%%%%%%%%%%%%%%%%%%%%%%%%%%%%%%%%
    meli             & \wordtype{noun}: woman, female, girl, wife, girlfriend                                        \\
    \dots{} meli     & \wordtype{adjective}: female, feminine, womanly                                               \\
    \wordrule %%%%%%%%%%%%%%%%%%%%%%%%%%%%%%%%%%%%%%%%%%%%%%%%%%%%%%%%%%%%%
    mije             & \wordtype{noun}: man, male, husband, boyfriend                                                \\
    \dots{} mije     & \wordtype{adjective}: male, masculine, manly                                                  \\
    \wordrule %%%%%%%%%%%%%%%%%%%%%%%%%%%%%%%%%%%%%%%%%%%%%%%%%%%%%%%%%%%%%
    nasa             & \wordtype{noun}: stupidity, foolishness, silliness, nonsense, idiocy, obtuseness, muddler     \\
    \dots{} nasa     & \wordtype{adjective}, \wordtype{adverb}: silly, crazy, foolish, drunk, strange, stupid, weird \\
    nasa (e \dots{}) & \wordtype{verb transitive}: to drive crazy, to make weird                                     \\
    \wordrule %%%%%%%%%%%%%%%%%%%%%%%%%%%%%%%%%%%%%%%%%%%%%%%%%%%%%%%%%%%%%
    seli             & \wordtype{noun}: fire, warmth, heat                                                           \\
    \dots{} seli     & \wordtype{adjective}, \wordtype{adverb}: hot, warm, cooked                                    \\
    seli (e \dots{}) & \wordtype{verb transitive}: to heat, to warm up, to cook                                      \\
    \wordrule %%%%%%%%%%%%%%%%%%%%%%%%%%%%%%%%%%%%%%%%%%%%%%%%%%%%%%%%%%%%%
    toki             & \wordtype{noun}: language, speech, tongue, lingo, jargon,                                     \\
    \dots{} toki     & \wordtype{adjective}, \wordtype{adverb}: speaking, eloquent, linguistic, verbal, grammatical  \\
    toki             & \wordtype{verb intransitive}: to talk, to chat, to communicate                                \\
    toki (e \dots{}) & \wordtype{verb transitive}: to speak, to talk, to say, to pronounce, to discourse             \\
\end{vocabularytable}

\subsection*{Adjectives}
\index{adjective}
\index{noun!compound}
\index{adjective!predicate}
\index{predicate adjective}
\index{pona@\texttp{pona}!predicate adjective}
\index{pona@\texttp{pona}!adjective}
We had already got to know predicate adjectives as part of a predicate phrase.
A predicate adjective describes the noun of the subject phrase.
In this example, the predicate adjective \texttp{pona} in the predicate phrase describes the noun \texttp{jan} in the subject phrase.

\begin{translationtable}
    jan li ' pona. & The person is good. \\
\end{translationtable}
%
Generally speaking, one can say that adjectives describe nouns.
As in other languages, adjectives can also be written directly with the noun.
In Toki Pona the adjectives come after the noun to be described are written.
This is exactly the opposite in English, but in other languages, such as Italian, this is normal.
Possible adjective slots are therefore located directly after nouns slots and, as described above, predicate adjectives in the predicate phrase.
Noun slots are possible at the beginning of a subject phrase, at the beginning of a predictive phrase as predicate phrases, and in object phrases.
This means that adjective slots are possible in subject phrases and predictive phrases.
Adjectives are similar to adverbs but in Toki Pona more complex.

The noun \texttp{jan} is described here with the adjective \texttp{pona}.

\begin{translationtable}
    jan pona & good person \\
             & friend
\end{translationtable}
%
A friend is nothing but a good person.
Since Toki Pona has a very small vocabulary, we often have to combine nouns with adjectives to say a certain term.
Here are further examples:

\begin{translationtable}
    jan pakala & an injured person, victim, etc.      \\
    ilo moku   & an eating utensil (fork/spoon/knife) \\
\end{translationtable}
%
\index{adjective!more than one}%
You should not use more than three adjectives after a noun.
One adjective should not be used more than once.

\begin{translationtable}
    jan utala           & soldier              \\
    jan utala nasa      & stupid soldier       \\
    jan utala nasa mute & many stupid soldiers \\
\end{translationtable}
%
As you might have noticed, \texttp{mute} as an adjective is at the end of the phrase.
The reason for this is that the phrases build as you go along, so the adjectives must be put into an organized, logical order.
Notice the differences in these two phrases.

\begin{translationtable}
    jan utala nasa & stupid soldier \\
    jan nasa utala & fighting fool  \\
\end{translationtable}
%
Here are some handy noun adjective combinations using words that you've already learned and that are fairly common.

\begin{translationtable}
    ike lukin             & ugly                                 \\
    pona lukin            & pretty, attractive                   \\
    jan ni li pona lukin. & That person is pretty.               \\
    jan ike               & enemy                                \\
    jan lawa              & leader                               \\
    jan lili              & child                                \\
    jan sewi              & saint, God, Flying Spaghetti Monster \\
    jan suli              & adult                                \\
    jan unpa              & lover, prostitute                    \\
    ma telo               & mud, swamp                           \\
    ma tomo               & city, town                           \\
    mi mute               & we, us                               \\
    ona mute              & they, them                           \\
    telo nasa             & alcohol, beer, wine                  \\
    tomo telo             & restroom                             \\
    ilo suno              & flashlight                           \\
\end{translationtable}
%
\index{predicate adjective!several}%
Several predicate adjectives are also possible.
However, it is usually not possible to distinguish between a predicate noun at the first position in the predicate phrase and a predicate adjective.
While \texttp{mute} in this example can only be an adjective, \texttp{pona} can be an adjective or a noun.

\begin{translationtable}
    jan li ' pona mute. & Man is very good.                        \\
                        & The human being is the many good things. \\
\end{translationtable}

\subsubsection*{Gender}
\index{gender}
\index{mije@\texttp{mije}}
\index{meli@\texttp{meli}}
Toki Pona doesn't have any grammatical gender like in most western languages.
However, some words in Toki Pona (such as \texttp{mama}) don't tell you which gender a person is, and so we use \texttp{mije} and \texttp{meli} to distinguish.

\begin{translationtable}
    mama      & a parent in general (mother or father) \\
    mama meli & mother                                 \\
    mama mije & father                                 \\
\end{translationtable}

\subsection*{Possessive Pronouns}
\index{possessive pronoun}
\index{pronoun!possessive}
\index{mi@\texttp{mi}!possessive pronoun}
\index{sina@\texttp{sina}!possessive pronoun}
\index{ona@\texttp{ona}!possessive pronoun}
A possessive pronoun expresses a property or affiliation and is placed after the corresponding (composite) noun.
This means that for a noun with adjectives, the possessive pronoun is placed after the adjectives.
For a noun without adjectives, the possessive pronoun is located after the noun.
In these examples are \texttp{mi}, \texttp{sina}, and \texttp{ona} possessive pronouns.

\begin{translationtable}
    tomo pona mi & my nice house     \\
    ma sina      & your country      \\
    telo ona     & his/her/its water \\
\end{translationtable}

\subsection*{The Demonstrative Pronoun \texttp{ni}}
\index{demonstrative pronoun}
\index{pronoun!demonstrative}
\index{ni@\texttp{ni}!demonstrative pronoun}
\index{ni@\texttp{ni}!like an adjective}
\index{ni@\texttp{ni}!like a noun}
The demonstrative pronoun is a kind of word with which the speaker refers to an  item of conversation.
The demonstrative pronoun \texttp{ni} can be used both like an adjective and like a noun.
A slot for an adjective demonstrative pronoun is therefore possible after a noun.

\begin{translationtable}
    jan ni li pona.        & This bloke is good.              \\
    jan li lukin e ijo ni. & The guy's looking at this thing. \\
\end{translationtable}
%
A noun demonstrative pronoun is used instead of the noun.
Slots for noun demonstrative pronouns therefore correspond to the positions of noun slots in the sentence.

\begin{translationtable}
    ni li pona.        & This is good.              \\
    jan li lukin e ni. & The guy looks at that one. \\
\end{translationtable}

\practice{
    What does a possessive pronoun replace?                   & \answer{It replaces an adjective.}                                            \\\wordrule
    What types of demonstrative pronouns are there?           & \answer{Adjective and noun demonstrative pronouns.}                           \\\wordrule
    What is more complex in Toki Pona: adjectives or adverbs? & \answer{Adjectives.}                                                          \\\wordrule
    What kind of words describe nouns?                        & \answer{Adjectives.}                                                          \\\wordrule
    What is the difference between adverbs and adjectives?    & \answer{Adverbs describe verbs and adjectives describe nouns.}                \\\wordrule
    Where are adjective slots located?                        & \answer{Only after nouns and as a predicate adjective in a predicate phrase.} \\\wordrule
    Can an adjective follow a predicate noun?                 & \answer{Yes, since a predicate noun is a noun.}                               \\
}

\practice[See how well you can read the following poem.]{
    \texttp{mi jo e kili.}             & \answer{I have a fruit.}              \\
    \texttp{ona li ' pona li ' lili.}  & \answer{It is good and is small.}     \\
    \texttp{mi moku lili e kili lili.} & \answer{I nibble on the small fruit.} \\
}

\practice[Try to translate these sentences.]{
    \answer{\texttp{jan lawa li moku e telo jaki.}}     & The leader drank dirty water.                 \\\wordrule
    \answer{\texttp{mi wile e ilo moku.}}               & I need a fork.                                \\\wordrule
    \answer{\texttp{jan ike li utala e ona mute.}}      & An enemy is attacking them.                   \\\wordrule
    \answer{\texttp{jan ike ni li jo e len nasa.}}      & That bad person has strange clothes.          \\\wordrule
    \answer{\texttp{mi mute li moku e telo nasa mute.}} & We drank a lot of vodka.                      \\\wordrule
    \answer{\texttp{jan lili li lukin e jan suli.}}     & Children watch adults.                        \\\wordrule
    \texttp{mi lukin e ni.}                             & \answer{I am looking at that.}                \\\wordrule
    \texttp{mi lukin sewi e tomo suli.}                 & \answer{I am looking up at the big building.} \\\wordrule
    \texttp{seli suno li seli e tomo mi.}               & \answer{The sun's warmth heats my home.}      \\\wordrule
    \texttp{jan lili li wile e telo kili.}              & \answer{Children want fruit juice.}           \\\wordrule
    \texttp{ona mute li nasa e jan suli.}\footnotemark  & \answer{They drove the adults crazy.}         \\\wordrule
    \texttp{mi kama e pakala.}                          & \answer{I caused an accident.}                \\
}

\footnotetext{Notice how even though \texttp{nasa} is typically an adjective, it is used as a verb here.}
