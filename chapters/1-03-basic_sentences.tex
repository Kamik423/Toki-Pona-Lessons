%!TEX root = ../toki-pona.tex

\section{Basic Sentences}
\subsection*{Vocabulary}
\index{jan@\texttp{jan}}
\index{li@\texttp{li}}
\index{mi@\texttp{mi}}
\index{moku@\texttp{moku}}
\index{ona@\texttp{ona}}
\index{pona@\texttp{pona}}
\index{sina@\texttp{sina}}
\index{suno@\texttp{suno}}
\index{suli@\texttp{suli}}
\index{telo@\texttp{telo}}
\index{apostrophe}

\begin{vocabularytable}
    jan                & \wordtype{noun}: person, people, human, being, somebody, anybody                                                                                                                  \\
    \dots{} jan        & \wordtype{adjective}, \wordtype{adverb}: human, somebody's, personal, of people                                                                                                   \\
    jan (e \dots{})    & \wordtype{verb transitive}: to personify, to humanize, to personalize                                                                                                             \\
    \wordrule %%%%%%%%%%%%%%%%%%%%%%%%%%%%%%%%%%%%%%%%%%%%%%%%%%%%%%%%%%%%%%%%%%%
    \dots{} li \dots{} & \wordtype{separator}: It separates the subject phrase, except \texttp{mi} and \texttp{sina}, from the predicate phrase. Don't use \texttp{li} before or after an other separator. \\
    \wordrule %%%%%%%%%%%%%%%%%%%%%%%%%%%%%%%%%%%%%%%%%%%%%%%%%%%%%%%%%%%%%%%%%%%
    mi                 & \wordtype{personal pronoun}: I, we                                                                                                                                                \\
    \dots{} mi         & \wordtype{possessive pronoun}: my, our                                                                                                                                            \\
    \dots{} e mi       & \wordtype{reflexive pronoun}: myself, ourselves                                                                                                                                   \\
    \wordrule %%%%%%%%%%%%%%%%%%%%%%%%%%%%%%%%%%%%%%%%%%%%%%%%%%%%%%%%%%%%%%%%%%%
    moku               & \wordtype{noun}: food, meal                                                                                                                                                       \\
    \dots{} moku       & \wordtype{adjective}, \wordtype{adverb}: eating                                                                                                                                   \\
    moku (e \dots{})   & \wordtype{verb transitive}: to eat, to drink, to swallow, to ingest, to consume                                                                                                   \\
    \wordrule %%%%%%%%%%%%%%%%%%%%%%%%%%%%%%%%%%%%%%%%%%%%%%%%%%%%%%%%%%%%%%%%%%%
    ona                & \wordtype{personal pronoun}: she, he, it, they                                                                                                                                    \\
    \dots{} ona        & \wordtype{possessive pronoun}: her, his, its                                                                                                                                      \\
    \dots{} e ona      & \wordtype{reflexive pronoun}: himself, herself, itself, themselves                                                                                                                \\
    \wordrule %%%%%%%%%%%%%%%%%%%%%%%%%%%%%%%%%%%%%%%%%%%%%%%%%%%%%%%%%%%%%%%%%%%
    pona               & \wordtype{noun}: good, simplicity, positivity                                                                                                                                     \\
    \dots{} pona       & \wordtype{adjective}, \wordtype{adverb}: good, simple, positive, nice, correct, right                                                                                             \\
    pona (e \dots{})   & \wordtype{verb transitive}: to improve, to fix, to repair, to make good                                                                                                           \\
    \wordrule %%%%%%%%%%%%%%%%%%%%%%%%%%%%%%%%%%%%%%%%%%%%%%%%%%%%%%%%%%%%%%%%%%%
    sina               & \wordtype{personal pronoun}: you                                                                                                                                                  \\
    \dots{} sina       & \wordtype{possessive pronoun}: yours                                                                                                                                              \\
    \dots{} e sina     & \wordtype{reflexive pronoun}: yourself, yourselves                                                                                                                                \\
    \wordrule %%%%%%%%%%%%%%%%%%%%%%%%%%%%%%%%%%%%%%%%%%%%%%%%%%%%%%%%%%%%%%%%%%%
    suno               & \wordtype{noun}: sun, light                                                                                                                                                       \\
    \dots{} suno       & \wordtype{adjective}, \wordtype{adverb}: sunny, sunnily                                                                                                                           \\
    suno (e \dots{})   & \wordtype{verb transitive}: to light, to illumine, to illuminate                                                                                                                  \\
    \wordrule %%%%%%%%%%%%%%%%%%%%%%%%%%%%%%%%%%%%%%%%%%%%%%%%%%%%%%%%%%%%%%%%%%%
    suli               & \wordtype{noun}: size                                                                                                                                                             \\
    \dots{} suli       & \wordtype{adjective}, \wordtype{adverb}: big, tall, long, adult, important                                                                                                        \\
    suli (e \dots{})   & \wordtype{verb transitive}: to enlarge, to lengthen                                                                                                                               \\
    \wordrule %%%%%%%%%%%%%%%%%%%%%%%%%%%%%%%%%%%%%%%%%%%%%%%%%%%%%%%%%%%%%%%%%%%
    telo               & \wordtype{noun}: water, liquid, juice, sauce                                                                                                                                      \\
    \dots{} telo       & \wordtype{adjective}, \wordtype{adverb}: wet, slobbery, moist, damp, humid, sticky, sweaty, dewy, drizzly                                                                         \\
    telo (e \dots{})   & \wordtype{verb transitive}: to water, to wash with water, to put water to, to melt, to liquify                                                                                    \\
    \wordrule %%%%%%%%%%%%%%%%%%%%%%%%%%%%%%%%%%%%%%%%%%%%%%%%%%%%%%%%%%%%%%%%%%%
    '                  & \wordtype{unofficial}: An apostrophe can identify a predicate that does not contain a verb.                                                                                       \\
\end{vocabularytable}

\subsection*{The Ambiguity of Toki Pona}
\index{ambiguity}
\index{suli@\texttp{suli}}
Do you see how several of the words in the vocabulary have multiple meanings?
For example, \texttp{suli} can mean either ``long'', ``tall'', ``big'', ``important'', or ``the size''.
By now, you might be wondering, ``What's going on? How can one word mean so many different things?''

Welcome to the world of Toki Pona!
The truth is that lots of words are like this in Toki Pona.
Because the language has such a small vocabulary and is so basic, the ambiguity is inevitable.
However, this vagueness is not necessarily a bad thing. Because of the vagueness, a speaker of Toki Pona is forced to focus on the very basic, unaltered aspect of things, rather than focusing on many minute details.

\index{singular}
\index{plural}
Another way that Toki Pona is ambiguous is that it can not specify whether a word is singular or plural.
For example, \texttp{jan} can mean either ``person'' or ``people''.
If you've decided that Toki Pona is too arbitrary and that not having plurals is simply the final straw, don't be so hasty.
Toki Pona is not the only language that doesn't specify whether a noun is plural or not.
Japanese, for example, does the same thing.

\index{Tense}
Toki Pona has no Tenses.
The verbs don't change.
If it's absolutely necessary, there are ways of saying that something happened in the past, present, or future.

As you can see in the vocabulary list, most words can be used in different word types.
They remain unchanged.
The word type is derived from the position in the sentence.
In this lesson, we will deal with nouns, pronouns, verbs, adjectives and a special separator.

\index{noun}
\index{adjective}
\index{verb}
A noun is a word for a person, place or thing.
An adjective is a word that describes a noun.
A verb describes an action.

\index{pronoun}
\index{pronoun!personal}
\index{pronoun!possessive}
Pronouns are proxies for different types of words.
They are used in the same place as the word to be represented and have the same grammatical characteristics as this one.
Pronouns are not words of content, but they denote persons or things by referring to the context.
Personal pronouns (I, you, \dots) represent nouns.
Possessive pronouns (my, your, \dots) represent adjectives.
In the next few lessons we will learn more about other types of pronouns.

\subsection*{The Personal Pronouns \texttp{mi} or \texttp{sina} as Subject}
\index{subject}
\index{sentence!declarative}
\index{nested subordinate clauses}
\index{comma}
\index{mi@\texttp{mi}!personal pronoun}
\index{sina@\texttp{sina}!personal pronoun}
\index{pona\texttp{pona}}
\index{moku\texttp{moku}}
With the personal pronoun \texttp{mi} or the personal pronoun \texttp{sina} at the beginning and a subsequent verb a simple sentence in Toki Pona is already complete.
A declarative sentence ends with a full stop.
Toki Pona has no nested subordinate clauses and nearly no commas.

\begin{translationtable}
    mi moku.   & I eat.   \\
    sina pona. & You fix. \\
\end{translationtable}
%
\index{subject phrase}
In these sentences personal pronouns \texttp{mi} and \texttp{sina} are in each case the subject phrase.
In Toki Pona, a subject phrase is always at the beginning of the sentence.
In these examples, the subject phrases consist of only one subject (\texttp{mi} or \texttp{sina}).

The subject is the carrier of the action, process, or state.
It is the most important addition to the verb in the sentence, a complete sentence always contains a subject.
You ask for the subject with whom or what.

\subsection*{Verbs as Predicates}
\index{predicate phrase}
\index{predicate}
\index{predicate!not a verb}
\index{sentence!statement}
\index{verb}
\index{verb vs\@. predicate}

The verbs \texttp{moku} and \texttp{pona} form the predicate phrase in these examples.
The predicate is a core element in a sentence and is the statement of the sentence.
No statement sentence is possible without a predicate.

In most languages, a predicate is formed by a verb, but this is not mandatory in all languages.
As we will soon see, in Toki Pona the predicate is not necessarily formed by a verb.
The difference between verb and predicate is that verb designates a word part and predicate designates a grammatical function.
A predicate and possible objects form a predicate phrase.

\subsection*{Nouns or Adjectives as Predicates}
\index{apostrophe}
\index{predicate adjectiv}
\index{predicate noun}
\index{adjective!predicate}
\index{noun!predicate}
\index{to be}
\index{be}
\index{slot}
\index{mi@\texttp{mi}!personal pronoun}
\index{sina@\texttp{sina}!personal pronoun}
\index{pona@\texttp{pona}}
\index{moku@\texttp{moku}}
\index{no-copula language}
\index{copula}

One of the first principles you'll need to learn about Toki Pona is that there is no form of the static verb ``to be'' like there is in English.
That's why the verb slot can be empty and after \texttp{mi} or \texttp{sina} can follow also a noun or adjective.
In these lessons, the term ``slot'' is used to indicate a valid position of a word type in the sentence.

Regular sentences can also be formed in other languages without a verb appearing in them.
Examples are Russian and Arabic.
These languages are called no-copula languages.

A copula is a word that connects the subject and predicate (``copulates'').
If a ``normal'' verb is the predicate, one does not need an additional copula.
It occurs only if a noun, pronoun or adjective is the predicate.
In English the verb ``to be'' serves as the copula.
A no-copula language, like Toki Pona, does not require a copula.

A noun then functions as a predicate noun or an adjective serves as predicate adjective.
But this noun or adjective does not become a verb.
An empty verb slot cannot, however, form a predicate phrase on its own.
A noun or adjective must follow.
That is, directly after \texttp{mi} or \texttp{sina} the sentence cannot be finished yet.

In no-copula languages, the word form usually indicates whether the predicate is a verb, noun or adjective.
This is not possible in Toki Pona.
In these lessons an apostrophe is used to indicate a subsequent noun or adjective.
But that's not an official rule.

\begin{translationtable}
    mi moku.     & I eat.        \\
    mi ' moku.   & I am food.    \\
    sina pona.   & You fix.      \\
    sina ' pona. & You are good. \\
\end{translationtable}
%
Because Toki Pona lacks ``to be'', the exact meaning is lost.
\texttp{moku} in this sentence could be a verb, or it could be a noun; just as \texttp{pona} could be an adjective or could be a verb.
In situations such as these, the listener must rely on context.
After all, how often do you hear someone say ``I am food.''?
I hope not very often! You can be fairly certain that \texttp{mi moku} means ``I'm eating''.

\subsection*{The Separator \texttp{li} }
\index{separator!li}
\index{predicate marker}
\index{li@\texttp{li}}
\index{ona@\texttp{ona}!personal pronoun}
For sentences that don't use the personal pronouns \texttp{mi} or \texttp{sina} as the subject, there is one small catch that you'll have to learn.
Look at how \texttp{li} is used.
\texttp{li} is a grammatical word that separates the subject phrase from the predicate phrase.
The predicate marker \texttp{li} is only used when the subject is not \texttp{mi} or \texttp{sina}.
Although the separator \texttp{li} might seem worthless right now, as you continue to learn Toki Pona you will see that some sentences could be very confusing if \texttp{li} weren't there.

\begin{translationtable}
    telo li pona.   & Water is cleaning.  \\
    suno li suno.   & The sun is shining. \\
    moku li ' pona. & The food is good.   \\
    ona li ' moku.  & It is food.         \\
\end{translationtable}
%
Is the verb slot empty, after \texttp{li} can follow a noun or adjective as well.
As already written, an empty verb slot cannot form a predicate phrase on its own.
A noun or adjective must follow.
That is, directly after \texttp{li} the sentence can not yet be finished or an object can follow.

\practice[Which word types are the bold words?]{
    \texttp{\textbf{mi} moku.}        & \showanswer{personal pronoun} \\\wordrule
    \texttp{\textbf{sina} pona.}      & \answer{personal pronoun}     \\\wordrule
    \texttp{\textbf{moku} li ' pona.} & \answer{noun}                 \\\wordrule
    \texttp{\textbf{ona} li ' moku.}  & \answer{personal pronoun}     \\\wordrule
    \texttp{moku \textbf{li} ' pona.} & \answer{separator}            \\
}

\newpage

\practice{
    What is a verb                                                       & \answer{A verb describes an action.}                                                \\\wordrule
    What is a noun?                                                      & \answer{A noun is a word for a person, place or thing.}                             \\\wordrule
    What is \texttp{li} used for?                                        & \answer{It separates the subject phrase from the predicate phrase.}                 \\\wordrule
    What does a personal pronoun replace?                                & \answer{It replaces a noun.}                                                        \\\wordrule
    How to recognize nouns, pronouns, verbs and adjectives in Toki Pona? & \answer{At their position in the sentence.}                                         \\\wordrule
    What is a subject?                                                   & \answer{The subject is the carrier of the action, process, or state.}                \\\wordrule
    After which subject phrases is \texttp{li} not used?                 & \answer{It is only used if the subject phrase is not \texttp{mi} or \texttp{sina}.} \\\wordrule
    Where does the subject stand in the sentence?                        & \answer{In Toki Pona it is always at the beginning of the sentence.}                \\\wordrule
    Can an empty verb slot alone form a predicate?                       & \answer{No!}                                                                        \\\wordrule
    When can a verb slot be empty?                                       & \answer{If the predicate is formed by a noun or adjective.}                         \\\wordrule
    What is a predicate?                                                 & \answer{It is a core element in a sentence and the statement of the sentence.}      \\\wordrule
    A complete sentence in Toki Pona always contains\dots                & \answer{\dots{} a subject and a predicate phrase.}                                          \\\wordrule
    What kinds of words can be used in Toki Pona to form a predicate?    & \answer{Verbs, nouns, or adjectives.}                                                \\\wordrule
    What is an adjective?                                                & \answer{An adjective is a word that describes a noun.}                              \\\wordrule
    Where are possible adjective slots?                                  & \answer{After a noun, after a pronoun, and after \texttp{li}.}                \\\wordrule
    Why can't a sentence be ended after \texttp{li}?                     & \answer{Because then the predicate is missing.}                                     \\
}

\practice[Try to translate these sentences. You can use the tool \textit{Toki Pona Parser} (\cite{www:rowa:02}) for spelling and grammar check.]{
    \answer{\texttp{jan li ' pona.}}   & People are good.                   \\\wordrule
    \answer{\texttp{mi moku.}}         & I'm eating.                        \\\wordrule
    \answer{\texttp{sina ' suli.}}     & You're tall.                       \\\wordrule
    \answer{\texttp{telo li ' pona.}}  & Water is simple.                   \\\wordrule
    \answer{\texttp{telo li ' suli.}}  & The lake is big.                   \\\wordrule
    \texttp{suno li ' suli.}           & \answer{The sun is big.}           \\\wordrule
    \texttp{mi ' suli.}                & \answer{I'm important. / I'm fat.} \\\wordrule
    \texttp{jan li moku.}              & \answer{Somebody is eating.}       \\
}