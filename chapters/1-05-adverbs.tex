%!TEX root = ../toki-pona-lessons.tex

\section{Verbs, Adverbs, Auxiliary Verbs}
\subsection*{Vocabulary}

\begin{supertabular}{p{2,5cm}|ll}
    \index{ike}
    \textbf{\dots ike}       &  & \textit{adjective}: bad, negative, wrong, evil, overly complex                 \\
    \textbf{\dots ike}       &  & \textit{adverb}: bad, negative, wrong, evil, overly complex                    \\
    \textbf{ike}             &  & \textit{noun}: negativity, badness, evil                                       \\
    \textbf{ike}             &  & \textit{verb intransitive}: to be bad, to suck                                 \\
    \textbf{ike (e \dots)}   &  & \textit{verb transitive}: to make bad, to worsen                               \\
                             &  &                                                                                \\
    \index{jaki}
    \textbf{\dots jaki}      &  & \textit{adjective}: dirty, gross, filthy, obscene                              \\
    \textbf{\dots jaki}      &  & \textit{adverb}: dirty, gross, filthy                                          \\
    \textbf{jaki}            &  & \textit{noun}: dirt, pollution, garbage, filth, feces                          \\
    \textbf{jaki (e \dots)}  &  & \textit{verb transitive}: to pollute, to dirty                                 \\
                             &  &                                                                                \\
    \index{lawa}
    \textbf{\dots lawa}      &  & \textit{adjective}: main, leading, in charge                                   \\
    \textbf{\dots lawa}      &  & \textit{adverb}: main, leading, in charge                                      \\
    \textbf{lawa}            &  & \textit{noun}: head, mind                                                      \\
    \textbf{lawa (e \dots)}  &  & \textit{verb transitive}: to lead, to control, to rule, to steer               \\
                             &  &                                                                                \\
    \index{lili}
    \textbf{\dots lili}      &  & \textit{adjective}: small, little, young, a bit, short, few, less              \\
    \textbf{\dots lili}      &  & \textit{adverb}: small, little, young, a bit, short, few, less                 \\
    \textbf{lili}            &  & \textit{noun}: smallness, youth, immaturity                                    \\
    \textbf{lili (e \dots)}  &  & \textit{verb transitive}: to reduce, to shorten, to shrink, to lessen          \\
                             &  &                                                                                \\
    \index{mute}
    \textbf{\dots mute}      &  & \textit{adjective}: many, very, much, several, a lot, abundant, numerous, more \\
    \textbf{\dots mute}      &  & \textit{adverb}: many, very, much, several, a lot, abundant, numerous, more    \\
    \textbf{mute}            &  & \textit{noun}: amount, quantity                                                \\
    \textbf{mute (e \dots)}  &  & \textit{verb transitive}: to make many or much                                 \\
                             &  &                                                                                \\
    \index{sewi}
    \textbf{\dots sewi}      &  & \textit{adjective}: superior, elevated, religious, formal                      \\
    \textbf{\dots sewi}      &  & \textit{adverb}: superior, elevated, religious, formal                         \\
    \textbf{sewi}            &  & \textit{noun}: high, up, above, top, over, on                                  \\
    \textbf{sewi}            &  & \textit{verb intransitive}: to get up                                          \\
    \textbf{sewi (e \dots)}  &  & \textit{verb transitive}: to lift                                              \\
                             &  &                                                                                \\
    \index{tomo}
    \textbf{\dots tomo}      &  & \textit{adjective}: urban, domestic, household                                 \\
    \textbf{\dots tomo}      &  & \textit{adverb}: urban, domestic, household                                    \\
    \textbf{tomo}            &  & \textit{noun}: indoor constructed space, e.g. house, home, room, building      \\
    \textbf{tomo (e \dots)}  &  & \textit{verb transitive}: to build, to construct, to engineer                  \\
                             &  &                                                                                \\
    \index{utala}
    \textbf{\dots utala}     &  & \textit{adjective}: fighting                                                   \\
    \textbf{\dots utala}     &  & \textit{adverb}: fighting                                                      \\
    \textbf{utala}           &  & \textit{noun}: conflict, disharmony, fight, war, battle, attack, violence      \\
    \textbf{utala (e \dots)} &  & \textit{verb transitive}: to hit, to strike, to attack, to compete against     \\
\end{supertabular}

\newpage

\subsection*{Adverbs}
\index{adverb}
\index{predicate phrase}

Adverbs refer to the circumstances in which an action takes place.
Since actions are described by verbs, adverbs describe verbs.
For example, in the phrase 'You sing well.' the verb 'singing' is described in more detail with the adverb 'well'.

In Toki Pona adverbs follow the verb they describe.
Possible adverb slots are therefore only available after verbs.
Adverbs cannot therefore stand after nouns, adjectives, prepositions or separators.

Since verbs belong to the predicate phrase, adverbs also belong to the predicate phrase.
In \textit{toki pona} a predicate phrase can contain a noun as predicate noun or an adjective as predicate adjective.
In this case the verb slot is empty, so there are no adverb slots in such a predicate phrase.

In this sentence the transitive verb \textit{lawa} with adverb \textit{pona} is described.

\begin{supertabular}{p{5,5cm}|ll}
    mi lawa pona e jan. &  & I lead people well. \\
\end{supertabular}

In the following sentences adverbs describe \textit{ike}, \textit{sewi}, \textit{mute}, \textit{lili} the respective verbs \textit{utala}, \textit{lukin}, \textit{wile}, \textit{lukin}.

\begin{supertabular}{p{5,5cm}|ll}
    mi utala ike.           &  & I fight badly.          \\
    sina lukin sewi e suno. &  & You look up at the sun. \\
    ona li wile mute e ni.  &  & He wants that a lot.    \\
    mi lukin lili e ona.    &  & I barely saw it.        \\
\end{supertabular}

You should not use more than three adverbs after a verb.
An adverb should not be used more than once.

\begin{supertabular}{p{5,5cm}|ll}
    ona li pona ike mute e ilo. &  & He was very bad at fixing the machine.          \\
    mi mute lukin mute e ma.    &  & I'm visibly increasing the size of the country. \\
\end{supertabular}

\subsection*{Auxiliary Verbs}
\index{auxiliary verb}
\index{verb!auxiliary}
\index{\textit{wile}!auxiliary verb}
\index{\textit{kama}!auxiliary verb}
\index{\textit{jo}!verb}
An auxiliary verb is placed in front of the main verb and supplements it.
An auxiliary verb belongs to the predicate phrase.

To say that you want to do something definite, use the auxiliary verb \textit{wile}.

\begin{supertabular}{p{5,5cm}|ll}
    mi wile lukin e ma.           &  & I want to see the countryside.  \\
    mi wile pakala e sina.        &  & I must destroy you.             \\
    ona li wile jo e ilo.         &  & He would like to have a tool.   \\
    sina kama e ni: mi wile moku. &  & You caused this: I want to eat. \\ && You made me hungry. \\
\end{supertabular}

Very often the auxiliary verb \textit{kama} is used together with the main verb \textit{jo}.

\begin{supertabular}{p{5,5cm}|ll}
    kama jo            &  & get              \\
    mi kama jo e telo. &  & I got the water. \\
\end{supertabular}

\newpage
\subsection*{Practice (Answers: Page~\pageref{'adverbs'})}

Please write down your answers and check them afterwards.

\begin{supertabular}{p{5,5cm}|ll}
    What are adverbs?                                           &  & \\
    Can an adverb be ranked according to a predicate noun?      &  & \\
    Where are slots for adverbs located?                        &  & \\
    What kind of words describes an action?                     &  & \\
    When does a predicate phrase contain slots for adverbs?     &  & \\
    What is an auxiliary verb used for?                         &  & \\
    Which phrase in the sentence can contain an auxiliary verb? &  & \\
\end{supertabular}

Which word types can represent the respective word in the sentence after the hyphen?
Example:

\begin{supertabular}{p{5,5cm}|ll}
    mi - mi moku. &  & personal pronoun \\
\end{supertabular}

\begin{supertabular}{p{5,5cm}|ll}
    kama - mi kama jo e telo.  &  & \\
    wile - mi wile lukin e ma. &  & \\
    ike - mi lawa ike e jan.   &  & \\
    jan - mi ' jan.            &  & \\
\end{supertabular}

Try to translate these sentences.
You can use the tool \textit{Toki Pona Parser} (\cite{www:rowa:02}) for spelling and grammar check.

\begin{supertabular}{p{5,5cm}|ll}
    jan li pona ilo e ilo.     &  & \\ % & English - Toki Pona
    sina lukin unpa mute e mi. &  & \\ % & English - Toki Pona
    jaki li jaki lili e mi.    &  & \\ % & English - Toki Pona
    sina len nasa jaki e sina. &  & \\ % & English - Toki Pona
    ilo li sewi e sewi.        &  & \\ % & English - Toki Pona
    ona li lawa utala e utala. &  & \\ % & English - Toki Pona
    mi wile unpa e ona.        &  & \\
    jan li wile jo e ma.       &  & \\
\end{supertabular}

\begin{supertabular}{p{5,5cm}|ll}
    She increases the property very badly. &  & \\ % & English - Toki Pona
    I want to have a lot of sex with you.  &  & \\ % & English - Toki Pona
    She was barely dressed.                &  & \\ % & English - Toki Pona
    The sun shines warmly on the land.     &  & \\ % & English - Toki Pona
    She's good.                            &  & \\ % & English - Toki Pona
    He wants to destroy the tool.          &  & \\
    She is thirsty.                        &  & \\
\end{supertabular}
