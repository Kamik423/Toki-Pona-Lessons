%!TEX root = ../toki-pona.tex

\section{Verbs, Adverbs, Auxiliary Verbs}
\subsection*{Vocabulary}
\index{ike@\texttp{ike}}
\index{jaki@\texttp{jaki}}
\index{kama@\texttp{kama}}
\index{lawa@\texttp{lawa}}
\index{len@\texttp{len}}
\index{lili@\texttp{lili}}
\index{mute@\texttp{mute}}
\index{ni@\texttp{ni}}
\index{sewi@\texttp{sewi}}
\index{tomo@\texttp{tomo}}
\index{utala@\texttp{utala}}

\begin{vocabularytable}
    \texttp{ike}                 & \wordtype{noun}: negativity, badness, evil                                                          \\
    \texttp{\dots{} ike}         & \wordtype{adjective}, \wordtype{adverb}: bad, negative, wrong, evil, overly complex                 \\
    \texttp{ike}                 & \wordtype{verb intransitive}: to be bad, to suck                                                    \\
    \texttp{ike (e \dots{})}     & \wordtype{verb transitive}: to make bad, to worsen                                                  \\
    \wordrule %%%%%%%%%%%%%%%%%%%%%%%%%%%%%%%%%%%%%%%%%%%%%%%%%%%%%%%%%%%%%
    \texttp{jaki}                & \wordtype{noun}: dirt, pollution, garbage, filth, feces                                             \\
    \texttp{\dots{} jaki}        & \wordtype{adjective}, \wordtype{adverb}: dirty, gross, filthy, obscene                              \\
    \texttp{jaki (e \dots{})}    & \wordtype{verb transitive}: to pollute, to dirty                                                    \\
    \wordrule %%%%%%%%%%%%%%%%%%%%%%%%%%%%%%%%%%%%%%%%%%%%%%%%%%%%%%%%%%%%%
    \texttp{kama}                & \wordtype{noun}: event, happening, chance, arrival, beginning                                       \\
    \texttp{\dots{} kama}        & \wordtype{adjective}, \wordtype{adverb}: coming, future                                             \\
    \texttp{kama}                & \wordtype{verb intransitive}: to come, to become, to arrive, to happen                              \\
    \texttp{kama \dots{}}        & \wordtype{auxiliary verb}: to become, to mange to                                                   \\
    \texttp{kama (e \dots{})}    & \wordtype{verb transitive}: to bring about, to summon                                               \\
    \texttp{kama jo (e \dots{})} & \wordtype{verb transitive}: to get                                                                  \\
    \wordrule %%%%%%%%%%%%%%%%%%%%%%%%%%%%%%%%%%%%%%%%%%%%%%%%%%%%%%%%%%%%%
    \texttp{lawa}                & \wordtype{noun}: head, mind                                                                         \\
    \texttp{\dots{} lawa}        & \wordtype{adjective}, \wordtype{adverb}: main, leading, in charge                                   \\
    \texttp{lawa (e \dots{})}    & \wordtype{verb transitive}: to lead, to control, to rule, to steer                                  \\
    \wordrule %%%%%%%%%%%%%%%%%%%%%%%%%%%%%%%%%%%%%%%%%%%%%%%%%%%%%%%%%%%%%
    \texttp{len}                 & \wordtype{noun}: clothing, cloth, fabric, network, internet                                         \\
    \texttp{\dots{} len}         & \wordtype{adjective}: dressed, clothed, costumed, dressed up                                        \\
    \texttp{len (e \dots{})}     & \wordtype{verb transitive}: to wear, to be dressed, to dress                                        \\                                                            &                                                            \\
    \wordrule %%%%%%%%%%%%%%%%%%%%%%%%%%%%%%%%%%%%%%%%%%%%%%%%%%%%%%%%%%%%%
    \texttp{lili}                & \wordtype{noun}: smallness, youth, immaturity                                                       \\
    \texttp{\dots{} lili}        & \wordtype{adjective}, \wordtype{adverb}: small, little, young, a bit, short, few, less              \\
    \texttp{lili (e \dots{})}    & \wordtype{verb transitive}: to reduce, to shorten, to shrink, to lessen                             \\
    \wordrule %%%%%%%%%%%%%%%%%%%%%%%%%%%%%%%%%%%%%%%%%%%%%%%%%%%%%%%%%%%%%
    \texttp{mute}                & \wordtype{noun}: amount, quantity                                                                   \\
    \texttp{\dots{} mute}        & \wordtype{adjective}, \wordtype{adverb}: many, very, much, several, a lot, abundant, numerous, more \\
    \texttp{mute (e \dots{})}    & \wordtype{verb transitive}: to make many or much                                                    \\
    \wordrule %%%%%%%%%%%%%%%%%%%%%%%%%%%%%%%%%%%%%%%%%%%%%%%%%%%%%%%%%%%%%
    \texttp{\dots{} ni}          & \wordtype{adjective demonstrative pronoun}: this, that                                              \\
    \texttp{ni}                  & \wordtype{noun demonstrative pronoun}: this, that                                                   \\
    \wordrule %%%%%%%%%%%%%%%%%%%%%%%%%%%%%%%%%%%%%%%%%%%%%%%%%%%%%%%%%%%%%
    \texttp{sewi}                & \wordtype{noun}: high, up, above, top, over, on                                                     \\
    \texttp{\dots{} sewi}        & \wordtype{adjective}, \wordtype{adverb}: superior, elevated, religious, formal                      \\
    \texttp{sewi}                & \wordtype{verb intransitive}: to get up                                                             \\
    \texttp{sewi (e \dots{})}    & \wordtype{verb transitive}: to lift                                                                 \\
    \wordrule %%%%%%%%%%%%%%%%%%%%%%%%%%%%%%%%%%%%%%%%%%%%%%%%%%%%%%%%%%%%%
    \texttp{tomo}                & \wordtype{noun}: indoor constructed space, e.g. house, home, room, building                         \\
    \texttp{\dots{} tomo}        & \wordtype{adjective}, \wordtype{adverb}: urban, domestic, household                                 \\
    \texttp{tomo (e \dots{})}    & \wordtype{verb transitive}: to build, to construct, to engineer                                     \\
    \wordrule %%%%%%%%%%%%%%%%%%%%%%%%%%%%%%%%%%%%%%%%%%%%%%%%%%%%%%%%%%%%%
    \texttp{utala}               & \wordtype{noun}: conflict, disharmony, fight, war, battle, attack, violence                         \\
    \texttp{\dots{} utala}       & \wordtype{adjective}, \wordtype{adverb}: fighting                                                   \\
    \texttp{utala (e \dots{})}   & \wordtype{verb transitive}: to hit, to strike, to attack, to compete against                        \\
\end{vocabularytable}

\newpage

\subsection*{Adverbs}
\index{adverb}
\index{predicate phrase}

Adverbs refer to the circumstances in which an action takes place.
Since actions are described by verbs, adverbs describe verbs.
For example, in the phrase ``You sing well.'' the verb ``singing'' is described in more detail with the adverb ``well''.

In Toki Pona adverbs follow the verb they describe.
Possible adverb slots are therefore only available after verbs.
Adverbs therefore cannot follow nouns, adjectives, prepositions, or separators.

Since verbs belong to the predicate phrase, adverbs also belong to the predicate phrase.
In Toki Pona a predicate phrase can contain a noun as predicate noun or an adjective as predicate adjective.
In this case the verb slot is empty, so these predicate phrases don't contain an adverb slot.

In this sentence the transitive verb \texttp{lawa} with adverb \texttp{pona} is described.

\begin{translationtable}
    \texttp{mi lawa pona e jan.} & I lead people well. \\
\end{translationtable}
%
In the following sentences the adverbs \texttp{ike}, \texttp{sewi}, \texttp{mute}, and \texttp{lili} describe the respective verbs \texttp{utala}, \texttp{lukin}, \texttp{wile}, and \texttp{lukin}.

\begin{translationtable}
    \texttp{mi utala ike.}           & I fight badly.          \\
    \texttp{sina lukin sewi e suno.} & You look up at the sun. \\
    \texttp{ona li wile mute e ni.}  & He wants that a lot.    \\
    \texttp{mi lukin lili e ona.}    & I barely saw it.        \\
\end{translationtable}
%
You should not use more than three adverbs after a verb.
An adverb should not be used more than once.

\begin{translationtable}
    \texttp{ona li pona ike mute e ilo.} & He was very bad at fixing the machine.          \\
    \texttp{mi mute lukin mute e ma.}    & I'm visibly increasing the size of the country. \\
\end{translationtable}
%
\subsection*{Auxiliary Verbs}
\index{auxiliary verb}
\index{verb!auxiliary}
\index{wile@\texttp{wile}!auxiliary verb}
\index{kama@\texttp{kama}!auxiliary verb}
\index{jo@\texttp{jo}!verb}
An auxiliary verb is placed in front of the main verb and supplements it.
An auxiliary verb belongs to the predicate phrase.

To say that you want to do something definite, use the auxiliary verb \texttp{wile}.

\begin{translationtable}
    \texttp{mi wile lukin e ma.}           & I want to see the countryside.  \\
    \texttp{mi wile pakala e sina.}        & I must destroy you.             \\
    \texttp{ona li wile jo e ilo.}         & He would like to have a tool.   \\
    \texttp{sina kama e ni: mi wile moku.} & You caused this: I want to eat. \\
                                           & You made me hungry.             \\
\end{translationtable}
%
Very often the auxiliary verb \texttp{kama} is used together with the main verb \texttp{jo}.

\begin{translationtable}
    \texttp{kama jo}            & get              \\
    \texttp{mi kama jo e telo.} & I got the water. \\
\end{translationtable}

\newpage

\practice{
    What are adverbs?                                           & \answer{Adverbs describe an action (verb).}                 \\\wordrule
    Can an adverb be ranked according to a predicate noun?      & \answer{No, this is not possible.}                          \\\wordrule
    Where are slots for adverbs located?                        & \answer{Only after verbs.}                                  \\\wordrule
    What kind of words describes an action?                     & \answer{Verbs.}                                             \\\wordrule
    When does a predicate phrase contain slots for adverbs?     & \answer{If the predicate phrase contains a verb.}           \\\wordrule
    What is an auxiliary verb used for?                         & \answer{It complements the main verb.}                      \\\wordrule
    Which phrase in the sentence can contain an auxiliary verb? & \answer{An auxiliary verb belongs to the predicate phrase.} \\
}

\practice[Which word types are the bold words?]{
    \texttp{\textbf{mi} moku.}            & \showanswer{personal pronoun}            \\\wordrule
    \texttp{mi \textbf{kama} jo e telo.}  & \answer{auxiliary verb}                  \\\wordrule
    \texttp{mi \textbf{wile} lukin e ma.} & \answer{auxiliary verb, transitive verb} \\\wordrule
    \texttp{mi lawa \textbf{ike} e jan.}  & \answer{adverb}                          \\\wordrule
    \texttp{mi ' \textbf{jan}.}           & \answer{adjective, noun}                 \\
}

\practice[Try to translate these sentences.]{
    \texttp{jan li pona ilo e ilo.}              & \answer{The guy usefully improves the tool.} \\\wordrule
    \texttp{sina lukin unpa mute e mi.}          & \answer{You're looking at me very sexily.}   \\\wordrule
    \texttp{jaki li jaki lili e mi.}             & \answer{The garbage disgusts me slightly.}   \\\wordrule
    \texttp{sina len nasa jaki e sina.}          & \answer{You dress disgustingly silly.}       \\\wordrule
    \texttp{ilo li sewi e sewi.}                 & \answer{The machine raises up the roof.}     \\\wordrule
    \texttp{ona li lawa utala e utala.}          & \answer{He leads the battle fightingly.}     \\\wordrule
    \texttp{mi wile unpa e ona.}                 & \answer{I want to have sex with him/her.}    \\\wordrule
    \texttp{jan li wile jo e ma.}                & \answer{People want to own land.}            \\\wordrule
    \answer{\texttp{ona li mute ike mute e jo.}} & She increases the property very badly.       \\\wordrule
    \answer{\texttp{mi wile unpa mute e sina.}}  & I want to have a lot of sex with you.        \\\wordrule
    \answer{\texttp{ona li len lili e ona.}}     & She was barely dressed.                      \\\wordrule
    % \answer{\texttp{suno li suno seli e ma.}}    & The sun shines warmly on the land.           \\\wordrule % removed because ``seli'' is only in the next chapter
    \answer{\texttp{ona li ' pona.}}             & She's good.                                  \\\wordrule
    \answer{\texttp{ona li wile pakala e ilo.}}  & He wants to destroy the tool.                \\\wordrule
    \answer{\texttp{ona li wile moku e telo.}}   & She is thirsty.                              \\
}
