%!TEX root = ../toki-pona-lessons.tex

\section{Introduction}

Sonja Lang created the language Toki Pona in the year 2001.
Her aim was minimalism.
Toki Pona consists of only about 120 words, which are not altered.
In accordance with the position in the sentence, the words can vary their significance.
To describe more detail you have to combine words.

It is not the goal of Toki Pona to describe complex issues.
Dissertations and scientific papers will never written in Toki Pona.
Lawyers, bureaucrats, theologians and politicians are warned of the side-effect of this language.

It is not the aim of Toki Pona to solve the communication problems in the world.
But you can learn this language in a month.
Toki Pona is easy in an intelligent way and yoga for the brain.
People who hate nested subordinate clauses and commas will certainly have fun with Toki Pona.

Maybe only one natural language can be compared to Toki Pona.
It is the language of the Pirah\'{a} (\cite{www:piraha:01}).
For example this language has no recursion.

Toki Pona has evolved since 2001.
Therefore these lessons are based on the tutorials from BJ Knight (jan Pije) \cite{www:Pije:01} (2003) and the official Toki Pona book \cite{www:tokipona.org} by Sonja Lang (2014).
But I tried not to take over mistakes and inaccuracies.
In my lessons, great importance is attached to the presentation of grammatical rules.
This avoids misunderstandings due to incorrect grammar.

So have fun with the lessons and learning of Toki Pona.
Memrise helps for learning vocabulary \cite{www:memrise:01}. Links related to Toki Pona can be found on the website \cite{www:rowa:01}.
A dictionary can be found here \cite{www:rowa:01}.

You can use the tool \textit{Toki Pona Parser} (\cite{www:rowa:02}) for spelling, grammar check and ambiguity check of Toki Pona sentences

\texttp{toki pona li ' pona, tawa sina.}