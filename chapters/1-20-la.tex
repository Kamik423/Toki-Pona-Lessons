%!TEX root = ../toki-pona.tex

\section{Conditional Sentences}
\subsection*{Vocabulary}

\begin{supertabular}{p{2,5cm}|ll}
    \index{ante!la}
    \textbf{ante la \dots}    &  & \textit{noun}: if difference, if variance, if disagreement                         \\
                              &  &                                                                                    \\
    \index{ike!la}
    \textbf{ike la \dots}     &  & \textit{noun}: if negativity, if badness, if evil                                  \\
                              &  &                                                                                    \\
    \index{ken!la}
    \textbf{ken la \dots}     &  & \textit{noun}: if possibility, if ability, if permission                           \\
                              &  &                                                                                    \\
    \index{kin!la}
    \textbf{kin la \dots}     &  & \textit{noun}: if reality, if fact                                                 \\
                              &  &                                                                                    \\
    \index{kipisi}
    \textbf{kipisi}           &  & \textit{noun} (unofficial): section, fragment, slice                               \\
    \textbf{kipisi (e \dots)} &  & \textit{verb transitive} (unofficial): to cut                                      \\
                              &  &                                                                                    \\
    \index{la}
    \textbf{\dots la \dots}   &  & \textit{separator}: A 'la' is between a conditional phrases and the main sentence. \\ &&  Don't use "la" before or after \\ && the other separators "e", "li", "pi", ".", "!", "?", ":", ",".  \\
                              &  &                                                                                    \\
    \index{pona!la}
    \textbf{pona la \dots}    &  & \textit{noun}: if good, if simplicity, if positivity                               \\
                              &  &                                                                                    \\
    \index{sama!la}
    \textbf{sama la \dots}    &  & \textit{noun}: in case of equality, if parity, on identity                         \\
                              &  &                                                                                    \\
\end{supertabular}

\newpage
\subsection*{Conditional Phrases}
\index{\textit{la}}
With the help of the separator \textit{la} a conditional sentence is formed.
In front of the separator \textit{la} there is the conditional phrase.
This is the condition.
In the English language, a condition is formed using the word' if'.
In Toki Pona the separator \textit{la} serves for this purpose.
After \textit{la} a complete main sentence begins.

\subsubsection*{Conditional phrases with a noun or pronoun}
A conditional phrase can have different structures.
In the simplest case, a conditional phrase consists of a single word.
This single word can only be a noun or pronoun.
So if there is only one word slot before \textit{la} it can only be filled with a noun or pronoun.

\begin{supertabular}{p{5,5cm}|ll}
    ilo li ' pakala.        &  & The tool is broken.       \\
    ken la ilo li ' pakala. &  & Maybe the tool is broken. \\
\end{supertabular}

The noun \textit{ken} means 'possibility'.
\textit{ken la} therefore means 'If there is a possibility' or better 'Maybe'.

\begin{supertabular}{p{5,5cm}|ll}
    ken la jan Lisa li jo e ona. &  & Maybe Lisa has it.        \\
    ken la ona li lape.          &  & Maybe he's alseep.        \\
    ken la mi ken tawa ma Elopa. &  & Maybe I can go to Europe. \\
\end{supertabular}

Here are further examples, each with one noun as a conditional phrase.

\begin{supertabular}{p{5,5cm}|ll}
    sama la sina en mi li utala ala. &  & We don't fight on parity.                \\
    ante la ni li ' ike.             &  & In case of deviations it is unfavorable. \\
    ike la sina moku e ni.           &  & In case of nausea swallow this.          \\
    pona la sina jo e mani.          &  & Fortunately, you have money.             \\
    tenpo la mi pali e ni.           &  & If there's time, I'll do it.             \\

\end{supertabular}

In this example, the conditional phrase consists of a conjunction and a pronoun.

\begin{supertabular}{p{5,5cm}|ll}
    taso ni la mi pilin pona. &  & But when that happens, I feel good. \\
\end{supertabular}

\subsubsection*{Composite Noun or Pronouns as Conditional Phrases}
A conditional phrase can be also a composite noun or Pronoun.
That is, the noun or pronoun followed by one or more adjectives or \textit{pi} phrases.
Optionally, a conjunct (\textit{anu}, \textit{en}, \textit{taso}) can be used before the noun or pronoun.

Typical examples of this are time specifications.
Time specifications as a conditional phrase define the time in which the statement of the main record takes place.
Literally translated, it would mean something like this: 'If time... is, then happens...'.

\begin{supertabular}{p{5,5cm}|ll}
    tenpo pini la mi ' weka.                &  & In the past, I was away.         \\
    tenpo ni la mi lon.                     &  & At this time, I am here.         \\
    tenpo kama la mi lape.                  &  & In the future, I'll sleep.       \\
    taso tenpo pimeja pini la mi kama nasa. &  & But, Last night, I became drunk. \\
\end{supertabular}

With a question pronoun \textit{seme} in a conditional-phrase it is possible to ask for age.

\begin{supertabular}{p{5,5cm}|ll}
    tenpo pi mute seme la sina sike e suno? &  & How old are you? \\
\end{supertabular}

Birthdays come once a year, and each time you have a birthday, you have gone around the sun one complete time.
To answer and tell someone how old you are, just replace the \textit{pi mute seme} with your age.

\begin{supertabular}{p{5,5cm}|ll}
    tenpo tu tu la mi sike e suno. &  & Four times (\textit{la}) I circled the sun. \\
\end{supertabular}

Here are further examples of compound nouns or pronouns as conditional phrases.
The first word in the conditional phrase is in each case a noun.

\begin{supertabular}{p{5,5cm}|ll}
    sama pi ni en ona la mi wile jo e ni tu. &  & If this and that is the same, I want both.  \\
    tawa mi la mi pilin pona.                &  & Am I in motion, I feel good.                \\
    tan ni la mi sona e nasin.               &  & If this is the cause, we know the solution. \\
    lon ona la mi ken lukin e ona.           &  & If it has suchness, we can see it.          \\
\end{supertabular}

In this example, the conditional phrase consists of a conjunction and a pronoun.

\begin{supertabular}{p{5,5cm}|ll}
    taso ni la mi pilin pona. &  & But when this happens, I feel good. \\
\end{supertabular}

Here are further examples with one noun each as a conditional phrase.

\begin{supertabular}{p{5,5cm}|ll}
    sama la sina en mi li utala ala. &  & In case of equality we don't fight.       \\
    ante la ni li ' ike.             &  & In case of deviations it is unfavourable. \\
    ike la sina moku e ni.           &  & If you feel nauseous, swallow this.       \\
    pona la sina jo e mani.          &  & Luckily, you have money.                  \\
\end{supertabular}

\subsubsection*{Complete Sentences as Conditional Phrases}
\index{if}
\index{when}
A conditional phrase can also be a complete sentence.

\begin{supertabular}{p{5,5cm}|ll}
    mama mi li ' moli la mi pilin ike.  &  & My parents die, I feel bad.          \\
    mi lape la ali li ' pona.           &  & When I'm asleep, everything is good. \\
    sina moku e telo nasa la sina nasa. &  & If you drink beer, you'll be silly.  \\
    sina ' moli la sina ken ala toki.   &  & If you are dead, you can't speak.    \\
    mi pali mute la mi pilin ike.       &  & When I work a lot, I feel bad.       \\
\end{supertabular}

Commas together with the separator \textit{la} are neither necessary nor useful.

\subsubsection*{The Question Pronoun \textit{seme} as Conditional Phrase}
\index{seme!in conditional phrase}
\index{conditional phrase!seme}

If the question pronoun \textit{seme} is used in a conditional-phrase, this means, 'Under what conditions is ... true?'.

\begin{supertabular}{p{5,5cm}|ll}
    seme la telo kama, tan sewi? &  & Under what conditions does it rain? \\
\end{supertabular}

\subsubsection*{Several Conditional Phrases in one sentence}
\index{\textit{la}!several}
It is possible to use \textit{la} two times in a sentence.
But please not more than two.

\begin{supertabular}{p{5,5cm}|ll}
    ken la tenpo pimeja la ni li ' pona. &  & Maybe in the night it will be ok. \\
\end{supertabular}

\subsubsection*{Conditional Phrases versus Prepositional Objects after the preposition \textit{lon} }
The (compound) noun of the prepositional object after the prepostion \textit{lon} can in some cases be placed before \textit{la} with nearly the same meaning.
This only applies to location and time specifications and if the sentence contains only one predicate phrase with only one prepositional object.

\begin{supertabular}{p{5,5cm}|ll}
    mi moku e telo, lon tenpo ni. &  & I drink now.                            \\
    tenpo ni la mi moku e telo.   &  & If it's now, I'll drink. / I drink now. \\
\end{supertabular}

The following sentence has two predicate phrases, each with a prepositional object with the preposition \textit{lon}.
None of the prepositional objects can be moved to before the separator \textit{la} without changing the statement.
The respective predicate phrase would be torn.

\begin{supertabular}{p{5,5cm}|ll}
    ona li pali, lon tomo pali li moku, lon tomo moku. &  & He works in the office and eats in the canteen. \\
\end{supertabular}

If the predicate is identical for all predicate phrases, prepositional objects with \textit{lon} can be moved before \textit{la}.

\begin{supertabular}{p{5,5cm}|ll}
    ona li moku, lon tenpo ni li moku, lon tenpo kama. &  & He eats now and he eats later. \\
    tenpo ni la tenpo kama la ona li moku.             &  & Now and later he eats.         \\
    tenpo ni en tenpo kama la ona li moku.             &  & Now and later he eats.         \\
\end{supertabular}

The other way around it is not possible to move all possible \textit{la} phrases after the preposition \textit{lon}.
For example, a conditional phrase before \textit{la} can consist of a complete sentence with a subject and predicate(s).
However, you cannot use a complete sentence as a prepositional object.
In the following examples, using conditional phrases as prepositional objects with the preposition \textit{lon} would be confusing.

\begin{supertabular}{p{5,5cm}|ll}
    lon ona la ni li ' pona, tawa mi. &  & If it exists, it's good for me.        \\
    sama ona la sina ken ante e ni.   &  & If it's the same, you can swap it.     \\
    ken la mi tawa.                   &  & Maybe I'll go.                         \\
    tawa mi la li ' pona, tawa mi.    &  & It's good for me when I'm on the move. \\
\end{supertabular}

\subsubsection*{Conditional Phrases versus indirect Objects after the Intransitive Verb \textit{lon} }
The (compound) noun of the indirect object after the intransitive verb \textit{lon} can in some cases be placed before \textit{la} with nearly the same meaning.

\begin{supertabular}{p{5,5cm}|ll}
    mi lon tenpo ni.    &  & I exist now. \\
    tenpo ni la mi lon. &  & Now I exist. \\
\end{supertabular}

The following sentence has two predicate phrases, each with the intransitive verb \textit{lon}.
Since the predicate (\textit{lon}) is the same for both predicate phrases, the indirect objects can be moved before \textit{la}.

\begin{supertabular}{p{5,5cm}|ll}
    ona li lon tenpo ni li lon tomo ni. &  & He's here during this time and in this house. \\
    tenpo ni la ona li lon tomo ni.     &  & At this time he's in the house .              \\
    tenpo ni la tomo ni la ona li lon.  &  & At this time and in this house he is.         \\
\end{supertabular}

The other way around it is not possible to move all possible \textit{la} phrases after the intransitive verb \textit{lon}.
For example, a conditional phrase before \textit{la} can consist of a complete sentence with a subject and predicate(s).
However, you cannot use a complete sentence as an indirect object.

\subsubsection*{Conditional Phrases versus Predicate Noun \textit{lon} or Predicate Adjective \textit{lon} }
After the separator \textit{li} a predicate noun \textit{lon} or a predicate adjective \textit{lon} can stand also.
Direct following words cannot be moved before \textit{lon} because they do not form an object.

\begin{supertabular}{p{5,5cm}|ll}
    ona li ' lon ala.           &  & It has no existence.     \\
    ona li ' lon pi nasin sewi. &  & It's a sacred existence. \\
\end{supertabular}

\subsubsection*{Conditional Phrases with Spatial Nouns}
\index{spatial noun!in conditional phrase}
If a (composite) noun of a prepositional object after the preposition \textit{lon} can also be placed before \textit{la} with (almost) the same meaning, then spatial nouns can also be used in a conditional phrase.

\begin{supertabular}{p{5,5cm}|ll}
    mi tawa, lon poka sina. &  & I'll walk beside you.                            \\
    poka sina la mi tawa.   &  & If at your side, I walk. / I'll walk beside you. \\
\end{supertabular}

If a (compound) noun of an indirect object after the intransitive verb \textit{lon} can also be placed before \textit{la} with (almost) the same meaning, then location-related nouns can also be used in a conditional phrase.

\begin{supertabular}{p{5,5cm}|ll}
    tomo li lon sinpin mi.    &  & The house is in front of me. \\
    sinpin mi la tomo li lon. &  & In front of me is the house. \\
\end{supertabular}

\subsection*{Miscellaneous}
\index{comparative}
\index{superlative}
\index{adjective!comparative}
\index{adjective!superlative}

\subsubsection*{comparative and superlative}
Now to use this concept in Toki Pona, you have to split your idea up into two separate sentences.
Here's how you'd say 'Lisa is better than Susan.'

\begin{supertabular}{p{5,5cm}|ll}
    jan Lisa li ' pona mute. ...  &  & Lisa is very good. ...      \\
    ... jan Susan li ' pona lili. &  & ... Susan is a little good. \\
\end{supertabular}

Make sense?
You say that one thing is very much of something, while you use another object as the basis for comparison and say that it's only a little bit of something.

\begin{supertabular}{p{5,5cm}|ll}
    mi ' suli mute. sina ' suli lili. &  & I'm bigger than you. \\
    mi moku mute. sina moku lili.     &  & I eat more than you. \\
\end{supertabular}

\subsubsection*{Headlines}
\index{headline}
Headings can be incomplete sentences and do not end with a punctuation mark.

\textit{tenpo mun nanpa luka luka wan} \\
tenpo ni li ike kin, lon ma Tosi. \\
suno li suli lili kin. \\
telo li kama, lon sewi. \\
kasi li moli. \\
waso li tawa. \\
tenpo seli o kama!

\newpage
\subsection*{Practice (Answers: Page~\pageref{'la'})}
Please write down your answers and check them afterwards.

\begin{supertabular}{p{5,5cm}|ll}
    What is a conditional phrase?                                      &  & \\
    What follows the separator \textit{la}?                            &  & \\
    What can a conditional phrase consist of?                          &  & \\
    Which word types can be at the beginning of a conditional phrase?  &  & \\
    Can the question pronoun \textit{seme} be in a conditional phrase? &  & \\
\end{supertabular}

Try to translate these sentences.
You can use the tool \textit{Toki Pona Parser} (\cite{www:rowa:02}) for spelling and grammar check.

\begin{supertabular}{p{5,5cm}|ll}
    Maybe Susan will come.                  &  & \\
    Last night I watched X-Files.           &  & \\
    If the enemy comes, burn these papers.  &  & \\
    Maybe he's in school.                   &  & \\
    I have to work tomorrow.                &  & \\
    When it's hot, I sweat. *               &  & \\
    Open the door.                          &  & \\
    Is the moon big tonight?                &  & \\
    Under what conditions will you do this? &  & \\
\end{supertabular}

\begin{supertabular}{p{5,5cm}|ll}
    tenpo suno ni la mun li pimeja ala pimeja e suno?             &  & \\
    ken la jan lili li wile moku e telo.                          &  & \\
    tenpo ali la o kama sona!                                     &  & \\
    sina sona e toki ni la sina sona e toki pona!                 &  & \\
    open la ala li lon!                                           &  & \\
    ken la tomo pi ona en sina pi jelo en loje li ' ike, tawa mi. &  & \\
    sina wile jo e ilo moli la sina wile moli e jan.              &  & \\
    jan nasa pi ilo moli li ken pana e ike.                       &  & \\
\end{supertabular}

* Think: "Heat is present, I emit fluid from my skin."

\begin{supertabular}{p{5,5cm}|ll}
    tenpo suno ni li tenpo suno pali nanpa luka.                                &  & \\
    tenpo suno ni la jan lili pi kama sona li tawa ala, tawa tomo pi kama sona. &  & \\
    ona li wile e ni: jan li pakala ala e ma e telo e kon.                      &  & \\
    tenpo kama la ona li wile lon kin.                                          &  & \\
\end{supertabular}
