%!TEX root = ../toki-pona.tex

\section{Relative Location Information}
\subsection*{Vocabulary}
\index{anpa@\texttp{anpa}}
\index{insa@\texttp{insa}}
\index{monsi@\texttp{monsi}}
\index{noka@\texttp{noka}}
\index{poka@\texttp{poka}}
\index{sewi@\texttp{sewi}}
\index{sinpin@\texttp{sinpin}}

\begin{vocabularytable}
    anpa             & \wordtype{noun}: bottom, lower part, under, below, floor, beneath                   \\
    \dots{} anpa     & \wordtype{adjective}: low, lower, bottom, down                                      \\
                     & \wordtype{adverb}: downstairs, below, deep, low, deeply                             \\
    anpa             & \wordtype{verb intransitive}: to prostrate oneself                                  \\
    anpa (e \dots{}) & \wordtype{verb transitive}: to defeat, to beat, to vanquish, to conquer, to enslave \\
    \wordrule %%%%%%%%%%%%%%%%%%%%%%%%%%%%%%%%%%%%%%%%%%%%%%%%%%%%%%%%%%%%%
    insa             & \wordtype{noun}: inside, inner world, centre, stomach                               \\
    \dots{} insa     & \wordtype{adjective}: inner, internal                                               \\
    \wordrule %%%%%%%%%%%%%%%%%%%%%%%%%%%%%%%%%%%%%%%%%%%%%%%%%%%%%%%%%%%%%
    monsi            & \wordtype{noun}: back, rear end, butt, behind                                       \\
    \dots{} monsi    & \wordtype{adjective}: back, rear                                                    \\
    \wordrule %%%%%%%%%%%%%%%%%%%%%%%%%%%%%%%%%%%%%%%%%%%%%%%%%%%%%%%%%%%%%
    noka             & \wordtype{noun}: leg, foot, organ of locomotion, bottom, lower part                 \\
    \dots{} noka     & \wordtype{adjective}: foot-, lower, bottom                                          \\
                     & \wordtype{adverb}: on foot                                                          \\
    \wordrule %%%%%%%%%%%%%%%%%%%%%%%%%%%%%%%%%%%%%%%%%%%%%%%%%%%%%%%%%%%%%
    poka             & \wordtype{noun}: side, hip, next to                                                 \\
    \dots{} poka     & \wordtype{adjective}: neighbouring                                                  \\
    \wordrule %%%%%%%%%%%%%%%%%%%%%%%%%%%%%%%%%%%%%%%%%%%%%%%%%%%%%%%%%%%%%
    sewi             & \wordtype{noun}: high, up, above, top, over, on                                     \\
    \dots{} sewi     & \wordtype{adjective}, \wordtype{adverb}: superior, elevated, religious, formal      \\
    sewi             & \wordtype{verb intransitive}: to get up                                             \\
    sewi (e \dots{}) & \wordtype{verb transitive}: to lift                                                 \\
    \wordrule %%%%%%%%%%%%%%%%%%%%%%%%%%%%%%%%%%%%%%%%%%%%%%%%%%%%%%%%%%%%%
    sinpin           & \wordtype{noun}: face, foremost, front, wall, chest, torso                          \\
    \dots{} sinpin   & \wordtype{adjective}: facial, frontal, anterior, vertical                           \\
\end{vocabularytable}

\subsection*{The Spatial Nouns \texttp{anpa}, \texttp{insa}, \texttp{monsi}, \texttp{noka}, \texttp{poka}, \texttp{sewi}, and \texttp{sinpin}}
\index{noun!spatial}
\index{spatial noun}
\index{anpa@\texttp{anpa}!spatial noun}
\index{insa@\texttp{insa}!spatial noun}
\index{monsi@\texttp{monsi}!spatial noun}
\index{noka@\texttp{noka}!spatial noun}
\index{poka@\texttp{poka}!spatial noun}
\index{sewi@\texttp{sewi}!spatial noun}
\index{sinpin@\texttp{sinpin}!spatial noun}
In Toki Pona relative location information is formed with special nouns.
These special nouns are called ``spatial nouns''.
In addition to the noun, adjectives, possessive pronouns, or demonstrative pronouns are required for the relative location information.

A spatial noun is preceded by either an intransitive verb or a preposition.
This means that relative location information is either in an indirect object or a prepositional object and is therefore part of a predicate phrase.

\subsubsection*{Spatial Nouns in an Indirect Object}
\index{spatial noun!indirect object}
\index{lon@\texttp{lon}!intransitive verb}
\index{lon@\texttp{lon}!preposition}
Usually the intransitive verb \texttp{lon} or preposition \texttp{lon} is used before spatial nouns.
If there is no verb before \texttp{lon}, \texttp{lon} cannot be a preposition.
In these examples the intransitive verb \texttp{lon} is used.

\begin{translationtable}
    pipi li lon anpa mi.       & The bug is underneath me.    \\
    telo suli li lon monsi mi. & The sea is behind me.        \\
    moku li lon insa mi.       & Food is inside me.           \\
    ma li lon noka mi.         & Land is under my feet.       \\
    ona li lon sewi mi.        & He is in my above.           \\
                               & He is above me.              \\
    tomo li lon sinpin mi.     & The house is in front of me. \\
\end{translationtable}

\subsubsection*{Spatial Nouns in a Prepositional Object}
\index{spatial noun!prepositional object}
The following examples contain a verb.
Hence, preposition \texttp{lon} is used.

\begin{translationtable}
    mi moku, lon poka sina.          & I'm eating beside or with you.     \\
    ona li pona e ilo, lon tomo ona. & He repairs the tools in his house. \\
\end{translationtable}
%
In this sentence the second \texttp{tawa} is a preposition and stands before the spatial noun \texttp{noka}.

\begin{translationtable}
    mi tawa e mi, tawa noka sina. & I bow before you. \\
\end{translationtable}

\subsection*{Further meanings of these words}
\index{poka@\texttp{poka}!adjective}
\subsubsection*{The transitive Verb \texttp{anpa}}
\index{anpa@\texttp{anpa}!verb}

\begin{translationtable}
    mi anpa e jan utala. & I defeated the warrior. \\
\end{translationtable}

\subsubsection*{The ``normal'' noun \texttp{poka}}
\index{poka@\texttp{poka}!noun}

\begin{translationtable}
    poka telo & water side, the beach \\
\end{translationtable}

\subsubsection*{The Adjektive \texttp{poka}}
\index{poka@\texttp{poka}!adjective}
\begin{translationtable}
    jan poka & neighbor, someone who is beside you \\
\end{translationtable}

\practice{
    How do you create relative location information in Toki Pona?     & \answer{With an indirect verb or a preposition and a compound spatial noun.} \\\wordrule
    What is a possessive pronoun?                                     & \answer{A possessive pronoun expresses a characteristic or affiliation.}     \\\wordrule
    % Where is a slot for a substantive demonstrative pronoun possible? & \answer{Instead of a noun.}                                                  \\\wordrule % removed. substantive demonstrative pronouns are mentioned nowhere
    Which separator is at the end of a declarative sentence?          & \answer{A full stop.}                                                        \\\wordrule
    What is a predicate adjective?                                    & \answer{An adjective that is used as predicate.}                             \\\wordrule
    In which sentence phrases can contain spatial nouns be found?     & \answer{In an indirect object or prepositional object.}                      \\
}

\practice[Try to translate these sentences.]{
    \answer{\texttp{jan pona mi li lon poka mi.}}    & My friend is beside me.                  \\\wordrule
    \answer{\texttp{suno li lon sewi mi.}}           & The sun is above me.                     \\\wordrule
    \answer{\texttp{ma li lon anpa mi.}}             & The land is beneath me.                  \\\wordrule
    \answer{\texttp{ijo ike li lon monsi mi.}}       & Bad things are behind me.                \\\wordrule
    \answer{\texttp{mi ' pona, tan ni: mi lon.}}     & I'm okay because I'm alive.\footnotemark \\\wordrule
    \answer{\texttp{mi lukin e ma, lon poka sina.}}  & I look at the land with you.             \\\wordrule
    \texttp{poka mi li ' pakala.}                    & \answer{My hip hurts.}                   \\\wordrule
    \texttp{mi kepeken poki li kepeken ilo moku.}    & \answer{I'm using a bowl and a spoon.}   \\\wordrule
    \texttp{jan li lon insa tomo.}                   & \answer{Somebody's inside the house.}    \\
}

\footnotetext{\texttp{lon} as a verb by itself means to exist, to be real, etc.}
