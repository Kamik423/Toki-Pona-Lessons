%!TEX root = ../toki-pona.tex

\section{Negation, Yes/No Questions}
\subsection*{Vocabulary}
\index{ala@\texttp{ala}}
\index{ken@\texttp{ken}}
\index{lape@\texttp{lape}}
\index{musi@\texttp{musi}}
\index{pali@\texttp{pali}}
\index{wawa@\texttp{wawa}}

\begin{vocabularytable}
    ala              & \wordtype{noun}: nothing, negation, zero                                       \\
    \dots{} ala      & \wordtype{adjective}: no, not, none, un-                                       \\
                     & \wordtype{adverb}: don't                                                       \\
    \wordrule %%%%%%%%%%%%%%%%%%%%%%%%%%%%%%%%%%%%%%%%%%%%%%%%%%%%%%%%%%%%%
    ken              & \wordtype{noun}: possibility, ability, power to do things, permission          \\
                     & \wordtype{verb intransitive}: can, is able to, is allowed to, may, is possible \\
    ken \dots{}      & \wordtype{auxiliary verb}: to can, may                                         \\
    ken (e \dots{})  & \wordtype{verb transitive}: to make possible, to enable, to allow, to permit   \\
    \wordrule %%%%%%%%%%%%%%%%%%%%%%%%%%%%%%%%%%%%%%%%%%%%%%%%%%%%%%%%%%%%%
    lape             & \wordtype{noun}: sleep, rest                                                   \\
    \dots{} lape     & \wordtype{adjective}: sleeping, of sleep, dormant                              \\
                     & \wordtype{adverb}: asleep                                                      \\
    lape             & \wordtype{verb intransitive}: to sleep, to rest                                \\
    lape (e \dots{}) & \wordtype{verb transitive}: to knock out                                       \\
    \wordrule %%%%%%%%%%%%%%%%%%%%%%%%%%%%%%%%%%%%%%%%%%%%%%%%%%%%%%%%%%%%%
    musi             & \wordtype{noun}: fun, playing, game, recreation, art, entertainment            \\
    \dots{} musi     & \wordtype{adjective}: artful, fun, recreational                                \\
                     & \wordtype{adverb}: cheerfully                                                  \\
    musi             & \wordtype{verb intransitive}: to play, to have fun                             \\
    musi (e \dots{}) & \wordtype{verb transitive}: to amuse, to entertain                             \\
    \wordrule %%%%%%%%%%%%%%%%%%%%%%%%%%%%%%%%%%%%%%%%%%%%%%%%%%%%%%%%%%%%%
    pali             & \wordtype{noun}: activity, work, deed, project                                 \\
    \dots{} pali     & \wordtype{adjective}: active, work-related, operating, working                 \\
                     & \wordtype{adverb}: actively, briskly                                           \\
    pali             & \wordtype{verb intransitive}: to act, to work, to function                     \\
    pali (e \dots{}) & \wordtype{verb transitive}: to do, to make, to build, to create                \\
    \wordrule %%%%%%%%%%%%%%%%%%%%%%%%%%%%%%%%%%%%%%%%%%%%%%%%%%%%%%%%%%%%%
    wawa             & \wordtype{noun}: energy, strength, power                                       \\
    \dots{} wawa     & \wordtype{adjective}: energetic, strong, fierce, intense, sure, confident      \\
                     & \wordtype{adverb}: strongly, powerfully                                        \\
    wawa (e \dots{}) & \wordtype{verb transitive}: to strengthen, to energize, to empower             \\
\end{vocabularytable}

\subsection*{Negation}
\index{negation}
\index{\textit{ala}!negation}
Sentence elements are negated by \texttp{ala}.

\subsubsection*{The Adverb \texttp{ala}}
\index{ala@\texttp{ala}!adverb}
In English, you negate a verb by adding ``not'' in front of the verb.
In Toki Pona you put the adverb \texttp{ala} after the verb.

\index{lazy}
\begin{translationtable}
    mi lape ala.           & I'm not sleeping.                \\
    mi musi ala.           & I'm not having fun. / I'm bored. \\
    mi wawa ala.           & I'm not strong. / I'm weak.      \\
    mi wile ala tawa musi. & I don't want to dance.           \\
    tawa musi              & dance (move entertainingly)      \\
    mi wile ala pali.      & I'm lazy.                        \\
\end{translationtable}

\newpage

\subsubsection*{The Adjective \texttp{ala}}
\index{ala@\texttp{ala}!adjective}
\begin{translationtable}
    jan ala li toki. & Nobody is talking. \\
\end{translationtable}

\subsubsection*{The Noun \texttp{ala}}
\index{ala@\texttp{ala}!noun}
\begin{translationtable}
    ala li ' jaki. & Nothing is dirty. \\
\end{translationtable}

\subsection*{Yes/no Questions with \texttp{ala}}
\index{question!yes,no}
\index{yes,no!question}
Yes/no questions in Toki Pona  are formed according to a simple pattern.
\texttp{ala} is appended to the part of the sentence that is asked for and that part of the sentence is repeated.
As a rule, this part of the sentence is the entire predicate (sentence statement).
The rest of the sentence structure does not change.
A question is ended with a question mark.

\subsubsection*{An Intransitive Verb as Subject of the Question}
If the yes/no question refers to an intransitive verb, the adverb \texttp{ala} is appended to it and the intransitive verb is repeated.
Consider the following example:

\begin{translationtable}
    sina tawa, tan mi. & You're leaving me. \\
\end{translationtable}
%
If we want to ask ``Are you leaving me?'', we append the adverb \texttp{ala} to the intransitive verb \texttp{tawa}.
Then we repeat the intransitive verb \texttp{tawa}.

\begin{translationtable}
    sina tawa ala tawa, tan mi? & Are you leaving me? \\
\end{translationtable}
%
Here are more examples.

\begin{translationtable}
    ona li lon ala lon tomo?      & Is he in the house?     \\
    sina kepeken ala kepeken ilo? & Are you using the tool? \\
    pona li kama ala kama?        & Is the good coming?     \\
    sina sewi ala sewi?           & Are you getting up?     \\
\end{translationtable}

\subsubsection*{A Transitive Verb as Subject of the Question}
If the yes/no question refers to a transitive verb, the adverb \texttp{ala} is appended to it and the transitive verb is repeated.

\begin{translationtable}
    sina pona ala pona e ilo?                & Are you fixing the tool?        \\
    sina pana ala pana e moku tawa jan lili? & Did you give food to the child? \\
    pipi li moku ala moku e kili?            & Are the bugs eating the fruit?  \\
    ona li mama ala mama e sina?             & Does she mother you?            \\
\end{translationtable}

\subsubsection*{An Auxiliary Verb as Subject of the Question}
As we have learned, the auxiliary verb and the verb together form the predicate.
If the yes/no question refers to an auxiliary verb, then the adverb \texttp{ala} is not attached to the predicate, but directly to the auxiliary verb.
Only the auxiliary verb is repeated.
Then the verb follows.

\begin{translationtable}
    sina wile ala wile moku?         & Do you want to eat?   \\
    sina ken ala ken lape?           & Can you sleep?        \\
    sina kama ala kama jo e pali ni? & Did you get this job? \\
\end{translationtable}

\newpage

\subsubsection*{A Predicate Noun as Subject of the Question}
We had already learned the difference between a verb and a predicate.
In Toki Pona sentences without verbs are possible.
Then nouns serve as predicate nouns or adjectives as predicate adjectives.

In the lessons of B. J. Knight (2003) and the official Toki Pona book of Sonja Lang \cite{www:tokipona.org} yes/no questions with \texttp{ala} are defined only with verbs.
But this contradicts their own examples as well as common practice.
For example one cannot formulate the question ``Is she a mother?''.
In these lessons I will therefore not adhere to this limitation.

If the yes/no question refers to a predicate noun, the adjective \texttp{ala} is added to it and the predicate noun is repeated.

\begin{translationtable}
    ona li ' mama ala ' mama ?                          & Is she a parent?             \\
    ni li ' jan ala ' jan?                              & Is this a person?            \\
    ni li ' kili ala ' kili?                            & Is this a banana?            \\
    ni li ' tomo pi telo nasa ala ' tomo pi telo nasa?  & Is this a pub?               \\
    ona li ' jan pi pona lukin ala ' jan pi pona lukin? & Is she an attractive person? \\
\end{translationtable}

\subsubsection*{An Predicate Adjective as Subject of the Question}
If the yes/no question refers to a predicate adjective, the adjective \texttp{ala} is added to it and the predicate adjective is repeated.

\begin{translationtable}
    sina ' pona ala ' pona?          & Are you OK?           \\
    mi ' pona ala ' pona, tawa sina? & Do you like me?       \\
    suno li ' suli ala ' suli?       & Is the sun big?       \\
    len sina li ' telo ala ' telo?   & Are your clothes wet? \\
\end{translationtable}

\subsection*{Yes/No Response}
\index{responding!yes,no}
\index{yes}
\index{no}
If you want to say ``yes'', you simply repeat the predicate or the auxiliary verb of the sentence.
If you want to say ``no'', you repeat the predicate or the auxiliary verb and add \texttp{ala} after it.

\begin{translationtable}
    sina wile ala wile moku?                           & Do you want to eat?                \\
    wile                                               & Yes.                               \\
    wile ala                                           & No.                                \\
    \wordrule %%%%%%%%%%%%%%%%%%%%%%%%%%%%%%%%%%%%%%%%%%%%%%%%%%%%%%%%%%%%%
    sina lukin ala lukin e kiwen?                      & Do you see the rock?               \\
    lukin                                              & Yes.                               \\
    lukin ala                                          & No.                                \\
    \wordrule %%%%%%%%%%%%%%%%%%%%%%%%%%%%%%%%%%%%%%%%%%%%%%%%%%%%%%%%%%%%%
    sina sona ala sona e toki mi?                      & Do you understand what I'm saying? \\
    sona                                               & Yes.                               \\
    sona ala                                           & No.                                \\
    \wordrule %%%%%%%%%%%%%%%%%%%%%%%%%%%%%%%%%%%%%%%%%%%%%%%%%%%%%%%%%%%%%
    ni li ' tomo pi telo nasa ala ' tomo pi telo nasa? & Is this a pub?                     \\
    tomo pi telo nasa.                                 & Yes.                               \\
    tomo pi telo nasa ala.                             & No.                                \\
\end{translationtable}

\newpage

\practice{
    Which separator is at the end of a question?        & \answer{A question mark.}                                                                  \\\wordrule
    How is a verb negated in Toki Pona?                 & \answer{By placing the adverb \texttp{ala} after the verb.}                                \\\wordrule
    How do you respond negatively to a yes/no question? & \answer{One repeats the predicate or the auxiliary of the question and adds \texttp{ala}.} \\\wordrule
    How do you respond positively to a yes/no question? & \answer{One repeats the predicate or the auxiliary of the question.}                       \\
}

\practice[Try to translate these sentences.]{
    \answer{\texttp{sina wile toki e tan, tawa mi.}}  & You have to tell me why.\footnotemark      \\\wordrule
    \answer{\texttp{pipi li lon ala lon poka mi?}}    & Is a bug beside me?                        \\\wordrule
    \answer{\texttp{mi ken ala lape.}}                & I can't sleep.                             \\\wordrule
    \answer{\texttp{mi wile ala toki, tawa sina.}}    & I don't want to talk to you.               \\\wordrule
    \answer{\texttp{ona li tawa ala, tawa telo.}}     & He didn't go to the lake.                  \\\wordrule
    \texttp{sina wile ala wile pali? wile ala.}       & \answer{Do you want to work? No.}          \\\wordrule
    \texttp{jan utala li seli ala seli e tomo?}       & \answer{Is the warrior burning the house?} \\\wordrule
    \texttp{jan lili li ken ala moku e telo nasa.}    & \answer{Children can't drink beer.}        \\\wordrule
    \texttp{sina kepeken ala kepeken ni?}             & \answer{Are you using that?}               \\\wordrule
    \texttp{sina ken ala ken kama?}                   & \answer{Can you come?}                     \\\wordrule
    \texttp{sina pona ala pona?}                      & \answer{Are you OK?}                       \\
}

\footnotetext{Think: ``You have to tell the reason to me.''}