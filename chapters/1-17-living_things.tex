%!TEX root = ../toki-pona-lessons.tex

\section{Living Things}
\subsection*{Vocabulary}

\begin{supertabular}{p{2,5cm}|ll}
    \index{akesi}
    \textbf{\dots akesi}      &  & \textit{adjective}: amphibian-, reptilian-, slimy                                                 \\
    \textbf{akesi}            &  & \textit{noun}: reptile, amphibian; non-cute animal                                                \\
                              &  &                                                                                                   \\
    \index{alasa}
    \textbf{\dots alasa}      &  & \textit{adjective}: hunting-, -hunting, hunting                                                   \\
    \textbf{alasa}            &  & \textit{noun}: hunting                                                                            \\
    \textbf{alasa (e \dots)}  &  & \textit{verb transitive}: to hunt, to forage                                                      \\
                              &  &                                                                                                   \\
    \index{kala}
    \textbf{\dots kala}       &  & \textit{adjective}: fish-                                                                         \\
    \textbf{kala}             &  & \textit{noun}: fish, marine animal, sea creature                                                  \\
                              &  &                                                                                                   \\
    \index{kasi}
    \textbf{\dots kasi}       &  & \textit{adjective}: vegetable, vegetal, biological, biologic, leafy                               \\
    \textbf{kasi}             &  & \textit{noun}: plant, vegetation, herb, leaf                                                      \\
    \textbf{kasi}             &  & \textit{verb intransitive}: to grow                                                               \\
    \textbf{kasi (e \dots)}   &  & \textit{verb transitive}: to plant, to grow                                                       \\
                              &  &                                                                                                   \\
    \index{moli}
    \textbf{\dots moli}       &  & \textit{adjective}: dead, dying, fatal, deadly, lethal, mortal, deathly, killing                  \\
    \textbf{\dots moli}       &  & \textit{adverb}: mortally                                                                         \\
    \textbf{moli}             &  & \textit{noun}: death, decease                                                                     \\
    \textbf{moli}             &  & \textit{verb intransitive}: to die, to be dead                                                    \\
    \textbf{moli (e \dots)}   &  & \textit{verb transitive}: to kill                                                                 \\
    kama \textbf{moli}        &  & \textit{intransitives Verb}: dieing                                                               \\
                              &  &                                                                                                   \\
    \index{monsuta}
    \textbf{\dots monsuta}    &  & \textit{adjective} (unofficial): fearful, afraid                                                  \\
    \textbf{monsuta}          &  & \textit{noun} (unofficial): monster, monstrosity, fearful thing, fright, mythical creatures, fear \\
                              &  &                                                                                                   \\
    \index{namako}
    \textbf{\dots namako}     &  & \textit{adjective}: spicy, piquant                                                                \\
    \textbf{namako}           &  & \textit{noun}: spice, something extra, food additive, accessory                                   \\
    \textbf{namako (e \dots)} &  & \textit{verb transitive}: to spice, to flavor, to decorate                                        \\
                              &  &                                                                                                   \\
    \index{pan}
    \textbf{pan}              &  & \textit{noun}: cereal, grain; barley, corn, oat, rice, wheat; bread, pasta                        \\
    \textbf{pan (e \dots)}    &  & \textit{verb transitive}: to sow                                                                  \\
                              &  &                                                                                                   \\
    \index{soweli}
    \textbf{\dots soweli}     &  & \textit{adjective}: animal                                                                        \\
    \textbf{soweli}           &  & \textit{noun}: animal, especially land mammal, lovable animal, beast                              \\
                              &  &                                                                                                   \\
    \index{waso}
    \textbf{\dots waso}       &  & \textit{adjective}: bird-                                                                         \\
    \textbf{waso}             &  & \textit{noun}: bird, bat; flying creature, winged animal                                          \\
\end{supertabular}

\newpage

\subsection*{Names of Living Things}
\subsubsection*{The Noun \textit{soweli}}
\index{\textit{soweli}!noun}
The noun \textit{soweli} is basically for all types of mammals.
The noun \textit{soweli} is used however also for meat of mammals, since there is no special word for meat.

\begin{supertabular}{p{5,5cm}|ll}
    soweli lili li ' ike, tawa mi. &  & I'm allergic to cats.    \\
    soweli ni li ' pona moku.      &  & This cow is good to eat. \\
\end{supertabular}

\subsubsection*{The Noun \textit{waso}}
\index{\textit{waso}!noun}
\index{\textit{soweli}!adjective}
The noun \textit{waso} includes all birds and flying animals.

\begin{supertabular}{p{5,5cm}|ll}
    waso wawa li tawa e ona, lon kon. &  & The eagle moves through the air. \\
    mi wile moku e waso.              &  & I want to eat chicken.           \\
\end{supertabular}

\subsubsection*{The Adjective \textit{soweli}}
\index{\textit{soweli}!adjective}
The names of the living beings can also be adjectives.

\begin{supertabular}{p{5,5cm}|ll}
    waso soweli li ' pimeja.    &  & The bat is black.  \\
    mi moku lili e moku soweli. &  & I eat little meat. \\
\end{supertabular}

\subsubsection*{The Noun \textit{akesi}}
\index{\textit{akesi}!noun}
The noun \textit{akesi} covers all of the reptiles, amphibians, dinosaurs and monsters.

\begin{supertabular}{p{5,5cm}|ll}
    akesi pi telo moli &  & venomous snakes, poisonous frogs \\
\end{supertabular}

\subsubsection*{The Adjective \textit{akesi}}
\index{\textit{akesi}!adjective}
The adjective \textit{akesi} means 'amphibian-', 'reptilian-' or 'slimy'.

\begin{supertabular}{p{5,5cm}|ll}
    tomo tawa akesi li tawa, lon ma li tawa, lon telo. &  & The amphibious vehicle drives on land and in the water. \\
\end{supertabular}

\subsubsection*{The Noun \textit{kala}}
\index{\textit{kala}!noun}
The noun \textit{kala} designates fish and other aquatic animals.

\begin{supertabular}{p{5,5cm}|ll}
    kalama pi kala ni li pakala e kala ali. &  & The noise of this fish disturbed all the fish. \\
\end{supertabular}

\subsubsection*{The Adjective \textit{kala}}
\index{\textit{kala}!adjective}

\begin{supertabular}{p{5,5cm}|ll}
    meli kala lili li tawa e ona, lon telo. &  & The mermaid floats in the water. \\
    kala wawa li moku e soweli kala.        &  & The shark eats the seal.         \\
\end{supertabular}

The first \textit{kala} in the last sentence is of course a noun.

\subsubsection*{The Noun \textit{pipi}}
\index{\textit{pipi}!noun}
The Noun \textit{pipi} is used for all types of bugs (spiders, ants, roaches, butterflies).

\begin{supertabular}{p{5,5cm}|ll}
    mi pakala e pipi ike. &  & I hurt the ugly bug. \\
\end{supertabular}

\subsubsection*{The Noun \textit{kasi}}
\index{\textit{kasi}!noun}
The noun \textit{kasi} is used to talk about all plants and plant-like things.

\begin{supertabular}{p{5,5cm}|ll}
    kasi kule             &  & flower            \\
    kasi suli             &  & trees, big shrubs \\
    kasi anpa             &  & grass             \\
    kasi nasa / kasi sona &  & hemp              \\
\end{supertabular}

\subsubsection*{The Adjective \textit{kasi}}
\index{\textit{kasi}!adjective}
The adjective \textit{kasi} means 'plant-based'.

\begin{supertabular}{p{5,5cm}|ll}
    ma kasi &  & forest, jungle \\
\end{supertabular}

\subsubsection*{The Transitive Verb \textit{kasi}}
\index{\textit{kasi}!verb}
The transitive verb \textit{kasi} means 'to plant'.

\begin{supertabular}{p{5,5cm}|ll}
    mi kasi e kasi kule, lon poki. &  & I'll plant the flower in the pot. \\
\end{supertabular}

\subsubsection*{The Intransitive Verb \textit{kasi}}
\index{\textit{kasi}!verb}
The intransitive verb \textit{kasi} means 'to grow'.

\begin{supertabular}{p{5,5cm}|ll}
    kasi suli li kasi, tawa sewi. &  & The tree grows into the sky. \\
\end{supertabular}

\subsection*{Animal Sounds and Communication}
\index{animal sounds}
\subsubsection*{The Noun \textit{mu}}
\index{mu!noun}

\begin{supertabular}{p{5,5cm}|ll}
    mu ni li ' ike a! &  & That barking is terrible! \\
    % mu! mu! mu! && Wow! Wow! Wow! \\
\end{supertabular}

\subsubsection*{The Adjective  \textit{mu}}
\index{mu!adjective}

\begin{supertabular}{p{5,5cm}|ll}
    kalama mu ni li ' pona, tawa mi. &  & I like this animal sound. \\
\end{supertabular}

\subsubsection*{The Transitive Verb \textit{mu}}
\index{mu!verb}

\begin{supertabular}{p{5,5cm}|ll}
    pipi li mu e kalama. &  & The cicadas are chirping noises. \\
\end{supertabular}

\subsubsection*{The Intransitive Verb \textit{mu}}
\index{mu!verb}

\begin{supertabular}{p{5,5cm}|ll}
    pipi li mu, tawa ona. &  & The beetles communicate with each other. \\
\end{supertabular}

\subsubsection*{The Adverb \textit{mu}}
\index{mu!adverb}

\begin{supertabular}{p{5,5cm}|ll}
    sina toki mu e ni. &  & You say that beastly. \\
\end{supertabular}

\subsection*{Miscellaneous}
\subsubsection*{The Noun \textit{pan}}
\index{\textit{pan}!noun}
The noun \textit{pan} refers to certain foods (cereals, grains; barley, maize, oats, rice, wheat, bread, pasta).

\begin{supertabular}{p{5,5cm}|ll}
    pan ni li ' moku ike. &  & This pasta is unappetizing. \\
\end{supertabular}

\subsubsection*{The Transitive Verb \textit{pan}}
\index{\textit{pan}!verb}

The transitive verb \textit{pan} means 'to sow' or 'to sow out'.

\begin{supertabular}{p{5,5cm}|ll}
    ona li pan e pan.   &  & They're sowing the grain. \\
    ona li pan ala pan? &  & Does he sow?              \\
\end{supertabular}

\subsubsection*{The Noun \textit{namako}}
\index{\textit{namako}!noun}
The noun \textit{namako} means 'spice', 'salt' or 'food additive'.

\begin{supertabular}{p{5,5cm}|ll}
    o pana e namako, tawa mi. &  & Give me some spice. \\
\end{supertabular}

\subsubsection*{The Adjective \textit{namako}}
\index{\textit{namako}!adjective}
The adjective \textit{namako} means 'spicy'.

\begin{supertabular}{p{5,5cm}|ll}
    mi moku e pan namako. &  & I eat the spicy bread. \\
\end{supertabular}

\subsubsection*{The Transitive Verb \textit{namako}}
\index{\textit{namako}!verb}
The transitive verb \textit{namako} means 'to spice'.

\begin{supertabular}{p{5,5cm}|ll}
    ona li namako ala namako? &  & Did she season?             \\
    meli mi li namako e moku. &  & My wife spices up the food. \\
\end{supertabular}

\subsubsection*{The Noun \textit{moli}}
\index{\textit{moli}!noun}
The noun \textit{moli} means 'the death'.

\begin{supertabular}{p{5,5cm}|ll}
    moli li ' ike, tawa jan ali. &  & Death is bad for all men. \\
    ona li anpa e moli.          &  & She defeated death.       \\
\end{supertabular}

\subsubsection*{The Adjective \textit{moli}}
\index{\textit{moli}!adjektive}
The adjective \textit{moli} means 'dead', 'fatal', or 'serious'.

\begin{supertabular}{p{5,5cm}|ll}
    pakala moli li kama, tawa sina. &  & The deadly battle comes to you. \\
\end{supertabular}

\subsubsection*{The Transitive Verb \textit{moli}}
\index{\textit{moli}!verb}
The transitive verb \textit{moli} means 'to kill'.

\begin{supertabular}{p{5,5cm}|ll}
    jan li moli e waso.          &  & The man killed the bird.   \\
    jan li moli ala moli e waso? &  & Did the man kill the bird? \\
\end{supertabular}

\subsubsection*{The Intransitive Verb \textit{moli}}
\index{\textit{moli}!verb}
The intransitive verb \textit{moli} means 'be dead'.
Mit dem Hilfsverb \textit{kama} means es 'die'.

\begin{supertabular}{p{5,5cm}|ll}
    soweli li kama ala kama moli? &  & Is the dog dying? \\
    soweli li kama moli.          &  & The dog dies.     \\
\end{supertabular}

\subsubsection*{The Adverb \textit{moli}}
\index{\textit{moli}!adverb}
The adverb \textit{moli} means 'deadly'.

\begin{supertabular}{p{5,5cm}|ll}
    akesi li pakala moli e soweli. &  & The monitor lizard bite deadly the goat. \\
\end{supertabular}

\subsubsection*{The Noun \textit{alasa}}
\index{\textit{alasa}!noun}
The noun \textit{alasa} means 'The hunting'.

\begin{supertabular}{p{5,5cm}|ll}
    alasa li pana e soweli, tawa mi. &  & The hunt brings me meat. \\
\end{supertabular}

\subsubsection*{The Adjective \textit{alasa}}
\index{\textit{alasa}!adjective}
The adjective \textit{alasa} means 'hunting-', '-hunting' or 'hunting'.

\begin{supertabular}{p{5,5cm}|ll}
    jan alasa pona li ' wawa. &  & A good hunter is strong. \\
\end{supertabular}

\subsubsection*{The Transitive Verb \textit{alasa}}
\index{\textit{alasa}!verb}
The transitive verb \textit{alasa} menas 'to hunt' or 'to forage'.

\begin{supertabular}{p{5,5cm}|ll}
    jan li alasa e soweli. &  & Somebody hunt a buffalo. \\
\end{supertabular}

\subsubsection*{The Noun \textit{monsuta}}
\index{\textit{monsuta}!noun}
The noun \textit{monsuta} means 'monster', 'mythical creatures' or 'fear'.

\begin{supertabular}{p{5,5cm}|ll}
    monsuta waso pi pan linja li pali e ali. &  & The Flying Spaghetti Monster has created the world. \\
\end{supertabular}

\subsubsection*{The Adjective \textit{monsuta}}
\index{\textit{monsuta}!adjective}
The adjective \textit{monsuta} means 'fearful' or 'afraid'.

\begin{supertabular}{p{5,5cm}|ll}
    ni li ' mije monsuta. &  & This is a fearful man. \\
\end{supertabular}

\newpage

\subsection*{Practice (Answers: Page~\pageref{'living_things'})}
Please write down your answers and check them afterwards.

\begin{supertabular}{p{5,5cm}|ll}
    Which separator is at the end of a question?             &  & \\
    In which cases is a comma used?                          &  & \\
    In which cases a colon is used?                          &  & \\
    Where are possible slots for prepositions in a sentence? &  & \\
\end{supertabular}

Try to translate these sentences.
You can use the tool \textit{Toki Pona Parser} (\cite{www:rowa:02}) for spelling and grammar check.

\begin{supertabular}{p{5,5cm}|ll}
    Is this a mammal?                  &  & \\
    I want a puppy.                    &  & \\
    Ahh! The dinosaur wants to eat me! &  & \\
    The mosquito bit me.               &  & \\
    Cows say moo.                      &  & \\
    Birds fly in air. *                &  & \\
    Let's eat fish.                    &  & \\
    Flowers are pretty. **             &  & \\
    I like plants.                     &  & \\
    Have you improved?                 &  & \\
\end{supertabular}

\begin{supertabular}{p{5,5cm}|ll}
    mama ona li kepeken kasi nasa. &  & \\
    akesi li pana e telo moli.     &  & \\
    pipi li moku e kasi.           &  & \\
    soweli mi li kama moli.        &  & \\
    jan Pawe o, mi wile ala moli.  &  & \\
    mi lon ma kasi.                &  & \\
    ona li kasi ala kasi?          &  & \\
\end{supertabular}

* Think: 'Birds go in air.' \\
** Think: 'Colorful plants are good to see.'
