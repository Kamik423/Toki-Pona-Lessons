%!TEX root = ../toki-pona.tex

\section{Prepositional Objects}
\subsection*{Vocabulary}
\index{ali@\texttp{ali}}
\index{pipi@\texttp{pipi}}
\index{sama@\texttp{sama}}
\index{sona@\texttp{sona}}
\index{tan@\texttp{tan}}

\begin{vocabularytable}
    ali                    & \wordtype{noun}: everything, anything, life, the universe                       \\
    \dots{} ali            & \wordtype{adjective}: all, every, complete, whole                               \\
                           & \wordtype{adverb}: always, forever, evermore, eternally                         \\
    \wordrule %%%%%%%%%%%%%%%%%%%%%%%%%%%%%%%%%%%%%%%%%%%%%%%%%%%%%%%%%%%%%
    pipi                   & \wordtype{noun}: bug, insect, spider                                            \\
    \wordrule %%%%%%%%%%%%%%%%%%%%%%%%%%%%%%%%%%%%%%%%%%%%%%%%%%%%%%%%%%%%%
    sama                   & \wordtype{noun}: equality, parity, equity, identity, par, sameness              \\
    \dots{} sama           & \wordtype{adjective}: same, similar, equal, of equal status or position         \\
    \dots{} sama           & \wordtype{adverb}: just as, equally, exactly the same, just the same, similarly \\
    \dots{} , sama \dots{} & \wordtype{preposition}: like, as, seem                                          \\
    sama (e \dots{})       & \wordtype{verb transitive}: to equate, to make equal, to make similar to        \\
    \wordrule %%%%%%%%%%%%%%%%%%%%%%%%%%%%%%%%%%%%%%%%%%%%%%%%%%%%%%%%%%%%%
    sona                   & \wordtype{noun}: knowledge, wisdom, intelligence, understanding                 \\
    \dots{} sona           & \wordtype{adjective}: knowing, cognizant, shrewd                                \\
    sona                   & \wordtype{verb intransitive}: to know, to understand                            \\
    sona (e \dots{})       & \wordtype{verb transitive}: to know, to understand, to know how to              \\
    sona \dots{}           & \wordtype{verb auxiliary}: to know how to                                       \\
    \wordrule %%%%%%%%%%%%%%%%%%%%%%%%%%%%%%%%%%%%%%%%%%%%%%%%%%%%%%%%%%%%%
    kama sona (e \dots{})  & \wordtype{verb transitive}: to learn, to study                                  \\
    \wordrule %%%%%%%%%%%%%%%%%%%%%%%%%%%%%%%%%%%%%%%%%%%%%%%%%%%%%%%%%%%%%
    tan                    & \wordtype{noun}: origin, cause                                                  \\
    \dots{} tan            & \wordtype{adjective}: causal                                                    \\
    \dots{} , tan \dots{}  & \wordtype{preposition}: from, by, because of, since                             \\
    tan                    & \wordtype{verb intransitive}: to come from, originate from, come out of         \\
\end{vocabularytable}

\subsection*{Prepositional Objects and Prepositions}
\index{object!prepositional}
\index{prepositional object}
\index{preposition}
The third object class in Toki Pona is the prepositional object.
A prepositional object begins with a preposition.
A preposition describes a relationship between other words in a sentence and stand in front of nouns or pronouns.
It is closely connected to the predicate.
The preposition determines the case.
The question of the prepositional object depends on the preposition used.
The slot for prepositions is only at the beginning of a prepositional object.
It is recommended that you put a comma before a preposition.
But that's not an official rule.

In the prepositional object the first slot after the preposition is always a noun or pronoun slot.
After that, optional slots for adjectives, possessive pronouns and demonstrative pronouns are possible.
In Toki Pona there is an optional prepositional object at the end of a sentence.
Possible direct or indirect objects are always in front of a prepositional object.
Like the other object types, a prepositional object is an optional part of a predicate phrase.

\index{kepeken@\texttp{kepeken}!preposition}%
The preposition \texttp{kepeken} means ``with'' or ``using''.

\begin{translationtable}
    mi moku, kepeken ilo moku.  & I eat using a fork/spoon/any type of eating utensil. \\
    mi lukin, kepeken ilo suno. & I look using a flashlight.                           \\
\end{translationtable}
%
\index{colon}%
\index{lon@\texttp{lon}!preposition}%
\index{wile@\texttp{wile}!verb}%
\index{wile@\texttp{wile}!auxiliary verb}%
\index{ni@\texttp{ni}}%
The preposition \texttp{lon} means ``be (located) in/at/on''.

\begin{translationtable}
    mi moku, lon tomo.           & I eat in the house.             \\
    mi telo e mi, lon tomo telo. & I bathe myself in the restroom. \\
\end{translationtable}
%
Since there is both preposition \texttp{lon} and the intransitive verb \texttp{lon}, the statement of the following sentences may be confusing.

\begin{translationtable}
    mi wile lon tomo. & I want to be at home.            \\
                      & I want (while being) in a house. \\
\end{translationtable}
%
The sentence has at least two possible translations.
The first translation states that the speaker wishes he were at home.
The second translation states that the speaker wants to do something while being in a house.
After a comma here only the preposition \texttp{lon} is possible.

\begin{translationtable}
    mi wile, lon tomo. & I want (while being) in a house. \\
\end{translationtable}
%
When you say, ``I want to be home.'' you have to divide the sentence with a colon into two sentences.

\begin{translationtable}
    mi wile e ni: mi lon tomo. & I want this: I'm at home. \\
                               & I want to be home.        \\
\end{translationtable}
%
Toki Pona often uses this \texttp{e ni:} trick.
Before and after the colon has to be complete sentences.
Toki Pona has no nested subordinate clauses.

\begin{translationtable}
    sina toki e ni, tawa mi: sina moku. & You told me that you are eating. \\
\end{translationtable}
%
\index{tawa@\texttp{tawa}!preposition}%
\index{tawa@\texttp{tawa}!verb}%
\index{tawa@\texttp{tawa}!adjective}%
In the last sentence there is the preposition \texttp{tawa} after the comma.

\begin{translationtable}
    mi toki, tawa sina.              & I talk to you.                  \\
    ona li lawa e jan, tawa ma pona. & He led people to the good land. \\
    ona li kama, tawa ma mi.         & He's coming to my country.      \\
\end{translationtable}
%
In the following sentences the first \texttp{tawa} is an intransitive verb.
The second \texttp{tawa} is a preposition and initiates the prepositional object.

\begin{translationtable}
    mi tawa, tawa tomo mi.          & I'm going to my house.       \\
    ona mute li tawa, tawa utala.   & They're going to the war.    \\
    sina wile tawa, tawa telo suli. & You want to go to the ocean. \\
    ona li tawa, tawa sewi kiwen.   & She's going up the rock.     \\
\end{translationtable}
%
In the following sentences the first \texttp{tawa} is an transitive verb.
The second \texttp{tawa} is a preposition.

\begin{translationtable}
    mi tawa e mi, tawa tomo mi. & I'm moving myself to my house.   \\
    mi tawa e kiwen, tawa sewi. & I'm moving the rock to the peak. \\
\end{translationtable}
%
\index{pona@\texttp{pona}!I like}%
\index{ike@\texttp{ike}!I don't like}%
In Toki Pona, to say that you (don't) like something, we have pattern, and the pattern use \texttp{tawa} as a preposition.
This is done according to the pattern ``it is good to me'' or ``it is bad to me''.

\begin{translationtable}
    ni li ' pona, tawa mi.       & That is good to me.         \\
                                 & I like that.                \\
    ni li ' ike, tawa mi         & That is bad to me.          \\
                                 & I don't like that.          \\
    kili li ' pona, tawa mi.     & I like fruit.               \\
    toki li ' pona, tawa mi.     & I like talking.             \\
                                 & I like languages.           \\
    utala li ' ike, tawa mi.     & I don't like wars.          \\
    telo suli li ' ike, tawa mi. & I don't like the ocean.     \\
    pipi li ' ike, tawa mi.      & I hate spiders.             \\
    ali li ' pona, tawa mi.      & Everything's fine for me.   \\
    ma ali li ' pona, tawa mi.   & All nations are good to me. \\
\end{translationtable}
%
\index{clauses}%
Toki Pona does not use clauses.
So for example, if you wanted to say ``I like watching the countryside'' it's best to split this into two sentences.

\begin{translationtable}
    mi lukin e ma. ni li ' pona, tawa mi. & I'm watching the countryside. This is good to me. \\
                                          & I like watching the countryside.                  \\
\end{translationtable}
%
Of course, you could choose to say this same sentence using other techniques.

\begin{translationtable}
    ma li pona lukin. & The countryside is good to look at. \\
\end{translationtable}
%
The preposition \texttp{tawa} can also mean ``for''.

\begin{translationtable}
    mi pona e tomo, tawa jan pakala. & I fixed the house for the disabled man. \\
\end{translationtable}
%
\clearpage

\index{tawa@\texttp{tawa}!adjective}
There are ambiguities since \texttp{tawa} can also be used as an adjective.
\texttp{tawa} is used as an adjective to make the phrase we use for ``car'', ``boat'', or ``airplane''.

\begin{translationtable}
    tomo tawa      & car (moving construction) \\
    tomo tawa telo & boat, ship                \\
    tomo tawa kon  & airplane, helicopter      \\
\end{translationtable}
%
Consider the following sentence.

\begin{translationtable}
    mi pana e tomo tawa sina. & ? \\
\end{translationtable}
%
After \texttp{mi pana e tomo}, both an adjective slot and a preposition slot are possible.

With the adjective \texttp{tawa} the sentence means ``I gave your car.''.
With the preposition \texttp{tawa}, however, the sentence means ``I gave the house to you.''.
You can insert a comma before \texttp{tawa} to force only a slot for preposition.
It is better to split the sentence.

\begin{translationtable}
    mi jo e tomo tawa sina. mi pana e ni tawa sina. & I have your car. I give it to you. \\
    ni li tomo. mi pana e ni tawa sina.             & This is a house. I give it to you. \\
\end{translationtable}
%
\index{kama@\texttp{kama}!intransitive verb}
In this set the intransitive verb \texttp{kama} and the preposition \texttp{tawa} is used.

\begin{translationtable}
    ona li kama, tawa tomo mi. & He came to my house. \\
\end{translationtable}
%
\index{sama@\texttp{sama}!preposition}
\index{sama@\texttp{sama}!adjective}
The preposition \texttp{sama} means ``like'', ``as'', or ``seem''.

\begin{translationtable}
    ona li lukin, sama pipi. & He looks like a bug. \\
\end{translationtable}
%
On the other hand, directly after the separator \texttp{li} no preposition can follow.
There would be no predicate.
The adjective \texttp{sama} is used here as a predicate adjective.

\begin{translationtable}
    jan ni li ' sama mi. & That person is like me. \\
\end{translationtable}
%
\index{\textit{tan}!preposition}
The preposition \textit{tan} means ``from'', ``by'', ``because of'', or ``since''.

\begin{translationtable}
    mi moku, tan ni: mi wile moku. & I eat because I'm hungry. \\
\end{translationtable}

\subsection*{Indirect Objects vs.\@ Prepositional Objects}
\index{object!indirect vs.\@ prepositional}
\index{indirect object!vs.\@ prepositional object}
\index{prepositional object!vs.\@ indirect object}
\index{intransitive verb!and auxiliary verb}
\index{auxiliary verb!and intransitive verb}
\index{kepeken@\texttp{kepeken}!preposition}
\index{kepeken@\texttp{kepeken}!intransitive verb}
Neither indirect objects nor prepositional objects are directly influenced by the predicate.
Prepositional objects are therefore a special form of indirect objects.
In the following example the indirect object \texttp{ilo ni} is used with the intransitive verb \texttp{kepeken}.

\begin{translationtable}
    mi pona e tomo tawa. mi kepeken ilo ni. & I repair the car. I use this tool. \\
\end{translationtable}
%
It is possible to formulate the statement shorter and more clearly, if the preposition \texttp{kepken} introduces the prepositional object \texttp{ilo ni}.

\begin{translationtable}
    mi pona e tomo tawa, kepeken ilo ni. & I repair the car with this tool. \\
\end{translationtable}
%
However, if one absolutely wants to use this tool, one must use the intransitive verb \texttp{kepeken}.
Auxiliary verbs can only be used with verbs and not with prepositions.
Before the intransitive verb \texttp{kepeken} auxiliary verb \texttp{wile} is used here.

\begin{translationtable}
    mi pona e tomo tawa. mi wile kepeken ilo ni. & I repair the car. I want to use this tool. \\
\end{translationtable}
%
\index{tawa@\texttp{tawa}!preposition}%
\index{tawa@\texttp{tawa}!intransitive verb}%
\index{tawa@\texttp{tawa}!transitive verb}%
Consider the intransitive verb \texttp{tawa}.

\begin{translationtable}
    mi tawa sina. & I'll go to you. \\
                  & I'll leave you. \\
\end{translationtable}
%
This sentence is ambiguous.
After \texttp{mi} both a noun (predicate noun) slot and an adjective slot (predicate adjective) are possible.

\begin{translationtable}
    mi tawa sina. & I am your movement. \\
\end{translationtable}
%
It is better to use a prepositional object.
If, as recommended in these lessons, a comma is placed before the preposition, the sentence becomes clearer.

\begin{translationtable}
    mi tawa, tawa sina. & I'll visit you. \\
    mi tawa, tan sina.  & I'll leave you. \\
\end{translationtable}
%
% On closer inspection it is noticeable that \texttp{tawa} here is no intransitive verb at all.
It is also possible to formulate the sentence with the reflexive pronoun \texttp{mi} as a direct object.
The first \texttp{tawa} is here a transitive verb.
The second \texttp{tawa} is a preposition.

\begin{translationtable}
    mi tawa e mi, tawa sina. & I move towards you. \\
\end{translationtable}

\practice{
    What is closely connected to a preposition?                                  & \answer{A preposition is closely connected to the verb.} \\\wordrule
    Which part of the sentence does the prepositional object belong to?          & \answer{It is an optional part of a predicate phrase.}   \\\wordrule
    Where are preposition slots located?                                         & \answer{At the beginning of a prepositional object.}     \\\wordrule
    At which position in the sentence is a prepositional object be located?      & \answer{At the end of a sentence.}                       \\\wordrule
    Which separators can be used to form composite sentences?                    & \answer{With the separators \texttp{li} and \texttp{e}.} \\\wordrule
    Which slots are possible in the second position in the prepositional object? & \answer{A noun or pronoun slot.}                         \\
}

\practice[Try to translate these sentences.]{
    \answer{\texttp{mi pona e ilo suno, kepeken ilo lili.}}        & I fixed the flashlight using a small tool.                                              \\\wordrule
    \answer{\texttp{toki pona li ' pona, tawa mi.}}                & I like Toki Pona.                                                                       \\\wordrule
    \answer{\texttp{mi mute li pana e moku, tawa ona mute.}}       & We gave them food.                                                                      \\\wordrule
    \answer{\texttp{mi wile tawa tomo ona, kepeken tomo tawa mi.}} & I want to go to his house using my car.                                                 \\\wordrule
    \answer{\texttp{jan li lukin, sama pipi.}}                     & People look like ants.                                                                  \\\wordrule
    \texttp{sina wile kama, tawa tomo toki.}                       & \answer{You should come to the chat room.}                                              \\\wordrule
    \texttp{jan li toki, kepeken toki pona, lon tomo toki.}        & \answer{People talk in/using Toki Pona in the chat room.}                               \\\wordrule
    \texttp{mi tawa, tawa tomo toki. ona li ' pona, tawa mi.}      & \answer{I go the chat room. It is good for me. \newline I like to go to the chat room.} \\\wordrule
    \texttp{sina kama jo e jan pona, lon ni.}                      & \answer{You will get friends there.}                                                    \\\wordrule
    \texttp{sama li ' pona.}                                       & \answer{Equality is good.}                                                              \\\wordrule % Was ``Equality is bad'', but that was an example before, added this for balance ;)
    \texttp{mi sona e tan.}                                        & \answer{I know the reason. / I know why.}                                               \\
}
