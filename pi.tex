%%%%%%%%%%%%%%%%%%%%%%%%%%%%%%%%%%%%%%%%%%%%%%%%%%%%%%%%%%%%%%%%%%%%%%%%%%
\section{Compound Nouns with \textit{pi}}
%%%%%%%%%%%%%%%%%%%%%%%%%%%%%%%%%%%%%%%%%%%%%%%%%%%%%%%%%%%%%%%%%%%%%%%%%%
%
%%%%%%%%%%%%%%%%%%%%%%%%%%%%%%%%%%%%%%%%%%%%%%%%%%%%%%%%%%%%%%%%%%%%%%%%%%
\subsection*{Vocabulary}
%%%%%%%%%%%%%%%%%%%%%%%%%%%%%%%%%%%%%%%%%%%%%%%%%%%%%%%%%%%%%%%%%%%%%%%%%%
%
\index{\textit{pi}}
\index{\textit{kalama}}
\index{\textit{kulupu}}
\index{\textit{nasin}}
\index{of}
\index{sound}
\index{noise}
\index{noise}
\index{group}
\index{community}
\index{society}
\index{road}
\index{way}
\index{doctrine}
\index{method}
\begin{supertabular}{p{2,5cm}|ll}
\textbf{\dots pi \dots } && \textit{separator}: 'pi' is used to build complex compound nouns. \\ && 'pi' separates a (pro)noun from another (pro)noun that has at least one adjective. \\ && After 'pi' could only be a noun or pronoun. \\ && Don't use 'pi' before or after \\ && the other separators 'e', 'la', 'li', 'o', '.', '!', '?', ':'.  \\ % no-dictionary 
 && \\ % no-dictionary
\textbf{\dots kalama} && \textit{adjective}: noisy, loud, rowdy \\ % no-dictionary
\textbf{kalama} && \textit{noun}: sound, noise, voice \\ 
\textbf{kalama} && \textit{verb intransitive}: to make noise \\ % no-dictionary
\textbf{kalama (e \dots)} && \textit{verb transitive}: to sound, to ring, to play (an instrument) \\ % no-dictionary
 && \\ % no-dictionary
\textbf{\dots kulupu} && \textit{adjective}: communal, shared, public, of the society \\ % no-dictionary
\textbf{kulupu} && \textit{noun}: group, community, society, company, people \\ % no-dictionary
\textbf{kulupu (e \dots)} && \textit{verb transitive}: to assemble, to call together, to convene \\ % no-dictionary
 && \\ % no-dictionary
\textbf{\dots nasin} && \textit{adjective}: systematic, habitual, customary, doctrinal \\ % no-dictionary
\textbf{nasin} && \textit{noun}: way, manner, custom, road, path, doctrine, system, method \\ % no-dictionary
 && \\ % no-dictionary
\end{supertabular} \\
%
%%%%%%%%%%%%%%%%%%%%%%%%%%%%%%%%%%%%%%%%%%%%%%%%%%%%%%%%%%%%%%%%%%%%%%%%%%
\index{\textit{pi}}
\subsection*{\textit{pi} separates a noun from another noun that has at least one adjective.}
%%%%%%%%%%%%%%%%%%%%%%%%%%%%%%%%%%%%%%%%%%%%%%%%%%%%%%%%%%%%%%%%%%%%%%%%%%

Until now we have learned to combine a noun and adjectives. 
Now we will learn how to combine two nouns including their adjectives into a composite noun. 
\textit{pi} is probably the most misused word in this language. 
At any rate, if you get frustrated while trying to learn about this word, just relax, try your best. 
After \textit{pi} have to be at least two words. 
For example \textit{pi} + noun + adjectiv or \textit{pi} + pronoun + adjectiv. 
\textit{pi} and \textit{li} or \textit{e} stand never together.
%
%%%%%%%%%%%%%%%%%%%%%%%%%%%%%%%%%%%%%%%%%%%%%%%%%%%%%%%%%%%%%%%%%%%%%%%%%%
\index{\textit{pi}!separate nouns}
\index{noun!separate nouns with \textit{pi}}
\index{water room}
\index{restroom}
\index{bar}
\index{pub}
\index{alcohol}
\subsubsection*{Compound words with \textit{pi}} 
%%%%%%%%%%%%%%%%%%%%%%%%%%%%%%%%%%%%%%%%%%%%%%%%%%%%%%%%%%%%%%%%%%%%%%%%%%
%
Now, you might remember that \textit{tomo telo} ('water room') is used to mean 'restroom'. 
You should also recall that \textit{nasa} means 'crazy', 'silly', 'stupid', and so on. 
Now, let's look at this sentence.

\begin{supertabular}{p{5,5cm}|ll}
mi tawa tomo telo nasa. && I went to the crazy restroom. \\ % no-dictionary
\end{supertabular}  

Okay, I think you'll agree with me when I say that that is just plain weird. 
It makes me think about some creepy restroom with neon lights lining the floor and a strobe light in every toilet stall. 
Now, the person who said this sentence had actually been trying to say that he had gone to a bar. 
As you probably recall, \textit{telo nasa} is used to mean 'alcohol'. 
So, a \textit{tomo} with \textit{telo nasa} would be a 'bar'. 
The only problem is that you can't fit \textit{tomo} and \textit{telo nasa} together, because it will mean 'crazy restroom,' as you just studied. 
The only way to fix this problem is to use \textit{pi}.

\begin{supertabular}{p{5,5cm}|ll}
tomo pi telo nasa && building for alcohol; a bar, pub \\
\end{supertabular}  

We're going to go over a bunch of examples using \textit{pi}; but, you need to be familiar with some of the compound noun combinations that we've learned. 

\begin{supertabular}{p{5,5cm}|ll}
jan pi ma tomo && a city-dweller \\
kulupu pi toki pona && the Toki Pona community \\
nasin pi toki pona && the ideology behind Toki Pona \\
jan lawa pi jan utala && commander, general \\
jan lawa pi tomo tawa kon && a pilot \\
jan pi nasin sewi Kolisu && a Christian \\
nasin sewi && religion \\
jan pi pona lukin && an attractive person \\
jan pi ike lukin && an ugly person \\
jan utala pi ma Losi li ike tawa ma ali. && Soldiers of Russia are bad for the world. \\
\end{supertabular}  
%
%%%%%%%%%%%%%%%%%%%%%%%%%%%%%%%%%%%%%%%%%%%%%%%%%%%%%%%%%%%%%%%%%%%%%%%%%%
\index{\textit{pi}!possessive}
\index{possessive!\textit{pi}}
\index{possession!\textit{pi}}
\subsubsection*{\textit{pi} and possessives} 
%%%%%%%%%%%%%%%%%%%%%%%%%%%%%%%%%%%%%%%%%%%%%%%%%%%%%%%%%%%%%%%%%%%%%%%%%%
%
If you wanted to say 'my house' you say \textit{tomo mi}. 
Similarly, 'your house' is \textit{tomo sina}. 
If you want to name a specific person who owns the house, you have to use \textit{pi}. 
Note that you can not say \textit{tomo Lisa}. 
That changes the whole meaning. 

\begin{supertabular}{p{5,5cm}|ll}
tomo pi jan Lisa && Lisa's house \\
kili pi jan Susan && Susan's fruit \\
ma pi jan Keli && Keli's country \\
len pi jan Lisa && Lisa's clothes \\
\end{supertabular}  

\index{pronoun!\textit{pi}}
\index{\textit{pi}!pronoun!plural}
\index{plural!pronoun!\textit{pi}}
Also, if you want to use the plural pronouns you have to use \textit{pi}.

\begin{supertabular}{p{5,5cm}|ll}
nimi pi mi mute && our names \\
tomo pi ona mute && their house \\
\end{supertabular}  

%%%%%%%%%%%%%%%%%%%%%%%%%%%%%%%%%%%%%%%%%%%%%%%%%%%%%%%%%%%%%%%%%%%%%%%%%%
\index{opposite!\textit{pi}}
\index{\textit{pi}!opposite}
\subsubsection*{Opposites with \textit{pi} and \textit{ala}}
%%%%%%%%%%%%%%%%%%%%%%%%%%%%%%%%%%%%%%%%%%%%%%%%%%%%%%%%%%%%%%%%%%%%%%%%%%

We also use \textit{pi} and \textit{ala} to express the opposite of some words. 
This could change the word type. In the first examples \textit{wawa} is a
adjectiv. But after \textit{pi} is \textit{wawa} a noun.

\begin{supertabular}{p{5,5cm}|ll}
jan wawa && a strong person \\
jan wawa ala && No strong people. \\ 
jan pi wawa ala && a person with weakness, a weak person \\
\end{supertabular}  

%%%%%%%%%%%%%%%%%%%%%%%%%%%%%%%%%%%%%%%%%%%%%%%%%%%%%%%%%%%%%%%%%%%%%%%%%%
\index{pi!several}
\subsubsection*{Several \textit{pi}-Phrases for one Compound Noun} 
%%%%%%%%%%%%%%%%%%%%%%%%%%%%%%%%%%%%%%%%%%%%%%%%%%%%%%%%%%%%%%%%%%%%%%%%%%

Similar to the other separate words \textit{li} and \textit{e} you can use several \textit{pi}. 
All additional \textit{pi}-phrases are connected to the first noun as well. 
\textit{pi} phrases are not nested. 
You can change the order. 
However you should avoid several \textit{pi} if you can. 
In the next lesson we will learn a way to avoid several pi.

\begin{supertabular}{p{5,5cm}|ll}
kulupu pi kalama musi pi ma Inli li pona. &&  The English rock band is good. \\ 
kulupu pi ma Inli pi kalama musi li pona. &&  The English rock band is good. \\ 
\end{supertabular}

% 
%%%%%%%%%%%%%%%%%%%%%%%%%%%%%%%%%%%%%%%%%%%%%%%%%%%%%%%%%%%%%%%%%%%%%%%%%%
\index{\textit{pi}!mistake}
\index{mistake!\textit{pi}}
\index{about!mistake!\textit{pi}}
\label{'mistakes_with_pi'}
\subsubsection*{Common mistakes with \textit{pi}}
%%%%%%%%%%%%%%%%%%%%%%%%%%%%%%%%%%%%%%%%%%%%%%%%%%%%%%%%%%%%%%%%%%%%%%%%%%
%
After \textit{pi} have to be at least two words. 
The word immediately after \textit{pi} is a noun or pronoun, followed by an adjectiv.
Do not separate adjectives, numbers or verbs by \textit{pi}.

\begin{supertabular}{p{5,5cm}|ll}
jan \sout{pi wawa} pi pona mute li kama. && Wrong! \\ % no-dictionary
\end{supertabular}

% People try to use \textit{pi} to mean 'about' (as in 'We talked about something.'). 
% While \textit{pi} can be used in this way, most people use it too much. 
%The 'about' is simply implied by the sentence. 
%
%\begin{supertabular}{p{5,5cm}|ll}
% mi toki jan. && I talked about people \\
%\end{supertabular}  
%
%Now here's another sentence which is correct and in which \textit{pi} is used to mean 'about'.
%
%\begin{supertabular}{p{5,5cm}|ll}
%mi toki pi jan ike. && I talked about bad people. \\
%\end{supertabular}  
%
%The reason that \textit{pi} can be used here is because \textit{jan ike} is its own singular, individual concept, and the combined phrase (\textit{jan ike}) acts on \textit{toki} as one thing; \textit{pi} is simply used to distinguish the \textit{jan ike} phrase. 
%If you left out \textit{pi}, both \textit{jan} and \textit{ike} would become adverbs, and the sentence would mean something really strange like 'Humanely, I talked evilly.' 
%
Another mistake is that people use \textit{pi} when they should use \textit{tan}. 

\begin{supertabular}{p{5,5cm}|ll}
mi kama tan ma Mewika. && I come from America. \\
\end{supertabular}  

At the beginning \textit{pi} is unfamiliar. 
But it helps to understand a sentence. \\
A \textit{pi} shows that after the \textit{pi} can only be a noun or pronoun.

%
%%%%%%%%%%%%%%%%%%%%%%%%%%%%%%%%%%%%%%%%%%%%%%%%%%%%%%%%%%%%%%%%%%%%%%%%%%
% \newpage
\subsection*{Miscellaneous}
\index{\textit{kalama}}
\index{\textit{kalama}!\textit{seme}}
\index{\textit{kalama}!\textit{musi}}
\index{\textit{musi}!\textit{kalama}}
\index{\textit{seme}!\textit{kalama}}
\index{sound}
\index{noise}
%%%%%%%%%%%%%%%%%%%%%%%%%%%%%%%%%%%%%%%%%%%%%%%%%%%%%%%%%%%%%%%%%%%%%%%%%%
%
%%%%%%%%%%%%%%%%%%%%%%%%%%%%%%%%%%%%%%%%%%%%%%%%%%%%%%%%%%%%%%%%%%%%%%%%%%
\subsubsection*{\textit{kalama}}
%%%%%%%%%%%%%%%%%%%%%%%%%%%%%%%%%%%%%%%%%%%%%%%%%%%%%%%%%%%%%%%%%%%%%%%%%%
%
Okay, \textit{kalama} is used to mean 'sound' or 'noise'.  \\
\begin{supertabular}{p{5,5cm}|ll}
kalama ni li ' seme? && What was that noise? \\
\end{supertabular}  

\textit{kalama} is usually combined with the word \textit{musi} to mean 'music' or 'song'. \\
\begin{supertabular}{p{5,5cm}|ll}
kalama musi li ' pona, tawa mi. && I like music. \\
\end{supertabular}  

\textit{kalama musi} precedes the names of specific songs. \\
\begin{supertabular}{p{5,5cm}|ll}
kalama musi 'Jingle Bells' li ' pona, tawa mi. && I like the song 'Jingle Bells'. \\
\end{supertabular}  

And we can use \textit{pi} to talk about music by a certain group or artist. \\
\begin{supertabular}{p{5,5cm}|ll}
kalama musi pi jan Elton-John li ' nasa. && Elton John's music is odd. \\
\end{supertabular}  

Finally, \textit{kalama} can be used as a verb. \\
\begin{supertabular}{p{5,5cm}|ll}
mi kalama, kepeken ilo. && I make noise using an instrument. \\
o kalama ala! && Don't make noise! \\
\end{supertabular}  
%
%%%%%%%%%%%%%%%%%%%%%%%%%%%%%%%%%%%%%%%%%%%%%%%%%%%%%%%%%%%%%%%%%%%%%%%%%%
\index{how}
\index{\textit{nasin}!\textit{seme}}
\index{\textit{seme}!\textit{nasin}}
\subsubsection*{Using \textit{nasin seme} to make 'how'}
%%%%%%%%%%%%%%%%%%%%%%%%%%%%%%%%%%%%%%%%%%%%%%%%%%%%%%%%%%%%%%%%%%%%%%%%%%
%
\begin{supertabular}{p{5,5cm}|ll}
sina pali e ni kepeken nasin seme? && How did you make this? \\
\end{supertabular}  
%
%
%%%%%%%%%%%%%%%%%%%%%%%%%%%%%%%%%%%%%%%%%%%%%%%%%%%%%%%%%%%%%%%%%%%%%%%%%%
\newpage
\subsection*{Practice 11 (Answers: Page~\pageref{'pi'})}
%%%%%%%%%%%%%%%%%%%%%%%%%%%%%%%%%%%%%%%%%%%%%%%%%%%%%%%%%%%%%%%%%%%%%%%%%%
%
Try to translate these sentences. 
You can use the tool \textit{Toki Pona Parser} (\cite{www:rowa:02}) for spelling and grammar check. 

\begin{supertabular}{p{5,5cm}|ll}
Keli's child is funny.    \\ % no-dictionary
I am a Toki Ponan.   \\  % no-dictionary
He is a good musician.   \\  % no-dictionary
The captain of the ship is eating.    \\ % no-dictionary
Meow.    \\ % no-dictionary
Enya's music is good.    \\ % no-dictionary
Which people of this group are important?    \\ % no-dictionary
Our house is messed up.    \\ % no-dictionary
How did she make that?    \\ % no-dictionary
I look at the land with my friend. && \\ % no-dictionary
Whom did you go with? &&  \\  % no-dictionary
 && \\ % no-dictionary
pipi pi ma mama mi li ' lili.  \\ % no-dictionary
kili pi jan Linta li ' ike.    \\ % no-dictionary
len pi jan Susan li ' jaki.    \\ % no-dictionary
mi sona ala e nimi pi ona mute.    \\ % no-dictionary
mi wile toki meli.    \\ % no-dictionary
sina pakala e ilo, kepeken nasin seme?    \\ % no-dictionary
jan Wasintan [Washington] li ' jan lawa pona pi ma Mewika.  \\   % no-dictionary
wile pi jan ike li pakala e ijo.    \\ % no-dictionary
\end{supertabular}  
%%%%%%%%%%%%%%%%%%%%%%%%%%%%%%%%%%%%%%%%%%%%%%%%%%%%%%%%%%%%%%%%%%%%%%%%%%
% eof
