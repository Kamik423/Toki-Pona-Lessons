%%%%%%%%%%%%%%%%%%%%%%%%%%%%%%%%%%%%%%%%%%%%%%%%%%%%%%%%%%%%%%%%%%%%%%%%%%
\section{Compound Nouns}
%%%%%%%%%%%%%%%%%%%%%%%%%%%%%%%%%%%%%%%%%%%%%%%%%%%%%%%%%%%%%%%%%%%%%%%%%%
%
%%%%%%%%%%%%%%%%%%%%%%%%%%%%%%%%%%%%%%%%%%%%%%%%%%%%%%%%%%%%%%%%%%%%%%%%%%
\subsection*{Vocabulary}
%%%%%%%%%%%%%%%%%%%%%%%%%%%%%%%%%%%%%%%%%%%%%%%%%%%%%%%%%%%%%%%%%%%%%%%%%%
%
\begin{supertabular}{p{2,5cm}|ll}
%
\index{kalama}
\textbf{\dots kalama} && \textit{adjective}: noisy, loud, rowdy \\ % no-dictionary
\textbf{kalama} && \textit{noun}: sound, noise, voice \\ 
\textbf{kalama} && \textit{verb intransitive}: to make noise \\ % no-dictionary
\textbf{kalama (e \dots)} && \textit{verb transitive}: to sound, to ring, to play (an instrument) \\ % no-dictionary
 && \\ % no-dictionary
%
\index{kulupu}
\textbf{\dots kulupu} && \textit{adjective}: communal, shared, public, of the society \\ % no-dictionary
\textbf{kulupu} && \textit{noun}: group, community, society, company, people \\ % no-dictionary
\textbf{kulupu (e \dots)} && \textit{verb transitive}: to assemble, to call together, to convene \\ % no-dictionary
 && \\ % no-dictionary
%
\index{nasin}
\textbf{\dots nasin} && \textit{adjective}: systematic, habitual, customary, doctrinal \\ % no-dictionary
\textbf{nasin} && \textit{noun}: way, manner, custom, road, path, doctrine, system, method \\ % no-dictionary
 && \\ % no-dictionary
%
\index{pi}
\textbf{\dots pi \dots } && \textit{separator}: 'pi' is used to build complex compound nouns. \\ && 'pi' separates a (pro)noun from another (pro)noun that has at least one adjective. \\ && After 'pi' could only be a noun or pronoun. \\ && Don't use 'pi' before or after \\ && the other separators 'e', 'la', 'li', 'o', '.', '!', '?', ':'.  \\ % no-dictionary 
\end{supertabular} \\
%
%%%%%%%%%%%%%%%%%%%%%%%%%%%%%%%%%%%%%%%%%%%%%%%%%%%%%%%%%%%%%%%%%%%%%%%%%%
\newpage
%
\subsection*{The Separator \textit{pi}}
%
\index{\textit{pi}}
\index{noun!compound}
\index{compound noun}
%%%%%%%%%%%%%%%%%%%%%%%%%%%%%%%%%%%%%%%%%%%%%%%%%%%%%%%%%%%%%%%%%%%%%%%%%%

So far we have learned how to combine a single noun with adjectives. 
Adjectives stand after the noun. 
This is exactly the opposite of the English language.
These possibilities are not sufficient for many terms. 
The English language knows compound nouns consisting of several nouns. 
In \textit{toki pona} this is also possible. 
It is possible to combine several nouns including their adjectives. 
However, the order is exactly the opposite here as in the English language. 
The main noun in English is at the end of the compound noun. 
For example the compound noun 'toothbrush'. Here 'brush' is the main noun. 
After all, it is a brush and not a tooth. 

In \textit{toki pona} the main noun is at the beginning.
This is followed by the supplementary nouns and their adjectives. 
The separator \textit{pi} serves to separate these supplementary nouns and to mark them as nouns. 
After the separator \textit{pi} must follow at least two words. 
For example \textit{pi} + noun + adjective or \textit{pi} + pronoun + adjective. 
That is, after the separator \textit{pi} only a noun or pronoun slot is possible. 

The separator \textit{pi} must not stand together with the separators \textit{li} or \textit{e}.
%
%%%%%%%%%%%%%%%%%%%%%%%%%%%%%%%%%%%%%%%%%%%%%%%%%%%%%%%%%%%%%%%%%%%%%%%%%%
\subsubsection*{General examples} 
%%%%%%%%%%%%%%%%%%%%%%%%%%%%%%%%%%%%%%%%%%%%%%%%%%%%%%%%%%%%%%%%%%%%%%%%%%
%
Now, you might remember that \textit{tomo telo} ('water room') is used to mean 'restroom'. 
You should also recall that \textit{nasa} means 'crazy', 'silly', 'stupid', and so on. 
Now, let's look at this sentence.

\begin{supertabular}{p{5,5cm}|ll}
mi tawa, tawa tomo telo nasa. && I went to the crazy restroom. \\ % no-dictionary
\end{supertabular}  

Okay, I think you'll agree with me when I say that that is just plain weird. 
It makes me think about some creepy restroom with neon lights lining the floor and a strobe light in every toilet stall. 
Now, the person who said this sentence had actually been trying to say that he had gone to a bar. 
As you probably recall, \textit{telo nasa} is used to mean 'alcohol'. 
So, a \textit{tomo} with \textit{telo nasa} would be a 'bar'. 
The only problem is that you can't fit \textit{tomo} and \textit{telo nasa} together, because it will mean 'crazy restroom,' as you just studied. 
The only way to fix this problem is to use the separator \textit{pi}.

\begin{supertabular}{p{5,5cm}|ll}
mi tawa, tawa tomo pi telo nasa. && I went to the pub. \\ 
\end{supertabular}  

We're going to go over a bunch of examples using \textit{pi}; but, you need to be familiar with some of the compound noun combinations that we've learned. 

\begin{supertabular}{p{5,5cm}|ll}
jan pi ma tomo && a city-dweller \\
kulupu pi toki pona && the Toki Pona community \\
nasin pi toki pona && the ideology behind Toki Pona \\
jan lawa pi jan utala && commander, general \\
jan lawa pi tomo tawa kon && a pilot \\
jan pi nasin sewi Kolisu && a Christian \\
jan pi pona lukin && an attractive person \\
jan pi ike lukin && an ugly person \\
jan utala pi ma Losi li ike, tawa ma ali. && Soldiers of Russia are bad for the world. \\
\end{supertabular} 

%
%%%%%%%%%%%%%%%%%%%%%%%%%%%%%%%%%%%%%%%%%%%%%%%%%%%%%%%%%%%%%%%%%%%%%%%%%%
\newpage
%
\subsubsection*{Possessives}
\index{possessive}
\index{property}
\index{ownership}
%%%%%%%%%%%%%%%%%%%%%%%%%%%%%%%%%%%%%%%%%%%%%%%%%%%%%%%%%%%%%%%%%%%%%%%%%%

In Toki Pona also compound nouns are used to identify property. 
If you wanted to say 'my house' you say \textit{tomo mi}. 
Similarly, 'your house' is \textit{tomo sina}. 
If you want to name a specific person who owns the house, you have to use the separator \textit{pi}. 

\begin{supertabular}{p{5,5cm}|ll}
tomo pi jan Lisa && Lisa's house \\
kili pi jan Susan && Susan's fruit \\
ma pi jan Keli && Keli's country \\
len pi jan Lisa && Lisa's clothes \\
\end{supertabular}  

Also, if you want to use the plural pronouns you have to use the separator \textit{pi}.

\begin{supertabular}{p{5,5cm}|ll}
nimi pi mi mute && our names \\
tomo pi ona mute && their house \\
\end{supertabular}  
%
%%%%%%%%%%%%%%%%%%%%%%%%%%%%%%%%%%%%%%%%%%%%%%%%%%%%%%%%%%%%%%%%%%%%%%%%%%
\subsubsection*{Opposites}
%
\index{opposite}
\index{negation}
\index{antonym}
\index{\textit{ala}!adjective}
%%%%%%%%%%%%%%%%%%%%%%%%%%%%%%%%%%%%%%%%%%%%%%%%%%%%%%%%%%%%%%%%%%%%%%%%%%

Composite nouns are also used to formulate the opposite of a word or group of words. 
The separator \textit{pi}, the word or group of words and the adjective \textit{ala} is used. 
This could change the word type. 
In the first examples \textit{wawa} is a adjectiv. 
But after the separator \textit{pi} only a noun or pronoun slot is possible.
So \textit{wawa} can only be a noun here.

\begin{supertabular}{p{5,5cm}|ll}
jan wawa && a strong person \\
jan pi wawa ala && a person with weakness, a weak person \\
jan wawa ala && No strong people. \\ 
\end{supertabular}  

%
%%%%%%%%%%%%%%%%%%%%%%%%%%%%%%%%%%%%%%%%%%%%%%%%%%%%%%%%%%%%%%%%%%%%%%%%%%
\subsubsection*{Whose} 
%
\index{whose}
%%%%%%%%%%%%%%%%%%%%%%%%%%%%%%%%%%%%%%%%%%%%%%%%%%%%%%%%%%%%%%%%%%%%%%%%%%

A compound noun is also used for questions of ownership. 
In this case after the separator \textit{pi} follows a noun \textit{jan} and the question pronoun \textit{seme} as representative of adjective. 

\begin{supertabular}{p{5,5cm}|ll}
ni li tomo pi jan seme? && Whose house is this? \\
\end{supertabular}

%
%%%%%%%%%%%%%%%%%%%%%%%%%%%%%%%%%%%%%%%%%%%%%%%%%%%%%%%%%%%%%%%%%%%%%%%%%%
\subsubsection*{Several \textit{pi} Phrases for one Compound Noun} 
%
\index{pi!several}
%%%%%%%%%%%%%%%%%%%%%%%%%%%%%%%%%%%%%%%%%%%%%%%%%%%%%%%%%%%%%%%%%%%%%%%%%%

The English language knows compound nouns consisting of more than two nouns. 
For example, the word 'open source software'.  
Here too, the last noun is the main noun. 
After all, it is software. 

In \textit{toki pona} several \textit{pi} phrases for a main noun are possible. 
This is similar to the other separators \textit{li} and \textit{e}. 
(Multiple predicate phrases (\textit{li}) belong to one subject. 
Several direct objects (\textit{e}) belong to one predicate.
Accordingly, all further \textit{pi} phrases are associated with the first noun. 
So \textit{pi} phrases are not nested. 
You can change the order. 
However you should avoid several \textit{pi} phrases if you can. 
In the next lesson we will learn a way to avoid several \textit{pi} phrases.

\begin{supertabular}{p{5,5cm}|ll}
kulupu pi kalama musi pi ma Inli li pona. &&  The English rock band is good. \\ 
kulupu pi ma Inli pi kalama musi li pona. &&  The English rock band is good. \\ 
\end{supertabular}

% 
%%%%%%%%%%%%%%%%%%%%%%%%%%%%%%%%%%%%%%%%%%%%%%%%%%%%%%%%%%%%%%%%%%%%%%%%%%
\newpage
%
\subsubsection*{Common mistakes with \textit{pi}}
\label{'mistakes_with_pi'}
%
%%%%%%%%%%%%%%%%%%%%%%%%%%%%%%%%%%%%%%%%%%%%%%%%%%%%%%%%%%%%%%%%%%%%%%%%%%

After the separator \textit{pi} have to be at least two words. 
The word immediately after the separator \textit{pi} is a noun or pronoun, followed by an adjectiv.

\begin{supertabular}{p{5,5cm}|ll}
jan \sout{pi wawa} pi pona mute li kama. && Wrong! \\ % no-dictionary
\end{supertabular}

The \textit{pi} before \textit{wawawa} is wrong. 
Right is: 

\begin{supertabular}{p{5,5cm}|ll}
jan wawa pi pona mute li kama. && A strong, very good man is coming. \\ 
\end{supertabular}

\index{\textit{tan}!preposition}
Another mistake is that people use the Separator \textit{pi} when they should use the preposition \textit{tan}. 

\begin{supertabular}{p{5,5cm}|ll}
mi kama, tan ma Mewika. && I come from America. \\
\end{supertabular}  

Do not separate adjectives, numbers or verbs by \textit{pi}.

At the beginning the separator \textit{pi} is unfamiliar. 
But it helps to understand a sentence. \\
A \textit{pi} shows that after the \textit{pi} can only be a noun or pronoun.

%
%%%%%%%%%%%%%%%%%%%%%%%%%%%%%%%%%%%%%%%%%%%%%%%%%%%%%%%%%%%%%%%%%%%%%%%%%%
\newpage
\subsection*{Miscellaneous}
%%%%%%%%%%%%%%%%%%%%%%%%%%%%%%%%%%%%%%%%%%%%%%%%%%%%%%%%%%%%%%%%%%%%%%%%%%
%
%%%%%%%%%%%%%%%%%%%%%%%%%%%%%%%%%%%%%%%%%%%%%%%%%%%%%%%%%%%%%%%%%%%%%%%%%%
\subsubsection*{\textit{kalama}}
%
\index{\textit{kalama}!noun}
%%%%%%%%%%%%%%%%%%%%%%%%%%%%%%%%%%%%%%%%%%%%%%%%%%%%%%%%%%%%%%%%%%%%%%%%%%
%
Okay, \textit{kalama} is used to mean 'sound' or 'noise'.  

\begin{supertabular}{p{5,5cm}|ll}
kalama ni li ' seme? && What was that noise? \\
\end{supertabular}  

\textit{kalama} is usually combined with the adjectiv \textit{musi} to mean 'music' or 'song'. 

\begin{supertabular}{p{5,5cm}|ll}
kalama musi li ' pona, tawa mi. && I like music. \\
\end{supertabular}  

\textit{kalama musi} precedes the names of specific songs. 

\begin{supertabular}{p{5,5cm}|ll}
kalama musi 'Jingle Bells' li ' pona, tawa mi. && I like the song 'Jingle Bells'. \\
\end{supertabular}  

And we can use the separator \textit{pi} to talk about music by a certain group or artist. 

\begin{supertabular}{p{5,5cm}|ll}
kalama musi pi jan Elton-John li ' nasa. && Elton John's music is odd. \\
\end{supertabular}  

\index{\textit{kalama}!verb}
Finally, \textit{kalama} can be used as a verb.

\begin{supertabular}{p{5,5cm}|ll}
mi kalama, kepeken ilo. && I make noise using an instrument. \\
o kalama ala! && Don't make noise! \\
\end{supertabular}  
%
%%%%%%%%%%%%%%%%%%%%%%%%%%%%%%%%%%%%%%%%%%%%%%%%%%%%%%%%%%%%%%%%%%%%%%%%%%
\subsubsection*{Using \textit{nasin seme} to make 'how'}
%
\index{how}
%%%%%%%%%%%%%%%%%%%%%%%%%%%%%%%%%%%%%%%%%%%%%%%%%%%%%%%%%%%%%%%%%%%%%%%%%%
%
\begin{supertabular}{p{5,5cm}|ll}
sina pali e ni kepeken nasin seme? && How did you make this? \\
\end{supertabular}  
%
%
%%%%%%%%%%%%%%%%%%%%%%%%%%%%%%%%%%%%%%%%%%%%%%%%%%%%%%%%%%%%%%%%%%%%%%%%%%
\newpage
%
\subsection*{Practice (Answers: Page~\pageref{'pi'})}
%%%%%%%%%%%%%%%%%%%%%%%%%%%%%%%%%%%%%%%%%%%%%%%%%%%%%%%%%%%%%%%%%%%%%%%%%%
%
Please write down your answers and check them afterwards. 

\begin{supertabular}{p{5,5cm}|ll}
Can the separator \textit{pi} be used to separate adjectives? &&  \\ % no-dictionary
Where is the main noun in \textit{toki pona} of a compound noun? && \\ % no-dictionary
How many words must at least be between the separator \textit{pi} and the next separator? &&  \\ % no-dictionary
Where can adjective slots after the separator \textit{pi} be located? &&  \\ % no-dictionary
How do you ask for the owner of an item? &&  \\ % no-dictionary
\end{supertabular}

Try to translate these sentences. 
You can use the tool \textit{Toki Pona Parser} (\cite{www:rowa:02}) for spelling and grammar check. 

\begin{supertabular}{p{5,5cm}|ll}
Keli's child is funny.    \\ % no-dictionary
I am a Toki Ponan.   \\  % no-dictionary
He is a good musician.   \\  % no-dictionary
The captain of the ship is eating.    \\ % no-dictionary
Meow.    \\ % no-dictionary
Enya's music is good.    \\ % no-dictionary
Which people of this group are important?    \\ % no-dictionary
Our house is messed up.    \\ % no-dictionary
How did she make that?    \\ % no-dictionary
I look at the land with my friend. && \\ % no-dictionary
Whom did you go with? &&  \\  % no-dictionary
\end{supertabular}

\begin{supertabular}{p{5,5cm}|ll}
pipi pi ma mama mi li ' lili.  \\ % no-dictionary
kili pi jan Linta li ' ike.    \\ % no-dictionary
len pi jan Susan li ' jaki.    \\ % no-dictionary
mi sona ala e nimi pi ona mute.    \\ % no-dictionary
mi wile toki meli.    \\ % no-dictionary
sina pakala e ilo, kepeken nasin seme?    \\ % no-dictionary
jan Wasintan [Washington] li ' jan lawa pona pi ma Mewika.  \\   % no-dictionary
wile pi jan ike li pakala e ijo.    \\ % no-dictionary
\end{supertabular}  
%%%%%%%%%%%%%%%%%%%%%%%%%%%%%%%%%%%%%%%%%%%%%%%%%%%%%%%%%%%%%%%%%%%%%%%%%%
% eof
