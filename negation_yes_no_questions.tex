%%%%%%%%%%%%%%%%%%%%%%%%%%%%%%%%%%%%%%%%%%%%%%%%%%%%%%%%%%%%%%%%%%%%%%%%%%
\section{Negation, Yes/No Questions}
%%%%%%%%%%%%%%%%%%%%%%%%%%%%%%%%%%%%%%%%%%%%%%%%%%%%%%%%%%%%%%%%%%%%%%%%%%
\subsection*{Vocabulary}
%%%%%%%%%%%%%%%%%%%%%%%%%%%%%%%%%%%%%%%%%%%%%%%%%%%%%%%%%%%%%%%%%%%%%%%%%%
\index{\textit{ala}}
\index{\textit{ali}}
\index{\textit{ken}}
\index{\textit{lape}}
\index{\textit{musi}}
\index{\textit{pali}}
\index{\textit{wawa}}
\index{not}
\index{none}
\index{nothing}
\index{everything}
\index{all}
\index{can}
\index{be able to}
\index{possibility}
\index{sleep}
\index{fun}
\index{amuse}
\index{game}
\index{fun}
\index{do}
\index{make}
\index{activity}
\index{work}
\index{know how}
\index{wisdom}
\index{strong}
\index{intense}
\index{energy}
\index{power}
\begin{supertabular}{p{2,5cm}|ll}
\textbf{\dots ala} && \textit{adjective}: no, not, none, un- \\ % no-dictionary
\textbf{\dots ala} && \textit{adverb}: don't \\ % no-dictionary
\textbf{ala} && \textit{noun}: nothing, negation, zero \\ % no-dictionary
 && \\ % no-dictionary
\textbf{\dots ali} && \textit{adjective}: all, every, complete, whole \\ % no-dictionary
\textbf{\dots ali} && \textit{adverb}: always, forever, evermore, eternally \\ % no-dictionary
\textbf{ali} && \textit{noun}: everything, anything, life, the universe \\ % no-dictionary
 && \\ % no-dictionary
\textbf{ken} && \textit{noun}: possibility, ability, power to do things, permission \\ % no-dictionary
\textbf{ken} && \textit{verb intransitive}: can, is able to, is allowed to, may, is possible \\ % no-dictionary
\textbf{ken \dots} && \textit{verb pre}: to can, may \\ % no-dictionary
\textbf{ken (e \dots)} && \textit{verb transitive}: to make possible, to enable, to allow, to permit \\ % no-dictionary
 && \\ % no-dictionary
\textbf{\dots lape} && \textit{adjective}: sleeping, of sleep, dormant \\ % no-dictionary
\textbf{\dots lape} && \textit{adverb}: asleep \\ % no-dictionary
\textbf{lape} && \textit{noun}: sleep, rest \\ % no-dictionary
\textbf{lape} && \textit{verb intransitive}: to sleep, to rest \\ % no-dictionary
\textbf{lape (e \dots)} && \textit{verb transitive}: to knock out \\ % no-dictionary
 && \\ % no-dictionary
\textbf{\dots musi} && \textit{adjective}: artful, fun, recreational \\ % no-dictionary
\textbf{\dots musi} && \textit{adverb}: cheerfully \\ % no-dictionary
\textbf{musi} && \textit{noun}: fun, playing, game, recreation, art, entertainment \\ % no-dictionary
\textbf{musi} && \textit{verb intransitive}: to play, to have fun \\ % no-dictionary
\textbf{musi (e \dots)} && \textit{verb transitive}: to amuse, to entertain \\ % no-dictionary
 && \\ % no-dictionary
\textbf{\dots pali} && \textit{adjective}: active, work-related, operating, working \\ % no-dictionary
\textbf{\dots pali} && \textit{adverb}: actively, briskly \\ % no-dictionary
\textbf{pali} && \textit{noun}: activity, work, deed, project \\ % no-dictionary
\textbf{pali} && \textit{verb intransitive}: to act, to work, to function \\ % no-dictionary
\textbf{pali (e \dots)} && \textit{verb transitive}: to do, to make, to build, to create \\ % no-dictionary
 && \\ % no-dictionary
\textbf{\dots wawa} && \textit{adjective}: energetic, strong, fierce, intense, sure, confident \\ % no-dictionary
\textbf{\dots wawa} && \textit{adverb}: strongly, powerfully \\ % no-dictionary
\textbf{wawa} && \textit{noun}: energy, strength, power \\ % no-dictionary
\textbf{wawa (e \dots)} && \textit{verb transitive}: to strengthen, to energize, to empower \\ % no-dictionary
 && \\ % no-dictionary
\end{supertabular} \\
%
%%%%%%%%%%%%%%%%%%%%%%%%%%%%%%%%%%%%%%%%%%%%%%%%%%%%%%%%%%%%%%%%%%%%%%%%%%
\index{negation}
\index{not}
\index{\textit{ala}!negation}
\subsection*{Negation}
%%%%%%%%%%%%%%%%%%%%%%%%%%%%%%%%%%%%%%%%%%%%%%%%%%%%%%%%%%%%%%%%%%%%%%%%%%
%
\textbf{\textit{ala} as an adverb}

In English, you make a verb negative by adding 'not' in front of the verb.
Toki Pona puts its negative word, \textit{ala}, after the verb. 

\index{lazy}
\begin{supertabular}{p{5,5cm}|ll}
mi lape ala. && I'm not sleeping. \\ 
mi musi ala. && I'm not having fun. / I'm bored. \\
mi wawa ala. && I'm not strong. / I'm weak. \\
mi wile ala tawa musi. && I don't want to dance. \\
tawa musi && dance (move entertainingly) \\
mi wile ala pali. && I'm lazy. \\
\end{supertabular} 

\index{\textit{ala}!adjective}
\index{adjective!\textit{ala}}
\textbf{\textit{ala} as an adjective}

\begin{supertabular}{p{5,5cm}|ll}
jan ala li toki. && Nobody is talking. \\
\end{supertabular} 

\index{\textit{ala}!\textit{ijo}}
\index{\textit{ijo}!\textit{ala}}
There's nothing wrong with putting \textit{ala} after the verb (which is \textit{toki} in this sentence), and in fact that's the more common way of doing it. 
However, you do have the option of using \textit{ala} after nouns, and so I just wanted to point that out. 

However, if you do that, try to remember not use \textit{ala} with \textit{ijo}. 
If not behind a verb, \textit{ala} already essentially means nothing by default, and so using \textit{ijo} just doesn't fit in.

\begin{supertabular}{p{5,5cm}|ll}
ala li ' jaki. && Nothing is dirty. \\
\end{supertabular} 
%
%%%%%%%%%%%%%%%%%%%%%%%%%%%%%%%%%%%%%%%%%%%%%%%%%%%%%%%%%%%%%%%%%%%%%%%%%%
\index{\textit{ali}}
\subsection*{\textit{ali}}
%%%%%%%%%%%%%%%%%%%%%%%%%%%%%%%%%%%%%%%%%%%%%%%%%%%%%%%%%%%%%%%%%%%%%%%%%%
%
However, despite the differences in meaning, \textit{ala} and \textit{ali} as adjectives are used the same way.

\index{travel}
\begin{supertabular}{p{5,5cm}|ll}
jan ali li wile tawa. && Everybody wants to travel. \\
ma ali li ' pona. && All nations are good. \\
jan utala ali li ' nasa. && All soldiers are stupid.
\end{supertabular} 

\index{\textit{ali}!\textit{ijo}}
\index{\textit{ijo}!\textit{ali}}
Also like \textit{ala}, it's best not to use \textit{ijo} and \textit{ali} together. 
By the way, \textit{ali li pona} is one of the Toki Pona proverbs, if you didn't know that.  
%
%%%%%%%%%%%%%%%%%%%%%%%%%%%%%%%%%%%%%%%%%%%%%%%%%%%%%%%%%%%%%%%%%%%%%%%%%%
\newpage
\index{question!yes,no}
\index{yes,no!question}
\subsection*{Yes/No Questions}
%%%%%%%%%%%%%%%%%%%%%%%%%%%%%%%%%%%%%%%%%%%%%%%%%%%%%%%%%%%%%%%%%%%%%%%%%%

To make yes/no questions, you say the verb, then the adverb \textit{ala}, then repeat the verb. 
An questions always ends in a question mark.

\begin{supertabular}{p{5,5cm}|ll}
sina ken ala ken lape? && Can you sleep? \\
ona li lon ala lon tomo? && Is he in the house? \\
ona li tawa ala tawa ma ike? && Did he go to the evil land? \\
sina pana ala pana e moku tawa jan lili? && Did you give food to the child? \\
pipi li moku ala moku e kili? && Are the bugs eating the fruit? \\
\end{supertabular} 

%
%%%%%%%%%%%%%%%%%%%%%%%%%%%%%%%%%%%%%%%%%%%%%%%%%%%%%%%%%%%%%%%%%%%%%%%%%%
\index{question!yes,no!answering}
\index{answering!question!yes,no}
\index{yes}
\index{no}
\subsection*{Answering}
%%%%%%%%%%%%%%%%%%%%%%%%%%%%%%%%%%%%%%%%%%%%%%%%%%%%%%%%%%%%%%%%%%%%%%%%%%

If you want to say 'yes', you simply repeat the verb of the sentence. 
If you want to say 'no', you repeat the verb and add the adverb \textit{ala}} after it. 

\begin{supertabular}{p{5,5cm}|ll}
sina wile ala wile moku? && Do you want to eat? \\ 
wile && Yes \\ % no-dictionary
wile ala && No \\ % no-dictionary
 && \\ % no-dictionary
sina lukin ala lukin e kiwen? && Do you see the rock? \\ 
lukin && Yes \\ % no-dictionary
lukin ala && No \\ % no-dictionary
 && \\ % no-dictionary
sina sona ala sona e toki mi? && Do you understand what I'm saying? \\ 
sona && Yes \\ % no-dictionary
sona ala && No \\ % no-dictionary
\end{supertabular}  


%%%%%%%%%%%%%%%%%%%%%%%%%%%%%%%%%%%%%%%%%%%%%%%%%%%%%%%%%%%%%%%%%%%%%%%%%%
% 
\subsection*{Problems with missing word for 'be'}
%%%%%%%%%%%%%%%%%%%%%%%%%%%%%%%%%%%%%%%%%%%%%%%%%%%%%%%%%%%%%%%%%%%%%%%%%%

We had already learned the difference between verb and predicate (see page~\pageref{'predicate'}). 
Since Toki Pona lacks the verb 'be', sentences without verb are possible. 
Then nouns serve as predicate nouns or adjectives as predicate adjectives. 
Yes/No questions with the adverb \textit{ala} are only possible with a verb. 
You cannot write the missing verb 'be', then \textit{ala} and then again write the missing verb 'be'.
For example 'Is she a mother?' can't be formulated that way. 

\begin{supertabular}{p{5,5cm}|ll}
ona li ' ala ' mama? && false \\ % no-dictionary
\end{supertabular} 

Yes or no answers are also not possible if no verb there. 

\begin{supertabular}{p{5,5cm}|ll}
"..." && Yes. (wrong) \\\ % no-dictionary
"' ala. && No. (wrong) \\\ % no-dictionary
\end{supertabular} 

We will later learn how to formulate yes/no questions with predicate nouns and predicate adjectives.
%
%%%%%%%%%%%%%%%%%%%%%%%%%%%%%%%%%%%%%%%%%%%%%%%%%%%%%%%%%%%%%%%%%%%%%%%%%%
\newpage
\subsection*{Practice 8 (Answers: Page~\pageref{'negation_yes_no_questions'})}
%%%%%%%%%%%%%%%%%%%%%%%%%%%%%%%%%%%%%%%%%%%%%%%%%%%%%%%%%%%%%%%%%%%%%%%%%%
%
Try to translate these sentences. 
You can use the tool \textit{Toki Pona Parser} (\cite{www:rowa:02}) for spelling and grammar check. 

\begin{supertabular}{p{5,5cm}|ll}
You have to tell me why. *  &&   \\ % no-dictionary
Is a bug beside me?  &&    \\ % no-dictionary
I can't sleep.  &&    \\ % no-dictionary
I don't want to talk to you.  &&    \\ % no-dictionary
He didn't go to the lake.   &&   \\ % no-dictionary
 && \\ % no-dictionary
sina wile ala wile pali? wile ala.  &&    \\ % no-dictionary
jan utala li seli ala seli e tomo?   &&   \\ % no-dictionary
jan lili li ken ala moku e telo nasa.   &&   \\ % no-dictionary
sina kepeken ala kepeken ni?  &&    \\ % no-dictionary
sina ken ala ken kama?   &&   \\ % no-dictionary
sina pona ala pona? &&   \\ % no-dictionary
\end{supertabular} 

* Think: 'You have to tell the reason to me.' 
%%%%%%%%%%%%%%%%%%%%%%%%%%%%%%%%%%%%%%%%%%%%%%%%%%%%%%%%%%%%%%%%%%%%%%%%%%
% eof
