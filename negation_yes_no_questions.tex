%%%%%%%%%%%%%%%%%%%%%%%%%%%%%%%%%%%%%%%%%%%%%%%%%%%%%%%%%%%%%%%%%%%%%%%%%%
\section{Negation, Yes/No Questions}
%%%%%%%%%%%%%%%%%%%%%%%%%%%%%%%%%%%%%%%%%%%%%%%%%%%%%%%%%%%%%%%%%%%%%%%%%%
\subsection*{Vocabulary}
%%%%%%%%%%%%%%%%%%%%%%%%%%%%%%%%%%%%%%%%%%%%%%%%%%%%%%%%%%%%%%%%%%%%%%%%%%
%
\begin{supertabular}{p{2,5cm}|ll}
%
\index{ala}
\textbf{\dots ala} && \textit{adjective}: no, not, none, un- \\ % no-dictionary
\textbf{\dots ala} && \textit{adverb}: don't \\ % no-dictionary
\textbf{ala} && \textit{noun}: nothing, negation, zero \\ % no-dictionary
 && \\ % no-dictionary
%
\index{ken}
\textbf{ken} && \textit{noun}: possibility, ability, power to do things, permission \\ % no-dictionary
\textbf{ken} && \textit{verb intransitive}: can, is able to, is allowed to, may, is possible \\ % no-dictionary
\textbf{ken \dots} && \textit{auxiliary verb}: to can, may \\ % no-dictionary
\textbf{ken (e \dots)} && \textit{verb transitive}: to make possible, to enable, to allow, to permit \\ % no-dictionary
 && \\ % no-dictionary
%
\index{lape}
\textbf{\dots lape} && \textit{adjective}: sleeping, of sleep, dormant \\ % no-dictionary
\textbf{\dots lape} && \textit{adverb}: asleep \\ % no-dictionary
\textbf{lape} && \textit{noun}: sleep, rest \\ % no-dictionary
\textbf{lape} && \textit{verb intransitive}: to sleep, to rest \\ % no-dictionary
\textbf{lape (e \dots)} && \textit{verb transitive}: to knock out \\ % no-dictionary
 && \\ % no-dictionary
%
\index{musi}
\textbf{\dots musi} && \textit{adjective}: artful, fun, recreational \\ % no-dictionary
\textbf{\dots musi} && \textit{adverb}: cheerfully \\ % no-dictionary
\textbf{musi} && \textit{noun}: fun, playing, game, recreation, art, entertainment \\ % no-dictionary
\textbf{musi} && \textit{verb intransitive}: to play, to have fun \\ % no-dictionary
\textbf{musi (e \dots)} && \textit{verb transitive}: to amuse, to entertain \\ % no-dictionary
 && \\ % no-dictionary
%
\index{pali}
\textbf{\dots pali} && \textit{adjective}: active, work-related, operating, working \\ % no-dictionary
\textbf{\dots pali} && \textit{adverb}: actively, briskly \\ % no-dictionary
\textbf{pali} && \textit{noun}: activity, work, deed, project \\ % no-dictionary
\textbf{pali} && \textit{verb intransitive}: to act, to work, to function \\ % no-dictionary
\textbf{pali (e \dots)} && \textit{verb transitive}: to do, to make, to build, to create \\ % no-dictionary
 && \\ % no-dictionary
%
\index{wawa} 
\textbf{\dots wawa} && \textit{adjective}: energetic, strong, fierce, intense, sure, confident \\ % no-dictionary
\textbf{\dots wawa} && \textit{adverb}: strongly, powerfully \\ % no-dictionary
\textbf{wawa} && \textit{noun}: energy, strength, power \\ % no-dictionary
\textbf{wawa (e \dots)} && \textit{verb transitive}: to strengthen, to energize, to empower \\ % no-dictionary
\end{supertabular} \\
%
%%%%%%%%%%%%%%%%%%%%%%%%%%%%%%%%%%%%%%%%%%%%%%%%%%%%%%%%%%%%%%%%%%%%%%%%%%
\newpage
%
\subsection*{Negation}
%
\index{negation}
\index{\textit{ala}!negation}
%%%%%%%%%%%%%%%%%%%%%%%%%%%%%%%%%%%%%%%%%%%%%%%%%%%%%%%%%%%%%%%%%%%%%%%%%%
%
\subsubsection*{The Adverb \textit{ala}}
%
\index{\textit{ala}!adverb}
%%%%%%%%%%%%%%%%%%%%%%%%%%%%%%%%%%%%%%%%%%%%%%%%%%%%%%%%%%%%%%%%%%%%%%%%%%
%
In English, you make a verb negative by adding 'not' in front of the verb.
In Toki Pona you put the adverb \textit{ala} after the verb. 

\index{lazy}
\begin{supertabular}{p{5,5cm}|ll}
mi lape ala. && I'm not sleeping. \\ 
mi musi ala. && I'm not having fun. / I'm bored. \\
mi wawa ala. && I'm not strong. / I'm weak. \\
mi wile ala tawa musi. && I don't want to dance. \\
tawa musi && dance (move entertainingly) \\
mi wile ala pali. && I'm lazy. \\
\end{supertabular} 

%
%%%%%%%%%%%%%%%%%%%%%%%%%%%%%%%%%%%%%%%%%%%%%%%%%%%%%%%%%%%%%%%%%%%%%%%%%%
\subsubsection*{The Adjective \textit{ala}}
%
\index{\textit{ala}!adjective}
%%%%%%%%%%%%%%%%%%%%%%%%%%%%%%%%%%%%%%%%%%%%%%%%%%%%%%%%%%%%%%%%%%%%%%%%%%
%
\begin{supertabular}{p{5,5cm}|ll}
jan ala li toki. && Nobody is talking. \\
\end{supertabular} 

%
%%%%%%%%%%%%%%%%%%%%%%%%%%%%%%%%%%%%%%%%%%%%%%%%%%%%%%%%%%%%%%%%%%%%%%%%%%
\subsubsection*{The Noun \textit{ala}}
%
\index{\textit{ala}!noun}
%%%%%%%%%%%%%%%%%%%%%%%%%%%%%%%%%%%%%%%%%%%%%%%%%%%%%%%%%%%%%%%%%%%%%%%%%%
%

\begin{supertabular}{p{5,5cm}|ll}
ala li ' jaki. && Nothing is dirty. \\
\end{supertabular} 

%
%%%%%%%%%%%%%%%%%%%%%%%%%%%%%%%%%%%%%%%%%%%%%%%%%%%%%%%%%%%%%%%%%%%%%%%%%%
% \newpage
\subsection*{Yes/No Questions with a verb and the adverb \textit{ala}}
%
\index{question!yes,no}
\index{yes,no!question}
%%%%%%%%%%%%%%%%%%%%%%%%%%%%%%%%%%%%%%%%%%%%%%%%%%%%%%%%%%%%%%%%%%%%%%%%%%

To make yes/no questions with a verb, you say the verb, then the adverb \textit{ala}, then repeat the verb. 
An questions always ends in a question mark.

\begin{supertabular}{p{5,5cm}|ll}
ona li lon ala lon tomo? && Is he in the house? \\
ona li tawa ala tawa, tawa ma ike? && Did he go to the evil land? \\
sina pana ala pana e moku tawa jan lili? && Did you give food to the child? \\
pipi li moku ala moku e kili? && Are the bugs eating the fruit? \\
ona li mama ala mama? && Does she mother (someone)? \\
\end{supertabular} 

%%%%%%%%%%%%%%%%%%%%%%%%%%%%%%%%%%%%%%%%%%%%%%%%%%%%%%%%%%%%%%%%%%%%%%%%%%
\subsection*{Yes/No Questions with an auxiliary verb and the adverb \textit{ala}}
%%%%%%%%%%%%%%%%%%%%%%%%%%%%%%%%%%%%%%%%%%%%%%%%%%%%%%%%%%%%%%%%%%%%%%%%%%
%
To make yes/no questions with an auxiliary verb, you say the auxiliary verb, then the adverb \textit{ala}, then repeat the auxiliary verb. 

\begin{supertabular}{p{5,5cm}|ll}
sina wile ala wile moku? && Do you want to eat? \\ 
sina ken ala ken lape? && Can you sleep? \\
sina kama ala kama jo e pali ni? && Did you get this job?
\end{supertabular} 
%
%%%%%%%%%%%%%%%%%%%%%%%%%%%%%%%%%%%%%%%%%%%%%%%%%%%%%%%%%%%%%%%%%%%%%%%%%%
\subsection*{Yes/No Answering}
%
\index{answering!yes,no}
\index{yes}
\index{no}
%%%%%%%%%%%%%%%%%%%%%%%%%%%%%%%%%%%%%%%%%%%%%%%%%%%%%%%%%%%%%%%%%%%%%%%%%%

If you want to say 'yes', you simply repeat the verb or the auxiliary verb of the sentence. 
If you want to say 'no', you repeat the verb or the auxiliary verb and add the adverb \textit{ala}} after it. 

\begin{supertabular}{p{5,5cm}|ll}
sina wile ala wile moku? && Do you want to eat? \\ 
wile && Yes \\ % no-dictionary
wile ala && No \\ % no-dictionary
% && \\ % no-dictionary
sina lukin ala lukin e kiwen? && Do you see the rock? \\ 
lukin && Yes \\ % no-dictionary
lukin ala && No \\ % no-dictionary
% && \\ % no-dictionary
sina sona ala sona e toki mi? && Do you understand what I'm saying? \\ 
sona && Yes \\ % no-dictionary
sona ala && No \\ % no-dictionary
\end{supertabular}  


%%%%%%%%%%%%%%%%%%%%%%%%%%%%%%%%%%%%%%%%%%%%%%%%%%%%%%%%%%%%%%%%%%%%%%%%%%
% 
\subsection*{Problems with missing word for 'be'}
%%%%%%%%%%%%%%%%%%%%%%%%%%%%%%%%%%%%%%%%%%%%%%%%%%%%%%%%%%%%%%%%%%%%%%%%%%

We had already learned the difference between verb and predicate (see page~\pageref{'predicate'}). 
Since Toki Pona lacks the verb 'be', sentences without verb are possible. 
Then nouns serve as predicate nouns or adjectives as predicate adjectives. 
Yes/No questions with the adverb \textit{ala} are only possible with a verb (see the offical Toki Pona book of Sonja Lang \cite{www:tokipona.org}). 
You cannot write the missing verb 'be', then \textit{ala} and then again write the missing verb 'be'.
For example 'Is she a mother?' can't be formulated that way. 

\begin{supertabular}{p{5,5cm}|ll}
ona li ' ala ' mama? && wrong \\ % no-dictionary
\end{supertabular} 

Yes or no answers are also not possible if no verb there. 

\begin{supertabular}{p{5,5cm}|ll}
' . && wrong \\\ % no-dictionary
' ala. && wrong \\\ % no-dictionary
\end{supertabular} 

We will later learn how to formulate yes/no questions with predicate nouns and predicate adjectives.
%
%%%%%%%%%%%%%%%%%%%%%%%%%%%%%%%%%%%%%%%%%%%%%%%%%%%%%%%%%%%%%%%%%%%%%%%%%%
\newpage
%
\subsection*{Practice (Answers: Page~\pageref{'negation_yes_no_questions'})}
%%%%%%%%%%%%%%%%%%%%%%%%%%%%%%%%%%%%%%%%%%%%%%%%%%%%%%%%%%%%%%%%%%%%%%%%%%
%
Please write down your answers and check them afterwards. 

\begin{supertabular}{p{5,5cm}|ll}
Which separator is at the end of a question? &&  \\ % no-dictionary
How is a yes/no question formulated in Toki Pona? &&   \\ % no-dictionary
What is to be considered for a predicate without a verb? &&   \\ % no-dictionary
How is a verb negated in Toki Pona? &&  \\ % no-dictionary
How do you answer in Toki Pona negative to a yes/no question? &&  \\ % no-dictionary
How do you answer positively to a yes/no question in Toki Pona? &&  \\ % no-dictionary
\end{supertabular} 

Try to translate these sentences. 
You can use the tool \textit{Toki Pona Parser} (\cite{www:rowa:02}) for spelling and grammar check. 

\begin{supertabular}{p{5,5cm}|ll}
You have to tell me why. *  &&   \\ % no-dictionary
Is a bug beside me?  &&    \\ % no-dictionary
I can't sleep.  &&    \\ % no-dictionary
I don't want to talk to you.  &&    \\ % no-dictionary
He didn't go to the lake.   &&   \\ % no-dictionary
\end{supertabular}

\begin{supertabular}{p{5,5cm}|ll}
sina wile ala wile pali? wile ala.  &&    \\ % no-dictionary
jan utala li seli ala seli e tomo?   &&   \\ % no-dictionary
jan lili li ken ala moku e telo nasa.   &&   \\ % no-dictionary
sina kepeken ala kepeken ni?  &&    \\ % no-dictionary
sina ken ala ken kama?   &&   \\ % no-dictionary
sina pona ala pona? &&   \\ % no-dictionary
\end{supertabular} 

* Think: 'You have to tell the reason to me.' 
%%%%%%%%%%%%%%%%%%%%%%%%%%%%%%%%%%%%%%%%%%%%%%%%%%%%%%%%%%%%%%%%%%%%%%%%%%
% eof
