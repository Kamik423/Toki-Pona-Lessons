%%%%%%%%%%%%%%%%%%%%%%%%%%%%%%%%%%%%%%%%%%%%%%%%%%%%%%%%%%%%%%%%%%%%%%%%%%
%%%%%%%%%%%%%%%%%%%%%%%%%%%%%%%%%%%%%%%%%%%%%%%%%%%%%%%%%%%%%%%%%%%%%%%%%%
\section{Answers}
%%%%%%%%%%%%%%%%%%%%%%%%%%%%%%%%%%%%%%%%%%%%%%%%%%%%%%%%%%%%%%%%%%%%%%%%%%
% \subsection*{Introduction} \\
% --
%%%%%%%%%%%%%%%%%%%%%%%%%%%%%%%%%%%%%%%%%%%%%%%%%%%%%%%%%%%%%%%%%%%%%%%%%%
% \subsection*{Pronunciation and the Alphabet} \\
%--

%%%%%%%%%%%%%%%%%%%%%%%%%%%%%%%%%%%%%%%%%%%%%%%%%%%%%%%%%%%%%%%%%%%%%%%%%%
\subsection*{Pronunciation, Alphabet and Punctuation Marks} 
\label{'pronunciation_alphabet'}
%
\begin{supertabular}{p{5,5cm}|ll}
What are separators? && Separators separate phrases from each other.  \\ % no-dictionary
Which phrase has no punctuation character at the end? && A heading (headline) has no punctuation character at the end. \\ % no-dictionary
Which separator is at the end of a declarative sentence? && A full stop. \\ % no-dictionary
When are official \textit{toki pona} words capitalized? && Never. \\ % no-dictionary
What is usually not allowed before or after a separator? && Another separator. \\ % no-dictionary
\end{supertabular} 

%%%%%%%%%%%%%%%%%%%%%%%%%%%%%%%%%%%%%%%%%%%%%%%%%%%%%%%%%%%%%%%%%%%%%%%%%%
\subsection*{Basic Sentences} 
\label{'basic_sentences'}
%
\begin{supertabular}{p{5,5cm}|ll}
What is a verb && A verb describes an action. \\ % no-dictionary
What is a noun? &&  A noun is a word for a person, place or thing. \\ % no-dictionary
What is \textit{li} used for?  &&  It separates the subject phrase from the predicate phrase.  \\ % no-dictionary
What does a personal pronoun replace? && It replaces a noun. \\ % no-dictionary
How to recognize nouns, pronouns, verbs and adjectives in \textit{toki pona}? && At their position in the sentence. \\ % no-dictionary
What is a subject?  &&   The subject is the carrier of the action, process or state. \\ % no-dictionary
After which subject phrases is \textit{li} not used?  &&   It is only used if the subject phrase is not \textit{mi} or \textit{sina}. \\ % no-dictionary
Where does the subject stand in the sentence?  &&  In Toki Pona it is always at the beginning of the sentence. \\ % no-dictionary
Can an empty verb slot alone form a predicate? && No!  \\ % no-dictionary
When can a verb slot be empty?  &&   If the predicate is formed by a noun or adjective.  \\ % no-dictionary
What is a predicate?  &&   It is a core element in a sentence and the statement of the sentence. \\ % no-dictionary
A complete sentence in \textit{toki pona} always contains\dots  &&  a subject and a predicate phrase.  \\ % no-dictionary
What kinds of words can be used in \textit{toki pona} to form a predicate? && Verbs, nouns or adjectives.  \\ % no-dictionary
What is an adjective?  &&  An adjective is a word that describes a noun.  \\ % no-dictionary
Where are possible adjective slots?  &&  After a noun, after a pronoun and according to \textit{li}. \\  % no-dictionary
Why can't a sentence be ended after \textit{li}? && Because then the predicate is missing. \\ % no-dictionary
\end{supertabular} 

\begin{supertabular}{p{5,5cm}|ll}
sina - sina pona. && personal pronoun \\ % no-dictionary
moku - moku li ' pona. && noun \ % no-dictionary
ona - ona li ' moku. && personal pronoun \\ % no-dictionary
li - moku li ' pona. && separator \\ % no-dictionary
\end{supertabular}

\begin{supertabular}{p{5,5cm}|ll}
People are good. && jan li ' pona. \\ % & English - Toki Pona
I'm eating. && mi moku. \\ % & English - Toki Pona
You're tall. && sina ' suli. \\ % & English - Toki Pona
Water is simple. && telo li ' pona. \\ % & English - Toki Pona
The lake is big. && telo li ' suli. \\ % & English - Toki Pona
 && \\ % no-dictionary
suno li ' suli. && The sun is big. \\
mi ' suli. && I'm important. / I'm fat. \\
jan li moku. && Somebody is eating. \\
\end{supertabular} 

\newpage
%%%%%%%%%%%%%%%%%%%%%%%%%%%%%%%%%%%%%%%%%%%%%%%%%%%%%%%%%%%%%%%%%%%%%%%%%%
\subsection*{Direct Objects} 
\label{'direct_objects_compund_sentences'}
%
\begin{supertabular}{p{5,5cm}|ll}
How to ask for the direct object? && With 'whom' or' what'. \\ % no-dictionary
What word type has a predicate before the separator \textit{e}?  && It is always a transitive verb. \\ % no-dictionary
To which phrase in the sentence belongs a direct object? && To the predicate phrase. \\ % no-dictionary
What kinds of words are possible after the separator \textit{e}?  && A noun or pronoun. \\ % no-dictionary
What is a predicate noun?  && A noun used as a predicate. \\ % no-dictionary
Where are possible slots for reflexive pronouns? && After the separator \textit{e}. \\ % no-dictionary
Is it possible to describe several properties of a subject with several \textit{e}? && No, because \textit{e} comes after a transitive verb. \\ % no-dictionary
How can you create multiple predicate phrases in a sentence?  && With several separators \textit{li}. \\ % no-dictionary
\end{supertabular} 

\begin{supertabular}{p{5,5cm}|ll}
I have a tool. && mi jo e ilo. \\ % & English - Toki Pona
She's eating fruit. && ona li moku e kili. \\ % & English - Toki Pona
Something is watching me. && ijo li lukin e mi. \\ % & English - Toki Pona
Pineapple is a food and is good. && kili li ' moku li ' pona. \\ % & English - Toki Pona
He washes himself. && ona li telo e ona.  \\ % & English - Toki Pona
 && \\ % no-dictionary
mi ' jan li ' suli. && I am somebody and am important. \\
\end{supertabular} 

%%%%%%%%%%%%%%%%%%%%%%%%%%%%%%%%%%%%%%%%%%%%%%%%%%%%%%%%%%%%%%%%%%%%%%%%%%
\subsection*{Verbs, Adverbs, Auxiliary Verbs} 
\label{'adverbs'}
%
\begin{supertabular}{p{5,5cm}|ll}
What are adverbs? && Adverbs describe an action (verb). \\ % no-dictionary
Can an adverb be ranked according to a predicate noun? && No, this is not possible.  \\ % no-dictionary
Where are slots for adverbs located? && Only after verbs. \\ % no-dictionary
What kind of words describes an action? && Verbs. \\ % no-dictionary
When does a predicate phrase contain slots for adverbs? && If the predicate phrase contains a verb. \\ % no-dictionary
What is an auxiliary verb used for?  && It complements the main verb.  \\ % no-dictionary
Which phrase in the sentence can contain an auxiliary verb? && An auxiliary verb belongs to the predicate phrase.  \\ % no-dictionary
\end{supertabular} 

\begin{supertabular}{p{5,5cm}|ll}
jan li pona ilo e ilo. && The guy improve useful the tool. \\ % & English - Toki Pona
sina lukin unpa mute e mi. && You're looking at me very sexy.   \\ % & English - Toki Pona
jaki li jaki lili e mi. && The garbage dirtys me something. \\ % & English - Toki Pona
sina len nasa jaki e sina. && You dress disgustingly silly. \\ % & English - Toki Pona
ilo li sewi sewi e sewi. && The machine raises up the roof. \\ % & English - Toki Pona
ona li lawa utala e utala. && He leads fightingly the battle. \\ % & English - Toki Pona
mi wile unpa e ona. && I want to have sex with him/her.  \\
jan li wile jo e ma. && People want to own land. \\
 && \\ % no-dictionary
She increases the property very badly. && ona li mute ike mute e jo. \\
I want to have a lot of sex with you. && mi wile unpa mute e sina. \\
She was barely dressed. && ona li len lili e ona. \\
The sun shines warmly on the land. && suno li suno seli e ma. \\
She's good. && ona li ' pona. \\
He wants to destroy the tool. && ona li wile pakala e ilo. \\ % & English - Toki Pona
She is thirsty. && ona li wile moku e telo. \\ % & English - Toki Pona
\end{supertabular} 

%\newpage
%%%%%%%%%%%%%%%%%%%%%%%%%%%%%%%%%%%%%%%%%%%%%%%%%%%%%%%%%%%%%%%%%%%%%%%%%%
\subsection*{Nouns, Adjectives} 
\label{'adjectives'}
%
\begin{supertabular}{p{5,5cm}|ll}
What does a possessive pronoun replace? && It replaces a adjective. \\ % no-dictionary
What types of demonstrative pronouns are there? && Adjective and noun demonstrative pronouns.  \\ % no-dictionary
What is more complex in Toki Pona: adjectives or adverbs? && adjectives.  \\ % no-dictionary
By what kind of words are nouns described? && By adjectives.  \\ % no-dictionary
What is the difference between adverbs and adjectives? && Adverbs describe verbs and adjectives describe nouns. \\ % no-dictionary
Where are adjective slots located? && Only after nouns and as a predicate adjective in a predicate phrase. \\ % no-dictionary
Can an adjective follow a predicate noun? && Yes, since a predicate noun is a noun.  \\ % no-dictionary
\end{supertabular} 

\begin{supertabular}{p{5,5cm}|ll}
mi jo e kili.  &&  I have a fruit. \\
ona li ' pona li ' lili. && It is good and is small. \\
mi moku lili e kili lili.  && I nibble (eat a little) the small fruit. \\
 && \\ % no-dictionary
The leader drank dirty water.  && jan lawa li moku e telo jaki. \\ % & English - Toki Pona
I need a fork.  && mi wile e ilo moku. \\ % & English - Toki Pona
An enemy is attacking them. && jan ike li utala e ona mute. \\ % & English - Toki Pona
That bad person has strange clothes.  && jan ike ni li jo e len nasa. \\ % & English - Toki Pona
We drank a lot of vodka.  && mi mute li moku e telo nasa mute. \\ % & English - Toki Pona
Children watch adults.  && jan lili li lukin e jan suli. \\ % & English - Toki Pona
 && \\ % no-dictionary
mi lukin e ni. && I am looking at that. \\ 
mi lukin sewi e tomo suli.  && I am looking up at the big building. \\
seli suno li seli e tomo mi.  && The sun's warmth heats my home.  \\
jan lili li wile e telo kili.  && Children want fruit juice. \\
ona mute li nasa e jan suli.  && They drove the adults crazy. \\
mi kama e pakala. && I caused an accident. \\
\end{supertabular} 

%%%%%%%%%%%%%%%%%%%%%%%%%%%%%%%%%%%%%%%%%%%%%%%%%%%%%%%%%%%%%%%%%%%%%%%%%%
\subsection*{Indirect Objects} 
\label{'indirect_objects'}
%
\begin{supertabular}{p{5,5cm}|ll}
How you can not ask for an indirect object? && You can't ask 'who' or 'what'. \\ % no-dictionary
Which object type is strongly influenced by the predicate? && The direct object.  \\ % no-dictionary
Which phrase in the sentence does the indirect object belong to? && To the predicate phrase. \\ % no-dictionary
What slot is in the first position in an indirect object? && A noun or pronoun slot. \\ % no-dictionary
What do you call verbs that don't affect an object? && They are intransitive verbs.  \\ % no-dictionary
What stands in front of an indirect object in Toki Pona? && An intransitive verb. \\ % no-dictionary
Where is a slot for an adjective demonstrative pronoun possible? && After a noun. \\ % no-dictionary
Where's an auxiliary verb slot? && An auxiliary verb is placed in front of the main verb. \\ % no-dictionary
\end{supertabular}

\begin{supertabular}{p{5,5cm}|ll}
This is for my friend.  && ni li tawa jan pona mi. \\ % & English - Toki Pona
The tools are in the container.  && ilo li lon poki. \\ % & English - Toki Pona
That bottle is in the dirt.  && poki ni li lon jaki. \\ % & English - Toki Pona
They are arguing. && ona mute li utala toki. \\ % & English - Toki Pona
\end{supertabular} 

%%%%%%%%%%%%%%%%%%%%%%%%%%%%%%%%%%%%%%%%%%%%%%%%%%%%%%%%%%%%%%%%%%%%%%%%%%
\subsection*{Prepositional Objects} 
\label{'prepositional_objects'}
%
\begin{supertabular}{p{5,5cm}|ll}
What is closely related to a preposition?  && A preposition is closely connected to the verb. \\ % no-dictionary
Which phrase in the sentence does the prepositional object belong to?  &&  It is an optional part of a predicate phrase.  \\ % no-dictionary
Where are preposition slots located?  &&  At the beginning of a prepositional object. \\ % no-dictionary
At which position in the sentence can a prepositional object be located? && At the end of a sentence. \\ % no-dictionary
Which separators can be used to form composite sentences?  &&  With the separators \textit{li} and \textit{e}. \\ % no-dictionary
Which slots are possible in the second position in the prepositional object?  &&  A noun or pronoun slot. \\ % no-dictionary
\end{supertabular}

\begin{supertabular}{p{5,5cm}|ll}
I fixed the flashlight using a small tool.  && mi pona e ilo suno, kepeken ilo lili. \\ % & English - Toki Pona
I like Toki Pona.  && toki pona li ' pona, tawa mi. \\ % & English - Toki Pona
We gave them food.  && mi mute li pana e moku, tawa ona mute. \\ % & English - Toki Pona
I want to go to his house using my car.  && mi wile tawa tomo ona, kepeken tomo tawa mi. \\ % & English - Toki Pona
People look like ants.  && jan li lukin, sama pipi. \\ % & English - Toki Pona
&& \\ % no-dictionary
sina wile kama, tawa tomo toki.  && You should come to the chat room. \\
jan li toki, kepeken toki pona, lon tomo toki.  && People talk in/using Toki Pona in the chat room. \\
mi tawa, tawa tomo toki. ona li ' pona, tawa mi.  && I go the chat room. It is good for me. \\
                                         && I like to go to the chat room. \\ % no-dictionary
sina kama jo e jan pona, lon ni.  && You will get friends there. \\
sama li ' ike. && Equality is bad. \\
mi sona e tan. && I know the reason. / I know why. \\
\end{supertabular} 

%%%%%%%%%%%%%%%%%%%%%%%%%%%%%%%%%%%%%%%%%%%%%%%%%%%%%%%%%%%%%%%%%%%%%%%%%%
\subsection*{Relative Location Information} 
\label{'other_prepositions'}
%
\begin{supertabular}{p{5,5cm}|ll}
How do you create relative location information in Toki Pona? && With an indirect verb or a preposition and a compound spatial noun. \\ % no-dictionary
What is a possessive pronoun? && A possessive pronoun expresses a characteristic or affiliation.  \\ % no-dictionary
Where is a slot for a substantive demonstrative pronoun possible? && Instead of a noun. \\ % no-dictionary
Which separator is at the end of a declarative sentence? && A full stop. \\ % no-dictionary
What is a predicate adjective? && An adjective that is used as predicate. \\ % no-dictionary
Which sentence phrases can contain spatial nouns be found? && In an indirect object or prepositional object.  \\ % no-dictionary
\end{supertabular}

\begin{supertabular}{p{5,5cm}|ll}
My friend is beside me. && jan pona mi li lon poka mi. \\ % & English - Toki Pona
The sun is above me. && suno li lon sewi mi. \\ % & English - Toki Pona
The land is beneath me. && ma li lon anpa mi. \\ % & English - Toki Pona
Bad things are behind me. && ijo ike li lon monsi mi. \\ % & English - Toki Pona
I'm okay because I'm alive. && mi ' pona, tan ni: mi lon. \\ % & English - Toki Pona
I look at the land with you.  && mi lukin e ma, lon poka sina. \\ % & English - Toki Pona
poka mi li ' pakala.  && My hip hurts. \\
mi kepeken poki li kepeken ilo moku.  && I'm using a bowl and a spoon. \\
jan li lon insa tomo.  && Somebody's inside the house. \\
\end{supertabular} 

%%%%%%%%%%%%%%%%%%%%%%%%%%%%%%%%%%%%%%%%%%%%%%%%%%%%%%%%%%%%%%%%%%%%%%%%%%
\subsection*{Negation Yes/No Questions} 
\label{'negation_yes_no_questions'}
%
\begin{supertabular}{p{5,5cm}|ll}
Which separator is at the end of a question? && A question mark. \\ % no-dictionary
How is a yes/no question formulated in Toki Pona? && The adverb \textit{ala} is added to the verb and the verb is repeated.  \\ % no-dictionary
What is to be considered for a predicate without a verb? &&  It is not possible to formulate a yes/no question with the adverb \textit{ala}. \\ % no-dictionary
How is a verb negated in Toki Pona? && By placing the adverb \textit{ala} after the verb.  \\ % no-dictionary
How do you answer in Toki Pona negative to a yes/no question? && One repeats the verb of the question and adds the adverb \textit{ala}. \\ % no-dictionary
How do you answer positively to a yes/no question in Toki Pona? && One repeats the verb of the question. \\ % no-dictionary
\end{supertabular} 

\begin{supertabular}{p{5,5cm}|ll}
You have to tell me why.  && sina wile toki e tan, tawa mi. \\ % & English - Toki Pona
Is a bug beside me?  && pipi li lon ala lon poka mi? \\ % & English - Toki Pona
I can't sleep.  && mi ken ala lape. \\ % & English - Toki Pona
I don't want to talk to you.  && mi wile ala toki, tawa sina. \\ % & English - Toki Pona
He didn't go to the lake.  && ona li tawa ala, tawa telo. \\ % & English - Toki Pona
sina wile ala wile pali? wile ala.  && Do you want to work? No. \\
jan utala li seli ala seli e tomo?  && Is the warrior burning the house? \\
jan lili li ken ala moku e telo nasa.  && Children can't drink beer. \\
sina kepeken ala kepeken ni?  && Are you using that? \\
sina ken ala ken kama?  && Can you come? \\
sina pona ala pona? && Do you fix (something)? \\
\end{supertabular} 

%%%%%%%%%%%%%%%%%%%%%%%%%%%%%%%%%%%%%%%%%%%%%%%%%%%%%%%%%%%%%%%%%%%%%%%%%%
\subsection*{Unofficial Words} 
\label{'unofficial_words'}
%
\begin{supertabular}{p{5,5cm}|ll}
What are proper names in Toki Pona? && Unofficial words, adjectives \\ % no-dictionary
Where are slots for predicate adjectives located? && After the separator \textit{li}. \\ % no-dictionary
How are names in \textit{toki pona} highlighted? && The first letter is a capital letter. \\ % no-dictionary
How is the original spelling of a name marked? && By quotation marks.  \\ % no-dictionary
Which slots can unofficial words fill? && Adjective slots.  \\ % no-dictionary
What kind of word type must unofficial words be used together with? && With a noun. \\ % no-dictionary
\end{supertabular}

\begin{supertabular}{p{5,5cm}|ll}
Susan is crazy.  && jan Susan li ' nasa. \\ % & English - Toki Pona
I come from Europe. && mi kama, tan ma suli Elopa. \\ % & English - Toki Pona
My name is Ken.  && mi ' jan Ken. / nimi mi li Ken. \\ % & English - Toki Pona
Hello, Lisa.  && jan Lisa o, toki! \\ % & English - Toki Pona
I want to go to Australia. && mi wile tawa, tawa ma suli Oselija. \\  % & English - Toki Pona
mi wile kama sona e toki Inli.  && I want to learn English. \\
jan Ana o pana e moku, tawa mi!  && Ana, give me food. \\
jan Mose o lawa e mi mute, tawa ma pona!  && Moses, lead us to the good land. \\
\end{supertabular} 

%%%%%%%%%%%%%%%%%%%%%%%%%%%%%%%%%%%%%%%%%%%%%%%%%%%%%%%%%%%%%%%%%%%%%%%%%%
\subsection*{Addressing People, Interjections, Commands} 
\label{'commands_interjections'}
%
\begin{supertabular}{p{5,5cm}|ll}
Which separator ends a command sentence (imperative)? && With an exclamation mark. \\ % no-dictionary
What is the subject of the command form if no one is addressed directly? && The interjection word \textit{o}. \\ % no-dictionary
How do you address people by name? && \textit{jan Name o,.... } \\ % no-dictionary
What do injections consist of? && A noun or an interjection word and an exclamation mark. \\ % no-dictionary
Which separator stands bevor  the predicate if someone is directly addressed in a command? && The separator \textit{o}. \\ % no-dictionary
Which separator ends an interjection (exclamation)? && With an exclamation mark. \\ % no-dictionary
\end{supertabular}

\begin{supertabular}{p{5,5cm}|ll}
Go!  && o tawa! \\ % & English - Toki Pona
Mama, wait.  && mama meli o awen! \\ % & English - Toki Pona
Hahaha! That's funny.  && a a a! ni li ' musi. \\ % & English - Toki Pona
F-ck! && pakala! \\ % & English - Toki Pona
Bye!  && mi tawa!  \\ % & English - Toki Pona
mu!  && meow, woof, moo, etc. \\
o tawa musi, lon poka mi!  && Dance with me! \\
tawa pona!  && Good bye (spoken by the person who's staying) \\
o pu! && Buy and read the official Toki Pona book! \\
\end{supertabular} 

%%%%%%%%%%%%%%%%%%%%%%%%%%%%%%%%%%%%%%%%%%%%%%%%%%%%%%%%%%%%%%%%%%%%%%%%%%
\subsection*{Questions} 
\label{'questions_using_seme'}
%

\begin{supertabular}{p{5,5cm}|ll}
How does the sentence structure change for a question in \textit{toki pona}? && The sentence structure does not change. \\ % no-dictionary
What kind of word has the word \textit{seme}? && It is a question pronoun.  \\ % no-dictionary
What is a reflexive pronoun? && A reflexive pronoun represents the subject in the direct object. \\ % no-dictionary
What can represent the word \textit{seme}? && Sentence parts or all word types (except separators).  \\ % no-dictionary
How do you ask for a person (who, whom)? && With the noun \textit{jan} and \textit{seme}. \\ % no-dictionary
How is a Why question asked? && With the preposition \textit{tan} and \textit{seme} as prepositional object. \\ % no-dictionary
How do you ask for an indirect object? && If \textit{seme} follows an intransitive verb. \\ % no-dictionary
How to ask for a prepositional object? && If \textit{seme} follows after a preposition. \\ % no-dictionary
Are there nested subordinate clauses in \textit{toki pona}? && No, there are none. \\ % no-dictionary
\end{supertabular}

\begin{supertabular}{p{5,5cm}|ll}
What do you want to do?  && sina wile pali e seme? \\ % & English - Toki Pona
Who loves you?  && jan seme li olin e sina? \\ % & English - Toki Pona
Does it sweeten? && ni li suwi ala suwi? \\ % & English - Toki Pona
I'm going to bed.  && mi tawa supa lape. \\ % & English - Toki Pona
Are more people coming?  && jan sin li kama ala kama? \\ % & English - Toki Pona
Give me a lollipop!  && o pana e suwi, tawa mi! \\ % & English - Toki Pona
Who's there?  && jan seme li lon? / jan seme li lon ni? \\ % & English - Toki Pona
Which bug hurt you?  && pipi seme li pakala e sina? \\ % & English - Toki Pona
He loves to eat.  && moku li pona, tawa ona. \\ % & English - Toki Pona
Pardon? && seme? \\ % & English - Toki Pona
jan Ken o, mi olin e sina.  && Ken, I love you. \\
ni li ' jan seme?  && Who is that? \\
sina lon seme?  && Where are you? \\ 
                &&   (lit: You in what?) \\ % no-dictionary
mi lon, tan seme?  && Why am I here? \\ 
                &&   (lit: I exist because-of what?) \\ % no-dictionary
jan seme li ' meli sina?  && Who is your girlfriend/wife? \\
sina tawa ma tomo, tan seme?  && Why did you go to the city? \\
sina wile tawa, tawa ma seme?  && What place do you want to go to? \\
\end{supertabular} 

%%%%%%%%%%%%%%%%%%%%%%%%%%%%%%%%%%%%%%%%%%%%%%%%%%%%%%%%%%%%%%%%%%%%%%%%%%
\subsection*{Compound Nouns} 
\label{'pi'}

\begin{supertabular}{p{5,5cm}|ll}
Can the separator \textit{pi} be used to separate adjectives? && No, it is not possible. \\ % no-dictionary
Where is the main noun in \textit{toki pona} of a compound noun? && At the beginning. \\ % no-dictionary
How many words must at least be between the separator \textit{pi} and the next separator? && Two words. \\ % no-dictionary
Where can adjective slots after the separator \textit{pi} be located? && On the second and following positions after the separator \textit{pi}. \\ % no-dictionary
How do you ask for the owner of an item? && item + \textit{pi} + \textit{jan} + \textit{seme} \\ % no-dictionary
\end{supertabular}

\begin{supertabular}{p{5,5cm}|ll}
Keli's child is funny.  && jan lili pi jan Keli li ' musi. \\ % & English - Toki Pona
I am a Toki Ponan.  && mi ' jan pi toki pona. \\ % & English - Toki Pona
He is a good musician.  && ona li ' jan pona pi kalama musi. \\ % & English - Toki Pona
The captain of the ship is eating.  && jan lawa pi tomo tawa telo li moku. \\ % & English - Toki Pona
Meow.  && mu! \\ % & English - Toki Pona
Enya's music is good.  && kalama musi pi jan Enja li ' pona. \\ % & English - Toki Pona
Which people of this group are important?  && jan seme pi kulupu ni li suli? \\ % & English - Toki Pona
Our house is messed up.  && tomo pi mi mute li ' pakala. \\ % & English - Toki Pona
How did she make that?  && ona li pali e ni, kepeken nasin seme? \\ % & English - Toki Pona
I look at the land with my friend.  && mi lukin e ma, lon poka pi jan pona mi. \\ % & English - Toki Pona
Whom did you go with?  && sina tawa, lon poka pi jan seme? \\ % & English - Toki Pona
pipi pi ma mama mi li ' lili. && The insects of my homeland are small. \\
kili pi jan Linta li ' ike.  && Linda's fruit is bad. \\
len pi jan Susan li ' jaki.  && Susan's clothes are dirty. \\
mi sona ala e nimi pi ona mute.  && I don't know their names. \\
mi wile toki meli.  && I want to talk about girls. \\
sina pakala e ilo, kepeken nasin seme?  && How did you break the tool? \\
jan Wasintan [Washington] li ' jan lawa pona pi ma Mewika.  && Washington was a good leader of America. \\
wile pi jan ike li pakala e ijo.  && The desires of evil people mess things up. \\
\end{supertabular}  

%%%%%%%%%%%%%%%%%%%%%%%%%%%%%%%%%%%%%%%%%%%%%%%%%%%%%%%%%%%%%%%%%%%%%%%%%%
\subsection*{Conjunctions \textit{kin} Temperature} 
\label{'conjunctions_temperature'}
%
\begin{supertabular}{p{5,5cm}|ll}
What are conjunctions? && Conjunctions connect words and phrases. \\ % no-dictionary
What is an answer-question? && The answer is already included in the question. \\ % no-dictionary
What is the difference between conjunctions and prepositions? && Conjunctions do not cause cases. \\ % no-dictionary
How is an answer-question formed in \textit{toki pona}? && The conjunction \textit{anu} and the question pronoun \textit{seme} is added. \\ % no-dictionary
Is there a comma before or after the conjunction \textit{taso}? && No, it is not. \\ % no-dictionary
What are alternative-questions? && A selection of several options is requested.  \\ % no-dictionary
What connects the conjunction \textit{taso}? && It refers to the previous sentence.  \\ % no-dictionary
What connects the conjunction \textit{en}? && It combines (composite) nouns or pronouns. \\ % no-dictionary
How is an alternative-question formed in \textit{toki pona}? && With the conjunction \textit{anu}. \\ % no-dictionary
How is a yes/no-question with predicate nouns or predicate adjectives formed in \textit{toki pona}? && An answer question is formulated. \\ % no-dictionary
\end{supertabular}

\begin{supertabular}{p{5,5cm}|ll}
Do you want to come or what?  && sina wile kama anu seme? \\ % & English - Toki Pona
Do you want food, or do you want water?  && sina wile e moku anu telo? \\ % & English - Toki Pona
I still want to go to my house.  && mi wile kin tawa, tawa tomo mi. \\ % & English - Toki Pona
This paper feels cold.  && lipu ni li ' lete, tawa mi. \\ % & English - Toki Pona
I like currency of other nations.  && mani pi ma ante li ' pona, tawa mi. \\ % & English - Toki Pona
I want to go, but I can't.  && mi wile tawa. taso mi ken ala. \\ % & English - Toki Pona
I'm alone.  && mi taso li lon. \\ % & English - Toki Pona
Do you like me?  && mi ' pona, tawa sina anu seme? \\ % & English - Toki Pona
This lake is cold. && telo ni li ' lete, tawa mi. \\ % & English - Toki Pona
 && \\ % no-dictionary
mi olin kin e sina.  && I still love you. / I love you too.\\
mi pilin e ni: ona li jo ala e mani.  && I think that he doesn't have money. \\
mi wile lukin e ma ante.  && I want to see other countries. \\
mi wile ala e ijo. mi lukin taso.  && I don't want anything. I'm just looking. \\
mi pilin lete.  && I'm cold. \\
 &&   (lit. "I feel cold.") \\ % no-dictionary
sina wile toki, tawa mije anu meli?  && Do you want to talk a male, or a female? \\
\end{supertabular} 
\newpage
%%%%%%%%%%%%%%%%%%%%%%%%%%%%%%%%%%%%%%%%%%%%%%%%%%%%%%%%%%%%%%%%%%%%%%%%%%
\subsection*{Colors} 
\label{'colors'}
%
\begin{supertabular}{p{5,5cm}|ll}
Which kinds of word are possible in the slot after the conjunction \textit{en}? && Noun or pronouns.  \\ % no-dictionary
How are color pattern of an item described in \textit{toki pona}? && Item + \textit{pi} + 1. colour + \textit{en} + 2. colour \dots \\ % no-dictionary
How are color tones described for which there is no word in \textit{toki pona}? && Through several words. \\ % no-dictionary
Which kinds of word are possible in the slot after the separator \textit{pi}? && Noun or pronouns.  \\ % no-dictionary
What kinds of words have the words for colors in \textit{toki pona}? && Adjectives and nouns. \\ % no-dictionary
\end{supertabular} 

\begin{supertabular}{p{5,5cm}|ll}
I don't see the blue bag.  && mi lukin ala e poki laso. \\ % & English - Toki Pona
A little green person came from the sky.  && jan laso jelo lili li kama, tan sewi. / \\ % & English - Toki Pona
A little green person came from the sky.  && jan jelo laso lili li kama, tan sewi. \\  % & English - Toki Pona
I like the color purple.  && kule loje laso li ' pona, tawa mi. / \\ % & English - Toki Pona
I like the color purple.  && kule laso loje li ' pona, tawa mi. \\ % & English - Toki Pona
The sky is blue.  && sewi li ' laso. \\ % & English - Toki Pona
Look at that red bug.  && o lukin e pipi loje ni!  \\ % & English - Toki Pona
I want the map.  && mi wile e sitelen ma. \\ % & English - Toki Pona
Do you watch The X-Files?  && sina lukin ala lukin e sitelen tawa X-Files? \\ % & English - Toki Pona
Which color do you like?  && kule seme li ' pona, tawa sina? \\ % & English - Toki Pona
Is it red? && ona li ' loje anu seme?  \\ % & English - Toki Pona
 && \\ % no-dictionary
ni li pimeja ala pimeja e suno? && Does that darken the sun? \\
suno li ' jelo.  && The sun is yellow. \\
telo suli li ' laso.  && The big water [ocean] is blue. \\
mi wile moku e kili loje.  && I want to eat a red fruit. \\
ona li kule e tomo tawa.  && He's painting the car. \\
len pi loje en laso pi meli sina li ' pona, tawa mi. && I like your wife's red and blue patterned dress. \\
 && \\ % no-dictionary
ma mi li ' pimeja. && My land is dark. \\
kalama ala li lon && No sound exists.\\
mi lape. mi sona. && I sleep. I know. \\
\end{supertabular} 

%%%%%%%%%%%%%%%%%%%%%%%%%%%%%%%%%%%%%%%%%%%%%%%%%%%%%%%%%%%%%%%%%%%%%%%%%%
\subsection*{Living Things} 
\label{'living_things'}
%
\begin{supertabular}{p{5,5cm}|ll}
Which separator is at the end of a question? && A question mark.  \\ % no-dictionary
In which cases is a comma used? && Addressing people: after \textit{o}. Optionally before prepositions. \\ % no-dictionary
In which cases a colon is used? && A colon is between an hint sentences and a sentences. \\ % no-dictionary
Where are possible slots for prepositions in a sentence? && At the beginning of a prepositional object. \\ % no-dictionary
\end{supertabular}

\begin{supertabular}{p{5,5cm}|ll}
Is this a mammal? && ni li ' soweli anu seme?  \\ % & English - Toki Pona
I want a puppy.  && mi wile e soweli lili. \\ % & English - Toki Pona
Ahh! The dinosaur wants to eat me!  && a! akesi li wile moku e mi! \\ % & English - Toki Pona
The mosquito bit me.  && pipi li moku e mi.  \\ % & English - Toki Pona
Cows say moo.  && soweli li toki e mu. \\ % & English - Toki Pona
Birds fly in air.  && waso li tawa, lon kon. \\ % & English - Toki Pona
Let's eat fish.  && mi mute o moku e kala! \\ % & English - Toki Pona
Flowers are pretty.  && kasi kule li ' pona lukin. \\ % & English - Toki Pona
I like plants.  && kasi li ' pona, tawa mi. \\ % & English - Toki Pona
Have you improved? && sina pona ala pona e sina? sina pona e sina anu seme? \\ % & English - Toki Pona
 && \\ % no-dictionary
mama ona li kepeken kasi nasa.  && His mother used pot. \\
akesi li pana e telo moli.  && The snake emitted venom ("deadly fluid"). \\
pipi li moku e kasi.  && Bugs eat plants. \\
soweli mi li kama moli.  && My dog is dying. \\
jan Pawe o, mi wile ala moli.  && Forrest, I don't want to die. \\
mi lon ma kasi.  && I'm in the forest. \\
ona li kasi ala kasi? && Is it growing?  \\
\end{supertabular} 

%%%%%%%%%%%%%%%%%%%%%%%%%%%%%%%%%%%%%%%%%%%%%%%%%%%%%%%%%%%%%%%%%%%%%%%%%%
\subsection*{The Body} 
\label{'the_body'}
%







\begin{supertabular}{p{5,5cm}|ll}
kepeken - mi kepeken ilo. && intransitive verb, noun \\ % no-dictionary
sina - sina pona ala pona? && transitive verb \ % no-dictionary
kama - mi kama jo e tomo tawa. && auxiliary verb \\ % no-dictionary
lon - mi lon tomo. && intransitive verb, adverb, adjective, noun \\ % no-dictionary
kepeken - mi pali e ni, kepeken ilo. && preposition \\ % no-dictionary
\end{supertabular}



















\begin{supertabular}{p{5,5cm}|ll}
Kiss me.  && o pilin e uta mi, kepeken uta sina! \\ % & English - Toki Pona
I need to pee.  && mi wile pana e telo jelo. \\ % & English - Toki Pona
My hair is wet.  && linja mi li ' telo. \\ % & English - Toki Pona
Something is in my eye.  && ijo li lon oko mi. \\ % & English - Toki Pona
I can't hear your talking.  && mi ken ala kute e toki sina. \\ % & English - Toki Pona
I need to crap.  && mi wile pana e ko jaki. \\ % & English - Toki Pona
That hole is big.  && lupa ni li ' suli. \\ % & English - Toki Pona
Is it a chain? && ona li ' linja anu seme? \\ % & English - Toki Pona
 && \\ % no-dictionary
selo pi jelo en laso pi akesi lili li ' pona, tawa mi. && I like the little lizard's green-blue skin. \\ % & English - Toki Pona
a! telo sijelo loje li kama tan nena kute mi!  && Ahh! Blood is coming from my ear! \\
selo mi li wile e ni: mi pilin e ona.  && My skin wants this: I touch it. \\
   && This is how we say that our skin itches. \\  % no-dictionary
o pilin e nena.  && Touch the button. \\
o moli e pipi, kepeken palisa.  && Kill the roach with the stick. \\
luka mi li ' jaki. mi wile telo e ona.  && My hands are dirty. I want to wash them. \\
o pana e sike, tawa mi.  && Give the ball to me. \\
mi pilin e seli sijelo sina.  && I feel your bodily warmth. \\
ona li selo ala selo? && Is it protecting? \\ 
\end{supertabular} 
%
\newpage
%%%%%%%%%%%%%%%%%%%%%%%%%%%%%%%%%%%%%%%%%%%%%%%%%%%%%%%%%%%%%%%%%%%%%%%%%%
\subsection*{Numbers} 
\label{'numbers'}
%
\begin{supertabular}{p{5,5cm}|ll}
I saw three birds.  && mi lukin e waso tu wan. \\ % & English - Toki Pona
Many people are coming.  && jan mute li kama. \\ % & English - Toki Pona
The first person is here.  && jan pi nanpa wan li lon. \\ % & English - Toki Pona
I own two cars.  && mi jo e tomo tawa tu. \\ % & English - Toki Pona
Some (but not a lot) of people are coming.  && jan mute lili li kama. \\ % & English - Toki Pona
Unite!  && o wan! \\ % & English - Toki Pona
Is this a part? && ni li ' wan anu seme? \\ % & English - Toki Pona
 && \\ % no-dictionary
mi weka e ijo tu ni.  && I got rid of those two things. \\
o tu.  && Break up. Split apart. \\
mi lukin e soweli luka.  && I saw five mammals. \\
mi ' weka.  && I was away. \\
ona li sike ala sike? && Is it rotating? \\
\end{supertabular} 

%%%%%%%%%%%%%%%%%%%%%%%%%%%%%%%%%%%%%%%%%%%%%%%%%%%%%%%%%%%%%%%%%%%%%%%%%%
\subsection*{Conditional Sentences} 
\label{'la'}
%

%\begin{supertabular}{p{5,5cm}|ll}
%la - ken la ni li pona. && separator \\ % no-dictionary
%ken - ken la mi tawa. && noun \\ % no-dictionary
%\end{supertabular}

\begin{supertabular}{p{5,5cm}|ll}
What is a conditional phrase? && It formulates a condition. \\  % no-dictionary
What follows the separator \textit{la}? && A complete main sentence.  \\  % no-dictionary
What can a conditional phrase consist of?  && It consists of a (composite) noun/pronoun or a complete sentence. \\  % no-dictionary
Which word types can be at the beginning of a conditional phrase? && Noun or pronoun. Optionally, there can be a conjunction before. \\  % no-dictionary
Can the question pronoun \textit{seme} be in a conditional phrase? && Yes, in a interrogative sentence.  \\  % no-dictionary
\end{supertabular}

\begin{supertabular}{p{5,5cm}|ll}
Maybe Susan will come.  && ken la jan Susan li kama. \\ % & English - Toki Pona
Last night I watched X-Files.  && tenpo pimeja pini la mi lukin e sitelen tawa X-Files. \\ % & English - Toki Pona
If the enemy comes, burn these papers.  && jan ike li kama la o seli e lipu ni! \\ % & English - Toki Pona
Maybe he's in school.  && ken la ona li lon tomo sona. \\ % & English - Toki Pona
I have to work tomorrow.  && tenpo suno kama la mi wile pali. \\ % & English - Toki Pona
When it's hot, I sweat.  && seli li lon la mi pana e telo, tan selo mi. \\ % & English - Toki Pona
Open the door.  && o open e lupa! \\ % & English - Toki Pona
The moon is big tonight.  && tenpo pimeja ni la mun li ' suli. \\ % & English - Toki Pona
Is the moon big tonight?  && tenpo pimeja ni la mun li ' suli anu seme? \\ % & English - Toki Pona
Under what conditions will you do this? && seme la sina pali e ni? \\ % & English - Toki Pona
 && \\ % no-dictionary
tenpo suno ni la mun li pimeja ala pimeja e suno? && Is there an eclipse today? \\
ken la jan lili li wile moku e telo.  && Maybe the baby is thirsty. \\
tenpo ali la o kama sona!  && Always learn!  \\
sina sona e toki ni la sina sona e toki pona!  && Figure this one out for yourself. :o) \\
open la ala li lon! && There was nothing in the beginning! \\
ken la tomo pi ona en sina pi jelo en loje li ' ike, tawa mi. &&  Maybe I don't like the yellow-red patterned house of her and you.  \\
\end{supertabular}  
%%%%%%%%%%%%%%%%%%%%%%%%%%%%%%%%%%%%%%%%%%%%%%%%%%%%%%%%%%%%%%%%%%%%%%%%%%
%\subsection*{Conclusion} \\
%-
%%%%%%%%%%%%%%%%%%%%%%%%%%%%%%%%%%%%%%%%%%%%%%%%%%%%%%%%%%%%%%%%%%%%%%%%%%
%%%%%%%%%%%%%%%%%%%%%%%%%%%%%%%%%%%%%%%%%%%%%%%%%%%%%%%%%%%%%%%%%%%%%%%%%%
% eof
