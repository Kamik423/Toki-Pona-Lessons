%%%%%%%%%%%%%%%%%%%%%%%%%%%%%%%%%%%%%%%%%%%%%%%%%%%%%%%%%%%%%%%%%%%%%%%%%%
\section{Indirect Objects}
%%%%%%%%%%%%%%%%%%%%%%%%%%%%%%%%%%%%%%%%%%%%%%%%%%%%%%%%%%%%%%%%%%%%%%%%%%
\index{Indirect Object}
\index{\textit{lon}}
\index{\textit{kepeken}}
\index{\textit{tawa}}
\index{\textit{kama}}       
\index{\textit{kiwen}}
\index{\textit{kon}}
\index{\textit{pana}}
\index{\textit{poki}}
\index{\textit{toki}}
\index{exist}
\index{on}
\index{at}
\index{in}
\index{be!in}
\index{be!at}
\index{be!on}
\index{with}
\index{using}
\index{go}
\index{move}
\index{for}
\index{come}
\index{happen}
\index{cause}
\index{stone}
\index{rock}
\index{air}
\index{atmosphere}
\index{spirit}
\index{wind}
\index{square}
\index{block}
\index{stairs}
\index{give}
\index{send}
\index{release}
\index{emit}
\index{container}
\index{bowl}
\index{glass}
\index{cup}
\index{box}
\index{language}
\index{talk}
\index{speak}
%%%%%%%%%%%%%%%%%%%%%%%%%%%%%%%%%%%%%%%%%%%%%%%%%%%%%%%%%%%%%%%%%%%%%%%%%%
\subsection*{Vocabulary}
%%%%%%%%%%%%%%%%%%%%%%%%%%%%%%%%%%%%%%%%%%%%%%%%%%%%%%%%%%%%%%%%%%%%%%%%%%
\begin{supertabular}{p{2,5cm}|ll}
\textbf{kepeken} && \textit{noun}: use, usage, tool \\ % no-dictionary
\textbf{\dots , kepeken \dots} && \textit{preposition}: with \\ % no-dictionary
\textbf{kepeken} && \textit{verb intransitive}: to use \\ % no-dictionary
\textbf{kepeken \dots} && \textit{auxiliary verb}: to use \\ % no-dictionary
 && \\ % no-dictionary
\textbf{\dots kiwen} && \textit{adjective}: hard, solid, stone-like, made of stone or metal \\ % no-dictionary
\textbf{\dots kiwen} && \textit{adverb}: hard, solid, stone-like, made of stone or metal \\ % no-dictionary
\textbf{kiwen} && \textit{noun}: hard thing, rock, stone, metal, mineral, clay \\ % no-dictionary
\textbf{kiwen (e \dots)} && \textit{verb transitive}: to solidify, to harden, to petrify, to fossilize \\ % no-dictionary
 && \\ % no-dictionary
\textbf{\dots kon} && \textit{adjective}: air-like, ethereal, gaseous \\ % no-dictionary
\textbf{\dots kon} && \textit{adverb}: air-like, ethereal, gaseous \\ % no-dictionary
\textbf{kon} && \textit{noun}: air, wind, smell, soul \\ % no-dictionary
\textbf{kon} && \textit{verb intransitive:}: to breathe \\ % no-dictionary
\textbf{kon (e \dots)} && \textit{verb transitive}: to blow away something, to puff away something \\ % no-dictionary
 && \\ % no-dictionary
\textbf{\dots lon} && \textit{adjective}: true, existing, correct, real, genuine \\ % no-dictionary
\textbf{lon} && \textit{noun}: existence, being, presence \\ % no-dictionary
\textbf{\dots , lon \dots} && \textit{preposition}: be (located) in/at/on \\ % no-dictionary
\textbf{lon} && \textit{verb intransitive}: to be there, to be present, to be real/true, to exist \\ % no-dictionary
 && \\ % no-dictionary
\textbf{\dots pana} && \textit{adjective}: generous \\ % no-dictionary
\textbf{pana} && \textit{noun}: giving, transfer, exchange \\ % no-dictionary
\textbf{pana (e \dots)} && \textit{verb transitive}: to give, to put, to send, to place, to release, to emit, to cause \\ % no-dictionary
 && \\ % no-dictionary
\textbf{poki} && \textit{noun}: container, box, bowl, cup, glass \\ % no-dictionary
\textbf{poki (e \dots)} && \textit{verb transitive}: to box up, to put in, to can, to bottle \\ % no-dictionary
 && \\ % no-dictionary
\textbf{\dots tawa} && \textit{adjective}: moving, mobile \\ % no-dictionary
\textbf{\dots tawa} && \textit{adverb}: moving, mobile \\ % no-dictionary
\textbf{tawa} && \textit{noun}: movement, transportation \\ % no-dictionary
\textbf{\dots , tawa \dots} && \textit{preposition}: to, in order to, towards, for, until \\ % no-dictionary
\textbf{tawa} && \textit{verb intransitive}: go to, walk, travel, move, leave \\ % no-dictionary
\textbf{tawa (e \dots)} && \textit{verb transitive}: to move, to displace \\ % no-dictionary
\end{supertabular} \\
%
%%%%%%%%%%%%%%%%%%%%%%%%%%%%%%%%%%%%%%%%%%%%%%%%%%%%%%%%%%%%%%%%%%%%%%%%%%
\newpage
\subsection*{Indirect Objects and Intransitive Verbs}
\index{Object!indirect}
\index{verb!intransitive}
\index{verb!transitive}
%%%%%%%%%%%%%%%%%%%%%%%%%%%%%%%%%%%%%%%%%%%%%%%%%%%%%%%%%%%%%%%%%%%%%%%%%%
%
We've already learned about direct objects. 
A direct object is most strongly influenced by the action (i. e. the transitive verb). 
Your can ask for direct object (accusative object) by' Who' or' What' (' What does she repair?'').
But, in the sentence, 'I am in the house.' is 'in the house' an indirect object because you can't ask for it by' Who' or' What'.
It is also not directly influenced by the predicate. 
A indirect object is part of the predicate phrase also. 
In the indirect object is the first slot always a noun or pronoun slot.
After that, optional slots for adjectives, possessive pronouns and demonstrative pronouns are possible. 

We've already learned transitive verbs. 
A transitive verb does something to the direct object. 
On the other hand, verbs that do not affect an object are called intransitive verbs. 
An intransitive verb is followed by either no object or an indirect object. 
In the sentences, 'I am.' and 'I am in the house.' is 'am' an intransitive verb. 
There is no \textit{e} between intrasitive verb and indirect object.

%%%%%%%%%%%%%%%%%%%%%%%%%%%%%%%%%%%%%%%%%%%%%%%%%%%%%%%%%%%%%%%%%%%%%%%%%%
\index{verb!\textit{lon}}
\index{\textit{lon}!verb}
%%%%%%%%%%%%%%%%%%%%%%%%%%%%%%%%%%%%%%%%%%%%%%%%%%%%%%%%%%%%%%%%%%%%%%%%%%
%
In this examples, \textit{lon} is used as an intransitive verb. 
Since there is no other predicate before \textit{lon} there must be a verb \textit{lon}.

\begin{supertabular}{p{5,5cm}|ll}
suno li lon sewi. && The sun is in the sky. \\
kili li lon poki. && The fruit is in the basket. \\
mi lon tomo. && I'm in the house. \\
\end{supertabular} 

%%%%%%%%%%%%%%%%%%%%%%%%%%%%%%%%%%%%%%%%%%%%%%%%%%%%%%%%%%%%%%%%%%%%%%%%%%
\index{verb!\textit{kepeken}}
\index{\textit{kepeken}!verb}
%
Using \textit{kepeken} as an intransitive verb.
 
\begin{supertabular}{p{5,5cm}|ll}
mi kepeken ilo. && I'm using tools. \\
sina wile kepeken ilo. && You have to use tools. \\
mi kepeken poki ni. && I'm using that cup. \\
\end{supertabular} 

In some other lessons \textit{kepken} is used as a transitive verb.
This is surely because with 'What' you can ask for the object after \textit{kepken}. 
As however the object is not directly influenced by the verb \textit{kepeken}, it is an indirect object and \textit{kepeken} an intransitive verb. 









%
%%%%%%%%%%%%%%%%%%%%%%%%%%%%%%%%%%%%%%%%%%%%%%%%%%%%%%%%%%%%%%%%%%%%%%%%%%
\newpage
\subsection*{Practice (Answers: Page~\pageref{'indirect_objects'})}
%%%%%%%%%%%%%%%%%%%%%%%%%%%%%%%%%%%%%%%%%%%%%%%%%%%%%%%%%%%%%%%%%%%%%%%%%%
%
Please write down your answers and check them afterwards. 

Try to translate these sentences. 
You can use the tool \textit{Toki Pona Parser} (\cite{www:rowa:02}) for spelling and grammar check. 

\begin{supertabular}{p{5,5cm}|ll}
This is for my friend.  &&  \\ % no-dictionary
The tools are in the container.  && \\ % no-dictionary
That bottle is in the dirt.  &&  \\ % no-dictionary
They are arguing. &&  \\ % no-dictionary
\end{supertabular} 

%
%%%%%%%%%%%%%%%%%%%%%%%%%%%%%%%%%%%%%%%%%%%%%%%%%%%%%%%%%%%%%%%%%%%%%%%%%%
% eof
