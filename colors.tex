%%%%%%%%%%%%%%%%%%%%%%%%%%%%%%%%%%%%%%%%%%%%%%%%%%%%%%%%%%%%%%%%%%%%%%%%%%
\section{Colors}
%%%%%%%%%%%%%%%%%%%%%%%%%%%%%%%%%%%%%%%%%%%%%%%%%%%%%%%%%%%%%%%%%%%%%%%%%%
%
%%%%%%%%%%%%%%%%%%%%%%%%%%%%%%%%%%%%%%%%%%%%%%%%%%%%%%%%%%%%%%%%%%%%%%%%%%
\subsection*{Vocabulary}
%%%%%%%%%%%%%%%%%%%%%%%%%%%%%%%%%%%%%%%%%%%%%%%%%%%%%%%%%%%%%%%%%%%%%%%%%%
%
\index{\textit{jelo}}
\index{\textit{kule}}
\index{\textit{laso}}
\index{\textit{loje}}
\index{\textit{pimeja}}
\index{\textit{sitelen}}
\index{\textit{walo}}
\index{yellow}
\index{paint}
\index{color}
\index{blue}
\index{red}
\index{black}
\index{dark}
\index{picture}
\index{image}
\index{draw}
\index{write}
\index{white}
\begin{supertabular}{p{5,5cm}|ll}
kule && color, to paint, to color \\                
sitelen && picture, image, to draw, to write \\
jelo && yellow \\
laso && blue \\
loje && red \\
pimeja && black, dark \\
walo && white \\
\end{supertabular} 
%
%%%%%%%%%%%%%%%%%%%%%%%%%%%%%%%%%%%%%%%%%%%%%%%%%%%%%%%%%%%%%%%%%%%%%%%%%%
\index{color}
\index{color!combination}
\index{purple}
\index{green}
\index{grey}
\subsection*{Color combinations}
\subsubsection*{jelo, laso, loje, pimeja and walo as adjectives}

%%%%%%%%%%%%%%%%%%%%%%%%%%%%%%%%%%%%%%%%%%%%%%%%%%%%%%%%%%%%%%%%%%%%%%%%%%
%
You can \textbf{mix different colors}. 
Toki Pona is about simplicity, so keep it basic. 

\begin{supertabular}{p{5,5cm}|ll}
laso loje && purple (reddish blue) \\
laso jelo && green (yellowish blue) \\
loje jelo && orange (yellowish red) \\
loje walo && pink (whitish red) \\
walo pimeja && grey (dark white) \\
\end{supertabular} 

Keep in mind that colors by themselves can't really follow any sort of logical pattern, so you're free to mix them around as you like. 

\begin{supertabular}{p{5,5cm}|ll}
laso loje / loje laso && reddish blue (purple \\
laso jelo / jelo laso && yellowish blue (green) \\
\end{supertabular} 
%
%% \includegraphics[scale=0.5]{sikekule3.png}
%
%%%%%%%%%%%%%%%%%%%%%%%%%%%%%%%%%%%%%%%%%%%%%%%%%%%%%%%%%%%%%%%%%%%%%%%%%%
\index{color!\textit{pi}}
\index{color!\textit{en}}
\index{\textit{pi}!color}
\index{\textit{en}!color}
\index{color!separate}
\subsubsection*{Using colors with \textit{pi}}
%%%%%%%%%%%%%%%%%%%%%%%%%%%%%%%%%%%%%%%%%%%%%%%%%%%%%%%%%%%%%%%%%%%%%%%%%%
%
Suppose that you have a shirt that have pattern with different colors (red and blue). 
However, you can't call it \textit{len loje laso}, because that means "purple shirt". 
So, we have to \textbf{use \textit{en} to separate the two colors}, and then we have to use \textbf{\textit{pi}} to show that the even though there are two different colors, they both modify the word "shirt".
\textit{loje} and \textit{laso} are nouns here. 

\begin{supertabular}{p{5,5cm}|ll}
len \textbf{pi} loje \textbf{en} laso && shirt of red and blue \\
\end{supertabular} 
%
%%%%%%%%%%%%%%%%%%%%%%%%%%%%%%%%%%%%%%%%%%%%%%%%%%%%%%%%%%%%%%%%%%%%%%%%%%
\index{\textit{kule}!\textit{seme}}
\index{what!color}
\index{color!what}
\subsection*{\textit{kule}}
%%%%%%%%%%%%%%%%%%%%%%%%%%%%%%%%%%%%%%%%%%%%%%%%%%%%%%%%%%%%%%%%%%%%%%%%%%
%
\index{\textit{kule}!noun}
\index{noun!\textit{kule}}
\textbf{\textit{kule} as a noun} \\
\begin{supertabular}{p{5,5cm}|ll}
ni li ' \textbf{kule seme?} && What color is that? \\
\end{supertabular} 

\index{\textit{kule}!verb}
\index{verb!\textit{kule}}
\index{paint}
\textbf{\textit{kule} as a verb} \\
\begin{supertabular}{p{5,5cm}|ll}
mi \textbf{kule} e lipu && I'm coloring the paper. \\
\end{supertabular} 
%
%%%%%%%%%%%%%%%%%%%%%%%%%%%%%%%%%%%%%%%%%%%%%%%%%%%%%%%%%%%%%%%%%%%%%%%%%%
\newpage
\subsection*{Miscellaneous}
\subsubsection*{sitelen}
%%%%%%%%%%%%%%%%%%%%%%%%%%%%%%%%%%%%%%%%%%%%%%%%%%%%%%%%%%%%%%%%%%%%%%%%%%
%
\index{\textit{sitelen}}
\index{picture}
\index{image}
\index{draw}
\index{write}
\index{write}
\index{picture!motion}
\index{motion picture}
\index{movie}
\index{TV show}
\index{show!TV}
\index{map}
\index{\textit{sitelen}!\textit{ma}}
\index{\textit{ma}!\textit{sitelen}}
% The word for today's miscellaneous section is \textbf{\textit{sitelen}}. 
sitelen as a \textbf{noun} means "\textbf{picture}" or "\textbf{image}". 
As a \textbf{verb}, it means to "\textbf{draw}" or to "\textbf{write}". 
\textit{sitelen} is most useful for the \textbf{compound nouns} that you can make with it. 
% \textit{sitelen tawa} ("motion picture") is used to mean either a "movie" or a "TV show". 

\begin{supertabular}{p{5,5cm}|ll}
\textbf{sitelen tawa}  && movie, TV show \\
\textbf{sitelen tawa} "Fahrenheit 9/11" li pona, tawa mi. && I like the movie "Fahrenheit 9/11". \\
\textbf{sitelen tawa} "Bowling for Columbine" li pona kin. && The movie "Bowling for Columbine" is also good. \\
\textbf{sitelen ma} && map \\
o pana e \textbf{sitelen ma}, tawa mi. && Give me the map. \\
\end{supertabular} 
%
%%%%%%%%%%%%%%%%%%%%%%%%%%%%%%%%%%%%%%%%%%%%%%%%%%%%%%%%%%%%%%%%%%%%%%%%%%
\subsection*{Practice 13 (Answers: Page~\pageref{'colors'})}
%%%%%%%%%%%%%%%%%%%%%%%%%%%%%%%%%%%%%%%%%%%%%%%%%%%%%%%%%%%%%%%%%%%%%%%%%%
%
Try translating these sentences. \\

\begin{supertabular}{p{5,5cm}|ll}
I don't see the blue bag. &&   \\ % no-dictionary
A little green person came from the sky. &&   \\ % no-dictionary
I like the color purple.  &&  \\ % no-dictionary
The sky is blue. &&   \\ % no-dictionary
Look at that red bug.  &&  \\ % no-dictionary
I want the map.  &&  \\ % no-dictionary
Do you watch The X-Files? &&  \\  % no-dictionary
Which color do you like?* &&  \\  % no-dictionary
 && \\ % no-dictionary
suno li ' jelo. &&   \\ % no-dictionary
telo suli li ' laso.  &&  \\ % no-dictionary
mi wile moku e kili loje.  &&  \\ % no-dictionary
ona li kule e tomo tawa. &&   \\ % no-dictionary
\end{supertabular} 

* Think: "Which color is good for you?" 

And now try reading this Toki Pona poem. 

\begin{supertabular}{p{5,5cm}|ll}
ma mi li ' pimeja. && \\ % no-dictionary
kalama ala li lon && \\ % no-dictionary
mi lape. mi sona. && \\ % no-dictionary
\end{supertabular} 
%%%%%%%%%%%%%%%%%%%%%%%%%%%%%%%%%%%%%%%%%%%%%%%%%%%%%%%%%%%%%%%%%%%%%%%%%%
% eof
