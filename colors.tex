%%%%%%%%%%%%%%%%%%%%%%%%%%%%%%%%%%%%%%%%%%%%%%%%%%%%%%%%%%%%%%%%%%%%%%%%%%
\section{Colors}
%%%%%%%%%%%%%%%%%%%%%%%%%%%%%%%%%%%%%%%%%%%%%%%%%%%%%%%%%%%%%%%%%%%%%%%%%%
%
%%%%%%%%%%%%%%%%%%%%%%%%%%%%%%%%%%%%%%%%%%%%%%%%%%%%%%%%%%%%%%%%%%%%%%%%%%
\subsection*{Vocabulary}
%%%%%%%%%%%%%%%%%%%%%%%%%%%%%%%%%%%%%%%%%%%%%%%%%%%%%%%%%%%%%%%%%%%%%%%%%%
%
\begin{supertabular}{p{2,5cm}|ll}
%
\index{jelo}
\textbf{\dots jelo} && \textit{adjective}: yellowish, yellowy \\ % no-dictionary
\textbf{jelo} && \textit{noun}: yellow, light green \\ % no-dictionary
 && \\ % no-dictionary
%
\index{kule}
\textbf{\dots kule} && \textit{adjective}: colourful, pigmented, painted \\ % no-dictionary
\textbf{kule} && \textit{noun}: color, colour, paint, ink, dye, hue \\ % no-dictionary
\textbf{kule (e \dots)} && \textit{verb transitive}: to paint, to color \\ % no-dictionary
 && \\ % no-dictionary
%
\index{laso}
\textbf{\dots laso} && \textit{adjective}: bluish, bluey \\ % no-dictionary
\textbf{laso} && \textit{noun}: blue, blue-green \\ % no-dictionary
 && \\ % no-dictionary
%
\index{loje}
\textbf{\dots loje} && \textit{adjective}: reddish, ruddy, pink, pinkish, gingery \\ % no-dictionary
\textbf{loje} && \textit{noun}: red \\ % no-dictionary
 && \\ % no-dictionary
%
\index{pimeja}
\textbf{\dots pimeja} && \textit{adjective}: black, dark \\ % no-dictionary
\textbf{pimeja} && \textit{noun}: darkness, shadows \\ % no-dictionary
\textbf{pimeja (e \dots)} && \textit{verb transitive}: to darken \\ % no-dictionary
 && \\ % no-dictionary
%
\index{sitelen}
\textbf{\dots sitelen} && \textit{adjective}: figurative, pictorial, metaphorical, metaphorisch \\ % no-dictionary
\textbf{\dots sitelen} && \textit{adverb}: pictorially \\ % no-dictionary
\textbf{sitelen} && \textit{noun}: picture, image, representation, symbol, mark, writing \\ % no-dictionary
\textbf{sitelen (e \dots)} && \textit{verb transitive}: to draw, to write \\ % no-dictionary
 && \\ % no-dictionary
%
\index{walo}
\textbf{\dots walo} && \textit{adjective}: white, whitish, light-coloured, pale \\ % no-dictionary
\textbf{walo} && \textit{noun}: white thing or part, whiteness, lightness \\ % no-dictionary
\textbf{walo (e \dots)} && \textit{verb transitive}: to whiten, to whitewash \\ % no-dictionary
%
\end{supertabular} \\
%
%
%%%%%%%%%%%%%%%%%%%%%%%%%%%%%%%%%%%%%%%%%%%%%%%%%%%%%%%%%%%%%%%%%%%%%%%%%%
\newpage
%
\subsection*{Color Combinations}
\subsubsection*{A Shade of Colour}
%
\index{color}
%%%%%%%%%%%%%%%%%%%%%%%%%%%%%%%%%%%%%%%%%%%%%%%%%%%%%%%%%%%%%%%%%%%%%%%%%%
%
In Toki Pona there are no words for the colors purple, green, grey, etc.
But you can create colors from several words.
One uses one of these nouns \textit{jelo}, \textit{laso}, \textit{loje}, \textit{pimeja} or \textit{walo}. 
Then use these adjectives \textit{jelo}, \textit{laso}, \textit{loje}, \textit{pimeja}, or \textit{walo}. 

\begin{supertabular}{p{5,5cm}|ll}
laso loje li ' pona, tawa mi. && Purple (reddish blue) is my favourite colour. \\
laso jelo li ' pona, tawa mi. && Green (yellowish blue) is my favourite colour. \\
loje jelo li ' pona, tawa mi. && Orange (yellowish red) is my favourite colour. \\
loje walo li ' pona, tawa mi. && Pink (whitish red) is my favourite colour. \\
walo pimeja li ' pona, tawa mi. && Grey (dark white) is my favourite colour. \\
\end{supertabular} 

It is also possible to form colors from a noun and several adjectives. 
The goal of Toki Pona is however the simplicity.
Therefore, avoid complex word compositions.
Incidentally, the order of the colours doesn't matter.

\begin{supertabular}{p{5,5cm}|ll}
laso loje  li ' pona, tawa mi. && Purple is my favourite colour.  \\ % no-dictionary
loje laso  li ' pona, tawa mi. && Purple is my favourite colour.  \\
\end{supertabular}

Colors are usually used as adjectives because they describe nouns. 
The adjectives \textit{loje} and \textit{laso} describe the noun \textit{len} here.

\begin{supertabular}{p{5,5cm}|ll}
len loje laso mi li ' pona, tawa mi. && Dieses lila T-Shirt gefällt mir. \\
\end{supertabular}

%
%% \includegraphics[scale=0.5]{sikekule3.png}
%
%%%%%%%%%%%%%%%%%%%%%%%%%%%%%%%%%%%%%%%%%%%%%%%%%%%%%%%%%%%%%%%%%%%%%%%%%%
\subsubsection*{Samples in Several Shades of Colour}
%
%%%%%%%%%%%%%%%%%%%%%%%%%%%%%%%%%%%%%%%%%%%%%%%%%%%%%%%%%%%%%%%%%%%%%%%%%%
%
Suppose that you have a shirt that have pattern with different colors (red and blue). 
However, you can't call it \textit{len loje laso}, because that means 'purple shirt'. 
The colours must be separated grammatically. 
Each color of the pattern is described with a noun and optional adjectives. 
To separate these color nouns with their adjectives we use the conjunction \textit{en}.
To separate the patterned item from its colours the separator serves \textit{pi}.
\textit{len}, \textit{loje} and \textit{laso} are nouns here. 

\begin{supertabular}{p{5,5cm}|ll}
len ni pi loje en laso li ' pona, tawa mi. && I like this red and blue patterned t-shirt. \\
\end{supertabular} 
%
%%%%%%%%%%%%%%%%%%%%%%%%%%%%%%%%%%%%%%%%%%%%%%%%%%%%%%%%%%%%%%%%%%%%%%%%%%
\subsection*{\textit{kule} as a Noun or as a as a Verb}
%
\index{\textit{kule}!noun}
\index{\textit{kule}!verb}
%%%%%%%%%%%%%%%%%%%%%%%%%%%%%%%%%%%%%%%%%%%%%%%%%%%%%%%%%%%%%%%%%%%%%%%%%%
%
\textit{kule} as noun means 'color'. 

\begin{supertabular}{p{5,5cm}|ll}
ni li ' kule seme? && What color is that? \\
\end{supertabular} 

As verb \textit{kule} means 'to dye'. 

\begin{supertabular}{p{5,5cm}|ll}
mi kule e lipu && I dye the dress. \\
\end{supertabular} 
%
%%%%%%%%%%%%%%%%%%%%%%%%%%%%%%%%%%%%%%%%%%%%%%%%%%%%%%%%%%%%%%%%%%%%%%%%%%
\newpage

\subsection*{\textit{ksitelen} as a Noun or as a as a Verb}
%
\index{\textit{sitelen}}
%%%%%%%%%%%%%%%%%%%%%%%%%%%%%%%%%%%%%%%%%%%%%%%%%%%%%%%%%%%%%%%%%%%%%%%%%%
%
\textit{sitelen}  as a noun means 'picture' or 'image'. 
As a verb, it means to 'draw' or to 'write'. 
\textit{sitelen} is most useful for the compound nouns that you can make with it. 

\begin{supertabular}{p{5,5cm}|ll}
sitelen tawa  && movie, TV show \\
sitelen tawa 'Fahrenheit 9/11' li pona, tawa mi. && I like the movie 'Fahrenheit 9/11'. \\
sitelen tawa 'Bowling for Columbine' li pona kin. && The movie 'Bowling for Columbine' is also good. \\
sitelen ma && map \\
o pana e sitelen ma, tawa mi. && Give me the map. \\
\end{supertabular} 
%
%%%%%%%%%%%%%%%%%%%%%%%%%%%%%%%%%%%%%%%%%%%%%%%%%%%%%%%%%%%%%%%%%%%%%%%%%%
\newpage{}
&
\subsection*{Practice (Answers: Page~\pageref{'colors'})}
%%%%%%%%%%%%%%%%%%%%%%%%%%%%%%%%%%%%%%%%%%%%%%%%%%%%%%%%%%%%%%%%%%%%%%%%%%
%
Please write down your answers and check them afterwards. 

Try to translate these sentences. 
You can use the tool \textit{Toki Pona Parser} (\cite{www:rowa:02}) for spelling and grammar check. 

\begin{supertabular}{p{5,5cm}|ll}
I don't see the blue bag. &&   \\ % no-dictionary
A little green person came from the sky. &&   \\ % no-dictionary
I like the color purple.  &&  \\ % no-dictionary
The sky is blue. &&   \\ % no-dictionary
Look at that red bug.  &&  \\ % no-dictionary
I want the map.  &&  \\ % no-dictionary
Do you watch The X-Files? &&  \\  % no-dictionary
Which color do you like?* &&  \\  % no-dictionary
 && \\ % no-dictionary
suno li ' jelo. &&   \\ % no-dictionary
telo suli li ' laso.  &&  \\ % no-dictionary
mi wile moku e kili loje.  &&  \\ % no-dictionary
ona li kule e tomo tawa. &&   \\ % no-dictionary
\end{supertabular} 

* Think: 'Which color is good for you?' 

And now try reading this Toki Pona poem. 

\begin{supertabular}{p{5,5cm}|ll}
ma mi li ' pimeja. && \\ % no-dictionary
kalama ala li lon && \\ % no-dictionary
mi lape. mi sona. && \\ % no-dictionary
\end{supertabular} 
%%%%%%%%%%%%%%%%%%%%%%%%%%%%%%%%%%%%%%%%%%%%%%%%%%%%%%%%%%%%%%%%%%%%%%%%%%
% eof
