%%%%%%%%%%%%%%%%%%%%%%%%%%%%%%%%%%%%%%%%%%%%%%%%%%%%%%%%%%%%%%%%%%%%%%%%%%
\section{Introduction}
%%%%%%%%%%%%%%%%%%%%%%%%%%%%%%%%%%%%%%%%%%%%%%%%%%%%%%%%%%%%%%%%%%%%%%%%%%
%
Sonja Lang created the language Toki Pona in the year 2001. 
Her aim was the minimalism. 
Toki Pona consists of only about 120 words, which are not altered. 
In accordance with the position in the sentence, the words can vary their significance. 
To describe more details you have to combine the words.

It is not the goal of Toki Pona to describe complex issues. 
Dissertations and scientific papers will never written in Toki Pona. 
Lawyers, bureaucrats, theologians and politicians are be warned of the side-effect of this language.

It is not he aim of Toki Pona to solve the communication problems in the world. 
But you can learn this language in a month. 
Toki Pona is easy in an intelligent way and yoga for the brain. 
People who hate nested subordinate clauses and commas will certainly have fun with Toki Pona.

Toki Pona has evolved since 2001. 
There are also slangs, which hinder the understanding. 
Therefore these lessons are based on the official Toki Pona book \cite{www:tokipona.org} by Sonja Lang (2014) and the updated tutorials from BJ Knight (jan Pije) \cite{www:Pije:01}.

So have fun with the lessons and learning of Toki Pona. 
For learning vocabulary helps Memrise \cite{www:memrise:01}. Related Links Toki Pona can be found on the website\cite{www:rowa:01}.

You can use the tool \textit{Toki Pona Parser} (\cite{www:rowa:02}) for spelling, grammar check and ambiguity check of Toki Pona sentences 


\textit{toki pona li pona, tawa sina.}
%
%%%%%%%%%%%%%%%%%%%%%%%%%%%%%%%%%%%%%%%%%%%%%%%%%%%%%%%%%%%%%%%%%%%%%%%%%%
% eof
