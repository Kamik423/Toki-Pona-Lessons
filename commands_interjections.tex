%%%%%%%%%%%%%%%%%%%%%%%%%%%%%%%%%%%%%%%%%%%%%%%%%%%%%%%%%%%%%%%%%%%%%%%%%%
\section{Addressing People, Interjections, Commands} 
%%%%%%%%%%%%%%%%%%%%%%%%%%%%%%%%%%%%%%%%%%%%%%%%%%%%%%%%%%%%%%%%%%%%%%%%%%
%
%%%%%%%%%%%%%%%%%%%%%%%%%%%%%%%%%%%%%%%%%%%%%%%%%%%%%%%%%%%%%%%%%%%%%%%%%%
\subsection*{Vocabulary}
%%%%%%%%%%%%%%%%%%%%%%%%%%%%%%%%%%%%%%%%%%%%%%%%%%%%%%%%%%%%%%%%%%%%%%%%%%
%
\begin{supertabular}{p{2,5cm}|ll}
%
\index{a}
\textbf{a} && \textit{interjection}: ah, ha, uh, oh, ooh, aw, well (emotion word) \\ % no-dictionary
\textbf{a a a!} && \textit{interjection}: laugh \\ % no-dictionary
 && \\ % no-dictionary
%
\index{awen}
\textbf{\dots awen} && \textit{adjective}: remaining, stationary, permanent, sedentary \\ % no-dictionary
\textbf{\dots awen} && \textit{adverb}: still, yet \\ % no-dictionary
\textbf{awen} && \textit{noun}: inertia, continuity, continuum, stay \\ % no-dictionary
\textbf{awen} && \textit{verb intransitive}: to stay, to wait,to remain \\ % no-dictionary
\textbf{awen (e \dots)} && \textit{verb transitive}: to keep \\ % no-dictionary
 && \\ % no-dictionary
%
\index{mu}
\textbf{\dots mu} && \textit{adjective}: animal nois- \\ % no-dictionary
\textbf{\dots mu} && \textit{adverb}: animal nois- \\ % no-dictionary
\textbf{mu!} && \textit{interjection}: woof! meow! moo! etc. (cute animal noise) \\ % no-dictionary
\textbf{mu} && \textit{noun}: animal noise \\ % no-dictionary
\textbf{mu} && \textit{verb intransitive}: to communicate animally \\ % no-dictionary
\textbf{mu (e \dots)} && \textit{verb transitive}: to make animal noise \\ % no-dictionary
 && \\ % no-dictionary
%
\index{o}
\textbf{o!} && \textit{interjection}: hey! (calling somebody's attention) \\ % no-dictionary
\textbf{\dots o, \dots} && \textit{interjection}: adressing people \\ % no-dictionary
\textbf{o \dots !} && \textit{subject}: An 'o' is used for imperative (commands). 'o' is the subject.  \\ % no-dictionary
\textbf{\dots o \dots !} && \textit{separator}: An 'o' is used for imperative (commands): 'o' replace 'li'. \\ % no-dictionary
 && \\ % no-dictionary
%
\index{pu}
\textbf{\dots pu} && \textit{adjective}: buying and interacting with the official Toki Pona book \\ 
\textbf{pu} && \textit{noun}: buying and interacting with the official Toki Pona book \\ 
\textbf{pu \dots} && \textit{auxiliary verb}: to buying and interacting with the official Toki Pona book \\ 
\textbf{pu} && \textit{verb intransitive}:  to buy and to read (the official Toki Pona book) \\
\textbf{pu (e \dots)} && \textit{verb transitive}: to apply (the official Toki Pona book) to \dots \\
 && \\ % no-dictionary
%
\index{ala!interjection}
\textbf{ala!} && \textit{interjection}: no! \\ % no-dictionary
 && \\ % no-dictionary
%
\index{ike!interjection}
\textbf{ike!} && \textit{interjection}: oh dear! woe! alas! \\ % no-dictionary
 && \\ % no-dictionary
%
\index{jaki!interjection}
\textbf{jaki!} && \textit{interjection}: ew! yuck! \\ % no-dictionary
 && \\ % no-dictionary
%
\index{pakala!interjection}
\textbf{pakala!} && \textit{interjection}: damn! fuck! \\ % no-dictionary
 && \\ % no-dictionary
%
\index{pona!interjection}
\textbf{pona!} && \textit{interjection}: great! good! thanks! OK! cool! yay! \\ % no-dictionary
 && \\ % no-dictionary
%
\index{toki!interjection}
\textbf{toki!} && \textit{interjection}: hello, hi, good morning, \\ % no-dictionary
\end{supertabular} \\
%
%


%%%%%%%%%%%%%%%%%%%%%%%%%%%%%%%%%%%%%%%%%%%%%%%%%%%%%%%%%%%%%%%%%%%%%%%%%%
\newpage
%
\subsection*{Vocativ (Addressing People)}
%
\index{vocativ}
\index{\textit{o}!vocativ}
\index{addressing people}
\index{attention}
%%%%%%%%%%%%%%%%%%%%%%%%%%%%%%%%%%%%%%%%%%%%%%%%%%%%%%%%%%%%%%%%%%%%%%%%%%
%
Sometimes you need to get a person's attention before you can talk to him. 
When you want to address someone like that before saying the sentence, you just follow this same pattern. 
\textit{jan} (name) \textit{o,} (sentence). 
Note the comma behind the interjection word \textit{o}. 

\begin{supertabular}{p{5,5cm}|ll}
jan Ken o, pipi li lon len sina. && Ken, a bug is on your shirt. \\
jan Keli o, sina ' pona lukin. && Kelly, you are pretty. \\
jan Mawen o, sina wile ala wile moku? && Marvin, are you hungry? \\
jan Tepani o, sina ' ike, tawa mi. && Steffany, I don't like you. \\
\end{supertabular} 
%
%%%%%%%%%%%%%%%%%%%%%%%%%%%%%%%%%%%%%%%%%%%%%%%%%%%%%%%%%%%%%%%%%%%%%%%%%%
\subsection*{Commands}
%
\index{imperativ}
\index{command}
\index{\textit{o}!command}
\index{\textit{o}!imperativ}
\index{\textit{o}!vocativ}
\index{\textit{o}!subject}
\index{\textit{o}!interjection}
\index{\textit{o}!separator}
\index{subject!\textit{o}}
\index{exclamation mark}
%%%%%%%%%%%%%%%%%%%%%%%%%%%%%%%%%%%%%%%%%%%%%%%%%%%%%%%%%%%%%%%%%%%%%%%%%%
%

The command form (imperative) is introduced with \textit{o} and ends with an exclamation mark.
The interjection word \textit{o} is the subject here.

\begin{supertabular}{p{5,5cm}|ll}
o pali! && Work! \\
o awen! && Wait! \\
o ' pona! && Be good! \\
o lukin e ni! && Watch this! \\
o tawa, tawa ma tomo, lon poka jan pona sina! && Go to the city with your friend! \\
\end{supertabular} 

We've learned how to address people and how to make commands; now let's put these two concepts together. 
Suppose you want to address someone and tell them to do something. 
Notice how one of the \textit{o}'s got dropped, as did the comma. 

\begin{supertabular}{p{5,5cm}|ll}
jan San o, ...  && John, ... \\ % no-dictionary
 ... o tawa tomo sina! && ... go to your house! \\ % no-dictionary
jan San o tawa tomo sina!  && John, go to your house! \\
 && \\ % no-dictionary
jan Ta o toki ala, tawa mi! && Todd, don't talk to me! \\
jan Sesi o moku e kili ni! && Jessie, eat this fruit!. \\
\end{supertabular} 

The separator \textit{o} replaces the separator \textit{li}. 
After the personal pronouns \textit{mi} and \textit{sina} also the separator \textit{o} is used. 

\begin{supertabular}{p{5,5cm}|ll}
sina o telo e sina! && Wash yourself! \\
\end{supertabular} 

This structure can also be used to make sentences like 'Let's go'.

\begin{supertabular}{p{5,5cm}|ll}
mi mute o tawa! && Let's go. \\
mi mute o ' musi! && Let's have fun. \\
\end{supertabular} 
%
%%%%%%%%%%%%%%%%%%%%%%%%%%%%%%%%%%%%%%%%%%%%%%%%%%%%%%%%%%%%%%%%%%%%%%%%%%
\subsection*{Interjections}
%
\index{interjection}
\index{laughter}
\inde{animal!noise}
\index{exclamation mark}
\index{\textit{a}}
%%%%%%%%%%%%%%%%%%%%%%%%%%%%%%%%%%%%%%%%%%%%%%%%%%%%%%%%%%%%%%%%%%%%%%%%%%
%
An interjection sentence makes conveys excitement.
Interjections sentences often consist only of a noun or an interjection word, e. g. \textit{a}, and end with an exclamation mark.

\begin{supertabular}{p{5,5cm}|ll}
jan Lisa o, toki! && Hello Lisa! \\
pona! && Yay! Good! Hoorah! \\
ike! && Oh no! Uh! oh! Alas! \\
pakala! && F-ck! D-mn! \\
% mu! && Meow! Woof! Grrr! Moo! (sounds made by animals) \\
a! && Ooh, Ahh! Unh! Oh! \\
a a a! && Hahaha! (laughter) \\
\end{supertabular} 

The interjection word \textit{a} adds emotion or stress. 
It can be used at the end of a sentence.
Use the Interjektion-Word \textit{a} sparingly!

\begin{supertabular}{p{5,5cm}|ll}
sina ' suli a! && You are so tall! \\
\end{supertabular} 

The interjection words \textit{o} and \textit{a} only used when the person makes you feel really emotional. 
For example, if you haven't seen a person for a long time or if you have sex and you still speak perfect Toki Pona. 

\begin{supertabular}{p{5,5cm}|ll}
jan Epi o a! && Oh Abbie! \\
\end{supertabular} 

%%%%%%%%%%%%%%%%%%%%%%%%%%%%%%%%%%%%%%%%%%%%%%%%%%%%%%%%%%%%%%%%%%%%%%%%%%
%
\subsection*{Salutations}
%
\index{salutation}
\index{interjection!salutation}
\index{exclamation mark}
%%%%%%%%%%%%%%%%%%%%%%%%%%%%%%%%%%%%%%%%%%%%%%%%%%%%%%%%%%%%%%%%%%%%%%%%%%
%
The second group of interjections are kind like salutations.
They usually consist of a noun, an optional adjective and an exclamation mark. 

\begin{supertabular}{p{5,5cm}|ll}
toki! && Hello!, Hi! \\
suno pona! && Good sun! Good day! \\
lape pona! && Sleep well! Have a good night! \\
moku pona! && Good food! Enjoy your meal! \\
mi tawa && I'm going. Bye! \\
tawa pona! && (in reply) Go well! Good bye! \\
kama pona! && Come well! Welcome! \\
musi pona! && Good fun! Have fun! \\
\end{supertabular}  

They can also consist of a complete sentence with an exclamation mark.

\begin{supertabular}{p{5,5cm}|ll}
jan Lisa o, toki! && Hello Lisa! \\
mi tawa && I'm going. Bye! \\
\end{supertabular}

%
%
%
%%%%%%%%%%%%%%%%%%%%%%%%%%%%%%%%%%%%%%%%%%%%%%%%%%%%%%%%%%%%%%%%%%%%%%%%%%
%
\subsection*{Animal Sounds and Communication}
%
\index{animal sounds}

%%%%%%%%%%%%%%%%%%%%%%%%%%%%%%%%%%%%%%%%%%%%%%%%%%%%%%%%%%%%%%%%%%%%%%%%%%
%
%%%%%%%%%%%%%%%%%%%%%%%%%%%%%%%%%%%%%%%%%%%%%%%%%%%%%%%%%%%%%%%%%%%%%%%%%%
\subsubsection*{The Noun \textit{mu}}
\index{mu!noun}

\begin{supertabular}{p{5,5cm}|ll}
mu ni li ' ike a! && That barking is terrible! \\
% mu! mu! mu! && Wow! Wow! Wow! \\
\end{supertabular} 

%
%%%%%%%%%%%%%%%%%%%%%%%%%%%%%%%%%%%%%%%%%%%%%%%%%%%%%%%%%%%%%%%%%%%%%%%%%%
\subsubsection*{The Adjective  \textit{mu}}
\index{mu!adjective}

\begin{supertabular}{p{5,5cm}|ll}
kalama mu ni li ' pona, tawa mi. && I like this animal sound.\\
\end{supertabular} 

%
%%%%%%%%%%%%%%%%%%%%%%%%%%%%%%%%%%%%%%%%%%%%%%%%%%%%%%%%%%%%%%%%%%%%%%%%%%
\subsubsection*{The Transitive Verb \textit{mu}}
\index{mu!verb}

\begin{supertabular}{p{5,5cm}|ll}
pipi li mu e kalama. && The cicadas are chirping noises.  \\
\end{supertabular} 

%
%%%%%%%%%%%%%%%%%%%%%%%%%%%%%%%%%%%%%%%%%%%%%%%%%%%%%%%%%%%%%%%%%%%%%%%%%%
\subsubsection*{The Intransitive Verb \textit{mu}}
\index{mu!verb}

\begin{supertabular}{p{5,5cm}|ll}
pipi li mu, tawa ona. && The beetles communicate with each other. \\
\end{supertabular} 

%
%%%%%%%%%%%%%%%%%%%%%%%%%%%%%%%%%%%%%%%%%%%%%%%%%%%%%%%%%%%%%%%%%%%%%%%%%%
\subsubsection*{The Adverb \textit{mu}}
\index{mu!adverb}

\begin{supertabular}{p{5,5cm}|ll}
sina toki mu e ni. && You say that beastly. \\
\end{supertabular} 

%
%
%
%
%%%%%%%%%%%%%%%%%%%%%%%%%%%%%%%%%%%%%%%%%%%%%%%%%%%%%%%%%%%%%%%%%%%%%%%%%%
\newpage
%
\subsection*{Practice (Answers: Page~\pageref{'commands_interjections'})}
%%%%%%%%%%%%%%%%%%%%%%%%%%%%%%%%%%%%%%%%%%%%%%%%%%%%%%%%%%%%%%%%%%%%%%%%%%
%
Please write down your answers and check them afterwards. 

\begin{supertabular}{p{5,5cm}|ll}
Which separator ends a command sentence (imperative)? &&  \\ % no-dictionary
What is the subject of the command form if no one is addressed directly? &&  \\ % no-dictionary
How do you address people by name? &&  \\ % no-dictionary
What do injections consist of? &&  \\ % no-dictionary
Which separator stands bevor  the predicate if someone is directly addressed in a command? &&  \\ % no-dictionary
Which separator ends an interjection (exclamation)? &&  \\ % no-dictionary
\end{supertabular}

Try to translate these sentences. 
You can use the tool \textit{Toki Pona Parser} (\cite{www:rowa:02}) for spelling and grammar check. 

\begin{supertabular}{p{5,5cm}|ll}
Go!  &&  \\ % no-dictionary
Mama, wait.  &&  \\ % no-dictionary
Hahaha! That's funny.  &&  \\ % no-dictionary
F-ck! &&  \\ % no-dictionary
Bye!  &&  \\ % no-dictionary
\end{supertabular}

\begin{supertabular}{p{5,5cm}|ll}
mu!  &&  \\ % no-dictionary
o tawa musi, lon poka mi!  &&  \\ % no-dictionary
tawa pona!  &&  \\ % no-dictionary
o pu! &&  \\ % no-dictionary
\end{supertabular} 

%
%%%%%%%%%%%%%%%%%%%%%%%%%%%%%%%%%%%%%%%%%%%%%%%%%%%%%%%%%%%%%%%%%%%%%%%%%%
% eof
