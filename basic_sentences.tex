%%%%%%%%%%%%%%%%%%%%%%%%%%%%%%%%%%%%%%%%%%%%%%%%%%%%%%%%%%%%%%%%%%%%%%%%%%
\section{Basic Sentences}
%%%%%%%%%%%%%%%%%%%%%%%%%%%%%%%%%%%%%%%%%%%%%%%%%%%%%%%%%%%%%%%%%%%%%%%%%%
\subsection*{Vocabulary}
%%%%%%%%%%%%%%%%%%%%%%%%%%%%%%%%%%%%%%%%%%%%%%%%%%%%%%%%%%%%%%%%%%%%%%%%%%
\index{\textit{jan}}
\index{\textit{mi}}
\index{\textit{moku}}
\index{\textit{sina}}
\index{\textit{suno}}
\index{\textit{telo}}
\index{\textit{pona}}
\index{\textit{suli}}
\index{\textit{li}}
\index{I}
\index{me}
\index{food}
\index{eat}
\index{drink}
\index{you}
\index{sun}
\index{light}
\index{water}
\index{liquid}
\index{good}
\index{simple}
\index{fix}
\index{repair}
\index{big}
\index{tall}
\index{long}
\index{important}
\begin{supertabular}{p{5,5cm}|ll}
mi &&  I, we, me, our  \\
sina &&  you, your  \\
jan && somebody, anybody, person  \\                     
li && separates a 3rd person subject \\ && from its verb \\
pona &&  good, simple, to fix, to repair \\
moku &&  food, eat, drink \\
suno &&  sun, light, to shine  \\
telo &&  water, liquid  \\
suli &&  big, tall, long, important \\
' && to be (unofficial) \\
\end{supertabular} 
%
%%%%%%%%%%%%%%%%%%%%%%%%%%%%%%%%%%%%%%%%%%%%%%%%%%%%%%%%%%%%%%%%%%%%%%%%%%
\index{singular}
\index{plural}
\subsection*{The Ambiguity of Toki Pona}
%%%%%%%%%%%%%%%%%%%%%%%%%%%%%%%%%%%%%%%%%%%%%%%%%%%%%%%%%%%%%%%%%%%%%%%%%%
%
Do you see how several of \textbf{the words} in the vocabulary \textbf{have multiple meanings}? 
For example, \textit{suli} can mean either "long", "tall", "big", "important" or "the size". 
By now, you might be wondering, "What's going on? How can one word mean so many different things?" 
Welcome to the world of Toki Pona! The truth is that lots of words are like this in Toki Pona. 
Because the language has such a small vocabulary and is so basic, the ambiguity is inevitable. 
However, this vagueness is not necessarily a bad thing. Because of the vagueness, a speaker of Toki Pona is forced to focus on the very basic, unaltered aspect of things, rather than focusing on many minute details. 

Another way that Toki Pona is ambiguous is that it \textbf{can not specify} whether a word is \textbf{singular} or \textbf{plural}. 
For example, \textit{jan} can mean either "person" or "people". 
If you've decided that Toki Pona is too arbitrary and that not having plurals is simply the final straw, don't be so hasty. 
Toki Pona is not the only language that doesn't specify whether a noun is plural or not. 
Japanese, for example, does the same thing. 

Toki Pona has no Tenses. 
The verbs don't change. 
If it's absolutely necessary, there are ways of saying that something happened in the past, present, or future. 
% 
%%%%%%%%%%%%%%%%%%%%%%%%%%%%%%%%%%%%%%%%%%%%%%%%%%%%%%%%%%%%%%%%%%%%%%%%%%
\index{\textit{mi}}
\index{\textit{sina}}
\index{\textit{mi}!subject}
\index{\textit{sina}!subject}
\index{subject!\textit{mi}}
\index{subject!\textit{sina}}
\index{verb!to be}
\index{be}
\subsection*{Sentences with \textit{mi} or \textit{sina }as the subject}
%%%%%%%%%%%%%%%%%%%%%%%%%%%%%%%%%%%%%%%%%%%%%%%%%%%%%%%%%%%%%%%%%%%%%%%%%%
%
Simply state \textit{mi} or \textit{sina}, then complete the sentence with a verb. 
These are the absolute simplest type of sentences in Toki Pona. 
A declarative sentence ends with a full stop. 
Toki Pona has no nested subordinate clauses and nearly no commas. 

\begin{supertabular}{p{5,5cm}|ll}
mi moku. && I eat. \\
sina pona. && You fix. \\
\end{supertabular} 

%%%%%%%%%%%%%%%%%%%%%%%%%%%%%%%%%%%%%%%%%%%%%%%%%%%%%%%%%%%%%%%%%%%%%%%%%%
\newpage
\subsubsection*{The Lack of the verb "to be"}
%%%%%%%%%%%%%%%%%%%%%%%%%%%%%%%%%%%%%%%%%%%%%%%%%%%%%%%%%%%%%%%%%%%%%%%%%%
%
One of the first principles you'll need to learn about Toki Pona is that there is \textbf{no form of the verb "to be"} like there is in English. 
That is why after a \textit{mi} or \textit{sina} can follow a noun or an adjective also. 
The missing verb "to be" is in these lessons marked with an apostrophe. 

\begin{supertabular}{p{5cm}|ll}
mi moku. && I eat.  \\
mi ' moku. && I am food. \\  % no-dictionary
sina pona. && You fix. \\
sina ' pona. && You are good. \\  % no-dictionary
\end{supertabular} 

Because Toki Pona lacks "to be", \textbf{the exact meaning is lost}. 
\textit{moku} in this sentence could be a verb, or it could be a noun; just as \textit{pona} could be an adjective or could be a verb. 
In situations such as these, the listener must rely on context. 
After all, how often do you hear someone say "I am food"? 
I hope not very often! You can be fairly certain that \textit{mi moku} means "I'm eating". 

%%%%%%%%%%%%%%%%%%%%%%%%%%%%%%%%%%%%%%%%%%%%%%%%%%%%%%%%%%%%%%%%%%%%%%%%%%
\index{\textit{li}}
\index{subject!\textit{mi}}
\index{subject!\textit{sina}}
\index{\textit{mi}!subject}
\index{\textit{sina}!subject}
\subsection*{Sentences without \textit{mi} or \textit{sina} as the subject}
%%%%%%%%%%%%%%%%%%%%%%%%%%%%%%%%%%%%%%%%%%%%%%%%%%%%%%%%%%%%%%%%%%%%%%%%%%
%
For sentences that don't use \textit{mi} or \textit{sina} as the subject, there is one small catch that you'll have to learn. 
Look at how \textit{li} is used. 
\textbf{\textit{li}} is a grammatical word that \textbf{separates the subject from its verb}. 
\textbf{It's only used when the subject is not \textit{mi} or \textit{sina}.} 
Although \textit{li} might seem worthless right now, as you continue to learn Toki Pona you will see that some sentences could be very confusing if \textit{li} weren't there. 

\begin{supertabular}{p{5cm}|ll}
telo \textbf{li} pona. && Water is cleaning. \\
suno \textbf{li} suno. && The sun is shining. \\
moku \textbf{li} ' pona. && The food is good. \\ % no-dictionary
\end{supertabular} 

Because of the lacks of "to be", after \textit{li} can not only be a verb, it can follow a noun or an adjective as well.
The missing verb "to be" is here marked with an apostrophe. 
%
%%%%%%%%%%%%%%%%%%%%%%%%%%%%%%%%%%%%%%%%%%%%%%%%%%%%%%%%%%%%%%%%%%%%%%%%%%
\subsection*{Practice 3 (Answers: Page~\pageref{'basic_sentences'})}
%%%%%%%%%%%%%%%%%%%%%%%%%%%%%%%%%%%%%%%%%%%%%%%%%%%%%%%%%%%%%%%%%%%%%%%%%%
%
Try translating these sentences. 

\begin{supertabular}{p{5cm}|ll}
People are good. && \\ % no-dictionary
I'm eating. &&  \\ % no-dictionary
You're tall. &&  \\ % no-dictionary
Water is simple. &&  \\ % no-dictionary
The lake is big. &&\\ % no-dictionary
 && \\
suno li suli. &&  \\% no-dictionary
mi ' suli. &&  \\% no-dictionary
jan li moku. &&  \\% no-dictionary
\end{supertabular} \\% no-dictionary
%
%%%%%%%%%%%%%%%%%%%%%%%%%%%%%%%%%%%%%%%%%%%%%%%%%%%%%%%%%%%%%%%%%%%%%%%%%%
% eof
