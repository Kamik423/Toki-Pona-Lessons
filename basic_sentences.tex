%%%%%%%%%%%%%%%%%%%%%%%%%%%%%%%%%%%%%%%%%%%%%%%%%%%%%%%%%%%%%%%%%%%%%%%%%%
\section{Basic Sentences}
%%%%%%%%%%%%%%%%%%%%%%%%%%%%%%%%%%%%%%%%%%%%%%%%%%%%%%%%%%%%%%%%%%%%%%%%%%
\subsection*{Vocabulary}
%%%%%%%%%%%%%%%%%%%%%%%%%%%%%%%%%%%%%%%%%%%%%%%%%%%%%%%%%%%%%%%%%%%%%%%%%%
\index{\textit{jan}}
\index{\textit{mi}}
\index{\textit{moku}}
\index{\textit{sina}}
\index{\textit{suno}}
\index{\textit{telo}}
\index{\textit{pona}}
\index{\textit{suli}}
\index{\textit{li}}
\index{I}
\index{me}
\index{food}
\index{eat}
\index{drink}
\index{you}
\index{sun}
\index{light}
\index{water}
\index{liquid}
\index{good}
\index{simple}
\index{fix}
\index{repair}
\index{big}
\index{tall}
\index{long}
\index{important}
\begin{supertabular}{p{2,5cm}|ll}
\textbf{\dots jan} && \textit{adjective}: human, somebody's, personal, of people \\ % no-dictionary
\textbf{\dots jan} && \textit{adverb}: human, somebody's, personal, of people \\ % no-dictionary
\textbf{jan} && \textit{noun}: person, people, human, being, somebody, anybody \\ % no-dictionary
\textbf{jan (e \dots)} && \textit{verb transitive}: to personify, to humanize, to personalize \\ % no-dictionary
 && \\ % no-dictionary
\textbf{\dots li \dots} && \textit{separator}: A 'li' is between any subject except 'mi' and 'sina' and its verb. \\ && Don't use 'li' before or after an other separator. \\ % no-dictionary
 && \\ % no-dictionary
\textbf{mi} && \textit{personal pronoun}: I, we  \\ 
\textbf{\dots mi} && \textit{possessive pronoun}: my, our \\  
\textbf{\dots e mi} && \textit{reflexive pronoun}: myself, ourselves  \\ 
 && \\ % no-dictionary
\textbf{\dots moku} && \textit{adjective}: eating \\ % no-dictionary
\textbf{\dots moku} && \textit{adverb}: eating \\ % no-dictionary
\textbf{moku} && \textit{noun}: food, meal \\ % no-dictionary
\textbf{moku (e \dots)} && \textit{verb transitive}: to eat, to drink, to swallow, to ingest, to consume \\ % no-dictionary
 && \\ % no-dictionary
\textbf{ona} && \textit{personal pronoun}: she, he, it, they \\ % no-dictionary
\textbf{\dots ona} && \textit{possessive pronoun}: her, his, its \\  % no-dictionary
\textbf{\dots e ona} && \textit{reflexive pronoun}: himself, herself, itself, themselves \\  
 && \\ % no-dictionary 
\textbf{\dots pona} && \textit{adjective}: good, simple, positive, nice, correct, right \\ % no-dictionary
\textbf{\dots pona} && \textit{adverb}: good, simple, positive, nice, correct, right \\ % no-dictionary
\textbf{pona} && \textit{noun}: good, simplicity, positivity \\ % no-dictionary
\textbf{pona (e \dots)} && \textit{verb transitive}: to improve, to fix, to repair, to make good \\ % no-dictionary
 && \\ % no-dictionary
\textbf{sina} && \textit{personal pronoun}: you \\ % no-dictionary
\textbf{\dots sina} && \textit{possessive pronoun}: yours \\  % no-dictionary
\textbf{\dots e sina} && \textit{reflexive pronoun}: yourself, yourselves \\  
 && \\ % no-dictionary
\textbf{\dots suno} && \textit{adjective}: sunny, sunnily \\ % no-dictionary
\textbf{\dots suno} && \textit{adverb}: sunny, sunnily \\ % no-dictionary
\textbf{suno} && \textit{noun}: sun, light \\ % no-dictionary
\textbf{suno (e \dots)} && \textit{verb transitive}: to light, to illumine \\ % no-dictionary
 && \\ % no-dictionary
\textbf{\dots suli} && \textit{adjective}: big, tall, long, adult, important \\ % no-dictionary
\textbf{\dots suli} && \textit{adverb}: big, tall, long, adult, important \\ % no-dictionary
\textbf{suli} && \textit{noun}: size \\ % no-dictionary
\textbf{suli (e \dots)} && \textit{verb transitive}: to enlarge, to lengthen \\ % no-dictionary
 && \\ % no-dictionary
\textbf{\dots telo} && \textit{adjective}: wett, slobbery, moist, damp, humid, sticky, sweaty, dewy, drizzly \\ % no-dictionary
\textbf{\dots telo} && \textit{adverb}: wett, slobbery, moist, damp, humid, sticky, sweaty, dewy, drizzly \\ % no-dictionary
\textbf{telo} && \textit{noun}: water, liquid, juice, sauce \\ % no-dictionary
\textbf{telo (e \dots)} && \textit{verb transitive}: to water, to wash with water, to put water to, to melt, to liquify \\ % no-dictionary
 && \\ % no-dictionary
\textbf{'} && \textit{unofficial}: For clarification the missing verb 'to be' can be marked with an apostrophe.  \\ % no-dictionary
\end{supertabular} \\
%
\newpage
%%%%%%%%%%%%%%%%%%%%%%%%%%%%%%%%%%%%%%%%%%%%%%%%%%%%%%%%%%%%%%%%%%%%%%%%%%
\index{singular}
\index{plural}
\index{Tense}
\index{noun}
\index{pronoun}
\index{personal pronoun}
\index{possessive pronoun}
\index{reflexive pronoun}
\index{adjective}
\index{verb}
\subsection*{The Ambiguity of Toki Pona}
%%%%%%%%%%%%%%%%%%%%%%%%%%%%%%%%%%%%%%%%%%%%%%%%%%%%%%%%%%%%%%%%%%%%%%%%%%
%
Do you see how several of the words in the vocabulary have multiple meanings? 
For example, \textit{suli} can mean either 'long', 'tall', 'big', 'important' or 'the size'. 
By now, you might be wondering, 'What's going on? How can one word mean so many different things?' 

Welcome to the world of Toki Pona! The truth is that lots of words are like this in Toki Pona. 
Because the language has such a small vocabulary and is so basic, the ambiguity is inevitable. 
However, this vagueness is not necessarily a bad thing. Because of the vagueness, a speaker of Toki Pona is forced to focus on the very basic, unaltered aspect of things, rather than focusing on many minute details. 

Another way that Toki Pona is ambiguous is that it can not specify whether a word is singular or plural. 
For example, \textit{jan} can mean either 'person' or 'people'. 
If you've decided that Toki Pona is too arbitrary and that not having plurals is simply the final straw, don't be so hasty. 
Toki Pona is not the only language that doesn't specify whether a noun is plural or not. 
Japanese, for example, does the same thing. 

Toki Pona has no Tenses. 
The verbs don't change. 
If it's absolutely necessary, there are ways of saying that something happened in the past, present, or future. 

As you can see in the vocabulary list, most words can be used in different word types. 
They remain unchanged. 
The word part is derived from the position in the sentence. 
In this lesson, we will deal with nouns, pronouns, verbs, adjectives and a special separator. 

A noun is a word for a person, place or thing. 
An adjective is a word that describes a noun. 
A verb describes an action. 

Pronouns are proxies for different types of words. 
They are used in the same place as the word to be represented and have the same grammatical characteristics as this one.
Pronouns are not words of content, but they denote persons or things by referring to the context. 
Personal pronouns (I, you, \dots) represent nouns. 
Possessive pronouns (my, your, \dots) represent adjectives. 
In the next few lessons we will learn more about other types of pronouns. 

% 
%%%%%%%%%%%%%%%%%%%%%%%%%%%%%%%%%%%%%%%%%%%%%%%%%%%%%%%%%%%%%%%%%%%%%%%%%%
\index{\textit{mi}}
\index{\textit{sina}}
\index{\textit{mi}!subject}
\index{\textit{sina}!subject}
\index{subject}
\index{subject phrase}
\index{subject!\textit{mi}}
\index{subject!\textit{sina}}
\index{predicate}
\index{predicate phrase}
\index{verb!to be}
\index{be}
\label{'predicate'}
\subsection*{The Personal Pronouns \textit{mi} or \textit{sina} as Subject}
%%%%%%%%%%%%%%%%%%%%%%%%%%%%%%%%%%%%%%%%%%%%%%%%%%%%%%%%%%%%%%%%%%%%%%%%%%
%
With the pronoun \textit{mi} or the pronoun \textit{sina} at the beginning and a subsequent verb a simple sentence in toki pona is already complete. 
A declarative sentence ends with a full stop. 
Toki Pona has no nested subordinate clauses and nearly no commas. 

\begin{supertabular}{p{5,5cm}|ll}
mi moku. && I eat. \\
sina pona. && You fix. \\
\end{supertabular} 

In these sentences pronouns \textit{mi} and \textit{sina} are in each case the subject phrase. 
In Toki Pona, a subject phrase is always at the beginning of the sentence. 
In these examples, the subject phrases consist of only one subject (\textit{mi} or \textit{sina}).

The subject is the carrier of the action, process or state. 
It is the most important addition to the verb in the sentence, a complete sentence always contains a subject. 
You ask for the subject with whom or what.

The verbs \textit{moku} and \textit{pona} form the predicate phrase in these examples.  
The predicate is a core element in a sentence and is the statement of the sentence.
In most languages, a predicate is formed by a verb, but this is not mandatory in all languages. 
As we will soon see, in Toki Pona the predicate is not necessarily formed by a verb. 
The difference between verb and predicate is that verb designates a word part and predicate designates a grammatical function.
A predicate and possible objects form a predicate phrase. 
%
%%%%%%%%%%%%%%%%%%%%%%%%%%%%%%%%%%%%%%%%%%%%%%%%%%%%%%%%%%%%%%%%%%%%%%%%%%
\index{apostrophe}
\subsubsection*{The Lack of the verb 'to be'}
%%%%%%%%%%%%%%%%%%%%%%%%%%%%%%%%%%%%%%%%%%%%%%%%%%%%%%%%%%%%%%%%%%%%%%%%%%
%
One of the first principles you'll need to learn about Toki Pona is that there is no form of the verb 'to be' like there is in English. 
By the missing 'to be' the verb slot can be empty and after \textit{mi} or \textit{sina} can follow also a noun or adjective. 
In these lessons, the term' slot' is used to indicate a valid position of a word type in the sentence.

Regular sentences can also be formed in other languages without a verb appearing in them. 
Examples are Russian and Arabic. 
A noun then functions as a predicate noun or an adjective serves as predicate adjective.
But this noun or adjective does not become a verb. 
An empty verb slot cannot, however, form a predicate phrase on its own. 
A noun or adjective must follow. 
That is, directly after \textit{mi} or \textit{sina} the sentence cannot be finished yet.

The missing verb 'to be' or the empty verb-slot is marked with an apostrophe in these lessons. 

\begin{supertabular}{p{5cm}|ll}
mi moku. && I eat.  \\
mi ' moku. && I am food. \\  
sina pona. && You fix. \\
sina ' pona. && You are good. \\  
\end{supertabular} 

Because Toki Pona lacks 'to be', the exact meaning is lost. 
\textit{moku} in this sentence could be a verb, or it could be a noun; just as \textit{pona} could be an adjective or could be a verb. 
In situations such as these, the listener must rely on context. 
After all, how often do you hear someone say 'I am food.'? 
I hope not very often! You can be fairly certain that \textit{mi moku} means 'I'm eating'. 
%
%%%%%%%%%%%%%%%%%%%%%%%%%%%%%%%%%%%%%%%%%%%%%%%%%%%%%%%%%%%%%%%%%%%%%%%%%%
\index{\textit{li}}
\index{subject!\textit{mi}}
\index{subject!\textit{sina}}
\index{\textit{mi}!subject}
\index{\textit{sina}!subject}
\subsection*{The Separator \textit{li} }
%%%%%%%%%%%%%%%%%%%%%%%%%%%%%%%%%%%%%%%%%%%%%%%%%%%%%%%%%%%%%%%%%%%%%%%%%%
%
For sentences that don't use the personal pronouns \textit{mi} or \textit{sina} as the subject, there is one small catch that you'll have to learn. 
Look at how \textit{li} is used. 
\textit{li} is a grammatical word that separates the subject phrase from the predicate phrase. 
Th predicate marker \textit{li} is only used when the subject is not \textit{mi} or \textit{sina}. 
Although the separator \textit{li} might seem worthless right now, as you continue to learn Toki Pona you will see that some sentences could be very confusing if \textit{li} weren't there. 

\begin{supertabular}{p{5cm}|ll}
telo li pona. && Water is cleaning. \\
suno li suno. && The sun is shining. \\
moku li ' pona. && The food is good. \\ 
ona li ' moku. && It is food. \\
\end{supertabular} 

Because of the lacks of 'to be', after \textit{li} can not only be a verb, it can follow a noun or adjective as well. 
As already written, an empty verb slot cannot form a predicate phrase on its own. 
A noun or adjective must follow. 
That is, directly after \textit{li} the sentence can not yet be finished or an object can follow.
%
\newpage
%%%%%%%%%%%%%%%%%%%%%%%%%%%%%%%%%%%%%%%%%%%%%%%%%%%%%%%%%%%%%%%%%%%%%%%%%%
\subsection*{Practice (Answers: Page~\pageref{'basic_sentences'})}
%%%%%%%%%%%%%%%%%%%%%%%%%%%%%%%%%%%%%%%%%%%%%%%%%%%%%%%%%%%%%%%%%%%%%%%%%%
%

Please write down your answers and check them afterwards. 

\begin{supertabular}{p{5cm}|ll}
What is a verb &&  \\ % no-dictionary
What is a noun? &&   \\ % no-dictionary
What is \textit{li} used for?  &&   \\ % no-dictionary
What does a personal pronoun replace? &&  \\ % no-dictionary
How to recognize nouns, pronouns, verbs and adjectives in \textit{toki pona}? &&  \\ % no-dictionary
What is a subject?  &&   \\ % no-dictionary
After which subject phrases is \textit{li} not used?  &&  \\ % no-dictionary
Where does the subject stand in the sentence?  &&    \\ % no-dictionary
Can an empty verb slot alone form a predicate? &&    \\ % no-dictionary
When can a verb slot be empty?  &&     \\ % no-dictionary
What is a predicate?  &&     \\ % no-dictionary
A complete sentence in \textit{toki pona} always contains\dots  &&     \\ % no-dictionary
What kinds of words can be used in \textit{toki pona} to form a predicate? &&   \\ % no-dictionary
What is an adjective?  &&    \\ % no-dictionary
Where are possible adjective slots?  &&    \\  % no-dictionary
Why can't a sentence be ended after \textit{li}? &&  \\ % no-dictionary
\end{supertabular} 

Try to translate these sentences. 
You can use the tool \textit{Toki Pona Parser} (\cite{www:rowa:02}) for spelling and grammar check. 

\begin{supertabular}{p{5cm}|ll}
People are good. && \\ % no-dictionary
I'm eating. &&  \\ % no-dictionary
You're tall. &&  \\ % no-dictionary
Water is simple. &&  \\ % no-dictionary
The lake is big. &&\\ % no-dictionary
 && \\ % no-dictionary
suno li ' suli. &&  \\% no-dictionary
mi ' suli. &&  \\% no-dictionary
jan li moku. &&  \\% no-dictionary
\end{supertabular} \\% no-dictionary
%
%%%%%%%%%%%%%%%%%%%%%%%%%%%%%%%%%%%%%%%%%%%%%%%%%%%%%%%%%%%%%%%%%%%%%%%%%%
% eof
