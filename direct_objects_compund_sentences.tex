%%%%%%%%%%%%%%%%%%%%%%%%%%%%%%%%%%%%%%%%%%%%%%%%%%%%%%%%%%%%%%%%%%%%%%%%%%
\section{Direct Objects, Compound Sentences}
%
%%%%%%%%%%%%%%%%%%%%%%%%%%%%%%%%%%%%%%%%%%%%%%%%%%%%%%%%%%%%%%%%%%%%%%%%%%
\subsection*{Vocabulary}
%%%%%%%%%%%%%%%%%%%%%%%%%%%%%%%%%%%%%%%%%%%%%%%%%%%%%%%%%%%%%%%%%%%%%%%%%%
%
\index{\textit{ilo}}
\index{\textit{kili}}
\index{\textit{ni}}
\index{\textit{ona}}
\index{\textit{pipi}}
\index{\textit{ma}}
\index{\textit{ijo}}             
\index{\textit{jo}}
\index{\textit{lukin}}
\index{\textit{pakala}}
\index{\textit{unpa}}
\index{\textit{wile}}
\index{\textit{e}}
\index{tool}
\index{device}
\index{machine}
\index{fruit}
\index{vegetable}
\index{this}
\index{that}
\index{he}
\index{she}
\index{it}
\index{bug}
\index{insect}
\index{spider}
\index{land}
\index{country}
\index{region}
\index{outside area}
\index{something}
\index{anything}
\index{stuff}
\index{thing}
\index{have}
\index{ownership}
\index{possession}
\index{see}
\index{look at}
\index{vision}
\index{sight}
\index{mess up}
\index{destroy}
\index{accident}
\index{sex}
\index{sexual}
\index{want}
\index{need}
\index{desire}
\begin{supertabular}{p{2,5cm}|ll}
\textbf{\dots ona} && \textit{possessive pronoun}: her, his, its \\  % no-dictionary
\textbf{ona} && \textit{pronoun}: she, he, it, they \\ % no-dictionary
 && \\ % no-dictionary
\textbf{\dots kili} && \textit{adjective}: fruity \\ % no-dictionary
\textbf{\dots kili} && \textit{adverb}: fruity \\ % no-dictionary
\textbf{kili} && \textit{noun}: fruit, pulpy vegetable, mushroom \\ % no-dictionary
 && \\ % no-dictionary
\textbf{\dots e \dots} && \textit{separator}: An 'e' introduces a direct object. \\ && Don't use 'e' before or after the other separators. \\ % no-dictionary
 && \\ % no-dictionary
%\textbf{\dots wile} && \textit{adjective}: necessary \\ % no-dictionary
%\textbf{\dots wile} && \textit{adverb}: necessary \\ % no-dictionary
\textbf{wile} && \textit{noun}: desire, need, will \\ % no-dictionary
\textbf{wile (e \dots)} && \textit{verb transitive}: to want, need, wish, have to, must, will, should \\ % no-dictionary
\textbf{wile \dots} && \textit{verb pre}: to want, need, wish, have to, must, will, should \\ % no-dictionary
 && \\ % no-dictionary
\textbf{\dots ilo} && \textit{adjective}: useful \\ % no-dictionary
\textbf{\dots ilo} && \textit{adverb}: usefully \\ % no-dictionary
\textbf{ilo} && \textit{noun}: tool, device, machine, thing used for a specific purpose \\ % no-dictionary
 && \\ % no-dictionary
\textbf{\dots ijo} && \textit{adjective}: of something \\ % no-dictionary
\textbf{\dots ijo} && \textit{adverb}: of something \\ % no-dictionary
\textbf{ijo} && \textit{noun}: thing, something, stuff, anything, object \\ % no-dictionary
\textbf{ijo (e \dots)} && \textit{verb transitive}: to objectify \\ % no-dictionary
 && \\ % no-dictionary
\textbf{\dots ni} && \textit{adjective}: this, that \\ % no-dictionary
\textbf{ni} && \textit{pronoun}: this, that \\ % no-dictionary
 && \\ % no-dictionary
\textbf{pipi} && \textit{noun}: bug, insect, spider \\ % no-dictionary
 && \\ % no-dictionary
\textbf{\dots ma} && \textit{adjective}: countrified, outdoor, alfresco, open-air \\ % no-dictionary
\textbf{ma} && \textit{noun}: land, earth, country, (outdoor) area \\ % no-dictionary
 && \\ % no-dictionary
\textbf{\dots jo} && \textit{adjective}: private, personal \\ % no-dictionary
\textbf{jo} && \textit{noun}: having, possessions, content \\ % no-dictionary
\textbf{jo (e \dots)} && \textit{verb transitive}: to have, to contain \\ % no-dictionary
 && \\ % no-dictionary
\textbf{\dots lukin} && \textit{adjective}: visual(ly) \\ % no-dictionary
\textbf{\dots lukin} && \textit{adverb}: visual(ly) \\ % no-dictionary
\textbf{lukin} && \textit{noun}: view, look, glance, sight, gaze, glimpse, seeing, vision \\ % no-dictionary
\textbf{lukin} && \textit{verb intransitive}: to look, to watch out, to pay attention \\ % no-dictionary
\textbf{lukin (e \dots)} && \textit{verb transitive}: to see, to look at, to watch, to read \\ % no-dictionary
\textbf{lukin \dots} && \textit{verb pre}: to seek to, try to, look for \\ % no-dictionary
 && \\ % no-dictionary
\textbf{\dots pakala} && \textit{adjective}: destroyed, ruined, demolished, shattered, wrecked \\ % no-dictionary
\textbf{\dots pakala} && \textit{adverb}: destroyed, ruined, demolished, shattered, wrecked \\ % no-dictionary
\textbf{pakala} && \textit{noun}: blunder, accident, mistake, destruction, damage, breaking \\ % no-dictionary
\textbf{pakala} && \textit{verb intransitive}: to screw up, to fall apart, to break \\ % no-dictionary
\textbf{pakala (e \dots)} && \textit{verb transitive}: to screw up, to ruin, to break, to hurt, to injure, to damage \\ % no-dictionary
 && \\ % no-dictionary
\textbf{\dots unpa} && \textit{adjective}: erotic, sexual \\ % no-dictionary
\textbf{\dots unpa} && \textit{adverb}: erotic, sexual \\ % no-dictionary
\textbf{unpa} && \textit{noun}: sex, sexuality \\ % no-dictionary
\textbf{unpa} && \textit{verb intransitive}: to have sex \\ % no-dictionary
\textbf{unpa (e \dots)} && \textit{verb transitive}: to have sex with, to sleep with, to fuck \\ % no-dictionary
\end{supertabular} \\
%
%%%%%%%%%%%%%%%%%%%%%%%%%%%%%%%%%%%%%%%%%%%%%%%%%%%%%%%%%%%%%%%%%%%%%%%%%%
\index{\textit{e}}
\index{object!direct}
\subsection*{Direct objects using \textit{e}}
%%%%%%%%%%%%%%%%%%%%%%%%%%%%%%%%%%%%%%%%%%%%%%%%%%%%%%%%%%%%%%%%%%%%%%%%%%
%
We saw how phrases such as \textit{mi moku} could have two potential meanings. 
'I'm eating' or 'I am food'. 
There is one way to specify that you want to say. 

\begin{supertabular}{p{5,5cm}|ll}
mi moku e kili. && I eat fruit. \\
\end{supertabular} 

Whatever is getting action done on itself is the 'direct object,' and in Toki Pona, we separate the verb and the direct object with \textit{e}}. 

\begin{supertabular}{p{5,5cm}|ll}
ona li lukin e pipi. && He's watching the bug. \\
\end{supertabular} 

Also we discussed how \textit{sina pona}, like \textit{mi moku}, has two possible meanings. 'You are good' or 'You're fixing'. 
Normally, it would mean 'You are good' simply because no one really says 'I'm fixing' without actually telling what it is that they are trying to fix. 

\begin{supertabular}{p{5,5cm}|ll}
ona li pona e ilo. && She's fixing the machine. \\
mi pona e ijo. && I'm fixing something. \\
\end{supertabular} 

Only a (composite) verb can stand in front of the \textit{e}. 
More specifically, it is a transitive verb. 
A transitive verb does something to the direct object. 
Your can ask for direct object by' Who' or' What' (' What does she repair?'').
The \textit{e} is at the beginning of the direct object. 
After the \textit{e} there is always a noun or pronoun. 
The direct object is part of the verb phrase. 
%
%%%%%%%%%%%%%%%%%%%%%%%%%%%%%%%%%%%%%%%%%%%%%%%%%%%%%%%%%%%%%%%%%%%%%%%%%%
% \newpage
\index{\textit{e}!\textit{wile}}
\index{\textit{wile}!\textit{e}}
\subsection*{Direct objects using \textit{e} with \textit{wile}}
%%%%%%%%%%%%%%%%%%%%%%%%%%%%%%%%%%%%%%%%%%%%%%%%%%%%%%%%%%%%%%%%%%%%%%%%%%
%
To say that you want to do something definite, use \textit{wile} and \textit{e}.
\textit{wile} is here an auxiliary verb. 
An auxiliary verb is placed in front of the main verb and supplements it. 
An auxiliary verb belongs to the verb phrase. 

\begin{supertabular}{p{5,5cm}|ll}
mi wile lukin e ma. && I want to see the countryside. \\
mi wile pakala e sina. && I must destroy you. \\
ona li wile jo e ilo. && He would like to have a tool. \\
\end{supertabular} 

As you can see, \textit{e} doesn't come until after the infinitive in these two sentences, rather than before it. 

%%%%%%%%%%%%%%%%%%%%%%%%%%%%%%%%%%%%%%%%%%%%%%%%%%%%%%%%%%%%%%%%%%%%%%%%%%
\newpage
\index{sentence!compound}
\index{sentence!compound!\textit{li}}
\index{sentence!compound!\textit{e}}
\index{\textit{li}!compound sentences}
\index{\textit{e}!compound sentences}
\label{'multiple_li'}
\subsection*{Compound sentences}
%
There are two ways to make compound sentences in Toki Pona; one way involves using \textit{li}, and the other way involves using \textit{e}. 
Since you've now studied both of these words, we'll cover how to use both of them to make compound sentences. 

%%%%%%%%%%%%%%%%%%%%%%%%%%%%%%%%%%%%%%%%%%%%%%%%%%%%%%%%%%%%%%%%%%%%%%%%%%
\subsubsection*{multiple \textit{li} technique}
%%%%%%%%%%%%%%%%%%%%%%%%%%%%%%%%%%%%%%%%%%%%%%%%%%%%%%%%%%%%%%%%%%%%%%%%%%
%

\begin{supertabular}{p{5,5cm}|ll}
pipi li lukin li unpa. && The bug looks and has sex. \\
\end{supertabular} 

By putting \textit{li} before each verb, you can show how the subject, which is \textit{pipi} in this case, does more than one thing. 
Each \textit{li} starts a new verb phrase. 

\begin{supertabular}{p{5,5cm}|ll}
mi moku li pakala. && I eat and destroy. \\
\end{supertabular} 

While \textit{li} is still omitted before \textit{moku} because the subject of the sentence is \textit{mi}, we still use it before the second verb, \textit{pakala}. 
Without the \textit{li} there, the sentence would be chaotic and confusing. 
Compound sentences with \textit{sina} follow this same pattern. 
\textit{li} phrases are not nested. You can change the order. \\
ona li moku li pona. = ona li pona li moku. 

Each verb phrase can of course contain direct objects. 

\begin{supertabular}{p{5,5cm}|ll}
mi moku e moku li lukin e ma. && I eat the food and look at the landscape. \\
\end{supertabular} 

The official Toki Pona book recommends to use only a verb phrase for \textit{mi} or \textit{sina} as subject. 
%
%%%%%%%%%%%%%%%%%%%%%%%%%%%%%%%%%%%%%%%%%%%%%%%%%%%%%%%%%%%%%%%%%%%%%%%%%%
\label{'multiple_e'}
\subsubsection*{multiple \textit{e} technique}
%%%%%%%%%%%%%%%%%%%%%%%%%%%%%%%%%%%%%%%%%%%%%%%%%%%%%%%%%%%%%%%%%%%%%%%%%%
%
The other type of compound sentence is used when there are several direct objects of the same verb.
In this case, all direct objects belong to a verb phrase. 

\begin{supertabular}{p{5,5cm}|ll}
mi moku e kili e telo. && I eat/drink fruit and water. \\
\end{supertabular} 

\textit{e} is used multiple times because \textit{kili} and \textit{telo} are both direct objects, and so \textit{e} must precede them both. 

\begin{supertabular}{p{5,5cm}|ll}
mi wile lukin e ma e suno. && I want to see the land and the sun. \\
\end{supertabular} 
\textit{e} phrases are not nested. You can change the order. \\
mi moku e moku e telo. = mi moku e telo e moku. 

We can combine several \textit{li} and \textit{e}.
We have two verb phrases with two direct objects each. 
However, it is better to use several short sentences. 

\begin{supertabular}{p{5,5cm}|ll}
mi moku e kili e telo li lukin e ma e jan. I eat fruits and water and see land and people. \\
\end{supertabular} 
%
\newpage
%%%%%%%%%%%%%%%%%%%%%%%%%%%%%%%%%%%%%%%%%%%%%%%%%%%%%%%%%%%%%%%%%%%%%%%%%%
\subsection*{Practice 4 (Answers: Page~\pageref{'direct_objects_compund_sentences'})}
%%%%%%%%%%%%%%%%%%%%%%%%%%%%%%%%%%%%%%%%%%%%%%%%%%%%%%%%%%%%%%%%%%%%%%%%%%
%
Try to translate these sentences. 
You can use the tool \textit{Toki Pona Parser} (\cite{www:rowa:02}) for spelling and grammar check. 

\begin{supertabular}{p{5,5cm}|ll}
I have a tool. &&  \\ % no-dictionary
She's eating fruit. &&  \\ % no-dictionary
Something is watching me. &&  \\ % no-dictionary
He wants to squish the spider. &&  \\ % no-dictionary
Pineapple is a food and is good. &&  \\ % no-dictionary
The bug is thirsty. && \\ % no-dictionary
  && \\ % no-dictionary
mi lukin e ni. &&  \\ % no-dictionary
mi wile unpa e ona. &&   \\ % no-dictionary
jan li wile jo e ma. &&  \\ % no-dictionary
mi ' jan li ' suli. &&  \\ % no-dictionary
\end{supertabular} 
%%%%%%%%%%%%%%%%%%%%%%%%%%%%%%%%%%%%%%%%%%%%%%%%%%%%%%%%%%%%%%%%%%%%%%%%%%
% eof
