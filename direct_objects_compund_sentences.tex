%%%%%%%%%%%%%%%%%%%%%%%%%%%%%%%%%%%%%%%%%%%%%%%%%%%%%%%%%%%%%%%%%%%%%%%%%%
\section{Direct Objects, Compound Sentences}
%
%%%%%%%%%%%%%%%%%%%%%%%%%%%%%%%%%%%%%%%%%%%%%%%%%%%%%%%%%%%%%%%%%%%%%%%%%%
\subsection*{Vocabulary}
%%%%%%%%%%%%%%%%%%%%%%%%%%%%%%%%%%%%%%%%%%%%%%%%%%%%%%%%%%%%%%%%%%%%%%%%%%
%
\index{\textit{ilo}}
\index{\textit{kili}}
\index{\textit{ni}}
\index{\textit{ona}}
\index{\textit{ma}}
\index{\textit{ijo}}             
\index{\textit{jo}}
\index{\textit{lukin}}
\index{\textit{pakala}}
\index{\textit{unpa}}
\index{\textit{wile}}
\index{\textit{e}}
\index{tool}
\index{device}
\index{machine}
\index{fruit}
\index{vegetable}
\index{this}
\index{that}
\index{he}
\index{she}
\index{it}
\index{bug}
\index{insect}
\index{spider}
\index{land}
\index{country}
\index{region}
\index{outside area}
\index{something}
\index{anything}
\index{stuff}
\index{thing}
\index{have}
\index{ownership}
\index{possession}
\index{see}
\index{look at}
\index{vision}
\index{sight}
\index{mess up}
\index{destroy}
\index{accident}
\index{sex}
\index{sexual}
\index{want}
\index{need}
\index{desire}
\begin{supertabular}{p{2,5cm}|ll}
\textbf{\dots e \dots} && \textit{separator}: An 'e' introduces a direct object. \\ && Don't use 'e' before or after the other separators. \\ % no-dictionary
 && \\ % no-dictionary
\textbf{\dots ijo} && \textit{adjective}: of something \\ % no-dictionary
\textbf{\dots ijo} && \textit{adverb}: of something \\ % no-dictionary
\textbf{ijo} && \textit{noun}: thing, something, stuff, anything, object \\ % no-dictionary
\textbf{ijo (e \dots)} && \textit{verb transitive}: to objectify \\ % no-dictionary
 && \\ % no-dictionary
\textbf{\dots ilo} && \textit{adjective}: useful \\ % no-dictionary
\textbf{\dots ilo} && \textit{adverb}: usefully \\ % no-dictionary
\textbf{ilo} && \textit{noun}: tool, device, machine, thing used for a specific purpose \\ % no-dictionary
 && \\ % no-dictionary
\textbf{\dots jo} && \textit{adjective}: private, personal \\ % no-dictionary
\textbf{jo} && \textit{noun}: having, possessions, content \\ % no-dictionary
\textbf{jo (e \dots)} && \textit{verb transitive}: to have, to contain \\ % no-dictionary
 && \\ % no-dictionary
\textbf{\dots kili} && \textit{adjective}: fruity \\ % no-dictionary
\textbf{\dots kili} && \textit{adverb}: fruity \\ % no-dictionary
\textbf{kili} && \textit{noun}: fruit, pulpy vegetable, mushroom \\ % no-dictionary
 && \\ % no-dictionary
\textbf{\dots lukin} && \textit{adjective}: visual(ly) \\ % no-dictionary
\textbf{\dots lukin} && \textit{adverb}: visual(ly) \\ % no-dictionary
\textbf{lukin} && \textit{noun}: view, look, glance, sight, gaze, glimpse, seeing, vision \\ % no-dictionary
\textbf{lukin} && \textit{verb intransitive}: to look, to watch out, to pay attention \\ % no-dictionary
\textbf{lukin (e \dots)} && \textit{verb transitive}: to see, to look at, to watch, to read \\ % no-dictionary
\textbf{lukin \dots} && \textit{auxiliary verb}: to seek to, try to, look for \\ % no-dictionary
 && \\ % no-dictionary
\textbf{\dots ma} && \textit{adjective}: countrified, outdoor, alfresco, open-air \\ % no-dictionary
\textbf{ma} && \textit{noun}: land, earth, country, (outdoor) area \\ % no-dictionary
 && \\ % no-dictionary
\textbf{\dots ni} && \textit{adjective demonstrative pronoun}: this, that \\ % no-dictionary
\textbf{ni} && \textit{noun demonstrative pronoun}: this, that \\ % no-dictionary
 && \\ % no-dictionary
\textbf{\dots pakala} && \textit{adjective}: destroyed, ruined, demolished, shattered, wrecked \\ % no-dictionary
\textbf{\dots pakala} && \textit{adverb}: destroyed, ruined, demolished, shattered, wrecked \\ % no-dictionary
\textbf{pakala} && \textit{noun}: blunder, accident, mistake, destruction, damage, breaking \\ % no-dictionary
\textbf{pakala} && \textit{verb intransitive}: to screw up, to fall apart, to break \\ % no-dictionary
\textbf{pakala (e \dots)} && \textit{verb transitive}: to screw up, to ruin, to break, to hurt, to injure, to damage \\ % no-dictionary
 && \\ % no-dictionary
\textbf{\dots unpa} && \textit{adjective}: erotic, sexual \\ % no-dictionary
\textbf{\dots unpa} && \textit{adverb}: erotic, sexual \\ % no-dictionary
\textbf{unpa} && \textit{noun}: sex, sexuality \\ % no-dictionary
\textbf{unpa} && \textit{verb intransitive}: to have sex \\ % no-dictionary
\textbf{unpa (e \dots)} && \textit{verb transitive}: to have sex with, to sleep with, to fuck \\ % no-dictionary
 && \\ % no-dictionary
%\textbf{\dots wile} && \textit{adjective}: necessary \\ % no-dictionary
%\textbf{\dots wile} && \textit{adverb}: necessary \\ % no-dictionary
\textbf{wile} && \textit{noun}: desire, need, will \\ % no-dictionary
\textbf{wile (e \dots)} && \textit{verb transitive}: to want, need, wish, have to, must, will, should \\ % no-dictionary
\textbf{wile \dots} && \textit{auxiliary verb}: to want, need, wish, have to, must, will, should \\ % no-dictionary
\end{supertabular} \\
%
\newpage
%%%%%%%%%%%%%%%%%%%%%%%%%%%%%%%%%%%%%%%%%%%%%%%%%%%%%%%%%%%%%%%%%%%%%%%%%%
\index{\textit{e}}
\index{object!direct}
\index{verb!transitive}
\index{transitive verb}
\subsection*{Transitive Verbs, the Separator \textit{e} and Direct Objects}
%%%%%%%%%%%%%%%%%%%%%%%%%%%%%%%%%%%%%%%%%%%%%%%%%%%%%%%%%%%%%%%%%%%%%%%%%%
%
We saw how phrases such as \textit{mi moku} could have two potential meanings. 
'I'm eating' or 'I am food'. 
There is one way to specify that you want to say. 

\begin{supertabular}{p{5,5cm}|ll}
mi moku e kili. && I eat fruit. \\
\end{supertabular} 

Also we discussed how \textit{sina pona}, like \textit{mi moku}, has two possible meanings. 'You are good' or 'You're fixing'. 
Normally, it would mean 'You are good' simply because no one really says 'I'm fixing' without actually telling what it is that they are trying to fix. 

\begin{supertabular}{p{5,5cm}|ll}
ona li pona e ilo. && She's fixing the machine. \\
mi pona e ijo. && I'm fixing something. \\
\end{supertabular} 

Only a (composite) verb can stand in front of the \textit{e}. 
More specifically, it is a slot for a transitive verb. 
A transitive verb does something to the direct object. 

The separator \textit{e} preface the direct object.
An object is an optional record supplement. 
A direct object is most strongly influenced by the action (i. e. the predicate).
Your can ask for direct object (accusative object) by' Who' or' What' (' What does she repair?'').
The direct object is part of the predicate phrase. 

In the direct object is the first slot after the separator \textit{e} is always a noun or pronoun slot.
In the above examples the noun slots were filled with \textit{kili} and \textit{ijo}. 

%
%%%%%%%%%%%%%%%%%%%%%%%%%%%%%%%%%%%%%%%%%%%%%%%%%%%%%%%%%%%%%%%%%%%%%%%%%%
\index{reflexive pronoun}
\index{pronoun!reflexive}
\subsection*{Reflexive Pronouns}
%%%%%%%%%%%%%%%%%%%%%%%%%%%%%%%%%%%%%%%%%%%%%%%%%%%%%%%%%%%%%%%%%%%%%%%%%%

A reflexive pronoun represents the subject in the direct object. 
So a slot for a reflective pronoun is located after the separator \textit{e}. 
In the following example, \textit{ona} is a reflexive pronoun, since it refers to the subject \textit{jan}. 

\begin{supertabular}{p{5,5cm}|ll}
jan li telo e ona. && A person washes himself. \\
\end{supertabular}

In this sentence the first \textit{mi} is a personal pronoun.
The \textit{mi} on the \textit{e} is a reflexive pronoun. 

\begin{supertabular}{p{5,5cm}|ll}
mi telo e mi. && I wash myself. \\
\end{supertabular}

Here a sentence with \textit{sina} as personal and reflective pronouns

\begin{supertabular}{p{5,5cm}|ll}
sina telo e sina. && You wash yourself. \\
\end{supertabular}

%
%%%%%%%%%%%%%%%%%%%%%%%%%%%%%%%%%%%%%%%%%%%%%%%%%%%%%%%%%%%%%%%%%%%%%%%%%%
\index{demonstrative pronoun}
\index{pronoun!demonstrative}
%\subsubsection*{The Demonstrative Pronoun \textit{ni}}
%%%%%%%%%%%%%%%%%%%%%%%%%%%%%%%%%%%%%%%%%%%%%%%%%%%%%%%%%%%%%%%%%%%%%%%%%%
%

The demonstrative pronoun is a kind of word with which the speaker refers to an  item of conversation. 
The demonstrative pronoun \textit{ni} can be used both like an adjective and like a noun.  
A slot for an adjective demonstrative pronoun is therefore possible after a noun. 

\begin{supertabular}{p{5,5cm}|ll}
jan ni li pona. && This bloke is good. \\
jan li lukin e ijo ni. && The guy's looking at this thing. \\
\end{supertabular}

A noun demonstrative pronoun is used instead of the noun. 
Slots for noun demonstrative pronouns therefore correspond to the positions of noun slots in the sentence. 

\begin{supertabular}{p{5,5cm}|ll}
ni li pona... && This is good. \\
jan li lukin e ni. .... the guy looks at that one. \\
\end{supertabular}


%
%%%%%%%%%%%%%%%%%%%%%%%%%%%%%%%%%%%%%%%%%%%%%%%%%%%%%%%%%%%%%%%%%%%%%%%%%%
% \newpage
\index{\textit{e}!\textit{wile}}
\index{\textit{wile}!\textit{e}}
\subsection*{Auxiliary Verbs}
%%%%%%%%%%%%%%%%%%%%%%%%%%%%%%%%%%%%%%%%%%%%%%%%%%%%%%%%%%%%%%%%%%%%%%%%%%
%
To say that you want to do something definite, use \textit{wile}.
\textit{wile} is here an auxiliary verb. 
An auxiliary verb is placed in front of the main verb and supplements it. 
An auxiliary verb belongs to the predicate phrase. 

\begin{supertabular}{p{5,5cm}|ll}
mi wile lukin e ma. && I want to see the countryside. \\
mi wile pakala e sina. && I must destroy you. \\
ona li wile jo e ilo. && He would like to have a tool. \\
\end{supertabular} 

As you can see, \textit{e} doesn't come until after the infinitive in these two sentences, rather than before it. 

%%%%%%%%%%%%%%%%%%%%%%%%%%%%%%%%%%%%%%%%%%%%%%%%%%%%%%%%%%%%%%%%%%%%%%%%%%
% \newpage
\index{sentence!compound}
\index{sentence!compound!\textit{li}}
\index{sentence!compound!\textit{e}}
\index{\textit{li}!compound sentences}
\index{\textit{e}!compound sentences}
\label{'multiple_li'}
\subsection*{Compound Sentences}
%
There are two ways to make compound sentences in Toki Pona; one way involves using \textit{li}, and the other way involves using \textit{e}. 
Since you've now studied both of these words, we'll cover how to use both of them to make compound sentences. 

%%%%%%%%%%%%%%%%%%%%%%%%%%%%%%%%%%%%%%%%%%%%%%%%%%%%%%%%%%%%%%%%%%%%%%%%%%
\subsubsection*{Several \textit{li} Separators}
%%%%%%%%%%%%%%%%%%%%%%%%%%%%%%%%%%%%%%%%%%%%%%%%%%%%%%%%%%%%%%%%%%%%%%%%%%
%

\begin{supertabular}{p{5,5cm}|ll}
ona li lukin li unpa. && He looks and has sex. \\
\end{supertabular} 

By putting \textit{li} before each verb, you can show how the subject, which is \textit{ona} in this case, does more than one thing. 
Each \textit{li} starts a new predicate phrase. 

\begin{supertabular}{p{5,5cm}|ll}
mi moku li pakala. && I eat and destroy. \\
\end{supertabular} 

While \textit{li} is still omitted before \textit{moku} because the subject of the sentence is \textit{mi}, we still use it before the second verb, \textit{pakala}. 
Without the \textit{li} there, the sentence would be chaotic and confusing. 
Compound sentences with \textit{sina} follow this same pattern. 
\textit{li} phrases are not nested. You can change the order. \\
ona li moku li pona. = ona li pona li moku. 

Each predicate phrase can of course contain direct objects. 

\begin{supertabular}{p{5,5cm}|ll}
mi moku e moku li lukin e ma. && I eat the food and look at the landscape. \\
\end{supertabular} 

The official Toki Pona book recommends to use only a predicate phrase for \textit{mi} or \textit{sina} as subject. 
%
%%%%%%%%%%%%%%%%%%%%%%%%%%%%%%%%%%%%%%%%%%%%%%%%%%%%%%%%%%%%%%%%%%%%%%%%%%
\label{'multiple_e'}
\subsubsection*{Several \textit{e} Separators}
%%%%%%%%%%%%%%%%%%%%%%%%%%%%%%%%%%%%%%%%%%%%%%%%%%%%%%%%%%%%%%%%%%%%%%%%%%
%
The other type of compound sentence is used when there are several direct objects of the same verb.
In this case, all direct objects belong to a predicate phrase. 

\begin{supertabular}{p{5,5cm}|ll}
mi moku e kili e telo. && I eat/drink fruit and water. \\
\end{supertabular} 

\textit{e} is used multiple times because \textit{kili} and \textit{telo} are both direct objects, and so \textit{e} must precede them both. 

\begin{supertabular}{p{5,5cm}|ll}
mi wile lukin e ma e suno. && I want to see the land and the sun. \\
\end{supertabular} 
\textit{e} phrases are not nested. You can change the order. \\
mi moku e moku e telo. = mi moku e telo e moku. 

We can combine several \textit{li} and \textit{e}.
We have two predicate phrases with two direct objects each. 
However, it is better to use several short sentences. 

\begin{supertabular}{p{5,5cm}|ll}
mi moku e kili e telo li lukin e ma e jan. I eat fruits and water and see land and people. \\
\end{supertabular} 

%
\newpage
%%%%%%%%%%%%%%%%%%%%%%%%%%%%%%%%%%%%%%%%%%%%%%%%%%%%%%%%%%%%%%%%%%%%%%%%%%
\subsection*{Practice 4 (Answers: Page~\pageref{'direct_objects_compund_sentences'})}
%%%%%%%%%%%%%%%%%%%%%%%%%%%%%%%%%%%%%%%%%%%%%%%%%%%%%%%%%%%%%%%%%%%%%%%%%%
%
Please write down your answers and check them afterwards. 

% What does a possessive pronoun replace? && It replaces a adjective. \\ % no-dictionary

Try to translate these sentences. 
You can use the tool \textit{Toki Pona Parser} (\cite{www:rowa:02}) for spelling and grammar check. 

\begin{supertabular}{p{5,5cm}|ll}
I have a tool. &&  \\ % no-dictionary
She's eating fruit. &&  \\ % no-dictionary
Something is watching me. &&  \\ % no-dictionary
He wants to destroy the tool. &&  \\ % no-dictionary
Pineapple is a food and is good. &&  \\ % no-dictionary
She is thirsty. && \\ % no-dictionary
He washes himself. &&  \\ % & English - Toki Pona
  && \\ % no-dictionary
mi lukin e ni. &&  \\ % no-dictionary
mi wile unpa e ona. &&   \\ % no-dictionary
jan li wile jo e ma. &&  \\ % no-dictionary
mi ' jan li ' suli. &&  \\ % no-dictionary
\end{supertabular} 
%%%%%%%%%%%%%%%%%%%%%%%%%%%%%%%%%%%%%%%%%%%%%%%%%%%%%%%%%%%%%%%%%%%%%%%%%%
% eof
